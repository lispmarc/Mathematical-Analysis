\documentclass{book}
\usepackage[english]{babel}
\usepackage{amsmath,amssymb,stmaryrd,enumerate,theorem,latexsym,makeidx}
\makeindex

%%%%%%%%%% Start TeXmacs macros
\catcode`\<=\active \def<{
\fontencoding{T1}\selectfont\symbol{60}\fontencoding{\encodingdefault}}
\catcode`\>=\active \def>{
\fontencoding{T1}\selectfont\symbol{62}\fontencoding{\encodingdefault}}
\newcommand{\Alpha}{\mathrm{A}}
\newcommand{\Leftrightarrowlim}{\mathop{\leftrightarrow}\limits}
\newcommand{\Rightarrowlim}{\mathop{\rightarrow}\limits}
\newcommand{\equallim}{\mathop{=}\limits}
\newcommand{\infixand}{\text{ and }}
\newcommand{\nin}{\not\in}
\newcommand{\nobracket}{}
\newcommand{\of}{:}
\newcommand{\tmdummy}{$\mbox{}$}
\newcommand{\tmmathbf}[1]{\ensuremath{\boldsymbol{#1}}}
\newcommand{\tmop}[1]{\ensuremath{\operatorname{#1}}}
\newcommand{\tmtextbf}[1]{\text{{\bfseries{#1}}}}
\newenvironment{enumeratealpha}{\begin{enumerate}[a{\textup{)}}] }{\end{enumerate}}
\newenvironment{itemizedot}{\begin{itemize} \renewcommand{\labelitemi}{$\bullet$}\renewcommand{\labelitemii}{$\bullet$}\renewcommand{\labelitemiii}{$\bullet$}\renewcommand{\labelitemiv}{$\bullet$}}{\end{itemize}}
\newenvironment{proof}{\noindent\textbf{Proof\ }}{\hspace*{\fill}$\Box$\medskip}
\newtheorem{axiom}{Axiom}
{\theorembodyfont{\rmfamily}\newtheorem{convention}{Convention}}
\newtheorem{corollary}{Corollary}
\newtheorem{definition}{Definition}
{\theorembodyfont{\rmfamily}\newtheorem{example}{Example}}
\newtheorem{lemma}{Lemma}
\newtheorem{notation}{Notation}
{\theorembodyfont{\rmfamily}\newtheorem{note}{Note}}
\newtheorem{proposition}{Proposition}
{\theorembodyfont{\rmfamily}\newtheorem{remark}{Remark}}
\newtheorem{theorem}{Theorem}
%%%%%%%%%% End TeXmacs macros

\begin{document}

TODO check next\chapter{Elements of set theory}

\section{Basic concepts about classes and sets}

Every book about mathematical subjects must be based on one form of set
theory. Because the focus of this book is on mathematical analysis instead of
the foundations of mathematics, I have decided to use Von Neumann's set theory
instead of the set theory of Fraenkel, Skolem and Zermelo. The benefit of Von
Neumann's theory is that it is nearer to the naive set theory of Cantor. This
book assumes that the basics of mathematical logic are understood, more
specifically that the reader knows the meaning of the following terms:
\begin{eqnarray*}
  \wedge & \tmop{meaning} & \tmop{and}\\
  \vee & \tmop{meaning} & \tmop{or}\\
  \neg & \tmop{meaning} & \tmop{not}\\
  \Rightarrow & \tmop{meaning} & \tmop{implies}\\
  \Leftrightarrow & \tmop{meaning} & \tmop{is} \tmop{equivalent} \tmop{with}\\
  \vdash, \vDash & \tmop{meaning} & \tmop{with}, \tmop{where}\\
  \forall & \tmop{meaning} & \prefixforall\\
  \exists & \tmop{meaning} & \tmop{there} \tmop{exists}\\
  \exists ! & \tmop{meaning} & \tmop{there} \tmop{exists} a \tmop{unique}
\end{eqnarray*}
and how to use them. Axiomatic set theory is based on two undefined concepts:
\tmtextbf{class} and the \tmtextbf{membership} relation between classes (noted
as $\in$). Intuitive you can think of a class as a collection and $x \in A$ to
mean that $x$ is part of the collection where $A$ stands for. We introduce
then axioms that state which are true statements about these undefined
concepts. Further we introduce different definitions that helps us to simplify
our notation. To start with, we define the concept of $\nin$ [not member of]

\begin{definition}
  Let A be a class then $x \nin A$ is equivalent with saying $\neg (x \in A)$.
  
\end{definition}

Next we define \tmtextbf{sets} and \tmtextbf{elements}, they are the same
thing. A \tmtextbf{set} or \tmtextbf{element} is something that is a member of
a class.

\begin{definition}
  \label{element is set}We say that a \tmtextbf{class} $x$ is a
  \tmtextbf{element} if $x \in A$ where $A$ is a class. Another name for a
  \tmtextbf{element} is a \tmtextbf{set}
\end{definition}

From here on we use the following convention: elements are noted in
\tmtextbf{lower-case} and classes are noted in \tmtextbf{upper-case}. Next we
define equality of classes.

\begin{definition}
  Let $A, B$ classes then we say that $A = B$ if and only if
  \[ \forall X \text{ we have } A \in X \Rightarrow B \in X \wedge B \in X
     \Rightarrow A \in X \]
  Less formally, two classes $A$ and $B$ are equal if every class that
  contains A or B must contains $B$ or $A$.
\end{definition}

Once we have defined equality we can define inequality

\begin{definition}
  Let $A$ and $B$ classes then $A \neq B$ is equivalent with $\neg (A = B)$
\end{definition}

If two classes are equal, we expect them to contain the same elements, this is
stated in the first set axiom.

\begin{axiom}[Axiom of extent]
  \label{axiom of extent}{\index{axiom of extent}}
  \[ A = B \Leftrightarrow [x \in A \Rightarrow x \in B \wedge x \in B
     \Rightarrow x \in A] \]
  Less formally $A$ is equal to $B$ if and only if ever element of $A$ is a
  element of $B$ and every element of $B$ is a element of $A$, in other words
  $A$ and $B$ have the same elements.
\end{axiom}

\begin{definition}
  Let $A$ and $B$ classes then $A$ is a sub-class of $B$ noted by $A \subseteq
  B$ iff
  \[ x \in A \Rightarrow x \in B \]
  So $A$ is a sub-class of $B$ iff every element of $A$ is also a element of
  $B$.
\end{definition}

\begin{definition}
  Let $A$ and $B$ classes then $A$ is a proper sub-class of $B$ noted by $A
  \subseteq B$ iff
  \[ x \in A \Rightarrow x \in B \wedge A \neq B \]
  So $A$ is a proper sub-class of $B$ iff $A$ is different from $B$ and every
  element of $A$ is also a element of $B$.
\end{definition}

\begin{theorem}
  \label{class properties (1)}Let $A, B, C$ be classes then the following
  holds:
  \begin{enumerate}
    \item A=A
    
    \item $A = B \Rightarrow B = A$
    
    \item $A = B \wedge B = C \Rightarrow A = C$
    
    \item $A \subseteq B \wedge B \subseteq A \Rightarrow A = B$
    
    \item $A \subseteq B \wedge B \subseteq C \Rightarrow A \subseteq C$
    
    \item $A = B \Rightarrow A \subseteq B$
  \end{enumerate}
\end{theorem}

\begin{proof}
  
  \begin{enumerate}
    \item $x \in A \Rightarrow x \in A$ and $x \in A \Rightarrow x \in A$ are
    obviously true, hence using the Axiom of Extent [axiom: \ref{axiom of
    extent}] it follows that $A = A$
    
    \item As $A = B$ we have using the Axiom of Extent [axiom: \ref{axiom of
    extent}] that $x \in A \Rightarrow x \in B \wedge x \in B \Rightarrow x
    \in A$ which is equivalent with $x \in B \Rightarrow x \in A \wedge x \in
    A \Rightarrow x \in B$. Using the Axiom of Extent [axiom: \ref{axiom of
    extent}] it follows that $B = A$
    
    \item As $A = B \wedge B = A$ we have by he Axiom of Extent [axiom:
    \ref{axiom of extent}] that
    \begin{eqnarray}
      x \in A & \Rightarrow & x \in B  \label{eq 1.1 001}\\
      x \in B & \Rightarrow & x \in A  \label{eq 1.2 001}\\
      x \in B & \Rightarrow & x \in C  \label{eq 1.3 001}\\
      x \in C & \Rightarrow & x \in B  \label{eq 1.4 001}
    \end{eqnarray}
    From [eq: \ref{eq 1.1 001}] and [eq: \ref{eq 1.3 001}] it follows that $x
    \in A \Rightarrow x \in C$ and from [eq: \ref{eq 1.4 001}] and [eq:
    \ref{eq 1.2 001}] it follows that $x \in C \Rightarrow x \in A$. Using the
    Axiom of Extent [axiom: \ref{axiom of extent}] it follows then that $A =
    C$.
    
    \item From $A \subseteq B \wedge B \subseteq A$ it follows that $x \in A
    \Rightarrow x \in B \wedge x \in B \Rightarrow x \in A$, so by the Axiom
    of Extent [axiom: \ref{axiom of extent}] we have $A = b$
    
    \item As $A \subseteq B \wedge B \subseteq C$ that $x \in A \Rightarrow x
    \in B$ and $x \in B \Rightarrow x \in C$ proving that $x \in A \Rightarrow
    x \in C$ or $A \subseteq C$
    
    \item If $x \in A$ then as $A = B$ we have by the axiom of extension
    [axiom: \ref{axiom of extent}] that $x \in B$, hence $A \subseteq B$.
  \end{enumerate}
  
\end{proof}

One way to create a new class is to specify a predicate that a object must
satisfies and then take the class of all objects that satisfies this
predicate. The problem with this construction is that it can lead to paradoxes
like the famous Russell paradox. Consider the predicate $R (x) = x \nin x$,
this predicate is true for $x$ if $x$ is not a member of itself and consider
the class that contains all classes that has not them self as member. Does
this class contain itself yes or no? If the class contain itself then by
definition $R (x)$ should be true so the class should not contain itself
leading to a contradiction. If the class does not contain itself then it
satisfies $R (x)$, hence it is a member of itself again leading to a
contradiction. So we can not test the predicate $R (x)$ for all classes and
thus can not define the class of all classes for which $R (x)$ is true. The
axiom of class construction allows us to create a new class in a safe way.

\begin{axiom}[Axiom of Construction]
  \label{axiom of construction}Let $P (x)$ be a statement about $x$ [using
  mathematical logic] then there exists a class $C$ such that $x \in C$ iff
  $x$ is a element and $P (x)$ is true.
  
  \begin{notation}
    This class $C$ is noted as $C = \{ x|P (x) \}$, note the use of lower
    cases for $x$, which is a visual indicator that $x$ is a element.
  \end{notation}
\end{axiom}

Note that that $C$ consists of \tmtextbf{elements} for which $P (x)$ is true,
it is not enough that $P (x)$ is true to belong to $C$. A object must belong
to a class [be a element or equivalently be a set] and $P (x)$ must be true to
be a member of $C$. Let's see how that solves Russell's paradox. Define the
class $R = \{ x|x \nin x \}$ [Russel's class] and check if $R \in R$ or $R
\nin R$ is true:
\begin{description}
  \item[$R \in R$] Then $R$ is a element and $R \nin R$ giving the
  contradiction $R \in R \wedge R \nin R$
  
  \item[$R \nin R$] Then $R$ is not a element or $R \in R$ which as $R \nin R$
  gives that $R$ is not a element
\end{description}
So we have that $R$ is not a element and indeed because of this that $R \nin
R$. You can ask yourself if there actually exists elements, none of the axioms
up to now can be used to get elements [or equivalent sets], for this we need
extra axioms.

The axiom of construction can be used as a way of creating a sub-class of a
given class.

\begin{definition}
  Let $A$ be a class and $P (x)$ a predicate then $\{ x \in A|P (x) \} = \{
  x|x \in A \wedge P (x) \}$
\end{definition}

Using the axiom of construction [axiom: \ref{axiom of construction}] we can
then define the universal class $\mathcal{U}$.

\begin{definition}[Universal class]
  \label{universal class}{\index{$\mathcal{U}$}}The universal class
  $\mathcal{U}$ is defined by $\mathcal{U}= \{ x|x = x \}$
\end{definition}

The universal class contains all the elements, as is expressed in the
following theorem.

\begin{theorem}
  \label{universal class property}If $x$ is a element then $x \in \mathcal{U}$
\end{theorem}

\begin{proof}
  Let $x$ be a element then, as $x = x$ [see theorem: \ref{class properties
  (1)}] we have that $x \in \mathcal{U}$
\end{proof}

We use now the axiom of construction to define the union and intersection of
two classes.

\begin{definition}
  {\index{$A \bigcup B$}}Let $A, B$ be two classes then the union of $A$ and
  $B$, noted as $A \bigcup B$ is defined by
  \[ A \bigcup B = \{ x|x \in A \vee x \in B \} \]
\end{definition}

\begin{definition}
  {\index{$A \bigcap B$}}Let $A, B$ be two classes then the union of $A$ and
  $B$, noted as $A \bigcap B$ is defined by
  \[ A \bigcap B = \{ x|x \in A \wedge x \in B \} \]
\end{definition}

Next we define the empty class, the class that does not contains a element.

\begin{definition}
  \label{class empty set definition}{\index{$\emptyset$}}The empty class
  $\varnothing$ is defined by
  \[ \varnothing = \{ x|x \neq x \} \]
\end{definition}

\begin{theorem}
  \label{empty set property}$\varnothing$ does not contains elements, meaning
  if $x$ is a element then $x \nin \varnothing$
\end{theorem}

\begin{proof}
  We proof this by contradiction, so assume that there exists a element $x \in
  \varnothing$ then $x \neq x$, contradicting $x = x$ [see theorem: \ref{class
  properties (1)}].
\end{proof}

\begin{theorem}
  \label{class empty set}If $A$ is a class then
  \begin{enumerate}
    \item $\varnothing \subseteq A$
    
    \item $A \subseteq \mathcal{U}$
    
    \item If $A \subseteq \varnothing$ then $A = \varnothing$
  \end{enumerate}
\end{theorem}

\begin{proof}
  
  \begin{enumerate}
    \item We proof this by contra-position, as $\varnothing \subseteq A$ is
    equivalent with $x \in \varnothing \Rightarrow x \in A$. We must proof
    that $x \nin A \Rightarrow x \nin \varnothing$. Well if $x \nin A$ then
    certainly $x \nin \varnothing$ [Theorem: \ref{empty set property}] so that
    $x \nin A \Rightarrow x \nin \varnothing$.
    
    \item If $x \in A$ then $x$ is a element, hence $x \in \mathcal{U}$ by
    [Theorem: \ref{universal class property}]
    
    \item By (1) we have $\varnothing \subseteq A$ which together with $A
    \subseteq \varnothing$ proves by [theorem: \ref{class properties (1)}]
    that $A = \varnothing$.
  \end{enumerate}
\end{proof}

We also have that every class with no elements is equal to the empty set
[there is only one empty set]

\begin{theorem}
  \label{class empty set is unique}If $A$ is a a class such that $x \in A$
  yields a contradiction then $A = \varnothing$
\end{theorem}

\begin{proof}
  Let $x \in A$ then we have a contradiction, so $x \in A$ must be false and
  thus $x \in A \Rightarrow x \in \varnothing$ is vacuously true which proves
  that $A \subseteq \varnothing$, combining this with [theorem: \ref{class
  empty set}, \ref{class properties (1)}] proves that $A = \varnothing$
\end{proof}

\begin{corollary}
  \label{class not empty sets}Let $A$ be a class such that $A \neq
  \varnothing$ then $\exists x$ such that $x \in A$
\end{corollary}

\begin{proof}
  We proof this by contradiction. Assume that $\forall x$ we have $x \nin A$
  then $x \in A$ yields the contradiction $x \in A \wedge x \nin A$, hence by
  [theorem: \ref{class empty set is unique}] $A = \varnothing$ which
  contradicts $A \neq \varnothing$.
\end{proof}

\begin{definition}
  Two classes $A, B$ are disjoint iff $A \bigcap B = \varnothing$
\end{definition}

We define now the complement of a class

\begin{definition}
  Let $A$ be a class then the complement of $A$ noted by $A^c$ is defined by
  \[ A^c = \{ x|x \nin A \} \]
\end{definition}

Something similar to the complement of a class is the difference between two
classes

\begin{definition}
  Let $A, B$ be classes then the difference between $A$ and $B$ noted by
  $A\backslash B$ is defined by
  \[ A\backslash B = \{ x|x \in A \wedge x \nin B \}
     \equallim_{\tmop{shorter} \tmop{notation}} \{ x \in A|x \in B \} \]
\end{definition}

We can express the difference of two classes using the intersection and the
complement.

\begin{theorem}
  \label{class difference}Let $A, B$ be classes then
  \[ A\backslash B = A \bigcap B^c \]
\end{theorem}

\begin{proof}
  Let $x \in A\backslash B$ then $x \in A \wedge x \nin B$ so that $x \in A
  \wedge x \in B^c$, further if $x \in A \bigcap B^c$ then $x \in A \wedge x
  \nin B$. Using then the axiom of extent [axiom: \ref{axiom of extent}].
\end{proof}

\section{Class operations}

\begin{theorem}
  \label{class intersection, union, inclusion}Let $A, B, C$ are classes then
  we have
  \begin{enumerate}
    \item $A \subseteq A \bigcup B$
    
    \item $B \subseteq A \bigcup B$
    
    \item $A \bigcap B \subseteq A$
    
    \item $A \bigcap B \subseteq B$
    
    \item $A\backslash B \subseteq A$
    
    \item $\left( A \bigcup B \right) \backslash C = (A\backslash C) \bigcup
    (B\backslash C)$
    
    \item If $C$ is a class such that $A \subseteq C$ and $B \subseteq C$ then
    $A \bigcup B \subseteq C$
    
    \item If C is a class such that $A \subseteq C$ and $D$ a class such that
    $B \subseteq D$ then $A \bigcup B \subseteq C \bigcup D$
    
    \item If $C$ is a class such that $C \subseteq A$ and $C \subseteq B$ then
    $C \subseteq A \bigcap B$
    
    \item If C is a class such that $A \subseteq C$ and $D$ a class such that
    $B \subseteq D$ then $A \bigcap B \subseteq C \bigcap D$
  \end{enumerate}
\end{theorem}

\begin{proof}
  
  \begin{enumerate}
    \item If $x \in A$ then $x \in A \vee x \in B$ proving that $x \in A
    \bigcup B$, hence $A \subseteq A \bigcup B$
    
    \item If $x \in B$ then $x \in A \vee x \in B$ proving that $x \in A
    \bigcup B$, hence $B \subseteq A \bigcup B$
    
    \item If $x \in A \bigcap B$ then $x \in A \wedge x \in B$, hence $x \in
    A$ so that $x \in A$, hence $A \bigcap B \subseteq A$
    
    \item If $x \in A \bigcap B$ then $x \in A \wedge x \in B$, hence $x \in
    B$ so that $x \in A$, hence $A \bigcap B \subseteq B$
    
    \item If $x \in A\backslash B$ then $x \in A \wedge x \nin B$ so that
    $A\backslash B \subseteq A$
    
    \item We have
    \begin{eqnarray*}
      \left( A \bigcup B \right) \backslash C & \equallim_{\text{[theorem:
      \ref{class difference}]}} & \left( A \bigcup B \right) \bigcap C^c\\
      & \equallim_{\text{[theorem: \ref{class class
      commutative,idempotent,associative,distributivity}]}} & \left( A \bigcap
      C^c \right) \bigcup \left( B \bigcap C^c \right)\\
      & \equallim_{\text{[theorem: \ref{class difference}]}} & (A\backslash
      C) \bigcup (B\backslash C)
    \end{eqnarray*}
    \item If $x \in A \bigcup B$ then $x \in A \Rightarrowlim_{A \subseteq C}
    x \in C$ or $x \in B \Rightarrowlim_{B \subseteq C} x \in C$ proving that
    $x \in C$
    
    \item Using (1) $A \subseteq C \bigcup D$ and $B \subseteq C \bigcup D$,
    so using (6) we have $A \bigcup B \subseteq C \bigcup D$
    
    \item If $x \in C$ then $x \in A$ and $x \in B$ so that $x \in A \bigcap
    B$
    
    \item If $x \in A \bigcap B$ then $x \in A \Rightarrowlim_{A \subseteq C}
    x \in C$ and $x \in B \Rightarrowlim_{B \subseteq D} x \in D$ hence $x \in
    C \bigcap D$.
  \end{enumerate}
  
\end{proof}

\begin{theorem}
  \label{class inclusion and union and intersection}If $A, B$ are classes then
  we have
  \begin{enumerate}
    \item $A \subseteq B$ if and only if $A \bigcup B = B$
    
    \item $A \subseteq B$ if and only if $A \bigcap B = A$
    
    \item $A\backslash (A\backslash B) = B$
  \end{enumerate}
\end{theorem}

\begin{proof}
  
  \begin{enumerate}
    \item
    \begin{description}
      \item[$\Rightarrow$] If $x \in A \bigcup B \Rightarrow x \in A
      \Rightarrowlim_{A \subseteq B} x \in B$ and thus $A \bigcup B \subseteq
      B$. From the previous theorem [theorem: \ref{class intersection, union,
      inclusion}] we have $B \subseteq A \bigcup B$ so by \ref{class
      properties (1)} we have $A \bigcup B = B$
      
      \item[$\Leftarrow$] If $A \bigcup B = B$ then $x \in A \Rightarrow x \in
      A \bigcup B \Rightarrowlim_{A \bigcup B = B} x \in B$ and thus $A
      \subseteq B$
    \end{description}
    \item
    \begin{description}
      \item[$\Rightarrow$] If $x \in A \Rightarrowlim_{A \subseteq B} x \in B
      \Rightarrow x \in A \wedge x \in B \Rightarrow x \in A \bigcap B$
      proving that $A \subseteq A \bigcap B$. From the previous theorem we
      have $A \bigcap B \subseteq A$ so by [theorem: \ref{class properties
      (1)}] we have $A \bigcap B = A$
      
      \item[$\Leftarrow$] If $A \bigcap B = A$ we have $x \in A \Rightarrow x
      \in A \bigcap B \Rightarrow (x \in A \wedge x \in B) \Rightarrow x \in
      B$ so $A \subseteq B$.
    \end{description}
    \item We have
    \begin{eqnarray*}
      A\backslash (A\backslash B) & \equallim_{\text{[theorem: \ref{class
      difference}]}} & A \bigcap (A\backslash B)^c\\
      & \equallim_{\text{[theorem: \ref{class difference}]}} & A \bigcap
      \left( A \bigcap B^c \right)^c\\
      & \equallim_{\text{[theorem: \ref{class de Morgan's law}]}} & A \bigcap
      \left( A^c \bigcup (B^c)^c \right)\\
      & \equallim_{\text{[theorem: \ref{class complement of comploment}]}} &
      A \bigcap \left( A^c \bigcup B \right)\\
      & \equallim_{\text{[theorem: \ref{class class
      commutative,idempotent,associative,distributivity}]}} & \left( A \bigcap
      A^c \right) \bigcup \left( A \bigcap B \right)\\
      & \equallim_{\text{[theorem: \ref{class difference}]}} & (A\backslash
      A) \bigcup \left( A \bigcap B \right)\\
      & \equallim_{\text{[theoremc: \ref{class universal and empotyset
      properties}]}} & \varnothing \bigcup \left( A \bigcap B \right)\\
      & \equallim_{\text{[theoremc: \ref{class universal and empotyset
      properties}]}} & A \bigcap B\\
      & \equallim_{\text{[theorem: \ref{class inclusion and union and
      intersection}]}} & B
    \end{eqnarray*}
  \end{enumerate}
  
\end{proof}

\begin{theorem}[Absorption Laws]
  \label{class absorption laws}If $A, B$ are classes then
  \begin{enumerate}
    \item $A \bigcup \left( A \bigcap B \right) = A$
    
    \item $A \bigcap \left( A \bigcup B \right) = A$
  \end{enumerate}
\end{theorem}

\begin{proof}
  
  \begin{enumerate}
    \item By [theorem: \ref{class intersection, union, inclusion} we have $A
    \bigcap B \subseteq A$, hence using [theorem: \ref{class inclusion and
    union and intersection}] we have that $A \bigcup \left( A \bigcap B
    \right) = A$
    
    \item By [theorem: \ref{class intersection, union, inclusion}] we have $A
    \subseteq A \bigcup B$, hence using [theorem: \ref{class inclusion and
    union and intersection}] we have that $A \bigcap \left( A \bigcup B
    \right) = A$
  \end{enumerate}
\end{proof}

\begin{theorem}
  \label{class complement of comploment}Let $A$ be a class then $(A^c)^c = A$
\end{theorem}

\begin{proof}
  If $x \in (A^c)^c$ then $x$ is a element and $x \nin A$ then $x \in A$ [for
  if $x \nin A$ we have $x \in A^c$]. If $x \in A$ then $x \nin A^c$ so that
  $x \in (A^c)^c$.
\end{proof}

\begin{theorem}[DeMorgan's Law]
  \label{class de Morgan's law}For all classes $A, B, C$ we have
  \begin{enumerate}
    \item $\left( A \bigcup B \right)^c = A^c \bigcap B^c$
    
    \item $\left( A \bigcap B \right)^c = A^c \bigcup B^c$
  \end{enumerate}
\end{theorem}

\begin{proof}
  
  \begin{enumerate}
    \item If $x \in \left( A \bigcup B \right)^c$ then $x \nin A \bigcup B$,
    so that $\neg (x \in A \vee x \in B) = x \nin A \wedge x \nin B$ proving
    that $x \in A^c \bigcap B^c$. If $x \in A^c \bigcap B^c$ then $x \nin A
    \wedge x \nin B = \neg (x \in A \vee x \in B)$, so that $x \nin A \bigcup
    B$ or $x \in \left( A \bigcup B \right)^c$. The proof follows then from
    the axiom of extent [axiom: \ref{axiom of extent}]
    
    \item If $x \in \left( A \bigcap B \right)^c$ then $x \nin A \bigcap B$,
    so that $\neg (x \in A \wedge x \in B) = x \nin A \vee x \nin B$ proving
    that $x \in A^c \bigcup B^c$. If $x \in A^c \bigcup B^c$ then $x \nin A
    \vee x \nin B = \neg (x \in A \wedge x \in B)$, so that $x \in \left( A
    \bigcap B \right)^c$. The proof follows then from axiom of extent [axiom:
    \ref{axiom of extent}]
  \end{enumerate}
\end{proof}

\begin{theorem}
  \label{class class commutative,idempotent,associative,distributivity}Let $A,
  B, C$ be classes then we have:
  \begin{description}
    \item[commutativity]
    \begin{enumerate}
      \item $A \bigcup B = B \bigcup A$
      
      \item $A \bigcap B = B \bigcap A$
    \end{enumerate}
    \item[idem potency] 
    \begin{enumerate}
      \item $A \bigcup A = A$
      
      \item $A \bigcap A = A$
    \end{enumerate}
    \item[associativity] 
    \begin{enumerate}
      \item $A \bigcup \left( B \bigcup C \right) = \left( A \bigcup B \right)
      \bigcup C$
      
      \item $A \bigcap \left( B \bigcap C \right) = \left( A \bigcap B \right)
      \bigcap C$
    \end{enumerate}
    \item[Distributivity]
    \begin{enumerate}
      \item $A \bigcup \left( B \bigcap C \right) = \left( A \bigcup B \right)
      \bigcap \left( A \bigcup C \right)$
      
      \item $A \bigcap \left( B \bigcup C \right) = \left( A \bigcap B \right)
      \bigcup \left( A \bigcap C \right)$
    \end{enumerate}
  \end{description}
\end{theorem}

\begin{proof}
  
  \begin{description}
    \item[$\tmop{commutatitivity}$]
    \begin{enumerate}
      \item This follows from [axiom: \ref{axiom of extent}] and
      \begin{eqnarray*}
        x \in A \bigcup B & \Leftrightarrow & x \in A \vee x \in B\\
        & \Leftrightarrow & x \in B \vee x \in A\\
        & \Leftrightarrow & x \in B \bigcup A
      \end{eqnarray*}
      \item This follows from [axiom: \ref{axiom of extent}] and
      \begin{eqnarray*}
        x \in A \bigcap B & \Leftrightarrow & x \in A \wedge x \in B\\
        & \Leftrightarrow & x \in B \wedge x \in A\\
        & \Leftrightarrow & x \in B \bigcap A
      \end{eqnarray*}
    \end{enumerate}
    \item[idem potency] 
    \begin{enumerate}
      \item This follows from [axiom: \ref{axiom of extent}] and
      \begin{eqnarray*}
        x \in A \bigcup A & \Leftrightarrow & x \in A \vee x \in A\\
        & \Leftrightarrow & x \in A
      \end{eqnarray*}
      \item This follows from [axiom: \ref{axiom of extent}] and
      \begin{eqnarray*}
        x \in A \bigcap A & \Leftrightarrow & x \in A \wedge x \in A\\
        & \Leftrightarrow & x \in A
      \end{eqnarray*}
    \end{enumerate}
    \item[associativity] 
    \begin{enumerate}
      \item This follows from [axiom: \ref{axiom of extent}] and
      \begin{eqnarray*}
        x \in A \bigcup \left( B \bigcup C \right) & \Leftrightarrow & x \in A
        \vee x \in B \bigcup C\\
        & \Leftrightarrow & x \in A \vee (x \in B \vee x \in C)\\
        & \Leftrightarrow & (x \in A \vee x \in B) \vee x \in C\\
        & \Leftrightarrow & x \in A \bigcup B \vee x \in C\\
        & \Leftrightarrow & x \in \left( A \bigcup B \right) \bigcup C
      \end{eqnarray*}
      \item This follows from [axiom: \ref{axiom of extent}] and
      \begin{eqnarray*}
        x \in A \bigcap \left( B \bigcap C \right) & \Leftrightarrow & x \in A
        \vee x \in B \bigcap C\\
        & \Leftrightarrow & x \in A \wedge (x \in B \wedge x \in C)\\
        & \Leftrightarrow & (x \in A \wedge x \in B) \wedge x \in C\\
        & \Leftrightarrow & x \in A \bigcap B \wedge x \in C\\
        & \Leftrightarrow & x \in \left( A \bigcap B \right) \bigcap C
      \end{eqnarray*}
    \end{enumerate}
    \item[Distributivity] 
    \begin{enumerate}
      \item This follows from [axiom: \ref{axiom of extent}] and
      \begin{eqnarray*}
        x \in A \bigcup \left( B \bigcap C \right) & \Leftrightarrow & x \in A
        \vee x \in B \bigcap C\\
        & \Leftrightarrow & x \in A \vee (x \in B \wedge x \in C)\\
        & \Leftrightarrow & (x \in A \vee x \in B) \wedge (x \in A \vee x \in
        C)\\
        & \Leftrightarrow & x \in A \bigcup B \wedge x \in A \bigcup C\\
        & \Leftrightarrow & x \in \left( A \bigcup B \right) \bigcap \left( A
        \bigcup C \right)
      \end{eqnarray*}
      \item This follows from [axiom: \ref{axiom of extent}] and
      \begin{eqnarray*}
        x \in A \bigcap \left( B \bigcup C \right) & \Leftrightarrow & x \in A
        \wedge x \in B \bigcup C\\
        & \Leftrightarrow & x \in A \wedge (x \in B \vee x \in C)\\
        & \Leftrightarrow & (x \in A \wedge x \in B) \vee (x \in A \wedge x
        \in C)\\
        & \Leftrightarrow & x \in A \bigcap B \wedge x \in A \bigcap C\\
        & \Leftrightarrow & x \in \left( A \bigcap B \right) \bigcup \left( A
        \bigcap C \right)
      \end{eqnarray*}
    \end{enumerate}
  \end{description}
\end{proof}

\begin{theorem}
  \label{class set difference and union , intersection}Let $A, B, C$ be
  classes then we have
  \begin{enumerate}
    \item $A\backslash \left( B \bigcup C \right) = (A\backslash B) \bigcap
    (A\backslash C) = (A\backslash B) \backslash C$
    
    \item $A\backslash \left( B \bigcap C \right) = (A\backslash B) \bigcup
    (A\backslash C)$
  \end{enumerate}
\end{theorem}

\begin{proof}
  
  \begin{enumerate}
    \item 
    \begin{eqnarray*}
      A\backslash \left( B \bigcup C \right) &
      \equallim_{\text{theorem:\ref{class difference}}} & A \bigcap \left( B
      \bigcup C \right)^c\\
      & \equallim_{\text{theorem: \ref{class de Morgan's law}}} & A \bigcap
      \left( B^c \bigcap C^c \right)\\
      & \equallim_{\text{associativity}} & \left( A \bigcap B^c \right)
      \bigcap C^c\\
      & \equallim_{\text{idem potency}} & \left( \left( A \bigcap A \right)
      \bigcap B^c \right) \bigcap C^c\\
      & \equallim_{\text{associativity}} & \left( A \bigcap \left( A \bigcap
      B^c \right) \right) \bigcap C^c\\
      & \equallim_{\text{commutativity}} & \left( \left( A \bigcap B^c
      \right) \bigcap A \right) \bigcap C^c\\
      & \equallim_{\text{associativity}} & \left( A \bigcap B^c \right)
      \bigcap \left( A \bigcap C^c \right)\\
      & \equallim_{\text{theorem:\ref{class difference}}} & (A\backslash B)
      \bigcap (A\backslash C)\\
      A\backslash \left( B \bigcup C \right) &
      \equallim_{\text{theorem:\ref{class difference}}} & A \bigcap \left( B
      \bigcup C \right)^c\\
      & \equallim_{\text{theorem: \ref{class de Morgan's law}}} & A \bigcap
      \left( B^c \bigcap C^c \right)\\
      & \equallim_{\text{associativity}} & \left( A \bigcap B^c \right)
      \bigcap C^c\\
      & \equallim_{\text{theorem:\ref{class difference}}} & (A\backslash B)
      \backslash C
    \end{eqnarray*}
    \item
    \begin{eqnarray*}
      A\backslash \left( B \bigcap C \right) &
      \equallim_{\text{theorem:\ref{class difference}}} & A \bigcap \left( B
      \bigcap C \right)^c\\
      & \equallim_{\text{theorem: \ref{class de Morgan's law}}} & A \bigcap
      \left( B^c \bigcup C^c \right)\\
      & \equallim_{\text{Distributivity}} &  \left( A \bigcap B^c \right)
      \bigcup \left( A \bigcap C^c \right)\\
      & \equallim_{\text{theorem:\ref{class difference}}} & (A\backslash B)
      \bigcup (A\backslash C)
    \end{eqnarray*}
  \end{enumerate}
\end{proof}

\begin{theorem}
  \label{class universal and empotyset properties}Let $A$ be a class then we
  have:
  \begin{enumerate}
    \item $\varnothing \bigcup A = A$
    
    \item $\varnothing \bigcap \varnothing = \varnothing$
    
    \item $A \bigcup \mathcal{U}=\mathcal{U}$
    
    \item $A \bigcap \mathcal{U}= A$
    
    \item $A\backslash A = \varnothing$
  \end{enumerate}
\end{theorem}

\begin{proof}
  
  \begin{enumerate}
    \item As $\varnothing \subseteq A$ [theorem: \ref{class empty set}] we
    have by [theorem: \ref{class inclusion and union and intersection}] that
    $\varnothing \bigcup A = A$
    
    \item As $\varnothing \subseteq A$ [theorem: \ref{class empty set}] we
    have by [theorem: \ref{class inclusion and union and intersection}] that
    $\varnothing \bigcap A = A$
    
    \item As $A \subseteq \mathcal{U}$ [theorem \ref{class empty set}] we have
    by [theorem: \ref{class inclusion and union and intersection}] that $A
    \bigcap \mathcal{U}= A$
    
    \item As $A \subseteq \mathcal{U}$ [theorem \ref{class empty set}] we have
    by [theorem: \ref{class inclusion and union and intersection}] that $A
    \bigcap \mathcal{U}= A$
    
    \item Let $x \in A\backslash A$ then $x \in A \wedge x \nin A$ a
    contradiction, so by [theorem: \ref{class empty set is unique}] we have
    that $A\backslash A = \varnothing$
  \end{enumerate}
\end{proof}

\section{Cartesian products}

If $a$ is a element we can use the axiom of construction [axiom: \ref{axiom of
construction}] to define the class $\{ x|x = a \}$, this leads to the
following definition.

\begin{definition}
  If $a$ is a element then $\{ a \} = \{ x | x = a \nobracket \}$ is a class
  containing only one element. The class $\{ a \}$ is called a
  \tmtextbf{singleton}.
\end{definition}

\begin{lemma}
  \label{element a=b=<gtr>{a}={b}}If $a, b$ are elements such that $a = b$
  then $\{ a \} = \{ b \}$
\end{lemma}

\begin{proof}
  If $z \in \{ a \}$ then $z = a$ which by $a = b$ and [theorem: \ref{class
  properties (1)}] proves that $z = b$ hence $z \in \{ b \}$. Likewise if $z
  \in \{ b \}$ then $z = b$ which by $a = b$ and [theorem: \ref{class
  properties (1)}] proves that $z = a$ hence $z \in \{ a \}$. Using the axiom
  of extent [axiom: \ref{axiom of extent}] it follows then that $\{ a \} = \{
  b \}$
\end{proof}

If $a, b$ are elements then we can use the axiom of construction [axiom:
\ref{axiom of construction}] to define the class $\{ x | x = a \vee x = b
\nobracket \}$ consisting of two elements. This leads to the following
definition.

\begin{definition}
  If $a, b$ are elements then $\{ a, b \} = \{ x | x = a \vee x = b \nobracket
  \}$ is called a \tmtextbf{unordered pair}.
\end{definition}

The next axiom ensures we can construct new elements from given elements.. It
allows us to create classes that has as members pairs of elements.

\begin{axiom}[Axiom of Pairing]
  \label{axiom of pairing}{\index{axiom of pairing}}If $a, b$ are elements
  then $\{ a, b \}$ is a element
\end{axiom}

\begin{lemma}
  \label{element {a,a}={a}}If $a$ is a element then $\{ a, a \} = \{ a \}$
\end{lemma}

\begin{proof}
  
  \begin{eqnarray*}
    x \in \{ a, a \} & \Leftrightarrow & x = a \vee x = a\\
    & \Leftrightarrow & x = a\\
    & \Leftrightarrow & x \in \{ a \}
  \end{eqnarray*}
\end{proof}

\begin{theorem}
  \label{element: {a} is a element}If $a$ is a element then $\{ a \}$ is a
  element
\end{theorem}

\begin{proof}
  As $a$ is a element we have by the axiom of pairing [axiom: \ref{axiom of
  pairing}] that $\{ a, a \}$ is a element, which as $\{ a \}
  \equallim_{\text{lemma: \ref{element {a,a}={a}}}} \{ a., a \}$ proves that
  $\{ a \}$ is a element.
\end{proof}

The following lemma characterize equality of unordered pairs and will be used
later to characterize equality of ordered pairs.

\begin{lemma}
  \label{element equalitiy of unordered pairs}If $x, y, x', y'$ are elements
  then
  \[ \{ x, y \} = \{ x', y' \} \text{ implies } (x = x' \wedge y = y') \vee
     (x = y' \wedge y = x') \]
\end{lemma}

\begin{proof}
  Lets's consider the following possible cases $x, y$:
  \begin{description}
    \item[$x = y$] Then $\{ x, y \} \equallim_{\text{lemma: \ref{element
    {a,a}={a}}}} \{ x \} = \{ x', y' \}$. From $x' \in \{ x', y' \} = \{ x \}$
    it follows that $x = x'$ and from $y' \in \{ x', y' \} = \{ x \}$ it
    follows that $y = x$. As $x = x'$ it follows from [theorem: \ref{class
    properties (1)}] that $y = x'$. So we have that $(x = x' \wedge y = y')$
    from which it follows that
    \[ (x = x' \wedge y = y') \vee (x = y' \wedge y = x') \]
    \item[$x \neq y$] Then as $x \in \{ x, y \} = \{ x', y' \}$ we have by
    [axiom: \ref{axiom of extent}] that $x \in \{ x', y' \}$, so by definition
    we have for $x$ either
    \begin{description}
      \item[$x = x'$] Then as $y \in \{ x, y \} = \{ x', y' \}$ we have by
      [axiom: \ref{axiom of extent}] that $y \in \{ x', y' \}$, so by
      definition we have for $y$ either:
      \begin{description}
        \item[$y = x'$] As $x = x' \Rightarrowlim_{\text{theorem: \ref{class
        properties (1)}}} x = y$ we contradict $x \neq y$ so this case does
        not apply
        
        \item[$y = y'$] Then $(x = x' \wedge y = y')$ hence $(x = x' \wedge y
        = y') \vee (x = y' \wedge y = x')$
      \end{description}
      \item[$x = y'$] Then as $y \in \{ x, y \} = \{ x', y' \}$ we have by
      [axiom: \ref{axiom of extent}] that $y \in \{ x', y' \}$, so by
      definition we have for $y$ either:
      \begin{description}
        \item[$y = x'$] Then $(x = y' \wedge y = x')$ hence $(x = x' \wedge y
        = y') \vee (x = y' \wedge y = x')$
        
        \item[$y = y'$] As $x = y' \Rightarrowlim_{\text{theorem: \ref{class
        properties (1)}}} x = y$ we contradict $x \neq y$ so this case does
        not apply
      \end{description}
    \end{description}
    So in all cases that apply we have
    \[ (x = x' \wedge y = y') \vee (x = y' \wedge y = x') \]
  \end{description}
\end{proof}

\begin{lemma}
  \label{element equality of unordered pairs (1)}If $x, y, x', y'$ are
  elements such that $(x = x' \wedge y = y') \vee (x = y' \wedge y = x')$ then
  $\{ x, y \} = \{ x, y' \}$
\end{lemma}

\begin{proof}
  Let $z \in \{ x, y \}$ then either:
  \begin{description}
    \item[$z = x$] then if $x = x' \wedge y = y'$ we have using [theorem:
    \ref{class properties (1)}] that $z = x'$, hence by definition $z \in \{
    x', y' \}$ and if $x = y' \wedge y = x'$ we have using \ [theorem:
    \ref{class properties (1)}] that $z = y'$, hence by definition $x \in \{
    x', y' \}$
    
    \item[$z = y$] then if $x = x' \wedge y = y'$ we have using [theorem:
    \ref{class properties (1)}] that $z = y'$, hence by definition $z \in \{
    x', y' \}$ and if $x = y' \wedge y = x'$ we have using \ [theorem:
    \ref{class properties (1)}] that $z = x'$, hence by definition $x \in \{
    x', y' \}$
  \end{description}
  which proves that
  \begin{equation}
    \label{eq 1.5.001} \{ x, y \} \subseteq \{ x', y' \}
  \end{equation}
  Let $z \in \{ x', y' \}$ then either:
  \begin{description}
    \item[$z = x'$] then if $x = x' \wedge y = y'$ we have using [theorem:
    \ref{class properties (1)}] that $z = x$, hence by definition $z \in \{ x,
    y \}$ and if $x = y' \wedge y = x'$ we have using \ [theorem: \ref{class
    properties (1)}] that $z = y$, hence by definition $x \in \{ x, y \}$
    
    \item[$z = y$] then if $x = x' \wedge y = y'$ we have using [theorem:
    \ref{class properties (1)}] that $z = y$, hence by definition $z \in \{ x,
    y \}$ and if $x = y' \wedge y = x'$ we have using \ [theorem: \ref{class
    properties (1)}] that $z = x$, hence by definition $x \in \{ x, y \}$
  \end{description}
  which proves that
  \begin{equation}
    \label{eq 1.6.001} \{ x', y' \} \subseteq \{ x, y \}
  \end{equation}
  Using [theorem: \ref{class properties (1)}] on [eq: \ref{eq 1.5.001},
  \ref{eq 1.6.001}] proves that
  \[ \{ x = y \} = \{ x' = y' \} \]
\end{proof}

The above lemma actually shows that the order of the elements in unordered
pairs do not matter, to remedy this we construct a ordered pair.

\begin{definition}
  \label{pair of elements}If $a, b$ are elements then
  \[ (a, b) = \{ \{ a \}, \{ a, b \} \} \]
  \begin{note}
    As $\{ a \}, \{ a, b \}$ are elements we have again that $\{ \{ a \}, \{
    a, b \} \}$ is a element, hence $(a, b)$ is also a element.
  \end{note}
\end{definition}

Next we show that the order of elements is important for a tuple

\begin{theorem}
  \label{pair equality of pairs}Let $x, y, x', y'$ are elements then
  \[ (x, y) = (x', y') \Leftrightarrow x = x' \wedge y = y' \]
\end{theorem}

\begin{proof}
  
  \begin{description}
    \item[$\Rightarrow$] If $(x, y) = (x', y')$ then by definition
    \[ \{ \{ x \}, \{ x, y \} \} = \{ \{ x' \}, \{ x', y' \} \} \]
    By [lemma: \ref{element equalitiy of unordered pairs}] we have either:
    \begin{description}
      \item[$\{ x \} = \{ x' \} \wedge \{ x, y \} = \{ x', y' \}$] then, as $x
      \in \{ x \},$ we have by definition $x = x'$, using \ [lemma:
      \ref{element equalitiy of unordered pairs}] again we have either:
      \begin{description}
        \item[$x = x' \wedge y = y'$] Then $x = x' \wedge y = y'$
        
        \item[$x = y' \wedge y = x'$] Then by [theorem: \ref{class properties
        (1)}] and $x = x'$ we have $y' = x'$ so that by \ [theorem: \ref{class
        properties (1)}] again $y = y'$. Hence we have $x = x' \wedge y = y'$
      \end{description}
      \item[$\{ x \} = \{ x', y' \} \wedge \{ x, y \} = \{ x' \}$] Then as
      $x', y' \in \{ x', y' \} = \{ x \}$ we have $x' = x \wedge y' = x$, as
      $x, y \in \{ x, y \} = \{ x' \}$ we have $x = x' \wedge y = x'$. Using
      [theorem: \ref{class properties (1)}] on $y' = x \wedge x = x' \wedge y
      = x'$ we have $y = y'$. Hence $x = x' \wedge y = y'$.
    \end{description}
    So in all cases we have
    \[ x = x' \wedge y = y' \]
    \item[$\Leftarrow$] As $x = x'$ it follows from [lemma: \ref{element
    a=b=<gtr>{a}={b}}] that $\{ x \} = \{ x' \}$, from $x = x' \wedge y = y'$
    we have by [lemma: \ref{element equality of unordered pairs (1)}] that $\{
    x, y \} = \{ x', y' \}$. Using [lemma: \ref{element equality of unordered
    pairs (1)}] gives then that $\{ \{ x \}, \{ x, y \} \} = \{ \{ x' \}, \{
    x', y' \} \}$ which by definition gives
    \[ (x, y) = (x', y') \]
  \end{description}
\end{proof}

We are now ready to define the Cartesian product of two classes, using the
axiom of construction [axiom: \ref{axiom of construction}].

\begin{definition}
  \label{cartesian product}{\index{cartesian product}}{\index{$A \times B$}}If
  $A, B$ are classes then the \tmtextbf{Cartesian product }of $A$ and $B$
  noted by $A \times B$ is defined as
  \[ A \times B = \{ z|z = (a, b) \wedge a \in A \wedge b \in B \} \]
  \begin{notation}
    Instead of writing $\{ z|z = (a, b) \wedge a \in A \wedge b \in A \}$ we
    use in the future the shorter notation $\{ (a, b) |a \in A \wedge b \in B
    \}$
  \end{notation}
\end{definition}

A special case of the Cartesian product is the Cartesian product of empty
sets.

\begin{example}
  \label{cartesian product of the empty set}$\varnothing = \varnothing \times
  \varnothing$
\end{example}

\begin{proof}
  If $z \in \varnothing \times \varnothing$ then there exists a $x, y \in
  \varnothing$ such that $z = (x, y)$ which contradict $x, y \nin \varnothing$
  \ [theorem: \ref{empty set property}] hence by \ref{class empty set is
  unique} we have $\varnothing \times \varnothing = \varnothing$.
\end{proof}

\begin{theorem}
  \label{cartesian product with enpty set}Let $A$ be a class then $A \times
  \varnothing = \varnothing$ and $\varnothing \times A = \varnothing$
\end{theorem}

\begin{proof}
  If $z \in A \times \varnothing$ then $z = (x, y)$ where $y \in \varnothing$,
  which contradicts $y \nin \varnothing$ [theorem: \ref{empty set property}],
  so using [theorem: \ref{class empty set is unique}] we have that
  \[ A \times \varnothing = \varnothing \]
  Likewise if $x \in \varnothing \times A$ then $z = (x, y)$ where $x \in
  \varnothing$, which contradicts $x \nin \varnothing$ \ [theorem: \ref{empty
  set property}], so using [theorem: \ref{class empty set is unique}] we have
  that
  \[ \varnothing \times A = \varnothing \]
  
\end{proof}

\begin{theorem}
  \label{cartesian product and inclusion}If $A, B, C, D$ are classes then we
  have:
  \begin{enumerate}
    \item If $A \subseteq B \wedge C \subseteq D$ then $A \times C \subseteq B
    \times D$
    
    \item Let $A \neq \varnothing \wedge C \neq \varnothing$ then if $A \times
    C \subseteq B \times D$ it follows that $A \subseteq B \wedge C \subseteq
    D$
    
    \item Let $A \neq \varnothing \wedge B \neq \varnothing \wedge C \neq
    \varnothing$ then $A \times C = B \times D \Leftrightarrow A = B \wedge C
    = D$
  \end{enumerate}
\end{theorem}

\begin{proof}
  
  \begin{enumerate}
    \item Let $z \in A \times C$ then there exists a $x \in A$ and $y \in C$
    such that $z = (x, y)$. As $A \subseteq B \wedge C \subseteq D$ it follows
    that $x \in B \wedge y \in D$ so that $z = (x, y) \in B \times D$/ Hence
    \[ A \times C \subseteq B \times D \]
    \item Let $x \in A$ then, as $C \neq \varnothing$, we have by \
    [corollary: \ref{class not empty sets}] the existence of a $y \in C$, then
    $(x, y) \in A \times C$ which as $A \times C \subseteq B \times D$ proves
    that $(x, y) \in B \times D$. By definition we have then that $x \in B$
    proving
    \[ A \subseteq B \]
    Likewise, let $y \in C$ then, as $A \neq \varnothing$ we have by \
    [corollary: \ref{class not empty sets}] the existence of a $x \in A$,
    hence $(x, y) \in A \times C$, which as $A \times C \subseteq B \times D$,
    proves $(x, y) \in B \times D$ and by definition $y \in D$. Hence
    \[ C \subseteq D \]
    \item 
    \begin{description}
      \item[$\Rightarrow$] First as $A \times C = B \times D$ we have by
      [theorem: \ref{class properties (1)}] that $A \times C \subseteq B
      \times D$, using (2) proves then that
      \begin{equation}
        \label{eq 1.7.001} A \subseteq B \wedge C \subseteq B
      \end{equation}
      Next as $A \times C = B \times D$ we have by [theorem: \ref{class
      properties (1)}] that $B \times D \subseteq A \times C$, using (2)
      proves then that
      \begin{equation}
        \label{eq 1.8.001} B \subseteq A \wedge C \subseteq D
      \end{equation}
      Combining then [eq \ref{eq 1.7.001}, \ref{eq 1.8.001}] with [theorem:
      \ref{class properties (1)}] proves
      \[ A = B \wedge C = D \]
      \item[$\Leftarrow$] As $A = B \wedge C = D$ we have by [theorem:
      \ref{class properties (1)}] that $A \subseteq B$, $C \subseteq D$, $B
      \subseteq A$, $D \subseteq C$ which using (1) gives that $A \times C
      \subseteq B \times D \wedge B \times D \subseteq A \times C$. Using
      [theorem: \ref{class properties (1)} it follows then that
      \[ A \times C = B \times D \]
    \end{description}
  \end{enumerate}
\end{proof}

\begin{theorem}
  \label{cartesian product properties (1)}Let A,B,C and $D$ be classes then we
  have
  \begin{enumerate}
    \item $A \times \left( B \bigcap C \right) = (A \times B) \bigcap (A
    \times C)$
    
    \item $A \times \left( B \bigcup C \right) = (A \times B) \bigcup (A
    \times C)$
    
    \item $(A \times B) \bigcap (C \times D) = \left( A \bigcap C \right)
    \times \left( B \bigcap D \right)$
    
    \item $\left( B \bigcap C \right) \times A = (B \times A) \bigcap (C
    \times A)$
    
    \item $\left( B \bigcup C \right) \times A = (B \times A) \bigcup (C
    \times A)$
    
    \item $(A \times B) \backslash (C \times D) = ((A\backslash C) \times B)
    \bigcup (A \times (B\backslash D))$
    
    \item $(A\backslash B) \times C = (A \times C) \backslash (B \times C)$
    
    \item $A \times (B\backslash C) = (A \times B) \backslash (A \times C)$
  \end{enumerate}
\end{theorem}

\begin{proof}
  
  \begin{enumerate}
    \item We have
    \begin{eqnarray*}
      z \in A \times \left( B \bigcap C \right) & \Leftrightarrow & z = (x, y)
      \wedge x \in A \wedge y \in \left( B \bigcap C \right)\\
      & \Leftrightarrow & z = (x, y) \wedge x \in A \wedge (y \in B \wedge y
      \in C)\\
      & \Leftrightarrow & (z = (x, y) \wedge x \in A \wedge y \in B) \wedge
      (z = (x, y) \wedge x \in A \wedge y \in C)\\
      & \Leftrightarrow & z \in A \times B \wedge z \in A \times C\\
      & \Leftrightarrow & z \in (A \times B) \bigcap (A \times C)
    \end{eqnarray*}
    \item We have
    \begin{eqnarray*}
      z \in A \times \left( B \bigcup C \right) & \Leftrightarrow & z = (x, y)
      \wedge x \in A \wedge y \in \left( B \bigcup C \right)\\
      & \Leftrightarrow & z = (x, y) \wedge x \in A \wedge (y \in B \vee y
      \in C)\\
      & \Leftrightarrow & (z = (x, y) \wedge x \in A \wedge y \in B) \vee (z
      = (x, y) \wedge x \in A \wedge y \in C)\\
      & \Leftrightarrow & z \in A \times B \vee z \in A \times C\\
      & \Leftrightarrow & z \in (A \times B) \bigcup (A \times C)
    \end{eqnarray*}
    \item We have
    \begin{eqnarray*}
      z \in (A \times B) \bigcap (C \times D) & \Leftrightarrow & z \in A
      \times B \wedge z \in C \times D\\
      & \Leftrightarrow & (z = (x, y) \wedge x \in A \wedge y \in B) \wedge
      (z = (x', y') \wedge x' \in C \wedge y' \in D)\\
      & \Leftrightarrowlim_{(x, y) = z = (x', y') \Rightarrow x = x', y = y'}
      & z = (x, y) \wedge x \in A \wedge y \in B \wedge x \in C \wedge y \in
      D\\
      & \Leftrightarrow & z = (x, y) \wedge (x \in A \wedge x \in C) \wedge
      (y \in B \wedge y \in D)\\
      & \Leftrightarrow & z = (x, y) \wedge \left( x \in A \bigcap C \right)
      \wedge \left( y \in B \bigcap D \right)\\
      & \Leftrightarrow & z \in \left( A \bigcap C \right) \times \left( B
      \bigcap D \right)
    \end{eqnarray*}
    \item We have
    \begin{eqnarray*}
      z \in \left( B \bigcap C \right) \times A & \Leftrightarrow & z = (x, y)
      \wedge x \in B \bigcap C \wedge y \in A\\
      & \Leftrightarrow & z = (x, y) \wedge x \in B \wedge x \in C \wedge y
      \in A\\
      & \Leftrightarrow & (z = (x, y) \wedge x \in B \wedge y \in A) \wedge
      (z = (x, y) \wedge x \in C \wedge y \in A)\\
      & \Leftrightarrow & z \in B \times A \wedge z \in C \times A\\
      & \Leftrightarrow & z \in (B \times A) \bigcap (C \times A)
    \end{eqnarray*}
    \item We have
    \begin{eqnarray*}
      z \in \left( B \bigcup C \right) \times A & \Leftrightarrow & z = (x, y)
      \wedge x \in B \bigcup C \wedge y \in A\\
      & \Leftrightarrow & z = (x, y) \wedge (x \in B \vee x \in C) \wedge y
      \in A\\
      & \Leftrightarrow & (z = (x, y) \wedge x \in B \wedge y \in A) \vee (z
      = (x, y) \wedge x \in C \wedge y \in A)\\
      & \Leftrightarrow & (z \in B \times A) \vee (z \in C \times A)\\
      & \Leftrightarrow & z \in (B \times A) \bigcup (C \times A)
    \end{eqnarray*}
    \item We have
    \begin{eqnarray*}
      z \in (A \times B) \backslash (C \times D) & \Leftrightarrow & \\
      (z = (x, y) \wedge x \in A \wedge y \in B) \wedge (x, y) \nin C \times D
      & \Leftrightarrow & \\
      (z = (x, y) \wedge x \in A \wedge y \in B) \wedge \neg (x \in C \wedge y
      \in D) & \Leftrightarrow & \\
      (z = (x, y) \wedge x \in A \wedge y \in B) \wedge (x \nin C \vee y \nin
      D) & \Leftrightarrow & \\
      (z = (x, y) \wedge x \in A \wedge y \in B \wedge x \nin C) \vee (z = (x,
      y) \wedge x \in A \wedge y \in B \wedge y \nin D) & \Leftrightarrow & \\
      z = (x, y) \wedge [(x, y) \in (A\backslash C) \times B \vee (x, y) \in A
      \times (B\backslash D)] & \Leftrightarrow & \\
      z \in ((A\backslash C) \times B) \bigcup (A \times (B\backslash D)) &
      \Leftrightarrow & 
    \end{eqnarray*}
    \item We have
    \begin{eqnarray*}
      (A \times C) \backslash (B \times C) & \equallim_{(6)} & ((A\backslash
      C) \times B) \bigcup (A \times (C\backslash C))\\
      & \equallim_{\text{[theorem: \ref{class universal and empotyset
      properties}]}} & ((A\backslash C) \times B) \bigcup (A \times
      \varnothing)\\
      & \equallim_{\text{[theorem: \ref{cartesian product with enpty set}}} &
      ((A\backslash C) \times B) \bigcup \varnothing\\
      & \equallim_{\text{[theorem: \ref{class universal and empotyset
      properties}]}} & (A\backslash C) \times B
    \end{eqnarray*}
    \item We have
    \begin{eqnarray*}
      (A \times B) \backslash (A \times C) & \equallim_{(6)} & ((A\backslash
      A) \times B) \bigcup (A \times (B\backslash C))\\
      & \equallim_{\text{[theorem: \ref{class universal and empotyset
      properties}]}} & (\varnothing \times B) \bigcup (A \times (B\backslash
      C))\\
      & \equallim_{\text{[theorem: \ref{cartesian product with enpty set}}} &
      \varnothing \bigcup (A \times (B\backslash C))\\
      & \equallim_{\text{[theorem: \ref{class universal and empotyset
      properties}]}} & A \times (B\backslash C)
    \end{eqnarray*}
  \end{enumerate}
\end{proof}

\section{Sets}

Remember that that another name for \tmtextbf{element} is \tmtextbf{set}
[definition: \ref{element is set}]. Up to now we have used the name
\tmtextbf{element}, because we want to think of a element as a member of a
class. However a element is also a class and can contain other elements. If we
want to stress the collection aspect then we use the word \tmtextbf{set}
instead of \tmtextbf{element}. The convention is to use uppercase to represent
a set and lower cases for a element. Of course set and element are the same
thing, we just want to stress different aspects of the same thing. Note that
we have two kinds of classes classes that are a member of another class and
classes that are not a member of a class. This leads to the following
definition.

\begin{definition}
  \label{set element proper class}A class $A$ is a \tmtextbf{set} [or
  \tmtextbf{element}] if there exists a class $B$ such that $A \in B$. A class
  that is never a member of another class is called a \tmtextbf{proper class}.
\end{definition}

Up to know we had axioms that given a element/set create a new element/set,
but we have not ensured the existence of a element/set. To this we must first
define the concept of a successor set.

\begin{definition}
  \label{set successor set}{\index{successor set}}A set $S$ is a
  \tmtextbf{successor set} iff
  \begin{enumerate}
    \item $\varnothing \in S$
    
    \item If $X \in S$ then $X \bigcup \{ X \} \in S$
  \end{enumerate}
\end{definition}

Of course nothing proves that successor set's exists, to ensure the existence
of a successor set we have the axiom of infinity.

\begin{axiom}[Axiom of Infinity]
  \label{axiom of infinity}{\index{axiom of infinity}}There exists a
  \tmtextbf{successor set}
\end{axiom}

This axiom ensures that we have at least one set. We can then use the other
axioms about elements/sets to create new elements. Later we will use the Axiom
of Infinity to create the Natural Numbers, form which we build all the other
numbers (integers, rationals, reals, complex numbers). The Axiom of Infinity
ensures also that the empty class is actually a set.

\begin{theorem}
  $\varnothing$ is a set
\end{theorem}

\begin{proof}
  The Axiom of Infinity [axiom: \ref{axiom of infinity}] ensures the existence
  of a successor set $S$. By definition we have then that $\varnothing \in S$
  which proves that $\varnothing$ is a set.
\end{proof}

So now we have two sets to start with, the successor set and the empty set. We
can use the Axiom of Pairing [axiom: \ref{axiom of pairing}] to create new
sets like singletons, unordered pairs and pairs. We introduce now extra axioms
to create new sets given existing sets.

\begin{axiom}[Axiom of Subsets]
  \label{axiom of subsets}{\index{axiom of subsets}}Every sub-class of a set
  is a set
\end{axiom}

As a application we proof that the intersection of two sets is a set

\begin{theorem}
  \label{set intersection of two sets is aset}Let $A, B$ be sets then $A
  \bigcap B$ is a set
\end{theorem}

\begin{proof}
  By [theorem: \ref{class intersection, union, inclusion}] we have that $A
  \bigcap B \subseteq A$, so by the axiom of infinity [axiom: \ref{axiom of
  infinity}] it follows that $A \bigcap B$ is a set.
\end{proof}

We define now a more general concept of union and intersection

\begin{definition}
  \label{class union}Let $\mathcal{A}$ be a class then using the Axiom of
  Construction [axiom: \ref{axiom of construction}] we define $\bigcup
  \mathcal{A}= \left\{ x| \exists y \in \mathcal{A} \text{ such that } x \in y
  \right\}$
\end{definition}

\begin{definition}
  \label{class intersection}Let $\mathcal{A}$ be a class then using the Axiom
  of Construction [axiom: \ref{axiom of construction}] we define $\bigcap
  \mathcal{A}= \left\{ x| \forall y \in \mathcal{A} \text{ we have } x \in y
  \right\}$
\end{definition}

\begin{example}
  \label{class trivial union intersection}Let $A$ be a class then
  \begin{enumerate}
    \item $\bigcup \{ A \} = A$
    
    \item $\bigcap \{ A \} = A$
    
    \item $\bigcup \varnothing = \varnothing$
  \end{enumerate}
\end{example}

\begin{proof}
  
  \begin{enumerate}
    \item 
    \begin{eqnarray*}
      x \in \bigcup \{ A \} & \Leftrightarrow & \exists y \in \{ A \} \text{
      with } x \in y\\
      & \Leftrightarrowlim_{y \in \{ A \} \Leftrightarrow y = A} & x \in A
    \end{eqnarray*}
    proving that
    \[ \bigcup \{ A \} = A \]
    \item 
    \begin{eqnarray*}
      x \in \bigcap \{ A \} & \Leftrightarrow & \forall y \in \{ A \} \text{
      we have } x \in y\\
      & \Leftrightarrowlim_{y \in \{ A \} \Leftrightarrow y = A} & x \in A
    \end{eqnarray*}
    proving that
    \[ \bigcap \{ A \} = A \]
    \item Assume that $x \in \varnothing$ then $\exists y \in \varnothing$
    such that $x \in y$ which lead by the definition of $\varnothing$
    [definition: \ref{class empty set definition}] to the contradiction that
    $y \neq y$.
  \end{enumerate}
\end{proof}

\begin{example}
  \label{class union{A,B}}Let $A$ and $B$ classes then
  \begin{enumerate}
    \item $\bigcup \{ A, B \} = A \bigcup B$
    
    \item $\bigcap \{ A, B \} = A \bigcap B$
  \end{enumerate}
\end{example}

\begin{proof}
  
  \begin{enumerate}
    \item 
    \begin{eqnarray*}
      x \in \bigcup \{ A, B \} & \Leftrightarrow & \exists y \in \{ A, B \}
      \text{ with } x \in y\\
      & \Leftrightarrowlim_{y \in \{ A, B \} \Leftrightarrow y = A \vee y =
      B} & x \in A \vee x \in B\\
      & \Leftrightarrow & x \in A \bigcup B
    \end{eqnarray*}
    proving that
    \[ \bigcup \{ A, B \} = A \bigcup B \]
    \item 
    \begin{eqnarray*}
      x \in \bigcap \{ A, B \} & \Leftrightarrow & \forall y \in \{ A, B \}
      \text{ with } x \in y\\
      & \Leftrightarrowlim_{y \in \{ A, B \} \Leftrightarrow y = A \vee y =
      B} & x \in A \wedge x \in B\\
      & \Leftrightarrow & x \in A \bigcap B
    \end{eqnarray*}
    proving that
    \[ \bigcap \{ A, B \} = A \bigcap B \]
  \end{enumerate}
  
\end{proof}

\begin{theorem}
  \label{class general intersection}If $\mathcal{A}$ is a class
  \begin{enumerate}
    \item If $A \in \mathcal{A}$ then $\bigcap \mathcal{A} \subseteq A$
    
    \item If $A \in \mathcal{A}$ then $A \subseteq \bigcup \mathcal{A}$
    
    \item If $\forall A \in \mathcal{A}$ we have $C \subseteq A$ then $C
    \subseteq \bigcap \mathcal{A}$
    
    \item If $\forall A \in \mathcal{A}$ we have $A \subseteq C$ then $\bigcup
    \mathcal{A} \subseteq C$
    
    \item If $\mathcal{A} \neq \varnothing$ then \ $\bigcap \mathcal{A}$ is a
    set
  \end{enumerate}
\end{theorem}

\begin{proof}
  
  \begin{enumerate}
    \item Let $A \in \mathcal{A}$ then if $x \in \bigcap \mathcal{A}$ we have
    by definition of $\bigcap \mathcal{A}$ that $x \in A$. Hence $\bigcap
    \mathcal{A} \subseteq A$
    
    \item If $x \in A$ then $\exists y \in \mathcal{A}$ such that $x \in y$
    [take $y = A$] so that $x \in \bigcup \mathcal{A}$
    
    \item If $x \in C$ then $\forall A \in \mathcal{A}$ we have as $C \in A$
    that $x \in A$ so that $x \in \bigcap \mathcal{A}$
    
    \item If $x \in \bigcup \mathcal{A}$ then $\exists A \in \mathcal{A}$ such
    that $x \in A$ which as $A \subseteq C$ proves that $x \in A$
    
    \item As $\mathcal{A} \neq \varnothing$ there exists a $A \in
    \mathcal{A}$, which by definition means that $A$ is a set. Using (1) we
    have $\bigcap \mathcal{A} \subseteq A$, applying then the Axiom of Subsets
    [axiom: \ref{axiom of subsets}] it follows that $\bigcap \mathcal{A}$ is a
    set.
  \end{enumerate}
\end{proof}

The above is not applicable for unions, however we state the Axiom of Unions
that will ensure that $\bigcup \mathcal{A}$ is a set if $\mathcal{A}$ is a set

\begin{axiom}[Axiom of Unions]
  \label{axiom of union}{\index{axiom of union}}If $\mathcal{A}$ is a set then
  $\bigcup \mathcal{A}$ is a set
\end{axiom}

A consequence of the above axiom is that the union of two sets is a set

\begin{theorem}
  \label{set union of two sets is a set}Let $A, B$ be tow sets then $A \bigcup
  B$ is a set
\end{theorem}

\begin{proof}
  Using the Axiom of Pairing [axiom: \ref{axiom of pairing}] we have that $\{
  A, B \}$ is a set. Further
  \begin{eqnarray*}
    x \in A \bigcup B & \Leftrightarrow & x \in A \vee x \in B\\
    & \Leftrightarrow & \exists C \in \{ A, B \} \text{ with } x \in C\\
    & \Leftrightarrow & \bigcup \{ A, B \}
  \end{eqnarray*}
  proving by the Axiom of Union [axiom: \ref{axiom of union}] we have that $A
  \bigcup B$ is a set.
\end{proof}

\begin{definition}
  \label{power set}{\index{power set}}{\index{$\mathcal{P} (A)$}}Let $A$ be a
  set then we use the Axiom of Construction to define $\mathcal{P} (A)$ by
  \[ \mathcal{P} (A) = \{ B|B \subseteq A \} \]
\end{definition}

We introduce now the Axiom of Power Sets to ensure that $\mathcal{P} (A)$ is a
set, called the \tmtextbf{power set} of $A$.

\begin{axiom}[Axiom of Power Sets]
  \label{axiom of power}{\index{axiom of power}}If $A$ is a set then
  $\mathcal{P} (A)$ is a set
\end{axiom}

\begin{theorem}
  \label{set restriction of a set of sets}If $A$ is a set and $P (X)$ a
  predicate then $\{ X|X \subseteq A \wedge P (X) \}$ is a set.
\end{theorem}

\begin{proof}
  If $B \in \{ X|X \subseteq A \wedge P (X) \}$ then $B \subseteq A$ so that
  $B \in \mathcal{P} (A)$, proving that
  \[ \{ X|X \subseteq A \wedge P (X) \} \subseteq \mathcal{P} (A) \]
  Using the Axiom of Power Sets [axiom: \ref{axiom of power}] $\mathcal{P}
  (A)$ is a set, so we can use the Axiom of Subsets to prove that $\{ X|X
  \subseteq A \wedge P (X) \}$ is a set.
\end{proof}

\begin{lemma}
  \label{set A*B is subset of P(P(AUB))}If A,B are classes then $A \times B
  \subseteq \mathcal{P} \left( \mathcal{P} \left( A \bigcup B \right) \right)$
\end{lemma}

\begin{proof}
  Let $z \in A \times B$ then there exists a $x \in A$ and a $y \in B$ so that
  $z = (x, y)$. Now $\tmop{if} e \in \{ x \}$ then $e = x$ proving that $e \in
  A$, hence we have, by definition of the union, that $\{ x \} \subseteq A
  \bigcup B$. By definition of the $\mathcal{P} \left( A \bigcup B \right)$
  set it follows then that
  \[ \{ x \} \in \mathcal{P} \left( A \bigcup B \right) \]
  Likewise if $e \in \{ x, y \}$ then either $e = x \Rightarrow e \in A$ or $e
  = y \Rightarrow e \in B$, hence ,by definition of the union, \ we have $\{
  x, y \} \subseteq A \bigcup B$. Using the definition $\mathcal{P} \left( A
  \bigcup B \right)$ we have then
  \[ \{ x, y \} \in \mathcal{P} \left( A \bigcup B \right) \]
  Now if $e \in \{ \{ x \}, \{ x, y \} \}$ then either $e = \{ x \} \in
  \mathcal{P} \left( A \bigcup B \right)$ or $e = \{ z, y \} \in \mathcal{P}
  \left( A \bigcup B \right)$ which proves that $\{ \{ x \}, \{ x, y \} \}
  \subseteq \mathcal{P} \left( A \bigcup B \right)$ or
  \[ z \in \{ \{ x \}, \{ x, y \} \} \in \mathcal{P} \left( \mathcal{P}
     \left( A \bigcup B \right) \right) \]
  giving finally
  \[ A \times B \subseteq \mathcal{P} \left( \mathcal{P} \left( A \bigcup B
     \right) \right) \]
\end{proof}

\begin{theorem}
  \label{set A*B}If $A$ and $B$ are sets then $A \times B$ is a set
\end{theorem}

\begin{proof}
  As $A, B$ are sets we have by [theorem: \ref{set union of two sets is a
  set}] that $A \bigcup B$ is a set, using the Axiom of Power sets [axiom:
  \ref{axiom of power}] it follows that $\mathcal{P} \left( A \bigcup B
  \right)$ is a set, using the Axiom of Power sets [axiom: \ref{axiom of
  power}] again proves that $\mathcal{P} \left( \mathcal{P} \left( A \bigcup B
  \right) \right)$ is a set. Finally by [lemma: \ref{set A*B is subset of
  P(P(AUB))}] we have that $A \times B \subseteq \mathcal{P} \left(
  \mathcal{P} \left( A \bigcup B \right) \right)$, which using the Axiom of
  Subsets [axiom: \ref{axiom of subsets}] proves that
  \[ A \times B \text{ is a set} \]
\end{proof}

\chapter{Partial Functions and Functions}

\section{Pairs and Triples}

Although we have already defined the concept of a pair, we can not simple
extend this to pairs (and later triples) of classes. If $A, B$ are pure
classes (classes that are not elements) then we can not just form $(A, B) = \{
A, \{ B \} \}$ because this would mean that $A, B$ are elements and not pure
classes. So we need another way of forming pairs, triples and so on.

\begin{definition}
  If $A, B$ are classes then $\langle A, B \rangle$ is defined by $\langle A,
  B \rangle = (A \times \{ \varnothing \}) \bigcup (B \times \{ \{ \varnothing
  \} \})$
\end{definition}

We show now that from $\langle A, B \rangle = \langle A',' B \rangle$ it
follows that $A = A' \wedge B = B'$, first we need some lemma's

\begin{lemma}
  \label{set emptyset is not set of empty set}We have $\varnothing \neq \{
  \varnothing \}$
\end{lemma}

\begin{proof}
  Assume that $\{ \varnothing \} = \varnothing$ then, as $\varnothing \in \{
  \varnothing \}$ it follows that $\varnothing$ which is a contradiction,
  hence
  \[ \varnothing \neq \{ \varnothing \} \]
  
\end{proof}

\begin{lemma}
  \label{<less>A,B<gtr>=<less>C,D<gtr>=<gtr>A=C,B=D}If $A, B, C, D$ are
  classes then $\langle A, B \rangle = \langle C, D \rangle \Leftrightarrow A
  = C \wedge B = D$
\end{lemma}

\begin{proof}
  
  \begin{description}
    \item[$\Rightarrow$] Assume that $\langle A, B \rangle = \langle C, D
    \rangle$ then by definition
    \begin{equation}
      \label{eq 2.1.001.1} (A \times \{ \varnothing \}) \bigcup (B \times \{
      \{ \varnothing \} \}) = (C \times \{ \varnothing \}) \bigcup (D \times
      \{ \{ \varnothing \} \})
    \end{equation}
    Let now $x \in A$ then $(x, \varnothing) \in (A \times \{ \varnothing \})$
    so that by the axiom of extent [axiom: \ref{axiom of extent}] and [eq:
    \ref{eq 2.1.001.1}]
    \[ (x, \varnothing) \in (C \times \{ \varnothing \}) \bigcup (D \times \{
       \{ \varnothing \} \}) \]
    which by the definition of the union gives
    \begin{equation}
      \label{eq 2.1.001} (x, \varnothing) \in C \times \{ \varnothing \} \vee
      (x, \varnothing) \in D \times \{ \{ \varnothing \} \}
    \end{equation}
    Now if $(x, \varnothing) \in D \times \{ \{ \varnothing \} \}$ then
    $\varnothing \in \{ \{ \varnothing \} \}$ or $\varnothing = \{ \varnothing
    \}$ which is impossible by [lemma: \ref{set emptyset is not set of empty
    set}] so that by [eq: \ref{eq 2.1.001}] we have $(x, \varnothing) \in C
    \times \{ \varnothing \}$, hence $x \in C$. This proves that
    \begin{equation}
      \label{eq 2.3.001.2} A \subseteq C
    \end{equation}
    Likewise, let $x \in C$ then $(x, \varnothing) \in (C \times \{
    \varnothing \})$ so that by the axiom of extent [axiom: \ref{axiom of
    extent}] and [eq: \ref{eq 2.1.001.1}
    \[ (x, \varnothing) \in (A \times \{ \varnothing \}) \bigcup (B \times \{
       \{ \varnothing \} \}) \]
    which by the definition of the union gives
    \begin{equation}
      \label{eq 2.2.001} (x, \varnothing) \in A \times \{ \varnothing \} \vee
      (x, \varnothing) \in B \times \{ \{ \varnothing \} \}
    \end{equation}
    Now if $(x, \varnothing) \in B \times \{ \{ \varnothing \} \}$ then
    $\varnothing \in \{ \{ \varnothing \} \}$ or $\varnothing = \{ \varnothing
    \}$ which is impossible by [lemma: \ref{set emptyset is not set of empty
    set}] so that by [eq: \ref{eq 2.2.001}] we have $(x, \varnothing) \in C
    \times \{ \varnothing \}$, hence $x \in A$. This proves that
    \begin{equation}
      \label{eq 2.5.001.2} C \subseteq A
    \end{equation}
    Combining [eq: \ref{eq 2.3.001.2}, \ref{eq 2.5.001.2}] with [theorem:
    \ref{class properties (1)}] proves
    \[ A = C \]
    Further if $x \in B$ then $(x, \{ \varnothing \}) \in B \times \{ \{
    \varnothing \} \}$ so that by the axiom of extent [axiom: \ref{axiom of
    extent}] and [eq: \ref{eq 2.1.001.1}]
    \[ (x, \{ \varnothing \}) \in (C \times \{ \varnothing \}) \bigcup (D
       \times \{ \{ \varnothing \} \}) \]
    or using the definition of the union that
    \begin{equation}
      \label{eq 2.6.001} (x, \{ \varnothing \}) \in C \times \{ \varnothing \}
      \vee (x, \{ \varnothing \}) \in D \times \{ \{ \varnothing \} \}
    \end{equation}
    If $(x, \{ \varnothing \}) \in C \times \{ \varnothing \}$ then $\{
    \varnothing \} \in \{ \varnothing \}$ or $\{ \varnothing \} = \varnothing$
    which is impossible by [lemma: \ref{set emptyset is not set of empty
    set}], so by [eq: \ref{eq 2.6.001}] we have that $(x, \{ \varnothing \})
    \in D \times \{ \{ \varnothing \} \}$, hence $x \in D$. This proves that
    \begin{equation}
      \label{eq 2.7.001} B \subseteq D
    \end{equation}
    Likewise, if $x \in D$ then $(x, \{ \varnothing \}) \in D \times \{ \{
    \varnothing \} \}$ so that by the axiom of extent [axiom: \ref{axiom of
    extent}] and [eq: \ref{eq 2.1.001.1}]
    \[ (x, \{ \varnothing \}) \in (A \times \{ \varnothing \}) \bigcup (B
       \times \{ \{ \varnothing \} \}) \]
    or using the definition of the union that
    \begin{equation}
      \label{eq 2.8.001} (x, \{ \varnothing \}) \in A \times \{ \varnothing \}
      \vee (x, \{ \varnothing \}) \in B \times \{ \{ \varnothing \} \}
    \end{equation}
    If $(x, \{ \varnothing \}) \in A \times \{ \varnothing \}$ then $\{
    \varnothing \} \in \{ \varnothing \}$ or $\{ \varnothing \} = \varnothing$
    which is impossible by [lemma: \ref{set emptyset is not set of empty
    set}], so by [eq: \ref{eq 2.8.001}] we have that $(x, \{ \varnothing \})
    \in B \times \{ \{ \varnothing \} \}$, hence $x \in B$. This proves that
    \begin{equation}
      \label{eq 2.9.001} D \subseteq B
    \end{equation}
    Combining [eq: \ref{eq 2.7.001}, \ref{eq 2.9.001}] with [theorem:
    \ref{class properties (1)}] proves
    \[ B = D \]
    \item[$\Leftarrow$] Assume that $A = C \wedge B = D$ then
    \begin{eqnarray*}
      x \in \langle A, B \rangle & \Leftrightarrow & x \in (A \times \{
      \varnothing \}) \bigcup (B \times \{ \{ \varnothing \} \})\\
      & \Leftrightarrow & x \in A \times \{ \varnothing \} \vee x \in B
      \times \{ \{ \varnothing \} \}\\
      & \Leftrightarrow & (x = (a, \varnothing) \wedge a \in A) \vee (x = (b,
      \{ \varnothing \}) \wedge b \in B)\\
      & \Leftrightarrowlim_{\text{[axiom: \ref{axiom of extent}]}} & (x = (a,
      \varnothing) \wedge a \in C) \vee (x = (b, \{ \varnothing \}) \wedge b
      \in D)\\
      & \Leftrightarrow & x \in (C \times \{ \varnothing \}) \bigcup (D
      \times \{ \{ \varnothing \} \})\\
      & \Leftrightarrow & e \in \langle C, D \rangle
    \end{eqnarray*}
    so that by the Axiom of Extent [axiom: \ref{axiom of extent}]
    \[ \langle A, B \rangle = \langle C, D \rangle \]
  \end{description}
\end{proof}

We can now easily extend $\langle A, B \rangle$ to a triple $\langle A, B, C
\rangle$.

\begin{definition}
  {\index{$\langle A, B, C \rangle$}}Let $A, B, C$ be classes then $\langle A,
  B, C \rangle$ is defined by
  \[ \langle A, B, C \rangle = \langle \langle A, B \rangle, C \rangle \]
\end{definition}

\begin{lemma}
  \label{<less>A,B,C<gtr>=<less>D,E,F<gtr>=<gtr>A=E,B=D,C=F}Let $A, B, C, D,
  E, F$ be classes then
  \[ \langle A, B, C \rangle = \langle D, E, F \rangle \Leftrightarrow A = D
     \wedge B = E \wedge C = F \]
\end{lemma}

\begin{proof}
  
  \begin{description}
    \item[$\Rightarrow$] Assume that $\langle A, B, C \rangle = \langle D, E,
    F \rangle$ then by definition $\langle \langle A, B \rangle, C \rangle =
    \langle \langle D, E \rangle, F \rangle$, by [lemma:
    \ref{<less>A,B<gtr>=<less>C,D<gtr>=<gtr>A=C,B=D}] then $C = F \wedge
    \langle A, B \rangle = \langle D, E \rangle$, using [lemma:
    \ref{<less>A,B<gtr>=<less>C,D<gtr>=<gtr>A=C,B=D}] again proves then $A = D
    \wedge B = E$.
    
    \item[$\Leftarrow$] Assume that $A = D \wedge B = E \wedge C = F$ then by
    [lemma: \ref{<less>A,B<gtr>=<less>C,D<gtr>=<gtr>A=C,B=D}] $\langle A, B
    \rangle = \langle D, E \rangle$, using [lemma:
    \ref{<less>A,B<gtr>=<less>C,D<gtr>=<gtr>A=C,B=D}] again we have $\langle
    \langle A, B \rangle, C \rangle = \langle \langle D, E \rangle, F \rangle$
    which by definition proves that
    \[ \langle A, B, C \rangle = \langle D, E, F \rangle \]
  \end{description}
\end{proof}

\section{Partial functions and Functions}

The concept of a function as a mapping of one value to a unique value is used
throughout mathematics, especially in analysis, which is essential a theory of
functions. Note that a function maps a value $x$ to a \tmtextbf{unique} value
$y$ which in the context of a set theory defines a pair $(x, y)$. This leads
to the following definition of a graph.

\subsection{Partial function}

\begin{definition}[Graph]
  \label{parttial function graph}A graph is a sub class of $\mathcal{U} \times
  \mathcal{U}$, or in other words a graph is a collection of pairs.
\end{definition}

\begin{definition}
  \label{partial function}A triple $\langle A, B, f \rangle$ where $A, B$ are
  classes and $f$ a graph is a \tmtextbf{partial function between $A$ and $B$}
  if
  \begin{enumerate}
    \item $f \subseteq A \times B$
    
    \item If $(x, y) \in f \wedge (x, y') \in f$ then $y = y'$
  \end{enumerate}
  We call $A$ the \tmtextbf{source} of the partial function, $B$ the
  \tmtextbf{destination} of the partial function and $f$ the \tmtextbf{graph}
  of the partial function.
\end{definition}

\begin{remark}
  Instead of writing $\langle A, B, f \rangle$ for a partial function between
  $A$ and $B$ we use the notation $f : A \rightarrow B$ or $A
  \xrightarrow{f}$B. Further the condition (2) ensures that only one value can
  be associated with $x$. So it is useful to use a special notation for this
  unique value, especially if we have a expression to calculate this unique
  value.
\end{remark}

\begin{definition}
  Let $f : A \rightarrow B$ be a partial function then $(x, y) \in f$ is
  equivalent with $y = f (x)$
\end{definition}

From now on we will use the Axiom of Construction [axiom: \ref{axiom of
construction}] to define different classes related to partial functions
without explicitly mentioning this. It is assumed that the reader understand
when to use this axiom.

\begin{definition}
  Let $f : A \rightarrow B$ be a partial function then its domain noted as
  $\tmop{dom} (f)$ and range noted as $\tmop{range} (f)$ is defined by
  \[ \tmop{dom} (f) = \left\{ x| \exists y \text{ such that } (x, y) \in f
     \right\} \]
  \[ \tmop{range} (f) = \left\{ y| \exists x \text{ such that } (x, y) \in f
     \right\} \]
\end{definition}

\begin{theorem}
  \label{partial function dom(f) range(f) as subclasses}If $f : A \rightarrow
  B$ is a partial function then $\tmop{dom} (f) \subseteq A$ and $\tmop{range}
  (f) \subseteq B$
\end{theorem}

\begin{proof}
  If $x \in \tmop{dom} (f)$ then $\exists y$ such that $(x, y) \in f
  \Rightarrowlim_{f \subseteq A \times B} (x, y) \in A \times B$ proving that
  $x \in A$, hence
  \[ \tmop{dom} (f) \subseteq A \]
  Further if $y \in \tmop{range} (f)$ then $\exists x$ such that $(x, y) \in f
  \Rightarrowlim_{f \subseteq A \times B} (x, y) \in A \times B$ proving that
  $y \in B$, hence
  \[ \tmop{range} (f) \subseteq B \]
\end{proof}

\begin{corollary}
  \label{partial function set domain range}If $A, B$ are sets and $f : A
  \rightarrow B$ a partial function then $\tmop{dom} (f)$ and $\tmop{range}
  (f)$ are sets
\end{corollary}

\begin{proof}
  Using [theorem: \ref{partial function dom(f) range(f) as subclasses}] we
  have that $\tmop{dom} (f) \subseteq A$ and $\tmop{range} (f) \subseteq B$,
  so applying the Axiom of Subsets [axiom: \ref{axiom of subsets}] proves that
  $\tmop{dom} (f)$ and $\tmop{range} (f)$ are sets.
\end{proof}

\begin{definition}
  \label{partial function image}{\index{image}}Let $f : A \rightarrow B$ be a
  partial function and $C$ a class such that $C \subseteq A$ then\tmtextbf{
  the image of $C$ by $f$ }noted as $f (C)$ is defined by
  \[ f (C) = \left\{ y| \exists x \in C \text{ such that } (x, y) \in f
     \right\} \]
\end{definition}

\begin{remark}
  Note that we use a conflicting notation here. On one hand $y = f (x)$ can be
  interpreted as $(x, y) \in f$, on the other hand it can also means that $y$
  is the image of $x$ by $f$. We adopt the following convention. If lower
  cases are used as in $y = f (x)$ we interpret this as $(x, y) \in f$ and if
  we use uppercase like in $f (C)$ we are talking about images. In case of
  doubt $(f) (C)$ always refers to the image.
\end{remark}

\begin{definition}
  \label{partial function preimage}Let $f : A \rightarrow B$ be a partial
  function and $C$ a class then \tmtextbf{the preimage of C by f} noted as
  $f^{- 1} (C)$ is defined by
  \[ f^{- 1} (C) = \left\{ x| \exists y \in C \text{ such that } (x, y) \in f
     \right\} \]
\end{definition}

\begin{note}
  In contrast with most text books we do not require that $C \subseteq B$,
  this will give us more flexibility if we compose partial functions.
\end{note}

\begin{theorem}
  \label{partial functions image/preimage properties}Let $f : A \rightarrow B$
  be a partial function, $C \subseteq A$ and $D$ a class then we have:
  \begin{enumerate}
    \item $f (C) \subseteq \tmop{range} (f)$
    
    \item $f^{- 1} (D) \subseteq \tmop{dom} (f)$
    
    \item $f (A) = \tmop{range} (f)$
    
    \item $f^{- 1} (B) = \tmop{dom} (f)$
    
    \item If $E \subseteq C$ then $f (E) \subseteq f (C)$
    
    \item If $E \subseteq D$ then $f^{- 1} (E) \subseteq f^{- 1} (D) $
  \end{enumerate}
  and if in addition $A, B$ are sets then $f (C)$ and $f^{- 1} (D)$ are sets
\end{theorem}

\begin{proof}
  
  \begin{enumerate}
    \item If $y \in f (C)$ then there exists a $x \in C$ such that $(x, y) \in
    f$, so $y \in \tmop{range} (f)$. Hence
    \[ f (C) \subseteq \tmop{range} (f) \]
    \item If $x \in f^{- 1} (D)$ then there exists a $y \in D$ such that $(x,
    y) \in f$, which proves that $x \in \tmop{dom} (f)$, hence
    \[ f^{- 1} (D) \subseteq \tmop{dom} (f) \]
    \item If $y \in \tmop{range} (f)$ then $\exists x$ such that $(x, y) \in
    f$, which as $f \subseteq A \times B$ proves that $x \in A$, hence $y \in
    f (A)$, or $\tmop{range} (f) \subseteq f (A)$. From (1) we have $f (A)
    \subseteq \tmop{range} (f)$, so using [theorem: \ref{class properties
    (1)}]
    \[ f (A) = \tmop{range} (f) \]
    \item If $x \in \tmop{dom} (f)$ then $\exists y$ such that $(x, y) \in f$,
    which as $f \subseteq A \times B$ proves that $y \in B$, giving $x \in
    f^{- 1} (B)$, hence $\tmop{dom} (f) \subseteq f^{- 1} (B)$. From (2) we
    have $f^{- 1} (B) \subseteq \tmop{dom} (f)$, so using [theorem: \ref{class
    properties (1)}]
    \[ f^{- 1} (B) = \tmop{dom} (f) \]
    \item If $y \in f (E)$ then $\exists x \in E$ such that $(x, y) \in f$, as
    $E \subseteq C$ we have $x \in C$ and still $(x, y) \in f$ so that $y \in
    f (C)$ proving
    \[ f (E) \subseteq f (C) \]
    \item If $x \in f^{- 1} (E)$ there $\exists y \in E$ such that $(x, y) \in
    f$, as $E \subseteq D$ we have $y \in D$ and still $(x, y) \in f$ so that
    $x \in f^{- 1} (D)$ proving
    \[ f^{- 1} (E) \subseteq f^{- 1} (D) \]
  \end{enumerate}
  Finally if $A$, $B$ are sets then using [theorem: \ref{partial function set
  domain range}] $\tmop{range} (f)$ and $\tmop{dom} (f)$ are sets, applying
  then the Axiom of Subsets [axiom: \ref{axiom of subsets}] proves that $f
  (C)$ and $f^{- 1} (D)$ are sets.
\end{proof}

Next we define the composition of two partial functions.

\begin{definition}[Composition of graphs]
  \label{partial function composition of graphs}{\index{$g \circ f$}}Let $f,
  g$ be two graphs then $f \circ g$ is defined by
  \[ g \circ f = \left\{ z|z = (x, y) \text{ such that $\exists u \text{ with
     $(x, u) \in f \wedge (u, y) \in g$}$} \right\} \]
\end{definition}

\begin{theorem}
  \label{partial function composition of partial functions}Let $f : A
  \rightarrow B$ and $g : C \rightarrow D$ be partial functions then
  \[ g \circ f : A \rightarrow D \]
  is a partial function. We call $g \circ f : A \rightarrow D$ the
  \tmtextbf{composiiton} of $f : A \rightarrow B$ and $g : C - D$
\end{theorem}

\begin{proof}
  If $(x, y) \in g \circ f$ then there exist a $u$ such that $(x, u) \in f$
  and $(u, y) \in g$, as $f$, $g$ are partial functions we have that $f
  \subseteq A \times B$ and $g \subseteq C \times D$. So $(x, u) \in A \times
  B$ and $(u, y) \in C \times D$. So $x \in A$ and $y \in D$ proving that $(x,
  y) \in A \times D$. Hence
  \[ g \circ f \subseteq A \times D \]
  Further if $(x, y) \in g \circ f \wedge (x, y') \in g \circ f$ then there
  exists $u, v$ such that $(x, u) \in f \wedge (u, y) \in g \wedge (x, v) \in
  f \wedge (v, y') \in g$. From $(x, u) \in f \wedge (x, v) \in f$ it follows
  [as $f$ is a partial function] that $u = v$. So $(u, y) \equallim_{\text{u=v
  and [theorem: \ref{pair equality of pairs}]}} (u, y') \in g$. Hence as $g$
  is a partial function it follows that $y = y'$. To summarize
  \[ \tmop{If} (x, y) \in g \circ f \wedge (x, y') \in g \circ f \text{ then }
     y = y' \]
  So all the requirements for $g \circ f : A \rightarrow D$ to be a partial
  function are satisfied.
\end{proof}

\begin{note}
  In contrast with most textbooks we do not require that $B = C$ in this
  theorem, there is no need for this because for partial functions $\tmop{dom}
  (f \circ g)$ can be different from $A$. Later we will compose functions and
  then we will need a extra condition for $C$.
\end{note}

\begin{theorem}[Associativity of Composition]
  \label{partial function associativity}Let $f : A \rightarrow B$, $g : C
  \rightarrow D$ and $h : E \rightarrow F$ be partial functions then $h \circ
  (g \circ f) = (h \circ g) \circ f$
\end{theorem}

\begin{proof}
  If $(x.z) \in h \circ (g \circ f)$ then $\exists u$ such that $(x, u) \in g
  \circ f$ and $(u, z) \in h$. As $(x, u) \in g \circ f$ there exists a $v$
  such that $(x, v) \in f$ and $(v, u) \in g$. As $(v, u) \in g \wedge (u, z)
  \in h$ we have that $(v, z) \in h \circ g$, as $(x, v) \in f$ it follows
  $(x, z) \in (h \circ g) \circ f$.
  
  If $(x, z) \in (h \circ g) \circ f$ there $\exists u$ such that $(x, u) \in
  f$ and $(u, z) \in h \circ g$. As $(u, z) \in h \circ g$ there $\exists v$
  such that $(u, v) \in g$ and $(v, z) \in h$. From $(x, u) \in f$ and $(u, v)
  \in g$ we have that $(x, v) \in g \circ f$. As $(v, z) \in h$ we have that
  $(x, z) \in h \circ (h \circ f)$.
  
  Using the Axiom of Extent [axiom: \ref{axiom of extent}] it follows that
  \[ h \circ (g \circ f) = (h \circ g) \circ f \]
\end{proof}

Let's look now at the domain and range of of the composition of two partial
functions.

\begin{theorem}
  \label{partial function domain range composition}Let $f : A \rightarrow B$
  and $g : C \rightarrow D$ be partial functions then for $g \circ f : A
  \rightarrow D$ we have
  \begin{enumerate}
    \item $\tmop{dom} (g \circ f) = \tmop{dom} (f) \bigcap f^{- 1} (\tmop{dom}
    (g))$
    
    \item $\tmop{range} (g \circ f) = g \left( \tmop{range} (f) \bigcap
    \tmop{dom} (g) \right)$
    
    \item $\tmop{range} (g \circ f) \subseteq \tmop{range} (g)$
  \end{enumerate}
\end{theorem}

\begin{proof}
  
  \begin{enumerate}
    \item If $x \in \tmop{dom} (g \circ f)$ then there exist a $z$ such that
    $(x, z) \in g \circ f$. So there exist a $y$ such that $(x, y) \in f$ and
    $ (y, z) \in g$, hence $x \in \tmop{dom} (f)$ and $y \in \tmop{dom} (g)
    \Rightarrowlim_{(x, y) \in f} x \in f^{- 1} (\tmop{dom} (g))$. So $x \in
    \tmop{dom} (f) \bigcap f^{- 1} (\tmop{dom} (g))$. Hence
    \begin{equation}
      \label{eq 2.10.001} \tmop{dom} (g \circ f) \subseteq \tmop{dom} (f)
      \bigcap f^{- 1} (\tmop{dom} (g))
    \end{equation}
    If $x \in \tmop{dom} (f) \bigcap f^{- 1} (\tmop{dom} \nobracket (g))
    \nobracket$ then $x \in \tmop{dom} (f)$ so that $\exists y$ such that $(x,
    y) \in f$ and $x \in f^{- 1} (\tmop{dom} (g))$ so that $\exists y' \in
    \tmop{dom} (g)$ such that $(x, y') \in f$. As $f$ is a partial function it
    follows that $y = y'$. So $y \in \tmop{dom} (g)$, from which it follows
    that $\exists z$ such that $(y, z) \in g$. As we have $(x, y) \in f$ and
    $(y, z) \in g$ it follows that $(x.z) \in g \circ f$ or $x \in \tmop{dom}
    (g \circ f)$. This proves that $\tmop{dom} (f) \bigcap f^{- 1} (\tmop{dom}
    (g)) \subseteq \tmop{dom} (g \circ f)$, combining this with [eq: \ref{eq
    2.10.001}] allows us to use [theorem: \ref{class properties (1)}] to get
    \[ \tmop{dom} (g \circ f) = \tmop{dom} (f) \bigcap f^{- 1} (\tmop{dom}
       (g)) \]
    \item If $z \in \tmop{range} (g \circ f)$ then there exists a $x \in A$
    such that $(x, z) \in g \circ f$, so there exist a $y$ such that $(x, y)
    \in f \wedge (y, z) \in g$. Then $y \in \tmop{range} (f)$ and $y \in
    \tmop{dom} (g)$ or $y \in \tmop{range} (f) \bigcap \tmop{dom} (g)$, which
    as $(y, z) \in g$ proves that $z \in g \left( \tmop{range} (f) \bigcap
    \tmop{dom} (g) \right)$. Hence
    \begin{equation}
      \label{eq 2.11.001} \tmop{range} (g \circ f) \subseteq g \left(
      \tmop{range} (f) \bigcap \tmop{dom} (g) \right)
    \end{equation}
    If $z \in g \left( \tmop{range} (f) \bigcap \tmop{dom} (g) \right)$ then
    $\exists y \in \tmop{range} (f) \bigcap \tmop{dom} (g)$ such that $(y, z)
    \in g$. From $y \in \tmop{range} (f)$ it follows that there exist a $x$
    such that $(x, y) \in f$. So $(x, z) \in g \circ f$ proving that $x \in
    \tmop{range} (g \circ f)$, hence $g \left( \tmop{range} (f) \bigcap
    \tmop{dom} (g) \right) \subseteq \tmop{range} (g \circ f)$. Combining this
    with [eq: \ref{eq 2.11.001}] allows us to use [theorem: \ref{class
    properties (1)}] to get
    \[ \tmop{range} (g \circ f) = g \left( \tmop{range} (f) \bigcap
       \tmop{dom} (g) \right) \]
    \item If $z \in \tmop{range} (g \circ f)$ then there exists a $x$ such
    that $(x, z) \in g \circ f$, so there exists a $y$ such that $(x, y) \in f
    \wedge (y, z) \in g$. Hence $z \in \tmop{range} (g)$.
  \end{enumerate}
\end{proof}

\begin{theorem}
  \label{partial function image preimage of compositions}If $f : A \rightarrow
  B$ and $g : C \rightarrow D$ are partial functions then we have
  \begin{enumerate}
    \item If $E \subseteq A$ then $(g \circ f) (E) = g (f (E))$
    
    \item If $E \subseteq D$ then $(g \circ f)^{- 1} (E) = f^{- 1} (g^{- 1}
    (E))$
  \end{enumerate}
\end{theorem}

\begin{proof}
  
  \begin{enumerate}
    \item If $z \in (g \circ f) (E)$ then there exists a $x \in E$ such that
    $(x, z) \in g \circ f$. So by definition there exist a $y$ such that $(x,
    y) \in f \wedge (y, z) \in g$. From $(x, y) \in f$ it follows that $y \in
    f (E)$ and as $(y, z) \in g$ it follows that $z \in g (f (E))$. Hence
    \begin{equation}
      \label{eq 2.12.002} (g \circ f) (E) \subseteq g (f (E))
    \end{equation}
    On the other hand if $z \in g (f (E))$ there exist a $y \in f (E)$ such
    that $(y, z) \in g$. As $y \in f (E)$ there exists a $x \in E$ such that
    $(x, y) \in f$. From $(x, y) \in f \wedge (y, z) \in g$ it follows that
    $(x, z) \in g \circ f$ so that [as $x \in E$] $z \in (g \circ f) (E)$.
    Proving $g (f (E)) \subseteq (g \circ f) (E)$, combining this with [eq
    \ref{eq 2.12.002}] and [theorem: \ref{class properties (1)}] gives
    \[ (g \circ f) (E) = g (f (E)) \]
    \item If $x \in (g \circ f)^{- 1} (E)$ then there exist a $z \in E$ such
    that $(x, z) \in g \circ f$, hence $\exists y$ such that $(x, y) \in f
    \wedge (y, z) \in g$. So by definition $y \in g^{- 1} (E)$ and as $(x, y)
    \in f$ it follows that $x \in f^{- 1} (g^{- 1} (E))$. Hence
    \begin{equation}
      \label{eq 2.13.002} (g \circ f)^{- 1} (E) \subseteq f^{- 1} (g^{- 1}
      (E))
    \end{equation}
    If $x \in f^{- 1} (g^{- 1} (E))$ then there exist a $y \in g^{- 1} (E)$
    such that $(x, y) \in f$, as $y \in g^{- 1} (E)$ then there exist a $z \in
    E$ such that $(y, z) \in g$. From $z \in E \wedge (x, y) \in f \wedge (y,
    z) \in g$ it follows that $x \in (g \circ f)^{- 1} (E)$ proving that $f^{-
    1} (g^{- 1} (E)) \subseteq (g \circ f)^{- 1} (E)$. Combining this with
    [eq: \ref{eq 2.13.002}] and [theorem: \ref{class properties (1)}] gives
    \[ (g \circ f)^{- 1} (E) = f^{- 1} (g^{- 1} (E)) \]
  \end{enumerate}
\end{proof}

\subsection{Functions}

\begin{definition}
  \label{function}{\index{function}}A partial function $f : A \rightarrow B$
  is a \tmtextbf{function} iff $\tmop{dom} (f) = A$
\end{definition}

So every function is also a partial function, hence statements about partial
functions applies also for functions. One special benefiit of functions is the
following.

\begin{theorem}
  \label{function preimage of image (1)}If $f : A \rightarrow B$ is a function
  then for $C \subseteq A$ we have $C \subseteq f^{- 1} (f (C))$.
\end{theorem}

\begin{proof}
  If $x \in C \subseteq A$ then as $A = \tmop{dom} (f)$ there exist a $y$ such
  that $(x, y) \in f$ so that $y \in f (C)$, which as $(x, y) \in f$ proves
  that $x \in f^{- 1} (f (C))$. Hence we have $C \subseteq f^{- 1} (f (C))$.
\end{proof}

\begin{proposition}
  \label{function condition (1)}A partial function $f : A \rightarrow B$ is a
  function iff $A \subseteq \tmop{dom} (f)$
\end{proposition}

\begin{proof}
  As $A \subseteq \tmop{dom} (f)$ and $\tmop{dom} (f) \subseteq A$ [theorem:
  \ref{partial function dom(f) range(f) as subclasses}] we have by [theorem:
  \ref{class properties (1)}] that
  \[ \tmop{dom} (f) = A \]
\end{proof}

\begin{example}
  \label{function between {0,1} and {A,B}}Let $A, B$ be elements and define $f
  = \{ (0, A), (1, B) \}$ then $f : \{ 0, 1 \} \rightarrow \{ A, B \}$ is a
  function 
\end{example}

\begin{proof}
  If $(x, y) \in f$ then
  \[ (x, y) = (0, A) \Rightarrow x = 1 \in \{ 0, 1 \} \wedge y = A \in \{ A, B
     \} \text{ so that } (x, y) \in \{ 0, 1 \} \times \{ A, B \} \]
  or
  \[ (x, y) = (1, B) \Rightarrow x = 1 \in \{ 0, 1 \} \wedge y = B \in \{ A, B
     \} \text{ so that } (x, y) \in \{ 0, 1 \} \times \{ A, B \} \]
  proving that
  \[ f \subseteq \{ 0, 1 \} \times \{ A, B \} \]
  If $(x, y), (x, y') \in f$ then for $(x, y)$ we have either:
  \begin{description}
    \item[$(x, y) = (0, A)$] Then $x = 0$ and $y = A$ so that $(x', y') = (0,
    y') \in f \Rightarrow y' = A$ hence $y = y'$
    
    \item[$(x, y) = (1, B)$] Then $x = 1$ and $y = B$ so that $(x', y') = (1,
    y') \in f \Rightarrow y' = B$ hence $y = y'$
  \end{description}
  which proves that
  \[ f : \{ 0, 1 \} \rightarrow \{ A, B \} \text{ is a partial function} \]
  If $x \in \{ 0, 1 \}$ then either $x = 0$ so that $(0, A) \in f$ or $x = 1$
  so that $(1, B) \in f$, so $\{ 0, 1 \} \subseteq \tmop{dom} (f)$. Using
  [proposition: \ref{function condition (1)}] it follows that
  \[ f : \{ 0, 1 \} \rightarrow \{ A, B \} \text{ is a function} \]
\end{proof}

Although the composition of functions $f : A \rightarrow B$ and $g : C
\rightarrow D$ is a partial function [see theorem: \ref{partial function
composition of partial functions}], it does not have to be a function as we
need the extra requirement that $\tmop{dom} (g \circ f) = A$. So we must have
a extra condition on $C$. This is expressed in the following theorem,

\begin{theorem}
  \label{function composition of functions is a fucntion}Let $f : A
  \rightarrow B$ and $g : C \rightarrow D$ functions with $f (A) \subseteq C$
  then $g \circ f : C \rightarrow D$ is also a function with $\tmop{range} (g
  \circ f) = g (\tmop{range} (f))$
\end{theorem}

\begin{proof}
  Using [theorem: \ref{partial function composition of partial functions}] we
  have that
  \[ g \circ f : A \rightarrow D \text{ is a partial function} \]
  Using [theorem: \ref{function preimage of image (1)}] we have that $A
  \subseteq f^{- 1} (f (A))$ and by [theorem: \ref{partial functions
  image/preimage properties}] together with $f (A) \subseteq C$ we have $f^{-
  1} (f (A)) \subseteq f^{- 1} (C)$ proving that
  \begin{equation}
    \label{eq 2.14.038} A \subseteq f^{- 1} (C)
  \end{equation}
  
  
  Further using [theorem: \ref{partial function domain range composition}] we
  have
  \begin{eqnarray*}
    \tmop{dom} (g \circ f) & = & \tmop{dom} (f) \bigcap f^{- 1} (\tmop{dom}
    (g))\\
    & \equallim_{f, g \text{ are functions}} & A \bigcap f^{- 1} (C)\\
    & \equallim_{\text{[theorem: \ref{eq 2.14.038}]}} & A
  \end{eqnarray*}
  which proves that
  \[ g \circ f \text{ is a function} \]
  Finally
  \begin{eqnarray*}
    \tmop{range} (g \circ f) & \equallim_{\text{[theorem: \ref{partial
    function domain range composition}}} & g \left( \tmop{range} (f) \bigcap
    \tmop{dom} (g) \right)\\
    & \equallim_{f \text{ is a function}} & g \left( \tmop{range} (f) \bigcap
    C \right)\\
    & \equallim_{\text{[theorem: \ref{partial functions image/preimage
    properties}]}} & g \left( f (A) \bigcap C \right)\\
    & \equallim_{f (A) \subseteq C} & g (f (A))\\
    & \equallim_{\text{[theorem: \ref{partial functions image/preimage
    properties}]}} & g (\tmop{range} (f))
  \end{eqnarray*}
\end{proof}

Next we define the class of all the graphs of functions between two classes

\begin{note}
  Be aware that some books calls partial functions functions and functions
  mappings. 
\end{note}

\begin{definition}
  \label{function B^A}{\index{$B^A$}}Let $A, B$ be two classes then we define
  the class $B^A$ [using the Axiom of Construction] as
  \[ B^A = \left\{ f|f : A \rightarrow B \text{ is a function} \right\} \]
\end{definition}

\begin{note}
  $B^A$ is not the class of functions between $A$ and $B$, but the class of
  graphs of functions between $A$ and $B$. This distinction is important
  because it makes the following theorem possible.
\end{note}

\begin{example}
  \label{function A^empty is empty}Let $A$ be a class then $A^{\varnothing} =
  \{ \varnothing \}$
\end{example}

\begin{proof}
  Let $f \in A^{\varnothing}$ then $f : \varnothing \Rightarrow A$ is a
  function, so that $f \subseteq \varnothing \times A = \varnothing$ or $f =
  \varnothing$
\end{proof}

\begin{lemma}
  \label{function extend target}If $f : A \rightarrow B$ is a function and $B
  \subseteq C$ then $f : A \rightarrow C$ is a function
\end{lemma}

\begin{proof}
  As $f : A \rightarrow B$ is a function we have $f \subseteq A \times B$
  which as by [theorem: \ref{cartesian product and inclusion}] $A \times B
  \subseteq A \times C$ means that $f \subseteq A \times C$'. Further as $f :
  A \rightarrow B$ is a function we we have also $\tmop{dom} (f) = A$ and if
  $(x, y), (x, y') \in f$ then $y = y'$. So by definition$f : A \rightarrow C$
  is a function.
\end{proof}

\begin{theorem}
  \label{function B^A and inclusion}Let $A, B, C$ be classes such that $B
  \subseteq C$ then $B^A \subseteq C^A$
\end{theorem}

\begin{proof}
  Let $f \in B^A$ then $f : A \rightarrow B$ is a function, using the above
  lemma [lemma: \ref{function extend target}] we have that $f : A \rightarrow
  C$ is a function, hence $f \in C^A$ proving that
  \[ B^A \subseteq C^A \]
\end{proof}

We have also the following relation between $A \times B$ and $B^C$

\begin{theorem}
  \label{function: A^B and sets}Let $A, B$ be two classes then we have:
  \begin{enumerate}
    \item $B^A \subseteq A \times B$
    
    \item If $A, B$ are sets then $B^A$ is a set
  \end{enumerate}
\end{theorem}

\begin{proof}
  
  \begin{enumerate}
    \item If $f \in B^A$ then $f : A \rightarrow B$ is a function so that $f
    \subseteq A \times B$ proving that $B^A \subseteq A \times B$
    
    \item If $A, B$ are sets then by [theorem: \ref{set A*B}] we have that $A
    \times B$ is a set. So using the Axiom of Subsets [axiom: \ref{axiom of
    subsets}] we have that $f$ is a set,
  \end{enumerate}
\end{proof}

\begin{theorem}
  \label{function power of intersection}Let $A, B, C$ be classes then $A^C
  \bigcap B^C = \left( A \bigcap B \right)^C$
\end{theorem}

\begin{proof}
  First by [theorem: \ref{class intersection, union, inclusion}] we have $A
  \bigcap B \subseteq A$ and $A \bigcap B \subseteq B$ it follows from the
  above theorem [theorem: \ref{function B^A and inclusion}] that $\left( A
  \bigcap B \right)^C \subseteq A^C$ and $\left( A \bigcap B \right)^C
  \subseteq B^C$. Applying then [theorem: \ref{class inclusion and union and
  intersection}] gives
  \begin{equation}
    \label{eq 2.14.009} \left( A \bigcap B \right)^C \subseteq A^C \bigcap B^C
  \end{equation}
  For the opposite inclusion, let $f \in A^C \bigcap B^C$ then $f \in A^C
  \wedge f \in B^C$ so that $f : C \rightarrow A$ and $f : C \rightarrow B$
  are functions. Then we have that $f \subseteq C \times A$ and $f \subseteq C
  \times B$ so that
  \[ f \subseteq (C \times A) \bigcap (C \times B)
     \equallim_{\text{\ref{cartesian product properties (1)}}} \left( C
     \bigcap C \right) \times \left( A \bigcap B \right)
     \equallim_{\text{[theorem: \ref{class class
     commutative,idempotent,associative,distributivity}]}} C \times \left( A
     \bigcap B \right) \]
  Further as $f : A \rightarrow C$ is a function we have $(x, y), (x, y') \in
  f$ and $\tmop{dom} (f) = C$ so that
  \[ f : C \rightarrow \left( A \bigcap B \right) \text{ is a function} \]
  proving that $f \in \left( A \bigcap B \right)^I$. So $A^C \bigcap B^C
  \subseteq \left( A \bigcap B \right)^C$ which combined with [eq: \ref{eq
  2.14.009}] gives
  \[ A^C \bigcap B^C = \left( A \bigcap B \right)^C \]
\end{proof}

We have the follow trivial fact about a function

\begin{proposition}
  \label{function range restriction}Let $f : A \rightarrow B$ be a function
  then if $\tmop{range} (f) \subseteq C$ we have that $f : A \rightarrow C$ is
  a function.
\end{proposition}

\begin{proof}
  If $(x, y) \in f$ then $y \in \tmop{range} (f)$ hence as $\tmop{range} (f)
  \subseteq C$ $y \in C$. As $f \subseteq A \times B$ we have also $x \in A$
  so that $(x, y) \in C \times B$. Hence $f \subseteq A \times C$, further if
  $(x, y), (x, y') \in f$ we have as $f : A \rightarrow B$ is a function that
  $y = y'$. So
  \[ f : A \rightarrow C \text{ is a partial function} \]
  As $\tmop{range} (f) = A$ (because $f : A \rightarrow B$ is a function] we
  have that $f : A \rightarrow C$ a function
\end{proof}

We have the following trivial proposition about the equality of two functions

\begin{proposition}
  \label{function equality (1)}Two functions $f : A \rightarrow B$ and $g : A
  \rightarrow B$ are equal if
  \[ [(x, y) \in f \Rightarrow (x, y) \in g \wedge (x, y) \in g \Rightarrow
     (x, y) \in f] \]
\end{proposition}

\begin{proof}
  Note that the statement $f : A \rightarrow B$ and $g : A \rightarrow B$ are
  equal is equivalent with $\langle A, B, f \rangle = \langle A, B, g
  \rangle$, which by \ref{<less>A,B,C<gtr>=<less>D,E,F<gtr>=<gtr>A=E,B=D,C=F}
  is equivalent with $A = A \wedge B = B \wedge f = g$, As $A = A$ and $B = B$
  are true this is equivalent with $f = g$. Now by the Axiom of Extent [axiom:
  \ref{axiom of extent}] we have that
  \[ f = g \Leftrightarrow [(x, y) \in f \Rightarrow (x, y) \in g \wedge (x,
     y) \in g \Rightarrow (x, y) \in f] \]
  
\end{proof}

If $f : A \rightarrow B$ is a function then for every $x \in A$ we have a
unique $y \in B$ such that $(x, y) \in f$. Furthermore in many cases we have
actually a expression valid for every $x \in A$ to calculate this unique
value. To express this we use the following notation.

\begin{definition}
  \label{function f(x)}If $f : A \rightarrow B$ is a function then
  \[ \tmmathbf{y = f (x)} \text{ or $\tmmathbf{f (x) = y}$ is equivalent with
     } \tmmathbf{(x, y) \in f} \]
  and
  \[ \tmmathbf{f (x) = E (x)} \text{ where $\tmmathbf{E (x)} \text{ is a
     expression depending on \tmtextbf{$x$} is equivalent with } \tmmathbf{(x,
     E (x)) \in f}$} \]
  Further if $D \subseteq B$ then \tmtextbf{$\tmmathbf{f (x) \in D}$} is the
  same as \tmtextbf{$\exists y \in D$ such that $y = f (x)$ or $(x, y) \in f$}
\end{definition}

\begin{example}
  Let $3 \cdot x + 1$ be the value associated with $x$, so $f = \{ z|z = (x, 3
  \cdot x + 1) \in f \wedge x \in A \}$, then we can use the following
  equivalent notations to define our function
  \[ f : A \rightarrow B \text{is defined by } x \rightarrow 3 \cdot x + 1 \]
  If we have defined a function $f : A \rightarrow B$ using a expression and
  we want to refer to the expression of the function we use the notation $f
  (x)$. Hence we define a function also as
  \[ f : A \rightarrow B \text{ is defined by } x \rightarrow f (x) = 3 \cdot
     x + 1 \]
  or
  \[ f : A \rightarrow B \tmop{is} \tmop{defined} \tmop{by} x \rightarrow f
     (x) \tmop{where} f (x) = 3 \cdot x + 1 \]
  or
  \[ f : A \rightarrow B \text{ is defined by } f (x) = 3 \ast x + 1 \]
  In all of the above cases we actually means that $\langle f, A, B \rangle$
  is a function with $f = \{ z|z = (x, 3 \cdot x + 1) \wedge x \in A \}$.
\end{example}

Using the above notation we can reformulate [proposition: \ref{function
equality (1)}] in a form that is easier to work with if we use expressions to
define a function.

\begin{proposition}
  \label{function equality (2)}Two functions $f : A \rightarrow B$ and $g : A
  \rightarrow B$ are equal if and only if
  \[ \forall x \in A \text{ } f (x) = g (x) \]
\end{proposition}

\begin{proof}
  Assume that $f : A \rightarrow B$ and $g : A \rightarrow B$ are equal then
  if $x \in A$ we have $\exists y \in B$ such that $(x, y) \in f$ or $y = f
  (x)$, using [proposition: \ref{function equality (1)}] we have also $(x, y)
  \in g$ hence $y = g (x)$ which proves that $f (x) = g (x)$.
  
  On the other hand assume that $\forall x \in A$ $f (x) = g (x)$ then if
  $(x, y) \in f$ we have $y = f (x) = g (x)$ so that $(x, y) \in g$. If $(x,
  y) \in g$ then $y = g (x) = f (x)$ or $(x, y) \in g$. Using [proposition:
  \ref{function equality (1)}] we have then that $f : A \rightarrow B$ and $g
  : A \rightarrow B$ are equal.
\end{proof}

Using the new notation, composition of function is written as

\begin{theorem}
  \label{function alternative for composition}If $f : A \rightarrow B$ and $g
  : C \rightarrow D$ are two functions with $f (A) \subseteq C$ then
  \[ (g \circ f) (x) = g (f (x)) \]
\end{theorem}

\begin{proof}
  Take $z = (g \circ f) (x)$ then $(x, z) \in g \circ f$ so that $\exists y$
  such that $(x, y) \in f$ and $(y, z) \in g$. Hence $y = f (x)$ and $z = g
  (y)$ so that $z = g (f (x))$, proving $(g \circ f) (x) = g (f (x))$.
\end{proof}

Image and pre-image can also be expressed in the new notation.

\begin{proposition}
  \label{function image preimage alternative}Let $f : A \rightarrow B$ a
  function, $C \subseteq A$ and $D \subseteq B$\quad then
  \begin{enumerate}
    \item $y \in f (C) \Leftrightarrow \exists x \in A$ such that $y = f (x)$
    
    \item $x \in f^{- 1} (D) \Leftrightarrow f (x) \in D$
  \end{enumerate}
\end{proposition}

\begin{proof}
  
  \begin{enumerate}
    \item 
    \begin{eqnarray*}
      y \in f (C) & \Leftrightarrow & \exists x \in C \text{ such that } (x,
      y) \in f\\
      & \Leftrightarrow & \exists x \in C \text{ such that } y = f (x)
    \end{eqnarray*}
    \item
    \begin{eqnarray*}
      x \in f^{- 1} (C) & \Leftrightarrow & \exists y \in D \text{ such that }
      (x, y) \in f\\
      & \Leftrightarrow & \exists y \in D \text{ such that } y = f (x)\\
      & \Leftrightarrow & f (x) \in D
    \end{eqnarray*}
  \end{enumerate}
\end{proof}

Let's now look at some example of functions:

\begin{example}[Empty Function]
  \label{function empty function}$\varnothing : \varnothing \rightarrow B$
\end{example}

\begin{proof}
  First $\varnothing \subseteq \varnothing \times B$ by [theorem: \ref{class
  empty set}], if $x \in \tmop{dom} (\varnothing)$ then $\exists y \in
  \varnothing$ such that $(x, y) \in \varnothing$ which is a contradiction, so
  by [theorem: \ref{class empty set is unique}] we have that $\tmop{dom}
  (\varnothing) = \varnothing$. And finally $(x, y) \in \varnothing \wedge (x,
  y') \in \varnothing \Rightarrow y = y'$ is satisfied vacuously as $(x, y)
  \in \varnothing \wedge (x, y') \in \varnothing$ is never true. 
\end{proof}

\begin{example}[Constant Function]
  \label{function constant function}Let $A$, $B$ classes and $c \in B$ then
  $C_c : A \rightarrow B$ is defined by $C_c (x) = c$ or formally $C_c = \{
  z|z = (x, c) |x \in A \} = A \times \{ c \}$
\end{example}

\begin{proof}
  If $(x, y) \in C_c$ then $x \in A$ and $y = c \in B$ so that $C_c \subseteq
  A \times B$. If $(x, y) \in C_c \wedge (x, y') \in C_c$ then $y = c \wedge
  y' = c$ so that $y = y'$. So
  \[ C_c : A \rightarrow B \text{ is a partial function} \]
  Finally if $x \in A$ then $(x, c) \in C_c$ so that $A \subseteq \tmop{dom}
  (C_c)$ which by [proposition: \ref{function condition (1)}] proves that
  \[ C_c : A \rightarrow B \tmop{is} a \tmop{function} \]
\end{proof}

\begin{example}[Characteristics Function]
  \label{function characteristics function}Let $A$ be a class and $B \subseteq
  A$ then $\mathcal{X}_{A, B} : A \rightarrow \{ 0, 1 \}$ is defined by
  $\mathcal{X}_{A, B} = (B \times \{ 1 \}) \bigcup ((A\backslash B) \times \{
  0 \})$ [so that $\mathcal{X}_{A, B} (x) = \left\{\begin{array}{l}
    1 \text{if } x \in B\\
    0 \text{ if } x \in A\backslash B
  \end{array}\right.$
\end{example}

\begin{proof}
  If $(x, y) \in \mathcal{X}_{A, B}$ then either $(x, y) \in (B \times \{ 1
  \}) \Rightarrow x \in B \Rightarrowlim_{B \subseteq A} x \in A$ and $y = 1
  \in \{ 0, 1 \}$ or $(x, y) \in ((A\backslash B), \{ 0 \}) \Rightarrow x \in
  A\backslash B \Rightarrow x \in A$ and $y = 1 \in \{ 0, 1 \}$ so that
  \[ \mathcal{X}_{A, B} \subseteq A \times \{ 0, 1 \} \]
  Also if $ (x, y), (x, y') \in \mathcal{X}_{A, B}$ then for $(x, y)$ we have
  either:
  \begin{description}
    \item[$(x, y) \in B \times \{ 1 \}$] then $x \in B$ so that $(x, y') \in B
    \times \{ 1 \}$ hence $y = 1 = y'$
    
    \item[$(x, y) \in (A\backslash B) \times \{ 0 \}$] then $x \in A\backslash
    B$ so that $(x, y') \in (A\backslash B) \times \{ 0 \}$ hence $y = 0 = y'$
  \end{description}
  or in all cases $y = y'$ and $x \in B \bigcup (A\backslash B) = A$. Hence
  $\mathcal{X}_{A, B} : A \rightarrow \{ 0, 1 \}$ is a function.
\end{proof}

\begin{example}[Identity Function]
  \label{function identity function}{\index{identity
  function}}{\index{$\tmop{Id}_A$}}Let $A$ be a class then $\tmop{Id}_A : A
  \rightarrow B$ is defined by
  \[ I_A = \{ z | z = (x, x) \wedge x \in A \} \]
\end{example}

\begin{proof}
  Trivially we have $\tmop{Id}_A \subseteq A \times A$. If $(x, y), (x, y')
  \in \tmop{Id}_A$ then $(x, y) = (x, x) = (x, y')$ proving that $y = x = y'$.
  Hence $I_d : A \rightarrow A$ is a partial function. Further if $x \in A$
  then $(x, x) \in \tmop{Id}_A$ so that $x \in \tmop{dom} (\tmop{Id}_A)$ or
  $\tmop{dom} (\tmop{Id}_A) \subseteq A$ which by [proposition: \ref{function
  condition (1)}] proves that
  \[ \tmop{Id}_A : A \rightarrow A \text{ is a function} \]
\end{proof}

\begin{proposition}
  \label{function composition of Id function}Let $f : A \rightarrow B$ be a
  partial function then $f = f \circ \tmop{Id}_A$ and $f = \tmop{Id}_B \circ
  f$
\end{proposition}

\begin{proof}
  
  \begin{enumerate}
    \item If $(x, y) \in f$ then as $f \subseteq A \times B$ we have $x \in A
    \wedge x \in B$, by the definition of $\tmop{Id}_A$ we have $(x, x) \in
    \tmop{Id}_A$, as $(x, y) \in f$ we have $(x, y) \in \tmop{Id}_A \circ f$.
    If $(x, y) \in f \circ \tmop{Id}_A$ then $\exists x'$ such that $(x, x')
    \in \tmop{Id}_A \wedge (x', y) \in f$. By definition of $\tmop{Id}_A$ we
    have that $\exists z \in A$ such that $(x, x') = (z, z)$ hence $x = x'$ so
    that $(x, y) \in f$. Using the Axiom of Extent [axiom: \ref{axiom of
    extent}] we have then that
    \[ f = f \circ \tmop{Id}_A \]
    \item If $(x, y) \in f$ then as $f \subseteq A \times B$ we have $x \in A
    \wedge x \in B$, by the definition of $\tmop{Id}_B$ we have $(y, y) \in
    \tmop{Id}_B$, so $(x, y) \in \tmop{Id}_B \circ f$. If $(x, y) \in
    \tmop{Id}_B \circ f$ then $\exists y'$ such that $(x, y') \in f \wedge (y,
    y')$, from the definition of $\tmop{Id}_B$ we have that $y = y'$ so that
    $(x, y) \in f$. Using the Axiom of Extent [axiom: \ref{axiom of extent}]
    we have then that
    \[ f = \tmop{Id}_B \circ f \]
  \end{enumerate}
\end{proof}

As a function $f : A \rightarrow B$ is a partial function with $\tmop{dom} (f)
= A$ we can refine [theorem: \ref{partial functions image/preimage
properties}].

\begin{theorem}
  \label{function image preimage}If $f : A \rightarrow B$ is a function $C
  \subseteq B$ and $D \subseteq B$ then we have
  \begin{enumerate}
    \item $f (C) \subseteq B$
    
    \item $f^{- 1} (D) \subseteq A$
    
    \item $f (A) = \tmop{range} (f)$
    
    \item $f^{- 1} (B) = A$
  \end{enumerate}
\end{theorem}

\begin{proof}
  This follows from \ \ref{partial functions image/preimage properties} taking
  in account that $A = \tmop{dom} (f)$
\end{proof}

\

\subsection{Injectivity, Surjectivity and bijectivity}

First we define injectivity and surjectivity of partial functions.

\begin{definition}
  \label{partial function injectivity and surjectivity}Let $f : A \rightarrow
  B$ be a partial function then we say that:
  \begin{enumerate}
    \item $f$ is \tmtextbf{injective} iff if $(x, y) \in f \wedge (x', y) \in
    f$ implies $x = x'$
    
    \item $f$ is \tmtextbf{surjective} iff $\tmop{range} (f) = B$
  \end{enumerate}
\end{definition}

\begin{proposition}
  \label{function surjection condition}A partial function $f : A \rightarrow
  B$ is surjective if $B \subseteq \tmop{range} (f)$
\end{proposition}

\begin{proof}
  By [theorem: \ref{partial function dom(f) range(f) as subclasses}]
  $\tmop{range} (f) \subseteq B$, so if $B \subseteq \tmop{range} (f)$ it
  follows from [theorem: \ref{class properties (1)}] that $B = \tmop{range}
  (f)$, proving surjectivity.
\end{proof}

Using the notation $y = f (x)$ is the same as $(x, y) \in f$ we have

\begin{theorem}
  \label{function injectivity, surjectivity}Let $f : A \rightarrow B$ be a
  function then
  \begin{enumerate}
    \item $f$ is injective if and only if $\forall x, x \in A$ with $f (x) = f
    (x')$ we have $x = x'$
    
    \item If $B \subseteq C$ and $f : A \rightarrow B$ is injective then $f :
    A \rightarrow C$ is injective
    
    \item $f$ is surjective if and only if \ $\forall y \in B$ there exists a
    $x \in A$ such that $y = f (x)$
  \end{enumerate}
\end{theorem}

\begin{proof}
  \quad
  \begin{enumerate}
    \item 
    \begin{description}
      \item[$\Rightarrow$] Let $x, x' \in A$ then if $y = f (x) = f (x')$ we
      have $(x, y) \in f$ and $(x', y)$ so that $x = x'$
      
      \item[$\Leftarrow$] If $(x, y) \in f$ and $(x', y) \in f$ then $y = f
      (x) \wedge y = f (x')$ so that $f (x) = f (x')$ so that $x = x'$
    \end{description}
    \item This is trivial because injectivity is a property of the graph of a
    function.
    
    \item 
    \begin{description}
      \item[$\Rightarrow$] As $B = \tmop{range} (f)$ we have $y \in B$ then
      $\exists x$ such that $(x, y) \in f \Rightarrow y = f (x)$ which as $f
      \subseteq A \times B$ proves that $x \in A$. So $\forall y \in B$
      $\exists x \in A$ such that $y = f (x)$
      
      \item[$\Leftarrow$] Let $y \in B$ then $\exists x \in A$ such that $y =
      f (x)$ or $(x, y) \in f$ proving that $B \subseteq \tmop{range} (f)$,
      using [proposition: \ref{function surjection condition}] we have that
      $f$ is surjective
    \end{description}
  \end{enumerate}
\end{proof}

\begin{example}
  \label{function inclusion function}{\index{$i_B$}}Let $A, B$ be classes, $B
  \subseteq A$ then $i_B : B \rightarrow A$ defined by $i_B = \{ (x, x) |x \in
  B \}$ is a injective function. This function is called the
  \tmtextbf{inclusion} function. 
\end{example}

\begin{proof}
  First if $(x, y) \in i_B$ then $\exists b \in B$ such that $(x, y) = (b, b)$
  so that $x = b \in B \wedge y = b \in B \subseteq A$ proving that
  \[ i_B \subseteq B \times A \]
  Further if $(x, y), (x, y') \in i_B$ then $\exists b, b' \in B$ such that
  $(x, y) = (b, b) \wedge (x, y') = (b', b')$, so that $x = b \wedge y = b
  \wedge x = b' \wedge y' = b'$, hence $y = y'$. So
  \[ i_B : B \rightarrow A \text{ is a partial function} \]
  If $x \in B$ then $(x, x) \in i_B$ proving that $A \subseteq \tmop{dom}
  (i_b)$ so using [proposition: \ref{function condition (1)}] it follows that
  \[ i_B : B \rightarrow A \text{ is a function} \]
  Finally if $(x, y), (x', y) \in i_B$ then there exists $b, b' \in B$ such
  that $(x, y) = (b, b) \wedge (x', y) = (b', b')$, so that $x = b \wedge y =
  b \wedge x' = b' \wedge y = b'$, hence $x = x'$, proving injectivity.
  
  \ 
\end{proof}

The following axiom ensures that the image of a set by a surjection is a set.

\begin{axiom}[Axiom of Replacement]
  \label{Axiom of Replacement}If $A$ is a set and $f : A \rightarrow B$ a
  surjection then $B$ is a set.
\end{axiom}

\begin{proposition}
  \label{function preimage of image}If $f : A \rightarrow B$ is a a function
  and $C \subseteq A$, $D \subseteq B$ then
  \begin{enumerate}
    \item $C \subseteq f^{- 1} (f (C))$
    
    \item If $f$ is injective then $C = f^{- 1} (f (C))$
    
    \item If $f$ is surjective then $D = f (f^{- 1} (D))$
  \end{enumerate}
\end{proposition}

\begin{proof}
  
  \begin{enumerate}
    \item This is stated in [theorem: \ref{function preimage of image (1)}]
    
    \item If $x \in f^{- 1} (f (C))$ then $\exists y \in f (C)$ such that $(y,
    x) \in f^{- 1}$, hence $(x, y) \in f$. As $y \in f (C)$ there exists a $x'
    \in C$ such that $(x', y) \in f$. Given that $f$ is injective it follows
    from $(x, y), (x', y) \in f$ that $x = x'$, so as $x' \in C$ it follow
    that $x \in C$. Hence $f^{- 1} (f (C)) \subseteq C$ which combined with
    (1) proves
    \[ C = f^{- 1} (f (C)) \]
    \item If $y \in f (f^{- 1} (D))$ then $\exists x \in f^{- 1} (D)$ such
    that $(x, y) \in f$, hence $\exists z \in D$ such that $(z, x) \in f^{- 1}
    \Rightarrow (x, z) \in f$, As $f$ is a function we have $y = z$ so that $y
    \in D$. Hence
    \begin{equation}
      \label{eq 2.15.013} f (f^{- 1} (D)) \subseteq D
    \end{equation}
    If $y \in D$ then as $f$ is a surjection there exist a $x \in A$ such that
    $(x, y) \in f$, hence $x \in f^{- 1} (D)$ proving that $y \in f (f^{- 1}
    (D))$. So $D \subseteq f (f^{- 1} (D))$ which together with [eq: \ref{eq
    2.15.013}] proves
    \[ D = f (f^{- 1} (D)) \]
  \end{enumerate}
\end{proof}

The importance of injectivity is that it allows us to define the inverse of a
partial function. First we define the inverse graph of the graph of a partial
function.

\begin{definition}
  \label{partial function inverse graph}Let $f : A \rightarrow B$ be a partial
  function then the \tmtextbf{inverse of the graph f} noted as $f^{- 1}$ is
  defined by
  \[ f^{- 1} = \left\{ z \of z = (z, y) \text{ where } (y, x) \in f \right\}
  \]
\end{definition}

\begin{theorem}
  \label{partial function inverse if injective}Let $f : A \rightarrow B$ be a
  \tmtextbf{injective }partial function then $f^{- 1} : B \rightarrow A$ is a
  partial function
\end{theorem}

\begin{proof}
  If $(x, y) \in f^{- 1}$ then $ (y, x) \in f$ which, as $f \subseteq A \times
  B$, gives $(y, x) \in A \times B$, so $x \in B \wedge y \in A$, proving $(x,
  y) \in B \times Y$. Hence
  \[ f^{- 1} \subseteq B \times A \]
  Further if $(x, y) \in f^{- 1}$ and $(x, y') \in f^{- 1}$ then $(y, x) \in f
  \wedge (y, x') \in f$ which, as $f$ is injectivity proves that $y = y'$. So
  all the conditions are satisfied to make $f^{- 1} : B \rightarrow A$ a
  partial function.
\end{proof}

\begin{note}
  The requirement that $f$ is injective is needed to make $f^{- 1}$ is a
  partial function. For example assume that $A = \{ 1, 2, 3 \}$, $B = \{ 10,
  20 \}$ and $f = \{ (1, 10), (2, 10), (3, 20) \}$ then $f^{- 1} = \{ (10, 1),
  (10, 2), (20, 3) \}$ which is not the graph of a partial function. 
\end{note}

If $f$ is a injective function then the above theorem ensures that $f^{- 1}$
is a partial function however \ $f^{- 1}$ can be a graph of a function if we
restrict the source of the inverse function.

\begin{theorem}
  \label{function injective inverse is a function}If $f : A \rightarrow B$ is
  a injective function then $f^{- 1} : f (A) \rightarrow A$ is a function
\end{theorem}

\begin{proof}
  First if $(x, y) \in f^{- 1}$ then $(y, x) \in f \subseteq A \times B$ so
  that $y \in A \wedge x \in B$, as $(y, x) \in f$ we have that $x \in f (A)$,
  hence $(x, y) \in f (A) \times A$. So $f^{- 1} \subseteq f (A) \times B$.
  Further if $(x, y), (x, y') \in f^{- 1}$ then $(y, x), (y', x) \in f$ which
  as $f$ is injective proves $y = y'$. Hence
  \[ f^{- 1} : f (A) \rightarrow A \text{ is a partial function} \]
  Further if $x \in f (A)$ then there exists a $y \in A$ such that $(y, x) \in
  f$, hence $(x, y) \in f^{- 1}$ so that $x \in \tmop{dom} (f^{- 1})$, proving
  that $f (A) \subseteq \tmop{dom} (f^{- 1})$. Hence
  \[ f^{- 1} : f (A) \rightarrow A \text{ is a function} \]
\end{proof}

\begin{corollary}
  \label{function injection condition}If $f : A \rightarrow B$ is a function,
  $A \neq \varnothing$ then $f : A \rightarrow B$ is injective if and only if
  there exist a function $g : B \rightarrow A$ such that $g \circ f =
  \tmop{Id}_A$
\end{corollary}

\begin{proof}
  
  \begin{description}
    \item[$\Rightarrow$] Using the above [theorem: \ref{function injective
    inverse is a function}] we have that $f^{- 1} : f (A) \rightarrow A$ is a
    function. As $A \neq 0$ there exist a $a \in A$ so we can consider the
    constant function $C_a : B\backslash f (A) \rightarrow A$ [see example:
    \ref{function constant function}]. As $f (A) \bigcap (B\backslash f (A)) =
    \varnothing$ and $B = f (A) \bigcup (B \setminus f (A))$ we have by
    [theorem: \ref{function combining functions (1)}] that
    \[ g = C_a \bigcup f^{- 1} \of B \rightarrow A \]
    is a function. If $(x, y) \in g \circ f$ $\tmop{then}${\exists}z such that
    $(x, z) \in f \wedge (z, y) \in g$. As $(x, z) \in f$ we have that $(z, x)
    \in f^{- 1} \subseteq C_a \bigcup f^{- 1} = g$, as also $(z, y) \in g$ and
    $g$ is function, we have that $y = x$ so that $(x, y) = (x, x) \in
    \tmop{Id}_A$ hence
    \[ g \circ f \subseteq \tmop{Id}_A \]
    Further if $(x, y) \in \tmop{Id}_A$ then $x = y$, as $x \in A = \tmop{dom}
    (f)$ there exist a $z \in B$ such that $(x, z) \in f \Rightarrow (z, x)
    \in f^{- 1} \subseteq C_a \bigcup f^{- 1} = g$ proving that $(x, y) = (x,
    x) \in g \circ f$. Hence
    \[ \tmop{Id}_A \subseteq g \circ f \]
    proving that
    \[ g \circ f = \tmop{Id}_A \]
    \item[$\Leftarrow$] Assume that there exists a function $g : B \rightarrow
    A$ such that $g \circ f = \tmop{Id}_A$ then
    \begin{eqnarray*}
      (x, y), (x', y) \in f \subseteq A \times B & \Rightarrowlim_{y \in B,
      \tmop{dom} (g) = B} & \exists z \vdash (y, z) \in g\\
      & \Rightarrow & (x, z), (x', z) \in g \circ f = \tmop{Id}_A\\
      & \Rightarrow & x = z = x'\\
      & \Rightarrow & x = x'
    \end{eqnarray*}
  \end{description}
\end{proof}

\begin{definition}
  \label{bijection}{\index{bijection}}A function $f : A \rightarrow B$ is a
  \tmtextbf{bijection} iff the function is \tmtextbf{injective} and
  \tmtextbf{surjective}.
\end{definition}

\begin{definition}
  \label{bijective classes}{\index{bijective}}Two classes $A$ and $B$ are
  bijective iff there exists a bijection between $A$ and $B$
\end{definition}

\begin{example}
  \label{function empty function bijection}The function $\varnothing :
  \varnothing \rightarrow \varnothing$ is a bijection.
\end{example}

\begin{proof}
  By [example: \ref{function empty function}] $\varnothing : \varnothing
  \rightarrow \varnothing$ is a function. To prove that is a bijection we
  have:
  \begin{description}
    \item[injectivity] $\forall (x, y), (x', y) \in \varnothing$ we have $x =
    x'$ is satisfied vacuously.
    
    \item[surjectivity] $\forall y \in \varnothing$ there exist a $x \in
    \varnothing$ such that $(x, y) \in \varnothing$ is satisfied vacuously.
  \end{description}
\end{proof}

\begin{example}
  \label{function identity map is a bijection}Let $A$ be a class then
  $\tmop{Id}_A : A \rightarrow A$ [example: \ref{function identity function}]
  is a bijection
\end{example}

\begin{proof}
  Let $(x, y) \in \tmop{Id}_A \wedge (x', y) \in \tmop{IdA}$ then $\exists z,
  z' \in A$ such that $(x, y) = (z, z) \wedge (x', y) = (z', z')$. So using
  [theorem: \ref{pair equality of pairs}] $x = z \wedge y = z \wedge x = z'
  \wedge y = z'$. Using [theorem: \ref{class properties (1)}] repeatedly gives
  then $x = x'$ proving that
  \[ \tmop{Id}_A \text{ is injective} \]
  If $y \in A$ then by definition $(y, y) \in \tmop{Id}_A$ so that
  $\tmop{range} (\tmop{Id}_A) \subseteq A$. Using [theorem: \ref{function
  surjection condition}] it follows that
  \[ \tmop{Id}_A \text{ is surjective} \]
\end{proof}

\begin{example}
  \label{function trivial bijection}Let $I = \{ 0 \}$ $B$ a class and take $f
  : I \rightarrow \{ B \}$ defined by $f = \{ (0, B) \}$ is a bijection
\end{example}

\begin{proof}
  As $0 \in \{ 0 \}$ and $B \in \{ B \}$ it follows that $(0, B) \in \{ 0 \}
  \times \{ B \}$, hence $f = \{ (0, B) \} \subseteq \{ 0 \} \times \{ B \}$.
  If $(x, y), (x, y') \in f = \{ 0 \} \times \{ B \}$ then $y = B = y'$,
  further $\tmop{dom} (f) = \{ 0 \} = I$. So we conclude that $f : \{ 0 \}
  \rightarrow \{ B \}$ is indeed a function. Further if $y \in \{ B \}$ then
  $y = B$ and as $(0, B) \in f$ it follows that $y \in \tmop{range} (f)$ or
  $\{ B \} \subseteq \tmop{range} (f)$, which by [theorem: \ref{function
  surjection condition}] proves that $f$ is surjective. Finally if $(x, y),
  (x', y) \in f = \{ (0, B) \}$ then $x = 0 = x'$ proving that $f : \{ 0 \}
  \rightarrow \{ B \}$ is a bijection.
\end{proof}

\begin{proposition}
  \label{function injectivity to bijection}If $f : A \rightarrow B$ is a
  injective function then $f : A \rightarrow f (A)$ is a bijection
\end{proposition}

\begin{proof}
  As injectivity is a property of the graph of a function,\quadthe function $f
  : A \rightarrow B$ is still injective. Further $\tmop{range} (f)
  \equallim_{\text{[theorem: \ref{partial functions image/preimage
  properties}]}} f (A)$ which proves surjectivity.
\end{proof}

\begin{theorem}
  \label{function bijection has a inverse}$\tmop{If} f : A \rightarrow B$ is a
  bijection then $f^{- 1} : B \rightarrow A$ is a function 
\end{theorem}

\begin{proof}
  As $f : A \rightarrow B$ is injective and surjective we have that $f (A) =
  B$ and by [theorem: \ref{function injective inverse is a function}] that
  $f^{- 1} \of f (A) \rightarrow B$ is a function. Hence $f^{- 1} : B
  \rightarrow A$ is a function.
\end{proof}

\begin{theorem}
  \label{function bijection f,f-1}If $f : A \rightarrow B$ is bijective then
  \begin{enumerate}
    \item $f \circ f^{- 1} = \tmop{Id}_B$
    
    \item $f^{- 1} \circ f = \tmop{Id}_A$
  \end{enumerate}
\end{theorem}

\begin{proof}
  First $f^{- 1} : B \rightarrow A$ is a function by [theorem: \ref{function
  bijection has a inverse}].
  \begin{enumerate}
    \item Let $(x, y) \in f \circ f^{- 1}$ then $\exists z$ such that $(x, z)
    \in f^{- 1} \Rightarrow (z, x)$ and $(z, y) \in f$. As $f^{- 1}$ is the
    graph of a function we have that $x = y$. Further from $(x, z) \in f^{- 1}
    \subseteq B \times A$ it follow that $x \in B$. Hence $(x, y) = (x, x) \in
    \tmop{Id}_B$, proving that
    \begin{equation}
      \label{eq 2.12.001} f \circ f^{- 1} \subseteq \tmop{Id}_B
    \end{equation}
    If $(x, y) \in \tmop{Id}_B$ then $\exists z \in B$ such that $(x, y) = (z,
    z)$ so that $x = y \in B$, As $B = \tmop{dom} (f^{- 1})$ there exists a
    $u$ such that $(y, u) \in f^{- 1} \Rightarrow (u, y) \in f$ so that $(y,
    y) \in f \circ f^{- 1} \Rightarrowlim_{x = y} (x, y) \in f \circ f^{- 1}$.
    So $\tmop{Id}_B \subseteq f \circ f^{- 1}$. Combining this with [eq:
    \ref{eq 2.12.001}] proves that
    \[ f \circ f^{- 1} = \tmop{Id}_B \]
    \item Let $(x, y) \in f^{- 1} \circ f$ then $\exists z$ such that $(x, z)
    \in f \Rightarrow (z, x) \in f^{- 1}$ and $(z, y) \in f^{- 1}$. As $f^{-
    1}$ is the graph of a function we have that $x = y$. Further from $(x, z)
    \in f \subseteq A \times B$ it follows that $x \in A$. Hence $(x, y) = (x,
    x) \in \tmop{Id}_A$, proving that
    \begin{equation}
      \label{eq 2.13.001} f^{- 1} \circ f \subseteq I_A
    \end{equation}
    If $(x, y) \in \tmop{Id}_A$ then $\exists z \in A$ such that $(x, y) = (z,
    z)$ so that $x = y \in A$, As $A = \tmop{dom} (f)$ there exists a $u$ such
    that $(x, u) \in f  \Rightarrow (u, x) \in f^{- 1}$ so that $(x, x) \in
    f^{- 1} \circ f \Rightarrowlim_{x = y} (x, y) \in f^{- 1} \circ f$. So
    $\tmop{Id}_B \subseteq f^{- 1} \circ f$. Combining this with [eq: \ref{eq
    2.13.001}] proves that
    \[ f^{- 1} \circ f = \tmop{Id}_A \]
  \end{enumerate}
\end{proof}

\begin{corollary}
  \label{function inverse function and f(x)}If $f : A \rightarrow B$ is
  bijection then
  \begin{enumerate}
    \item $\forall x \in A$ we have $(f^{- 1}) (f (x)) = x$
    
    \item $\forall y \in B$ we have $f ((f^{- 1}) (y)) = y$
  \end{enumerate}
\end{corollary}

\begin{proof}
  
  \begin{enumerate}
    \item If $x \in A$ then $(f^{- 1}) (f (x)) = ((f^{- 1}) \circ f) (x)
    \equallim_{\text{[theorem: }} \tmop{Id}_A (x) = x$
    
    \item If $y \in B$ then $f ((f^{- 1}) (y)) \equallim_{\text{[theorem: }}
    \tmop{Id}_B (y) = y$
  \end{enumerate}
\end{proof}

\begin{corollary}
  \label{function bijection condition (2)}Let $f : A \rightarrow B$ a function
  then the following are equivalent:
  \begin{enumerate}
    \item $f : A \rightarrow B$ is a bijection
    
    \item There exists a function $g : B \rightarrow A$ such that $f \circ g =
    \tmop{id}_B$ and $g \circ f = \tmop{Id}_A$
  \end{enumerate}
\end{corollary}

\begin{proof}
  \quad
  \begin{description}
    \item[$1 \Rightarrow 2$] This follows from [theorem: \ref{function
    bijection f,f-1}] by taking $g = f^{- 1}$
    
    \item[$2 \Rightarrow 1$] Let $(x, y), (x', y) \in f \subseteq A \times B$,
    as $y = \tmop{dom} (g)$ there exists a $z$ such that $(y, z) \in g$, hence
    $(x, z), (x', z) \in g \circ f = \tmop{Id}_A$ so that $x = z = x'$ proving
    that
    \[ f : A \rightarrow B \text{ is injective} \]
    Further if $y \in B$ then $(y, y) \in \tmop{Id}_B = f \circ g$ so there
    exists a $z \in A$ such that $(y, z) \in g$ and $(z, y) \in f$. Proving
    that $B \subseteq \tmop{range} (f)$ so by [proposition: \ref{function
    surjection condition}]
    \[ f : A \rightarrow B \text{ is a surjection} \]
  \end{description}
\end{proof}

The inverse of a bijection is again a bijection

\begin{corollary}
  \label{function bijection and inverse}If $f : A \rightarrow B$ is a
  bijection then $f^{- 1} : B \rightarrow A$ is a bijection 
\end{corollary}

\begin{proof}
  If $f : A \rightarrow B$ is a bijection then by [theorem: \ref{function
  bijection f,f-1}] $f \circ f^{- 1} = \tmop{Id}_B$ and $f^{- 1} \circ f =
  \tmop{Id}_A$ which by [theorem: \ref{function bijection condition (2)}]
  proves that $f^{- 1} : B \rightarrow A$ is a bijection.
\end{proof}

\begin{proposition}
  \label{function inverse of a bijection is unique}If $f : A \rightarrow B$ is
  a bijection then we have:
  \begin{enumerate}
    \item If $g : B \rightarrow A$ is such that $f \circ g = \tmop{Id}_B$ and
    $g \circ f = \tmop{Id}_A$ then $g = f^{- 1}$
    
    \item $(f^{- 1})^{- 1} = f$
  \end{enumerate}
\end{proposition}

\begin{proof}
  
  \begin{enumerate}
    \item We have
    \begin{eqnarray*}
      f \circ g = \tmop{Id}_B & \Rightarrow & f^{- 1} \circ (f \circ g) = f^{-
      1} \circ \tmop{Id}_B\\
      & \Rightarrowlim_{\text{[proposition: \ref{function composition of Id
      function}}} & f^{- 1} \circ (f \circ g) = f^{- 1}\\
      & \Rightarrowlim_{\text{[theorem: \ref{partial function
      associativity}]}} & (f^{- 1} \circ f) \circ g = f^{- 1}\\
      & \Rightarrowlim_{\text{[function: \ref{function bijection f,f-1}]}} &
      \tmop{Id}_B \circ g = f^{- 1}\\
      & \Rightarrowlim_{\text{[proposition: \ref{function composition of Id
      function}}} & g = f^{- 1}
    \end{eqnarray*}
    \item We have
    \begin{eqnarray*}
      (x, y) \in (f^{- 1})^{- 1} & \Leftrightarrow & (y, x) \in f^{- 1}\\
      & \Leftrightarrow & (x, y) \in f
    \end{eqnarray*}
    which by the Axiom of Extent [axiom: \ref{axiom of extent}] proves
    \[ (f^{- 1})^{- 1} = f \]
  \end{enumerate}
\end{proof}

Composition preserves injectivity, surjectivity and $\tmop{bijectivity}$.

\begin{theorem}
  \label{function composition injectivity, surjectivity and bijectivity}We
  have
  \begin{enumerate}
    \item $\tmop{If}$ $f : A \rightarrow B$ and $g : C \rightarrow D$ are
    injective functions with $f (A) \subseteq C$ then $g \circ f : A
    \rightarrow D$ is a injective function.
    
    \item If $f : A \rightarrow B$ and $g : C \rightarrow D$ are injective
    functions with $f (A) \subseteq C$ then $g \circ f : A \rightarrow g (f
    (A))$ is a bijective function.
    
    \item If $f : A \rightarrow B$ is a function and $g : C \rightarrow D$ a
    surjective function so that $f (A) = C$ then $g \circ f : A \rightarrow D$
    is a surjective function.
    
    \item If $f : A \rightarrow B$ is a injective function and $g : C
    \rightarrow D$ a bijective function so that $f (A) = C$ then $g \circ f :
    A \rightarrow D$ is a bijective function.
    
    \item If $f : A \rightarrow B$ is a injective function and $g : C
    \rightarrow D$ a bijective function \ so that $f (A) = C$ then $(g \circ
    f)^{- 1} = f^{- 1} \circ g^{- 1}$.
  \end{enumerate}
\end{theorem}

\begin{proof}
  
  \begin{enumerate}
    \item Let $(x, z), (x', z) \in g \circ f$ then $\exists u, v$ such that
    \[ (x, u) \in f \wedge (x', v) \in f \wedge (u, y) \in g \wedge (v, y) \in
       g \]
    As $g$ is injective we have $u = v$, but that means from the above that
    $(x, u) \in f \wedge (x', u) \in f$, which as $f$ is injective proves
    \[ x = x' \]
    \item Using (1) we have that $g \circ f : A \rightarrow D$ is injective so
    that by [theorem: \ref{function injectivity to bijection}] $g \circ f : A
    \rightarrow (g \circ f) (A)$ is a bijection. Further by [theorem:
    \ref{partial function image preimage of compositions}] $(g \circ f) (A) =
    g (f (A))$ so that $g \circ f : A \rightarrow g (f (A))$ is a bijection.
    
    \item Let $z \in D$ then as $g$ is surjective there $\exists y \in C$ such
    that $(y, z) \in g$. As $f (A) = C$ there exists a $x \in A$ such that
    $(x, y) \in f$. But then $(x, z) \in g \circ f$ proving that $g \circ f$
    is surjective.
    
    \item Using (1) and (2) proves that $g \circ f : A \rightarrow D$ is
    injective and surjective and thus by definition bijective.
    
    \item By (3) $g \circ f$ is a bijection, so by [theorem: \ref{function
    bijection f,f-1}] we have that
    \begin{eqnarray*}
      (g \circ f)^{- 1} \circ (g \circ f) = \tmop{Id}_A &
      \Rightarrowlim_{\text{[associativity: \ref{partial function
      associativity}] }} & ((g \circ f)^{- 1} \circ g) \circ f = \tmop{Id}_A\\
      & \Rightarrow & (((g \circ f)^{- 1} \circ g) \circ f) \circ f^{- 1} =
      \tmop{Id}_A \circ f^{- 1}\\
      & \Rightarrowlim_{\text{[proposition: \ref{function composition of Id
      function}]}} & (((g \circ f)^{- 1} \circ g) \circ f) \circ f^{- 1} =
      f^{- 1}\\
      & \Rightarrowlim_{\text{[associativity: \ref{partial function
      associativity}] }} & ((g \circ f)^{- 1} \circ g) \circ (f \circ f^{- 1})
      = f^{- 1}\\
      & \Rightarrowlim_{\text{[theorem: \ref{function bijection f,f-1}]}} &
      ((g \circ f)^{- 1} \circ g) \circ \tmop{Id}_B = f^{- 1}\\
      & \Rightarrowlim_{\text{[proposition: \ref{function composition of Id
      function}]}} & (g \circ f)^{- 1} \circ g = f^{- 1}\\
      & \Rightarrow & ((g \circ f)^{- 1} \circ g) \circ g^{- 1} = f^{- 1}
      \circ g^{- 1}\\
      & \Rightarrowlim_{\text{[associativity: \ref{partial function
      associativity}] }} & (g \circ f^{- 1}) \circ (g \circ g^{- 1}) = f^{- 1}
      \circ g^{- 1}\\
      & \Rightarrowlim_{\text{[theorem: \ref{function bijection f,f-1}]}} &
      (g \circ f)^{- 1} \circ \tmop{Id}_A = f^{- 1} \circ g^{- 1}\\
      & \Rightarrowlim_{\text{[proposition: \ref{function composition of Id
      function}]}} & (g \circ f)^{- 1} = f^{- 1} \circ g^{- 1}
    \end{eqnarray*}
  \end{enumerate}
\end{proof}

In the special case that $B = C$ we have

\begin{corollary}
  \label{function composition injectivity, surjectivity and bijectivity (1)}We
  have
  \begin{enumerate}
    \item $\tmop{If}$ $f : A \rightarrow B$ and $g : B \rightarrow C$ are
    injective functions then $g \circ f : A \rightarrow C$ is a injective
    function.
    
    \item If $f : A \rightarrow B$ and $g : B \rightarrow C$ are surjective
    functions then $g \circ f : A \rightarrow C$ is a surjective function.
    
    \item If $f : A \rightarrow B$ and $g : B \rightarrow C$ are bijective
    function then $g \circ f : A \rightarrow C$ is a bijective function.
    
    \item If $f : A \rightarrow B$ and $g : B \rightarrow C$ are bijective
    function \ then $(g \circ f)^{- 1} = f^{- 1} \circ g^{- 1}$.
  \end{enumerate}
\end{corollary}

\begin{proof}
  
  \begin{enumerate}
    \item This follows from [theorem: \ref{function composition injectivity,
    surjectivity and bijectivity} (1)] because $f (A) \subseteq B$.
    
    \item This follows from [theorem: \ref{function composition injectivity,
    surjectivity and bijectivity} (2)] because if $f$ is surjective we have $f
    (A) = B$.
    
    \item This follows from (1) and (2)
    
    \item This follows from [theorem: \ref{function composition injectivity,
    surjectivity and bijectivity} (4)] because if $f$ is bijective, hence
    surjective, we have $f (A) = B$
  \end{enumerate}
\end{proof}

The following is a example of a bijection between a class and the class of
functions in this set.

\begin{theorem}
  \label{function and power}Let $A$ be a class then there exists a bijection
  between $A$ and $A^{\{ 0 \}}$
\end{theorem}

\begin{proof}
  Given $x \in A$ define the function $f_x : \{ 0 \} \rightarrow \{ x \}$
  where $f_x = \{ (0, x) \}$ [see [example: \ref{function trivial bijection}]
  to prove that this is a function (even a bijection)]. So $f_x \in \{ x
  \}^{\{ 0 \}},$ which as $\{ x \} \subseteq A$ proves by [theorem:
  \ref{function B^A and inclusion}] that $f_x \in A^{\{ 0 \}}$. Define now $f
  = \left\{ z|z = (x, f_x) \text{ where } x \in A \right\}$. If $(x, y) \in f$
  we have $x \in A$ and thus $y = f_x \in A^{\{ 0 \}}$ hence $(x, y) \in A
  \times A^{\{ 0 \}}$. Also if $(x, y), (x, y') \in A$ then $y = f_x$ and $y'
  = f_x$ so that $y = y'$. Further for every $x \in A$ we have by the
  definition of $f$ that $(x, f_x) \in f$. So we conclude that
  \[ f : A \rightarrow A^{\{ 0 \}} \text{ is a function} \]
  Assume now that $(x, y), (x', y) \in f$ then $f_x = y = f_{x'}$, so that $\{
  (0, x) \} = \{ (0, x') \}$, hence $(0, x) = (0, x')$, from which it follows
  that $x = x'$. this proves that
  \[ f : A \rightarrow A^{\{ 0 \}} \text{ is a injective function} \]
  If $y \in A^{\{ 0 \}}$ then $y : \{ 0 \} \rightarrow A$ is a function, hence
  $0 \in \{ 0 \} = \tmop{dom} (y)$, so there exists a $z$ such that $(0, z)
  \in y \subseteq \{ 0 \} \times A$ proving that $z \in A$. Hence
  \begin{equation}
    \label{eq 2.16.005} \{ (0, z) \} \subseteq y \wedge z \in A \text{}
  \end{equation}
  If $(u, v) \in y \subseteq \{ 0 \} \times A$ then $u = 0$ so that $(0, u)
  \in y$, which, as $(0, z) \in y$ and $y$ is a function, proves that $u = z$
  or $(u, v) = (0, z) \in \{ (0, z) \}$. So $y \subseteq \{ (0, z) \}$ which
  combined with [eq: \ref{eq 2.16.005}] proves that $\{ (0, z) \} = y$. As
  $f_z = \{ (0, z) \} = y$ we have that $(z, y) \in f$ which proves that
  \[ f \text{ is a surjection} \]
\end{proof}

\begin{theorem}
  \label{function P(A)=2^A}If $A$ is a class then there is a bijection between
  $\mathcal{P} (A)$ and $\{ 0, 1 \}^A$ where $0 = \varnothing$ and $1 = \{
  \varnothing \}$ are different elements.
\end{theorem}

\begin{proof}
  Define $\gamma : \mathcal{P} (A) \rightarrow \{ 0, 1 \}^A$ by $\gamma =
  \left\{ z|z = (B, \mathcal{X}_{A, B}) \text{ where } B \in \mathcal{P} (A)
  \right\}$ where $\mathcal{X}_{A, B} = (B \times \{ 1 \}) \bigcup
  ((A\backslash B) \times \{ 0 \})$ is the graph of the Characteristic
  function [example: \ref{function characteristics function}]. If $(B, f) \in
  \gamma$ then \ $B \in \mathcal{P} (A)$ and $f =\mathcal{X}_{A, B}$, as $B
  \in \mathcal{P} (A) \Rightarrow B \subseteq A$ it follow using [example:
  \ref{function characteristics function}] that $\mathcal{X}_{A, B} : A
  \rightarrow \{ 0, 1 \}$ is a function. So $(B, f) \in \{ 0, 1 \}^A$ giving
  \[ \gamma \subseteq \mathcal{P} (A) \times (\{ 0, 1 \}^A) \]
  If $(B, f), (B, g) \in \gamma$ then $f =\mathcal{X}_{A, B}$ and $g
  =\mathcal{X}_{A, B}$ so that $f = g$, also by the definition of $\gamma$ we
  have that $\tmop{dom} (\gamma) =\mathcal{P} (A)$, hence
  \[ \gamma : \mathcal{P} (A) \rightarrow \{ 0, 1 \}^A \text{ is a function}
  \]
  If $(B, f), (B', f) \in \gamma$ then $\mathcal{X}_{A, B} =\mathcal{X}_{A,
  B'}$ so that
  \begin{eqnarray*}
    x \in B & \Leftrightarrow & \mathcal{X}_{A, B} (x) = 1\\
    & \Leftrightarrowlim_{\mathcal{X}_{A, B} =\mathcal{X}_{A, B'}} &
    \mathcal{X}_{A, B'} (x) = 1\\
    & \Leftrightarrow & x \in B'
  \end{eqnarray*}
  proving that $B = B'$. Hence
  \[ \gamma : \mathcal{P} (A) \rightarrow \{ 0, 1 \}^A \text{ is injective}
  \]
  Let $f \in \{ 0, 1 \}^A$, define $B = \{ x \in A| (x, 1) \in f \} \subseteq
  A$, then $B \in \mathcal{P} (A)$.
  
  If $(x, y) \in f$ then we have for x either:
  \begin{description}
    \item[$x \in B$] Then $(x, 1) \in f$ and as $(x, y) \in f$ we have that $y
    = 1$ so that $(x, y) = (x, 1) \in \mathcal{X}_{A, B}$
    
    \item[$x \nin B$] Then $(x, 0) \in f$ and as $(x, y) \in f$ we have that
    $y = 0$ so that $(x, y) = (x, 0) \in \mathcal{X}_{A, B}$ [as $x \in
    A\backslash B$]
  \end{description}
  proving that
  \begin{equation}
    \label{eq 2.17.006} f \subseteq \mathcal{X}_{A, B}
  \end{equation}
  If $(x, y) \in \mathcal{X}_{A, B}$ then we have for $x$ either:
  \begin{description}
    \item[$x \in B$] Then as $(x, 1) \in \mathcal{X}_{A, B}$ we must have that
    $y = 1$, using the definition of $B$ we have also $(x, 1) \in f
    \Rightarrow (x, y) \in f$
    
    \item[$x \nin B$] Then $x \in A\backslash B$ so that $(x, 0) \in
    \mathcal{X}_{A, B}$ hence we must have that $y = 0$. As $(x, 0) \in f$ [if
    $(x, 1) \in f$ then $x \in B$ a contradiction] it follows that $(x, y) =
    (x, 0) \in f$
  \end{description}
  proving that $\mathcal{X}_{A, B} \subseteq f$, which combined with \ref{eq
  2.17.006} gives
  \begin{equation}
    \label{eq 2.18.006} \mathcal{X}_{A, B} = f
  \end{equation}
  So given $f \in \{ 0, 1 \}^A$ we have found a $B \in \mathcal{P} (A)$ such
  that $\mathcal{X}_{A, B} \equallim_{\text{[eq: \ref{eq 2.18.006}]}} f$,
  hence $(B, f) \in \gamma$ proving that
  \[ \gamma : \mathcal{P} (A) \rightarrow \{ 0, 1 \}^A \text{ is a
     surjective} \]
\end{proof}

\subsection{Restriction of a Function/Partial Function}

Sometimes we only want to work with functions whose graphs satisfies certain
conditions. It could be that the graph of a function does not satisfies these,
but that the restriction of this graph to a sub-class satisfies the
conditions. For example, the function $f : \mathbb{R} \rightarrow \mathbb{R}$
defined by $f (x) = \left\{\begin{array}{l}
  1 \text{ if } x < 1\\
  0 \text{ if } 1 \leqslant x
\end{array}\right.$ is not continuous, as it is discontinuous at $1.$ However
restricting this function to $\mathbb{R}\backslash \{ 1 \}$ produces a
continuous function. \ This is the idea of the next definition

\begin{definition}
  \label{function restriction of a graph}{\index{$f_{|C}$}}Let $f : A
  \rightarrow B$ be a partial function and $C \subseteq A$ a sub-class of $A$
  then the restriction of $f$ to $C$ noted by $f_{|C}$ is defined by
  \[ f_{|C} = \{ z|z = (x, y) \in f \wedge x \in C \} = f \bigcap (C \times B)
  \]
  which defines the partial function
  \[ f_{|C} : C \rightarrow B \]
\end{definition}

\begin{proof}
  We must of course proof that $\{ z|z = (x, y) \in f \wedge x \in C \} = f
  \bigcap (C \times B)$ and that $f_{|C} : C \rightarrow B$ is indeed a
  partial function. If $(x, y) \in \{ z|z = (x, y) \in f \wedge x \in C \}$
  then $(x, y) \in f \subseteq A \times B \Rightarrow y \in B$ and $x \in C$,
  so that $(x, y) \in f \wedge (x, y) \in C \times B$, hence $(x, y) \in f
  \bigcap (C \times B)$. If $(x, y) \in f \bigcap (C \times B)$ then $(x, y)
  \in f \wedge (x, y) \in C \times B \Rightarrow x \in C$, proving that $(x,
  y) \in \{ z|z = (x, y) \in f \wedge x \in C \}$. So we have that
  \[ f_{|C} = \{ z|z = (x, y) \in f \wedge x \in C \} = f \bigcap (C \times
     B) \]
  From the above it follows, using [theorem: \ref{class intersection, union,
  inclusion}], that
  \[ f_{|C} \subseteq C \times B \]
  Finally, if $(x, y), (x, y') \in f_{|C}$ then $(x, y), (x, y') \in f$ so
  that $y = y'$. Hence we have that $f_{|C} : C \rightarrow B$ is a partial
  function.
\end{proof}

\begin{theorem}
  \label{function combining functions (1)}Let $f : A \rightarrow C$ and $g : B
  \rightarrow C$ be two partial functions such that $A \bigcap B =
  \varnothing$ then
  \begin{enumerate}
    \item $f \bigcup g : A \bigcup B \rightarrow C$ is a partial function
    
    \item $f = \left( f \bigcup g \right)_{|A}$ and $g = \left( f \bigcup g
    \right)_{|B}$
    
    \item $\tmop{dom} \left( f \bigcup g \right) = \tmop{dom} (f) \bigcup
    \tmop{dom} (g)$
    
    \item If $f : A \rightarrow C$ and $g : B \rightarrow C$ are functions
    then $f \bigcup g : A \bigcup B \rightarrow C$ are functions
  \end{enumerate}
\end{theorem}

\begin{proof}
  
  \begin{enumerate}
    \item As $f : A \rightarrow C$ and $g : B \rightarrow C$ are functions we
    have that $f \subseteq A \times C$ and $g \subseteq B \times C$ so that by
    [theorem: \ref{class intersection, union, inclusion}]
    \[ f \bigcup g \subseteq (A \times C) \bigcup (B \times C)
       \equallim_{\text{[theorem: \ref{cartesian product properties (1)}]}}
       \left( A \bigcup B \right) \times C \]
    Let $(x, y), (x, y') \in f \bigcup g$. Assume that $y \neq y'$ then we can
    not have that $(x, y), (x, y') \in f$ for then, as $f$ is a function, we
    would have $y = y'$, likewise we can not have that $(x, y), (x, y') \in
    g$, for then, as $g$ is a function, we would have that $y = y'$. So we
    must that either $(x, y) \in f \wedge (x, y') \in g$ or $(x, y) \in g
    \wedge (x, y') \in f$, but then we would have $x \in A \bigcap B$ which
    contradicts $A \bigcap B = \varnothing$. So we must have that $y = y'$.
    Summarized
    \[ \text{If } (x, y), (x, y) \in f \bigcup g \tmop{then} \tmop{we}
       \tmop{have} y = y' \]
    \item As $f \subseteq A \times C$ we have by [theorem :\ref{class
    intersection, union, inclusion}] that
    \[ f \bigcap (B \times C) \subseteq (A \times C) \bigcap (B \times C)
       \equallim_{\text{[theorem: \ref{cartesian product properties (1)}]}}
       \left( A \bigcap B \right) \times C = \varnothing \times C
       \equallim_{\text{[theorem: \ref{cartesian product with enpty set}}}
       \varnothing \]
    proving using [theorem: \ref{class empty set}] that
    \begin{equation}
      \label{eq 2.21.017} f \bigcap (B \times C) = \varnothing
    \end{equation}
    As \ $g \subseteq B \times C$ we have by [theorem :\ref{class
    intersection, union, inclusion}] that
    \[ g \bigcap (A \times C) \subseteq (B \times C) \bigcap (A \times C)
       \equallim_{\text{[theorem: \ref{cartesian product properties (1)}]}}
       \left( A \bigcap B \right) \times C = \varnothing \times C
       \equallim_{\text{[theorem: \ref{cartesian product with enpty set}}}
       \varnothing \]
    proving using [theorem: \ref{class empty set} that
    \begin{equation}
      \label{eq 2.22.017} g \bigcap (A \times C) = \varnothing
    \end{equation}
    Further we have
    \begin{eqnarray*}
      \left( f \bigcup g \right)_{|A} & = & \left( f \bigcup g \right) \bigcap
      (A \times C)\\
      & \equallim_{\text{[theorem: \ref{class class
      commutative,idempotent,associative,distributivity}}} & \left( f \bigcap
      (A \times C) \right) \bigcup \left( g \bigcap (A \times C) \right)\\
      & \equallim_{\text{[eq: \ref{eq 2.22.017}]}} & \left( f \bigcap (A
      \times C) \right) \bigcup \varnothing\\
      & \equallim_{\text{[theorem: \ref{class universal and empotyset
      properties}]}} & f \bigcap (A \times C)\\
      & \equallim_{f \subseteq A \times C \tmop{snd} \left[ \tmop{theorem} :
      \ref{class inclusion and union and intersection} \right]} & f\\
      \left( f \bigcup g \right)_{|B} & = & \left( f \bigcup g \right) \bigcap
      (B \times C)\\
      & \equallim_{\text{[theorem: \ref{class class
      commutative,idempotent,associative,distributivity}}} & \left( f \bigcap
      (B \times C) \right) \bigcup \left( g \bigcap (B \times C) \right)\\
      & \equallim_{\text{[eq: \ref{eq 2.21.017}]}} & \varnothing \bigcup
      \left( \bigcap (B \times C) \right)\\
      & \equallim_{\text{[theorem: \ref{class universal and empotyset
      properties}]}} & g \bigcap (B \times C)\\
      & \equallim_{g \subseteq B \times C \tmop{snd} \left[ \tmop{theorem} :
      \ref{class inclusion and union and intersection} \right]} & g
    \end{eqnarray*}
    \item 
    \begin{eqnarray*}
      x \in \tmop{dom} \left( f \bigcup g \right) & \Leftrightarrow & \exists
      y \text{ such that } (x, y) \in f \bigcup g\\
      & \Leftrightarrow & \exists y \text{ such that } (x, y) \in f \vee (x,
      y) \in g\\
      & \Rightarrow & x \in \tmop{dom} (f) \vee x \in \tmop{dom} (g)\\
      & \Rightarrow & x \in \tmop{dom} (f) \bigcup \tmop{dom} (g)\\
      x \in \tmop{dom} (f) \bigcup \tmop{dom} (g) & \Rightarrow & x \in
      \tmop{dom} (f) \vee x \in \tmop{dom} (g)\\
      & \Rightarrow & \left( \exists y \text{ such that } (x, y) \in f
      \right) \vee \left( \exists y' \text{ such that } (x, y) \in g \right)\\
      & \Rightarrow & \left( \exists y \text{ such that } (x, y) \in f
      \bigcup g \right) \vee \left( \exists y' \text{ such that } (x, y) \in f
      \bigcup g \right)\\
      & \Rightarrow & x \in \tmop{dom} \left( f \bigcup g \right)
    \end{eqnarray*}
    so
    \[ \tmop{dom} \left( f \bigcup g \right) = \tmop{dom} (f) \bigcup
       \tmop{dom} (g) \]
    \item As $f : A \rightarrow C$ and $g : B \rightarrow C$ are functions we
    have that $A = \tmop{dom} (f)$, $B = \tmop{dom} (g)$. So that
    \[ \tmop{dom} \left( f \bigcup g \right) \equallim_{(3)} \tmop{dom} (f)
       \bigcup \tmop{dom} (g) = A \bigcup B \]
    proving that
    \[ f \bigcup g : A \bigcup B \rightarrow C \text{ is a function} \]
  \end{enumerate}
\end{proof}

\begin{corollary}
  \label{function combining functions (2)}Let $f : A \rightarrow B$ and $g : C
  \rightarrow D$ be functions such that $A \bigcap C = \varnothing$ then
  \[ f \bigcup g : A \bigcup C \rightarrow B \bigcup D \]
  is a function.
\end{corollary}

\begin{proof}
  Using [theorem: \ \ref{function extend target}] we have that $f : A
  \rightarrow B \bigcup D$ and $g : C \rightarrow B \bigcup D$ are functions.
  Applying then the previous theorem [theorem: \ref{function combining
  functions (1)}] proves that $f \bigcup g : A \bigcup C \rightarrow B \bigcup
  D$ is a function.
\end{proof}

\begin{corollary}
  \label{function combining bijections}Let $f : A \rightarrow B$ and $g : C
  \rightarrow D$ be bijections with $A \bigcap C = \varnothing$ and $B \bigcap
  D = \varnothing$ then
  \[ f \bigcup g : A \bigcup C \rightarrow B \bigcup D \]
  is a bijection.
\end{corollary}

\begin{proof}
  Using the previous theorem [theorem: \ref{function combining functions (2)}]
  we have that $f \bigcup g : A \bigcup C \rightarrow B \bigcup D$ is a
  function. Now we have:
  \begin{description}
    \item[injectivity] If $(x, y), (x', y) \in f \bigcup g \subseteq \left( A
    \bigcup C \right) \times \left( B \bigcup D \right)$ we have the following
    possibilities for $y$:
    \begin{description}
      \item[$y \in B$] As $f \subseteq A \times B$ and $g \subseteq C \times
      D$ we can not have $(x, y), (x', y) \in g$ [for then $y \in D
      \Rightarrow y \in B \bigcap D = \varnothing$], as $g$ is injective we
      have $x = x'$.
      
      \item[$y \in D$] As $f \subseteq A \times B$ and $g \subseteq C \times
      D$ we can not have $(x, y), (x', y) \in f$ [for then $y \in B
      \Rightarrow y \in B \bigcap D = \varnothing$], as $f$ is injective we
      have $x = x'$.
    \end{description}
    so in all cases we have $x = x'$ proving injectivity of $f \bigcup g : A
    \bigcup C \rightarrow B \bigcup D$.
    
    \item[surjectivity] If $y \in B \bigcup D$ then we have either:
    \begin{description}
      \item[$y \in B$] Then as $f$ is surjective there exist a $x \in A
      \subseteq A \bigcup C$ such that $(x, y) \in f \subseteq f \bigcup g$.
      
      \item[$y \in D$] Then as $g$ is surjective there exist a $x \in C
      \subseteq A \bigcup C$ such that $(x, y) \in g \subseteq f \bigcup g$.
    \end{description}
    proving that in all cases there exist a $x \in A \bigcup C$ such that $(x,
    y) \in f \bigcup g$.
  \end{description}
\end{proof}

\begin{corollary}
  \label{function extending funtion domain}Let $f : A \rightarrow B$ be a
  function $a, b$ elements such that $a \nin A$ then
  \[ g : A \bigcup \{ a \} \rightarrow B \bigcup \{ b \} \text{ defined by } g
     = \{ (a, b) \} \bigcup f \]
  is a function.
  
  \begin{note}
    A alternative definition of $g$ is $g (x) = \left\{\begin{array}{l}
      b \text{ if } x = a\\
      f (x) \text{ if } x \in A
    \end{array}\right.$
  \end{note}
\end{corollary}

\begin{proof}
  Using [example: \ref{function constant function}] we have that $C_b : \{ a
  \} \rightarrow \{ b \}$ where $C_b = \{ (x, b) |x \in \{ a \} \} = \{ (a, b)
  \}$ is a function. As $A \bigcap \{ a \}$ we can use the previous corollary
  [corollary: \ref{function combining functions (2)}] so that
  \[ h : A \bigcup \{ a \} \rightarrow B \bigcup \{ b \} \text{ where } h =
     \{ (a, b) \} \bigcup f \text{ is a function} \]
\end{proof}

\begin{theorem}
  \label{function restricted function properties}Let $f : A \rightarrow B$ be
  a partial function and $C \subseteq A$ a sub-class of $A$ then we have:
  \begin{enumerate}
    \item $\tmop{dom} (f_{|C}) = C \bigcap \tmop{dom} (f)$
    
    \item $\tmop{range} (f_{|C}) = f (C)$
    
    \item If $D \subseteq C$ then $f_{|C} (D) = f (D)$ and $(f_{|C})_{|D} =
    f_{|D}$
    
    \item If $E \subseteq B$ then $(f_{|C})^{- 1} (E) = C \bigcap f^{- 1} (E)$
    
    \item If $f : A \rightarrow B$ is injective then $f_{|C} : C \rightarrow
    B$ is injective
    
    \item If $f : A \rightarrow B$ is bijective and $x \in A$ then
    $f_{|A\backslash \{ a \}} : A\backslash \{ a \} \rightarrow B\backslash \{
    b \}$ where $(a, b) \in f$ is a bijection.
  \end{enumerate}
\end{theorem}

\begin{proof}
  
  \begin{enumerate}
    \item If $x \in \tmop{dom} (f_{|C})$ then there exists a $y$ such that
    $(x, y) \in f_{|C}$, hence $x \in C$ and $(x, y) \in f$ or $x \in C$ and
    $x \in \tmop{dom} (f)$, so that $x \in C \bigcap \tmop{dom} (f)$. Hence
    \begin{equation}
      \label{eq 2.14.001} \tmop{dom} (f_{|C}) \subseteq C \bigcap \tmop{dom}
      (f)
    \end{equation}
    Further if $x \in C \bigcap \tmop{dom} (f)$ then $x \in C$ and $x \in
    \tmop{dom} (f)$, so there exists a $y$ such that $(x, y) \in f$, hence
    $(x, y) \in f_{|C}$ or $x \in \tmop{dom} (f_{|C})$. So $C \bigcap
    \tmop{dom} (f) \subseteq \tmop{dom} (f_{|C})$ which together with [eq:
    \ref{eq 2.14.001}] gives
    \[ \tmop{dom} (f_{|C}) = C \bigcap \tmop{dom} (f) \]
    \item If $y \in \tmop{range} (f_{|C})$ then $\exists x$ such that $(x, y)
    \in f_{|C}$, hence $(x, y) \in f$ and $x \in C$, so that $y \in f (C)$. On
    the other hand if $y \in f (C)$ there exists a $x \in C$ such that $(x, y)
    \in f$, hence $(x, y) \in f_{|C}$ so that $y \in \tmop{range} (f_{|C})$.
    Hence using the Axiom of Extent [axiom: \ref{axiom of extent}] we have
    \[ \tmop{range} (f_{|C}) = f (C) \]
    \item If $y \in f_{|C} (D)$ then $\exists x \in D$ such that $(x, y) \in
    f_{|C}$, hence $(x, y) \in f$ so that $y \in f (D)$. On the other hand if
    $y \in f (D)$ then $\exists x \in D$ such that $(x, y) \in f$, which as $x
    \in D \subseteq C \Rightarrow x \in C$ proves that $(x, y) \in f_{|C}$, so
    $y \in f_{|C} (D)$. Hence using the Axiom of Extent [axiom: \ref{axiom of
    extent}] we have
    \[ f_{|C} (D) = f (D) \]
    Further
    \[ (f_{|C})_{|D} = \left( f \bigcap (C \times B) \right) \bigcap (D
       \times B) \equallim_{D \times B \subseteq C \times B} f \bigcap (D
       \times B) = f_{|D} \]
    \item If $x \in (f_{|C})^{- 1} (E)$ then there exist a $y \in E$ such that
    $(x, y) \in f_{|C}$, hence $x \in C$ and $(x, y) \in f \Rightarrow x \in
    f^{- 1} (E)$, so that $x \in C \bigcap f^{- 1} (E)$. \ Further if $x \in C
    \bigcap f^{- 1} (E)$ then $x \in C$ and $x \in f^{- 1} (E)$, so there
    exist a $y \in E$ such that $(x, y) \in f \Rightarrowlim_{x \in C} (x, y)
    \in f_{|C}$, hence $x \in (f_{|C})^{- 1} (E)$. Hence using the Axiom of
    Extent [axiom: \ref{axiom of extent}] we have
    \[ (f_{|C})^{- 1} (E) = C \bigcap f^{- 1} (E) \]
    \item If $(x, y), (x', y) \in f_{|C}$ then as $f_{|C} \subseteq f$ we have
    $ (x, y), (x', y) \in f$ which as $f$ is injective proves $y = y'$
    
    \item Let $(x, y) \in f_{|A\backslash \{ a \}} = f \bigcap ((A\backslash
    \{ a \}) \times B)$ then $(x, y) \in f$ and $x \neq a$. Assume that $y =
    b$ then $(x, b) \in f$ and as $(a, b) \in f$ we must have by injectivity
    of $f$ that $x = a$ contradicting $x \neq a$, hence we have that $y \in
    B\backslash \{ b \}$. So we have that $f_{|A\backslash \{ a \}} \subseteq
    (A\backslash \{ a \}) \times ((B\backslash \{ b \}))$ and
    \[ f_{|A\backslash \{ a \}} : A\backslash \{ a \} \rightarrow B\backslash
       \{ b \} \text{ is a function} \]
    If $(x, y), (x', y) \in f_{|A\backslash \{ a \}}$ then $(x, y), (x', y)
    \in f \Rightarrowlim_{f \text{ is injective}} x = x'$ proving that
    $f_{|A\backslash \{ a \}}$ is injective. So
    \[ f_{|A\backslash \{ a \}} \text{ is injective} \]
    Finally if $y \in B\backslash \{ b \}$ then as $f : A \rightarrow B$ is a
    bijection hence surjective, there exist a $x \in A$ such that $(x, y) \in
    f$. Assume that $x = a$ then $(a, y) \in f$ which as $(a, b) \in f$ gives
    $y = b$ contradicting $y \in B\backslash \{ b \}$. Hence we must have $x
    \in A\backslash \{ a \}$ so that $(x, y) \in f \bigcap ((A\backslash \{ a
    \}) \times B) = f_{|A\backslash \{ a \}}$ proving that
    \[ f : A\backslash \{ a \} \rightarrow B\backslash \{ b \} \text{ is
       surjective} \]
  \end{enumerate}
\end{proof}

\begin{theorem}
  \label{function restriction and domain}Let $f : A \rightarrow B$ be a
  \tmtextbf{partial} function then $f_{| \tmop{dom} (f)} = f$
\end{theorem}

\begin{proof}
  If $(x, y) \in f$ then by definition $x \in \tmop{dom} (f)$ hence $(x, y)
  \in f_{| \tmop{dom} (f)}$, further if $(x, y) \in f_{| \tmop{dom} (f)}$ then
  $(x, y) \in f$ and $x \in \tmop{dom} (f)$, so evidently $(x, y) \in f$.
  Hence using the Axiom of Extent [axiom: \ref{axiom of extent}] we have
  \[ f_{| \tmop{dom} (f)} = f \]
\end{proof}

\begin{theorem}
  \label{function inverse and restriction}Let $f : A \rightarrow B$ be a
  injective \tmtextbf{partial} function and $C \subseteq A$ then $(f^{-
  1})_{|f (C)} = (f_{|C})^{- 1}$
\end{theorem}

\begin{proof}
  Let $(x, y) \in (f^{- 1})_{|f (C)}$ then $x \in f (C)$ and $(x, y) \in f^{-
  1} \Rightarrow (y, x) \in f$, as $x \in f (C)$ there exists a $z \in C$ such
  that $(z, x) \in f$. As $f$ is injective we have that $z = y$, proving that
  $y \in C$, which as $(y, x) \in f$ gives $(y, x) \in f_{|C}$ so that $(x, y)
  \in (f_{|C})^{- 1}$. Hence
  \begin{equation}
    \label{eq 2.17.004} (f^{- 1})_{|f (C)} \subseteq (f_{|C})^{- 1}
  \end{equation}
  If $(x, y) \in (f_{|C})^{- 1}$ then $(y, x) \in f_{|C}$ so that $y \in C$
  and $(y, x) \in f$. Hence $x \in f (C)$ and as $(y, x) \in f$ gives $(x, y)
  \in f^{- 1}$ we have $(x, y) \in (f^{- 1})_{|f (C)}$. This proves that
  $(f_{|C})^{- 1} \subseteq (f^{- 1})_{|f (C)}$, combing this with [eq:
  \ref{eq 2.17.004}] gives:
  \[ (f^{- 1})_{f (C)} = (f_{|C})^{- 1} \]
\end{proof}

\begin{theorem}
  \label{function composition and restriction}Let $f : A \rightarrow B$ and $g
  : C \rightarrow D$ be \tmtextbf{partial} functions and $E \subseteq A$ then
  \[ (g \circ f)_{|E} = g_{|f (E)} \circ f_{|E} \]
\end{theorem}

\begin{proof}
  Let $(x, z) \in (f \circ g)_{|E}$ then $(x, z) \in f \circ g$ and $x \in E$.
  Hence $\exists y$ such that $(x, y) \in f \wedge (y, z) \in g$, as $x \in E$
  $(x, y) \in f_{|E}$. From $x \in E$ and $(x, y) \in f$ it follows also that
  $y \in f (E)$, hence as $(y, z) \in g$ we have that $(y, z) \in g_{|f (E)}$.
  From $(x, y) \in f_{|E}$ and $(y, z) \in g_{|f (E)}$ it follows that $(x, z)
  \in g_{|f (E)} \circ f_{|E}$ so that
  \begin{equation}
    \label{eq 2.17.003} (g \circ f)_{|E} \subseteq g_{|f (E)} \circ f_{|E}
  \end{equation}
  If $(x, z) \in g_{|f (E)} \circ f_{|E}$ then there exists a $y$ such that
  $(x, y) \in f_{|E}$ and $(y, z) \in g_{|f (E)}$, so $x \in E$, $(x, y) \in
  f$, $y \in f (E)$ and $(y, z) \in g$. Hence $x \in E$ and $(x, z) \in g
  \circ f$ proving that $(x, z) \in (g \circ f)_{|E} .$ So $g_{|f (E)} \circ
  f_{|E} \subseteq (g \circ f)_{|E}$ which combined with [eq: \ref{eq
  2.17.003}] gives
  \[ (g \circ f)_{|E} = g_{|f (E)} \circ g_{|E} \]
\end{proof}

\begin{theorem}
  \label{function restriction of a function}Let $f : A \rightarrow B$ and $C
  \subseteq A$ a sub-class of $A$ then $f_{|C} : C \rightarrow B$ is a
  function.
\end{theorem}

\begin{proof}
  Using [definition: \ref{function restriction of a graph}] we have that
  $f_{|C} : C \rightarrow B$ is a partial function, as by [theorem:
  \ref{function restricted function properties}] $\tmop{dom} (f_{|C}) = C
  \bigcap \tmop{dom} (f) \equallim_{f \text{ is a function}} C \bigcap A
  \equallim_{C \subseteq A} C$, it follows that $f_{|C} : C \rightarrow B$ is
  a function.
\end{proof}

The following theorem will be used for manifolds later

\begin{theorem}
  Let $f : A \rightarrow B$ and $g : C \rightarrow D$ be injections then we
  have
  \begin{enumerate}
    \item $f : A \rightarrow f (A)$ and $g : C \rightarrow f (C)$ are
    bijections
    
    \item $\tmop{dom} (f \circ g^{- 1}) = g \left( A \bigcap C \right)$
    
    \item $f \circ g^{- 1} : g \left( A \bigcap C \right) \rightarrow f \left(
    A \bigcap C \right)$ is a bijection
    
    \item $f \circ g^{- 1} = (f \circ g^{- 1})_{|g \left( A \bigcap C \right)}
    = f_{|A \bigcap C} \circ (g^{- 1})_{|g \left( A \bigcap C \right)} = f_{|
    \left( A \bigcap C \right)} \circ \left( g_{|A \bigcap C} \right)^{- 1}$
  \end{enumerate}
\end{theorem}

\begin{proof}
  
  \begin{enumerate}
    \item This follows from [proposition: \ref{function injectivity to
    bijection}]
    
    \item If $z \in \tmop{dom} (f \circ g^{- 1})$ then $\exists x$ such that
    $(z, x) \in f \circ g^{- 1}$, hence $\exists y$ such that $(z, y) \in g^{-
    1} \text{ and } (y, z) \in f$, from which it follows that $(y, z) \in g$
    and $(y, z) \in f$. As $g \subseteq C \times B$ and $f \subseteq A \times
    B$ it follows that $y \in A$ and $y \in C$ so that $y \in A \bigcap C$, as
    $(y, z) \in g$ we have $z \in g \left( A \bigcap C \right)$. This proves
    \begin{equation}
      \label{eq 2.16.003} \tmop{dom} (g \circ f^{- 1}) \subseteq g \left( A
      \bigcap C \right)
    \end{equation}
    If $z \in g \left( A \bigcap C \right)$ then $\exists y \in A \bigcap C$
    such that $(y, z) \in g$, hence $(z, y) \in g^{- 1}$. As $f$ is a function
    we have that $A = \tmop{dom} (f)$, hence as $y \in A \bigcap C \Rightarrow
    y \in A$, there exists a $x$ such that $(y, x) \in f$. As $(z, y) \in g^{-
    1}$ we have $(z, x) \in f \circ g^{- 1}$ proving that $z \in \tmop{dom} (f
    \circ g^{- 1})$. Hence $g \left( A \bigcap C \right) \subseteq \tmop{dom}
    (g \circ f^{- 1})$ which combined with [eq: \ref{eq 2.16.003}].
    \[ \tmop{dom} (g \circ f^{- 1}) = g \left( A \bigcap C \right) \]
    \item 
    \begin{description}
      \item[injectivity] If $(x, y), (x', y) \in f \circ g^{- 1}$ then
      $\exists z, z'$ such that $ (x, z), (x', z') \in f$ and $(z, y), (z', y)
      \in g^{- 1}$. Hence $(y, z), (y, z') \in g$ so that $z = z'$ [as $g^{-
      1}$ is a function] hence $(x, z), (x', z) \in f$ giving $x = x'$.
      
      \item[$\tmop{surjectivity}$] If $y \in f \left( A \bigcap C \right)$
      then $\exists x \in A \bigcap C$ such that $ (x, y) \in f$. As $A
      \bigcap C \subseteq C$ we have that $x \in C$, so as $g : C \rightarrow
      B$ is a function there exist a $z$ such that $(x, z) \in g$, hence $(z,
      x) \in g^{- 1}$. As $(x, y) \in f$ it follows that $(z, y) \in f \circ
      g^{- 1}$.
    \end{description}
    \item We have
    \begin{eqnarray*}
      (f \circ g^{- 1}) & \equallim_{\text{[theorem: \ref{function restriction
      and domain}]}} & (f \circ g^{- 1})_{\tmop{dom} (f \circ g^{- 1})}\\
      & \equallim_{(1) \text{}} & (f \circ g^{- 1})_{g \left( A \bigcap C
      \right)}\\
      & \equallim_{\text{[theorem: \ref{function composition and
      restriction}]}} & f_{|g^{- 1} \left( g \left( A \bigcap C \right)
      \right)} \circ (g^{- 1})_{g \left( A \bigcap C \right)}\\
      & \equallim_{g \text{ is injective and [theorem: \ref{function preimage
      of image}]}} & f_{|A \bigcap C} \circ (g^{- 1})_{|g \left( A \bigcap C
      \right)}\\
      & \equallim_{\text{[theorem: \ref{function inverse and restriction}]}}
      & f_{|A \bigcap C} \circ \left( g_{|A \bigcap C} \right)^{- 1}
    \end{eqnarray*}
  \end{enumerate}
\end{proof}

\subsection{Set operations and (Partial) Functions}

\begin{theorem}
  \label{function properties (1)}Let $f : A \rightarrow B$ be a function then
  we have
  \begin{enumerate}
    \item If $C, D \subseteq A$ with $C \subseteq D$ then $f (C) \subseteq f
    (D)$
    
    \item If $C, D \subseteq B$ with $C \subseteq D$ then $f^{- 1} (C)
    \subseteq f^{- 1} (D)$
    
    \item If $C, D \subseteq B$ then $f^{- 1} (C\backslash D) = f^{- 1} (C)
    \backslash f^{- 1} (D)$
    
    \item If $D \subseteq B$ then $f^{- 1} (B\backslash D) = A\backslash f^{-
    1} (D)$
    
    \item If $C, D \subseteq A$ then $f (C) \backslash f (D) \subseteq f
    (C\backslash D)$
    
    \item If $C, D \subseteq A$ and $f$ is \tmtextbf{injective} then $f (C)
    \backslash f (D) = f (C\backslash D)$
  \end{enumerate}
\end{theorem}

\begin{proof}
  
  \begin{enumerate}
    \item Let $y \in f (C)$ then there exist a $x \in C$ such that $(x, y) \in
    f$, as $C \subseteq D$ we have $x \in D$ so that $y \in f (D)$
    
    \item If $x \in f^{- 1} (C)$ there exists a $y \in C$ such that $(x, y)
    \in f$, as $C \subseteq D$ then $y \in D$ so that $x \in f^{- 1} (D)$
    
    \item If $x \in f^{- 1} (C\backslash D)$ then $\exists y \in C\backslash
    D$ such that $(x, y) \in f$. As $y \in C\backslash D$ we have that $y \in
    C$ and $y \nin D$, from $y \in C$ it follows that $x \in f^{- 1} (C)$.
    Assume that also $x \in f^{- 1} (D)$ then $\exists y' \in D$ such that
    $(x, y') \in f$ which, as $f$ is a function and $(x, y) \in f$, proves
    that $y = y'$, hence $y \in D$ contradicting $y \nin D$, so we must have
    $x \nin f^{- 1} (D)$, hence $x \in f (C) \backslash f (D)$ proving
    \begin{equation}
      \label{eq 2.17.002} f^{- 1} (C\backslash D) \subseteq f^{- 1} (C)
      \backslash f^{- 1} (D)
    \end{equation}
    If $x \in f^{- 1} (C) \backslash f^{- 1} (D)$ then $x \in f^{- 1} (C)$ and
    $x \nin f^{- 1} (D)$. As $x \in f^{- 1} (C)$ there exists a $y \in C$ such
    that $(x, y) \in f$. Assume that $y \in D$, then as $(x, y) \in f$ we have
    $x \in f^{- 1} (D)$ contradicting $x \nin f^{- 1} (D)$, so we must have $y
    \nin D$. Hence $y \in C\backslash D$ which proves that $x \in f^{- 1}
    (C\backslash D)$ or $f^{- 1} (C) \backslash f^{- 1} (D) \subseteq f^{- 1}
    (C\backslash D)$. Combining this with [eq: \ref{eq 2.17.002}] proves
    \[ f^{- 1} (C\backslash D) = f^{- 1} (C) \backslash f^{- 1} (D) \]
    \item As $D \subseteq B \subseteq B$ we have by (3) that
    \begin{eqnarray*}
      f^{- 1} (B\backslash D) & = & f^{- 1} (B) \backslash f^{- 1} (D)\\
      & \equallim_{\text{[theorem: \ref{function image preimage}]}} &
      A\backslash f^{- 1} (D)
    \end{eqnarray*}
    \item If $y \in f (C) \backslash f (D)$ then $y \in f (C)$ and $y \nin f
    (D)$. From $y \in f (C)$ it follows that $\exists x \in C$ such that $(x,
    y) \in f$. Assume that $x \in D$ then as $(x, y) \in f$ we have $y \in f
    (D)$ contradicting $y \nin f (D)$, so we must have $x \nin D$, proving
    that $x \in C\backslash D$. Hence $y \in f (C\backslash D)$ or
    \[ f (C) \backslash f (D) \subseteq f (C\backslash D) \]
    \item If $y \in f (C\backslash D)$ then $\exists x \in C\backslash D$ such
    that $x \in C$, $x \nin D$ and $(x, y) \in f$. From $x \in C$ it follows
    that $y \in f (C)$. Assume that $y \in f (D)$ then there exist a $x' \in
    D$ such that $(x', y) \in f$, as $f$ is \tmtextbf{injective} we have $x =
    x'$ so that $x \in D$ contradicting $x \nin D$, hence $y \nin f (D)$. This
    proves that $y \in f (C) \backslash f (D)$ or $f (C\backslash D) \subseteq
    f (C) \backslash f (D)$ which combined with (3) gives
    \[ f (C) \backslash f (D) = f (C\backslash D) \]
  \end{enumerate}
\end{proof}

\begin{theorem}
  \label{function function and intersection and union}If $f : A \rightarrow B$
  is a function, $E, F \subseteq A$ and $C, D \subseteq B$ then we have
  \begin{enumerate}
    \item $f \left( E \bigcup F \right) = f (E) \bigcup f (F)$
    
    \item $f^{- 1} \left( C \bigcup D \right) = f^{- 1} (C) \bigcup f^{- 1}
    ()$
    
    \item $f \left( E \bigcap F \right) \subseteq f (E) \bigcap f (F)$
    
    \item If $f$ is injective then $f \left( E \bigcap F \right) = f (E)
    \bigcap f (F)$
    
    \item $f^{- 1} \left( C \bigcap D \right) = f^{- 1} (C) \bigcap f^{- 1}
    (D)$
  \end{enumerate}
\end{theorem}

\begin{proof}
  
  \begin{enumerate}
    \item Let $y \in f \left( E \bigcup F \right)$ then there exist a $x \in E
    \bigcup F$ with $(x, y) \in f$. \ So $x \in E$ proving that $y \in f (E)$
    or $x \in F$ proving $y \in f (F)$. So it follows that $y \in f (E)
    \bigcup f (F)$ or
    \begin{equation}
      \label{eq 2.18.002} f \left( E \bigcup F \right) \subseteq f (E) \bigcup
      f (F)
    \end{equation}
    If $y \in f (E) \bigcup f (F)$ then we have the following possibilities
    \begin{description}
      \item[$y \in f (E)$] Then $\exists x \in E$ such that $(x, y) \in f$. As
      by the definition of a union $x \in E \bigcup F$, it follows that $y \in
      f \left( E \bigcup F \right)$
      
      \item[$y \in f (F)$] Then $\exists x \in F$ such that $(x, y) \in f$. As
      by the definition of a union $x \in E \bigcup F$, it follows that $y \in
      f \left( E \bigcup F \right)$
    \end{description}
    So in all cases we have $y \in f \left( E \bigcup F \right)$. Hence $f (E)
    \bigcup f (F) \subseteq f \left( E \bigcup F \right)$ which combined with
    [eq: \ref{eq 2.18.002}] proves
    \[ f \left( E \bigcup F \right) = f (E) \bigcup f (F) \]
    \item If $x \in f^{- 1} \left( C \bigcup D \right)$ there exists a $y \in
    C \bigcup D$ such that $(x, y) \in f$. From $y \in C \bigcup D$ we have $y
    \in C$ hence $x \in f^{- 1} (C)$ or $y \in D$ hence $x \in f^{- 1} (D)$.
    So $x \in f^{- 1} (C) \bigcup f^{- 1} (D)$ proving
    \begin{equation}
      \label{eq 2.19.002} f^{- 1} \left( C \bigcup D \right) \subseteq f^{- 1}
      (C) \bigcup f^{- 1} (D)
    \end{equation}
    If $x \in f^{- 1} (C) \bigcup f^{- 1} (D)$ then we have the following
    possibilities to consider:
    \begin{description}
      \item[$x \in f^{- 1} (C)$] Then $\exists y \in C$ such that $(x, y) \in
      f$. As by the definition of a union $y \in C \bigcup D$ it follows that
      $x \in f^{- 1} \left( C \bigcup D \right)$
      
      \item[$x \in f^{- 1} (D)$] Then $\exists y \in D$ such that $(x, y) \in
      f$. As by the definition of a union $y \in C \bigcup D$ it follows that
      $x \in f^{- 1} \left( C \bigcup D \right)$
    \end{description}
    So in all cases we have $x \in f^{- 1} \left( C \bigcup D \right)$,
    proving $f^{- 1} (C) \bigcup f^{- 1} (D) \subseteq f^{- 1} \left( C
    \bigcup D \right)$ which combined with [eq \ref{eq 2.19.002}] proves
    \[ f^{- 1} \left( C \bigcup D \right) = f^{- 1} (C) \bigcup f^{- 1} (D)
    \]
    
    
    \item If $y \in f \left( E \bigcap F \right)$ then $\exists x \in E
    \bigcap F$ such that $(x, y) \in f$. From $x \in E \bigcap F$ we have that
    $x \in E$ hence $y \in f (E)$ and $x \in F$, so that $y \in f (F)$. Hence
    $y \in f (E) \bigcap f (F)$ or
    \[ f \left( E \bigcap F \right) \subseteq f (E) \bigcap f (F) \]
    \item Using (3) we have that
    \begin{equation}
      \label{eq 2.20.002} f \left( E \bigcap F \right) \subseteq f (E) \bigcap
      f (F)
    \end{equation}
    Let $y \in f (E) \bigcap f (F)$ then we have $y \in f (E)$ so that
    $\exists x \in E$ such that $(x, y) \in f$ and $y \in f (F)$ so that
    $\exists x' \in F$ such that $(x', y) \in f$. As $f$ is injective and $(x,
    y), (x', y) \in f$ we have $x = x'$ so that $x \in E \bigcap F$, proving
    that $f (E) \bigcap f (F) \subseteq f \left( E \bigcap F \right)$.
    Combining this result with [eq: \ref{eq 2.20.002}] gives
    \[ f \left( E \bigcap F \right) = f (E) \bigcap f (F) \]
    \item If $x \in f^{- 1} \left( C \bigcap D \right)$ then $\exists y \in C
    \bigcap D$ such that $y \in C$, so that $x \in f^{- 1} (C)$ and $y \in D$,
    so that $x \in f^{- 1} (D)$. Hence $x \in f^{- 1} (C) \bigcap f^{- 1} (D)$
    proving
    \begin{equation}
      \label{eq 2.21.002} f^{- 1} \left( C \bigcap D \right) \subseteq f^{- 1}
      (C) \bigcap f^{- 1} (D)
    \end{equation}
    If $x \in f^{- 1} (C) \bigcap f^{- 1} (D)$ then $x \in f^{- 1} (C)$ so
    there exists a $y \in C$ such that $(x, y) \in f$ and $x \in f^{- 1} (D)$
    so $\exists y' \in D$ such that $(x, y') \in f$. As $f$ is a function $y =
    y'$ proving $y \in C \bigcap D$, hence $x \in f^{- 1} \left( C \bigcap D
    \right)$. So $f^{- 1} (C) \bigcap f^{- 1} (D) \subseteq f^{- 1} \left( C
    \bigcap D \right)$, combining this with [eq: \ref{eq 2.21.002}] gives
    \[ f^{- 1} \left( C \bigcap D \right) = f^{- 1} (C) \bigcap f^{- 1} (D)
    \]
  \end{enumerate}
\end{proof}

Up to now we define a function $f : A \rightarrow B$ by specifying what the
classes $f, A, B$ are. However in many cases we have a parameterized
expression [based on function calls and operators) to define $f$. Then we have
the following

\begin{proposition}
  \label{function simple definition}Let $A, B$ be classes and suppose that
  there exists a parameterized expression $F (x)$ that calculates a
  \tmtextbf{unique} value for \tmtextbf{every} $x \in A$ then we can define
  the function $f : A \rightarrow B$ by $f = \{ z|z = (x, F (x)) \wedge x \in
  A \}$
\end{proposition}

\begin{proof}
  If $(x, y), (x, y') \in f$ then there exists $a, a' \in A$ such that $(x, y)
  = (a, F (a)) \wedge (x, y') \in (a', F (a'))$, hence $x = a \wedge x = a'
  \wedge y = f (a) \wedge y' = F (a') \Rightarrow a = a' \wedge y' = F (a)
  \wedge y = F (a)$ proving that $y = y'$. So
  \[ f : A \rightarrow B \text{ is a partial function} \]
  If $x \in A$ then as $F (x)$ is defined on every $x \in A$ we have that $(x,
  F (x)) \in f$ so that $x \in \tmop{dom} (f)$. So $A \subseteq \tmop{dom}
  (f)$ we have by \ref{function condition (1)} that
  \[ f : A \rightarrow B \text{ is a function} \]
\end{proof}

This leads to a notation that we will gradually start to use

\begin{notation}
  \label{function simple definition notation}The function definition $f : A
  \rightarrow B$ defined by $f (x) = F (x)$ [where $E (x)$ is a parameterized
  expression that calculates a unique value for every $x \in A$] is equivalent
  with
  \[ f = \{ z|z = (x, Ex) \wedge x \in E \} = \{ (x, E (x)) |x \in X \} \]
\end{notation}

\begin{example}
  $f : \mathbb{R} \rightarrow \mathbb{R}$ is defined by $f (x) = \cos (4 \cdot
  x)$
\end{example}

\subsection{Indexed sets}

In many cases we have to deal with sets indexed by a index, which is actually
a function in another form. We will use this in toplogy and vector spaces.

\begin{definition}[indexed set]
  \label{function indexed set}Let $f : I \rightarrow A$ be a surjection then
  $A$ is called a \tmtextbf{indexed set} and noted as
  \[ A = \{ f (i) |i \in I \} \text{ or } A = \{ f_i |i \in I \} \]
  So
  \[ x \in A \Leftrightarrow \exists i \in I \text{ such that } x = f (i)
     \text{ or } x = f_i \]
  $I$ is called the index of the indexed set $\{ f_i |i \in I \}$.
\end{definition}

\begin{definition}[unique indexed set]
  \label{function indexed set unique}$A = \{ f_i | i \in I \nobracket \}$ is a
  \tmtextbf{unique indexed set }if $f : I \rightarrow A$ is a bijection. So
  \[ x \in A \Leftrightarrow \exists i \in I \text{ such that } x = f (i)
     \text{ or } x = f_i \]
  and
  \[ \text{If } x_i = x_j \text{ then } i = j \]
\end{definition}

\begin{example}
  \label{function indexed set example}Every set can be writen as a unique
  indexed set indexed by itself. So if $A$ is a set then $A = \{ \tmop{Id}_X
  (i) |i \in I \}$.
\end{example}

\section{Families}

\subsection{Family}

We introduce now the idea of a indexed family which is essential a function of
a class to another class. It is essential another notation for a function
where the emphasis is on the objects in a collection and a way of indexing
these objects and less on the function itself

\begin{definition}
  \label{family}{\index{$\{ A_i \}_{i \in I}$}}A Let $I, B$ be classes then a
  family
  \[ \tmmathbf{\{ x_i \}_{i \in I} \subseteq B} \]
  is actually a function
  \[ \tmmathbf{f : I \rightarrow B} \]
  Further \tmtextbf{$x_i$} is another notation for $\tmmathbf{f (i)}$ so that
  \tmtextbf{$y = f_i$} is equivalent with $\tmmathbf{y = f (i)}$ or
  \tmtextbf{$(i, y) \in f$}
\end{definition}

\begin{note}
  In the above definition $\{ x_i \}_{i \in I}$ only make sense if you specify
  what the defining function is. To avoid excessive notation, we assume that
  if we write \tmtextbf{$\{ x \}_{i \in I} \subseteq B$} that the defining
  function is $\tmmathbf{x : I \rightarrow B}$. However this is sometimes not
  feasible and in that case we state what the defining function of $\{ x_i
  \}_{i \in I}$ is. 
\end{note}

\begin{note}
  In most cases we use the notation $\{ x_i \}_{i \in I} \subseteq B$ to
  indicate three parts of a function namely $x : I \rightarrow B$ where $x$ is
  the graph of a function, $I$ is the domain and $B$ is the target of the
  function. In some cases we are not interested in the target of the function
  [we can always assume then that the target is the universal class
  $\mathcal{U}$] in this case we just use the notation $\{ x_i \}_{i \in I}$
  for the family.
\end{note}

\begin{example}
  \label{family empty family}The empty function $\varnothing : \varnothing
  \rightarrow V$ [see example: \ref{function empty function}] defines a family
  that is noted as $\{ \varnothing_i \}_{i \in \varnothing} \subseteq V$.
  Further if $\{ x_i \}_{i \in \varnothing} \subseteq V$ is a family where the
  index set is empty then $\{ x_i \}_{i \in \varnothing} \subseteq V = \{
  \varnothing_i \}_{i \in \varnothing} \subseteq V$
\end{example}

\begin{proof}
  $\{ x_i \}_{i \in \varnothing} \subseteq V$ is defined by the function $x :
  \varnothing \rightarrow V$, as $x \subseteq \varnothing \times V =
  \varnothing$ we have that $x = \varnothing$ so that $\{ x_i \}_{i \in
  \varnothing} \subseteq V = \{ \varnothing_i \}_{i \in \varnothing} \subseteq
  V$.
\end{proof}

\begin{proposition}
  \label{family empty family condition}For the family $\{ x_i \}_{i \in I}
  \subseteq \varnothing$ we have $I = \varnothing$ so that $\{ x_i \}_{i \in
  I} \subseteq \varnothing = \{ \varnothing_i \}_{i \in \varnothing} \subseteq
  \varnothing$
\end{proposition}

\begin{proof}
  Let $f : I \rightarrow \varnothing$ be the function that defines the family
  then as $f$ is a function we have that $f (I) = \varnothing$. So if $x \in
  I$ then $\exists y \in \varnothing$ such that $(x, y) \in f \subseteq I
  \times \varnothing = \varnothing$ a contradiction, hence we must have $I =
  \varnothing$.
\end{proof}

\begin{example}
  \label{family {x}xeA}Let $A, B$ be classes then family $\{ (\tmop{Id}_A)_a
  \}_{a \in A} \subseteq B$ defined by the function $\tmop{Id}_A : A
  \rightarrow B$ is noted as $\{ x \}_{x \in A}$.
\end{example}

We can now define the concept of a sub family

\begin{definition}
  \label{family definition (2)}Let $\{ A_i \}_{i \in I} \subseteq B$ be a
  family of objects in $B$ defined by the function $f : I \rightarrow B$ and
  $J \subseteq I$ then $\{ A_i \}_{i \in J} \subseteq B$ is the family defined
  by the function $f_{|J} : J \rightarrow B$ [see: theorem: \ref{function
  restriction of a function} for the proof that $f_J : I \rightarrow B$ is a
  function]
\end{definition}

\begin{definition}
  \label{family definition (3)}Let $I, J, A, B$ be classes such that $I
  \bigcap J = \varnothing$ and
  \[ \{ x \}_{i \in I} \subseteq A \text{ defined by the function } f : I
     \rightarrow A \]
  \[ \{ y_i \}_{i \in J} \subseteq B \text{ defined by the function } g : J
     \rightarrow B \]
  then $\{ z_i \}_{i \in I \bigcup J} \subseteq A \bigcup B$ defined by $z_i =
  \left\{\begin{array}{l}
    A_i \tmop{if} i \in I\\
    B_i \text{ if } i \in J
  \end{array}\right.$ is the family defined by the function
  \[ f \bigcup g : I \bigcup J \rightarrow A \bigcup B \]
  [see theorem: \ref{function combining functions (2)} for the proof that $f
  \bigcup g : I \bigcup J \rightarrow A \bigcup B$ is indeed a function]
\end{definition}

\begin{definition}
  \label{family index set is a product}If $I, J$ are classes then $\{ x_{i, j}
  \}_{(i, j) \in I \times J} \subseteq A$ is defined by a function $x : I
  \times J \rightarrow A$, based on this we can define the following families:
  \begin{enumerate}
    \item $\forall i \in I$ $\{ x_{i, j} \}_{j \in J}$ is defined by the
    function $x_{i, \ast} : J \rightarrow A$ where $x_{i, \ast} (j) = x (i, j)
    = x_{i, j}$
    
    \item $\forall j \in J$ $\{ x_{i, j} \}_{i \in I}$ is defined by the
    function $x_{\ast, j} : I \rightarrow A$ where $x_{\ast, j} (i) = x (i, j)
    = x_{i, j}$
  \end{enumerate}
\end{definition}

Composition of functions can also also be represented via the above family
notation,

\begin{definition}
  \label{family and function composition}If you have a function $\tmmathbf{f :
  I \rightarrow J}$ and a family $\tmmathbf{\{ x_j \}_{j \in J}}
  \tmmathbf{\subseteq A}$ [defined by the function $\tmmathbf{g : J
  \rightarrow A}$] then
  \[ \tmmathbf{\{ x_{f (i)} \}_{i \in I}} \]
  is the family represented by the function
  \[ \tmmathbf{g \circ f : I \rightarrow A} \]
\end{definition}

So a family is just another notation for a function. We introduce also a new
notation for the range of this function.

\begin{definition}
  \label{family range}{\index{$\{ A_i |i \in I \}$}}If $\{ x_i \}_{i \in I}$
  is a family of objects in B [standing for the function $f : I \rightarrow
  B$] then we define $\{ x_i |i \in I \}$ by
  \[ \{ x_i |i \in I \} = \tmop{range} (f) = f (I) \]
\end{definition}

The motivation for this definition is the following theorem

\begin{theorem}
  \label{family range (1)}If $\{ x_i \}_{i \in I} \subseteq B$ is a family of
  objects in $B$ with associated function $f$ then
  \[ x \in \{ x_i |i \in I \} \Leftrightarrow \exists i \in I \text{ such
     that } x = x_i \]
\end{theorem}

\begin{proof}
  As $\{ x_i \}_{i \in I} \subseteq B$ is equivalent with $f : I \rightarrow
  B$ we have
  \begin{eqnarray*}
    z \in \{ x_i |i \in I \} & \Leftrightarrowlim_{\tmop{define}} & z \in
    \tmop{range} (x)\\
    & \Leftrightarrow & \exists i \tmop{with} (i, z) \in f\\
    & \Leftrightarrowlim_{f \subseteq I \times B} & \exists i \text{ with $i
    \in I \wedge (i, z) \in f$}\\
    & \Leftrightarrow & \exists i \in I \text{ with } (i, z) \in f\\
    & \Leftrightarrow & \exists i \in I \text{ with } z = f (i)\\
    & \Leftrightarrow & \exists i \in I \text{ with z=x\_i}
  \end{eqnarray*}
\end{proof}

\begin{theorem}
  \label{family set}If $\{ x_i \}_{i \in I} \subseteq B$ is a family such that
  $I$ and $B$ are sets then $\{ x_i |i \in I \}$ is a set
\end{theorem}

\begin{proof}
  $\{ x_i \}_{i \in I} \subseteq B$ is actually the function $x : I
  \rightarrow B$ where $\tmop{range} (x) = \{ x_i |i \in I \}$. As $I$and $B$
  are sets, it follows from [theorem: \ref{partial function set domain range}]
  that $\tmop{range} (x)$ is a set, hence $\{ x_i |i \in I \}$ is a set.
\end{proof}

Up to now we consider a family as a indexed collection of objects. What is
actually a object, in set theory it is a class which can be either a set or a
proper class. A class is a collection so we can talk about the union of these
collection. The convention is then to use upper case instead of lower case. If
we want to deal with the union and intersection of the objects [considered as
collections] in the family we use also a different notation.

\begin{notation}
  \label{family union (1)}If $\{ A_i \}_{i \in I} \subseteq B$ is a family of
  objects in $B$ [standing for the function $A : I \rightarrow B$] then
  $\bigcup_{i \in I} A_i$ is defined by
  \[ \bigcup_{i \in I} A_i = \bigcup \{ \tmop{range} (A) \} \text{
     [definition: \ref{class union}]} \]
\end{notation}

\begin{definition}
  A family $\{ A_i \}_{i \in I} \subseteq B$ is \tmtextbf{pairwise disjoint}
  iff $\forall i, j \in I$ with $i \neq j$ we have $A_i \bigcap A_j =
  \varnothing$.
\end{definition}

\begin{notation}
  If $\{ A_i \}_{i \in I} \subseteq B$ is pairwise disjoint and we want to
  indicate this fact when we write the union of the family then we use the
  notation $\bigsqcup_{i \in I} A_i$. So $\bigsqcup_{i \in I} A_i$ is actually
  the same as $\bigcup_{i \in I} A_i$, but also relating the information that
  $\{ A_i \}_{i \in I}$ is pairwise disjoint.
\end{notation}

Using this new notation we have the following characterization of the union

\begin{theorem}
  \label{family union (2)}If $\{ A_i \}_{i \in I} \subseteq B$ is a family of
  objects in $B$ then
  \[ x \in \bigcup_{i \in I} A_i \Leftrightarrow \exists i \in I \text{ such
     that } x \in A_i \]
\end{theorem}

\begin{proof}
  As $\{ A_i \}_{i \in I} \subseteq B$ is actually the function $A : I
  \rightarrow B$ where $\bigcup_{i \in I} A_i = \bigcup \tmop{range} (A)$.
  Then we have
  \begin{eqnarray*}
    x \in \bigcup_{i \in I} A_i & \Leftrightarrowlim_{\text{definition}} & x
    \in \bigcup \tmop{range} (A)\\
    & \Leftrightarrowlim_{\text{[definition: \ref{class union}]}} & \exists y
    \in \tmop{range} (A) \text{ such that } x \in y\\
    & \Leftrightarrow & \exists i \text{such that } (i, y) \in A \text{ and }
    x \in y\\
    & \Leftrightarrowlim_{A \subseteq I \times B} & \exists i \in I \text{
    such that } (x, y) \in A \infixand x \in y\\
    & \Leftrightarrow & \exists i \in I \text{ such that $y = A_i \text{ and
    $x \in y$}$}\\
    & \Leftrightarrow & \exists i \in I \text{ such that $x \in A_i$}
  \end{eqnarray*}
\end{proof}

\begin{corollary}
  \label{family union of family set and surjections}If $\{ A_j \}_{j \in J}
  \subseteq B$ is a family and $f : I \rightarrow J$ is a surjection then
  \[ \bigcup_{j \in J} A_j = \bigcup_{i \in I} A_{f (i)} \]
\end{corollary}

\begin{proof}
  If $x \in \bigcup_{i \in J} A_j$ then by [theorem: \ref{family union (2)}]
  there exist a $j \in J$ such that $x \in A_j = A (j)$. As $f$ is surjective
  we have by [theorem: \ref{function injectivity, surjectivity}] that there
  exist a $i \in I$ such that $j = f (i)$. Hence $x \in A (f (i)) = (A \circ
  f) (i)$. So by [theorem: \ref{family union (2)}] and the definition of
  $\bigcup_{i \in I} A_{f (i)}$ we have $x \in \bigcup_{i \in I} A_{f (i)}$.
  Hence
  \begin{equation}
    \label{eq 2.29.012} \bigcup_{j \in J} A_j \subseteq \bigcup_{i \in I} A_{f
    (i)}
  \end{equation}
  If $x \in \bigcup_{i \in I} A_{f (i)}$ then there exist a $i \in I$ such
  that $x \in (A \circ f) (i)$, which, as using [theorem: \ref{partial
  function domain range composition}] \ $(A \circ f) (i) \in \tmop{range}
  (A)$, means that there exists a $j \in J$ such that $A_j = (A \circ f) (i)$.
  Hence $x \in A_j$ proving by [theorem: \ref{family union (2)}] that $x \in
  \bigcup_{j \in J} A_j$. So $\bigcup_{i \in I} A_{f (i)} \subseteq \bigcup_{j
  \in J} A_j$ which combined with [eq: \ref{eq 2.29.012}] gives
  \[ \bigcup_{j \in J} A_j = \bigcup_{i \in I} A_{f (i)} \]
  
\end{proof}

\begin{theorem}
  \label{family union condition set}If $\{ A_i \}_{i \in I} \subseteq B$ is a
  family of objects in $B$ where $I$ and $B$ are sets then $\bigcup_{i \in I}
  A_i$ is a set.
\end{theorem}

\begin{proof}
  As $\{ A_i \}_{i \in I} \subseteq B$ is another way of saying $A : I
  \rightarrow B$ and $I$ and $B$ are sets, it follows from [theorem:
  \ref{partial function set domain range}] that $\tmop{range} (A)$ is a set.
  Using the Axiom of Union [axiom: \ref{axiom of union}] $\bigcup \tmop{range}
  (A)$ is a set, so by definition $\bigcup_{i \in I} A_i$. is a set.
\end{proof}

\begin{example}
  \label{family union of a empty set}Let $\{ A_i \}_{i \in \varnothing}
  \subseteq B$ be the family defined by $A = \varnothing$ [the empty function
  $\varnothing : \varnothing \rightarrow B$ [see example: \ref{function empty
  function}[] then $\bigcup_{i \in \varnothing} A_i = \varnothing$
\end{example}

\begin{proof}
  Let $y \in \tmop{range} (A) = \tmop{range} (\varnothing)$ then $x$ such that
  $(x, y) \in \varnothing$, a contradiction. Hence $\tmop{range} (A) =
  \varnothing$. So
  \[ \bigcup_{i \in \varnothing} A = \bigcup \tmop{range} (A) = \bigcup
     \varnothing \equallim_{\text{\ref{class trivial union intersection}}}
     \varnothing \]
\end{proof}

\begin{definition}
  \label{family intersection(1)}{\index{$\bigcap_{i \in I} A_i$}}If $\{ A_i
  \}_{i \in I} \subseteq B$ is a family of objects in $B$ [standing for the
  function $A : I \rightarrow B$] then $\bigcap_{i \in I} A_i$ is defined by
  \[ \bigcap_{i \in I} A_i = \bigcap \tmop{range} (A)  \text{[definition:
     \ref{class intersection}]} \]
\end{definition}

\begin{theorem}
  \label{family intersection (2)}If $\{ A_i \}_{i \in I} \subseteq B$ then $x
  \in \bigcap_{i \in I} A_i \Leftrightarrow \forall i \in I \text{ we have } x
  \in A_i$
\end{theorem}

\begin{proof}
  $\{ A_i \}_{i \in I} \subseteq B$ is actually the function $A : I
  \rightarrow B$ where $\bigcap_{i \in I} A_i = \bigcap \tmop{range} (A)$.
  \begin{eqnarray*}
    x \in \bigcap_{i \in I} A_i & \Leftrightarrowlim_{\text{definition}} & x
    \in \bigcap \tmop{range} (A)\\
    & \Leftrightarrowlim_{\text{[definition: \ref{class intersection}]}} &
    \forall y \in \tmop{range} (A) \text{ we have } x \in y\\
    & \Leftrightarrowlim_{y \in \tmop{range} (A) \Leftrightarrow \exists i
    \text{ with } (i, y) \in A} & \forall i \in I \text{ } \tmop{with} (i, y)
    \in A \text{ we have } x \in y\\
    & \Leftrightarrow & \forall i \in I \text{ with } y = A_i \text{ we have
    } x \in y\\
    & \Leftrightarrow & \forall i \in I \text{ we have } x \in A_i
  \end{eqnarray*}
\end{proof}

\begin{theorem}
  \label{family intersection is a set}If $\{ A_i \}_{i \in I} \subseteq B$ is
  a family of objects in $B$ such that $I \neq \varnothing$ then $\bigcap_{i
  \in I} A_i$ is a set.
\end{theorem}

\begin{proof}
  $\{ A_i \}_{i \in I} \subseteq B$ is actually the function $A : I
  \rightarrow B$ where $\bigcap_{i \in I} A_i = \bigcap \tmop{range} (A)$. As
  $I \neq \varnothing$ there exists a $i \in I$. Given that $A$ is a function
  it follows that $\tmop{dom} (A) = I$, so there exists a $y$ such that $(i,
  y) \in A$ or $y \in \tmop{range} (A)$. So $\varnothing \neq \tmop{range} (A)
  $ which by [theorem: \ref{class intersection}] proves that $\bigcap
  \tmop{range} (A)$ is a set, hence $\bigcap_{i \in I} A_i$ is a set.
\end{proof}

\begin{example}
  \label{family trivial}Let $I = \{ 0 \}$, $B$ a class and take $A : I
  \rightarrow \{ B \}$ defined by $A = \{ (0, B) \}$, defining the family $\{
  A_i \}_{i \in \{ 0 \}} \subseteq \{ B \}$ where $A_0 = B$. For this family
  we have $\bigcap_{i \in \{ 0 \}} A_i = B$ and $\bigcup_{i \in \{ 0 \}} A_i =
  B$
\end{example}

\begin{proof}
  Using [example: \ref{function trivial bijection}] it follows that $A : I
  \rightarrow \{ B \}$ is bijection, hence a function, so that $\{ A_i \}_{i
  \in \{ 0 \}} \subseteq \{ B \}$ is a well defined family. Further as $A$ is
  a bijection we have that
  \[ \tmop{range} (A) = \{ B \} \]
  Finally
  \[ \bigcup_{i \in \{ 0 \}} A_i = \bigcup \tmop{range} (A) = \bigcup \{ B \}
     \equallim_{\text{[example: \ref{class trivial union intersection}]}} A \]
  and
  \[ \bigcap_{i \in \{ 0 \}} A_i = \bigcap \tmop{range} (A) = \bigcap \{ B \}
     \equallim_{\text{[example: \ref{class trivial union intersection}]}} A \]
\end{proof}

\begin{example}
  \label{family union{A,B}}Let $C, D$ classes, $I = \{ 0, 1 \}$ and take $A :
  I \rightarrow \{ C, D \}$ defined by $A = \{ (0, C), (1, D) \}$ [see
  example: \ref{function between {0,1} and {A,B}}], defining the family $\{
  A_i \}_{i \in \{ 0, 1 \}} \subseteq \{ C, D \}$ where $A_0 = C$ and $A_1 =
  D$. For this family we have $\bigcup_{i \in \{ 0, 1 \}} A_i = C \bigcup D$
  and $\bigcap_{i \in \{ 0, 1 \}} A_i = C \bigcap D$.
\end{example}

\begin{proof}
  If $y \in \tmop{range} (A)$ then $\exists x$ such that $(x, y) \in A = \{
  (0, C), (1, D) \}$, so that $(x, y) = (0, C) \Rightarrow y = C$ or $(x, y) =
  (1, D) \Rightarrow y = D$, proving that $x \in \{ C, D \}$. Further if $y
  \in \{ C, D \}$ then $y = C \Rightarrow (0, C) \in A \Rightarrow y \in
  \tmop{range} (A)$ or $y = D \Rightarrow (1, D) \in A \Rightarrow y \in
  \tmop{range} (A)$. So we have
  \[ \tmop{range} (A) = \{ C, D \} \]
  Finally
  \[ \bigcup_{i \in \{ 0, 1 \}} A_i = \bigcup \tmop{range} (A) \equallim
     \bigcup \{ C, D \} \equallim_{\text{[example: \ref{class union{A,B}}]]}}
     C \bigcup D \]
  and
  \[ \bigcap_{i \in \{ 0, 1 \}} A_i = \bigcap \tmop{range} (A) \equallim
     \bigcap \{ C, D \} \equallim_{\text{[example: \ref{class union{A,B}}]]}}
     C \bigcap D \]
  
\end{proof}

\subsection{Properties of the union and intersection of families}

To save space, from now on we use [theorem: \ref{family union (2)}] and
[theorem: \ref{family intersection (2)}] about union and intersection of
families without explicit referring to these theorems.

\begin{theorem}
  \label{family properties (1)}If $\{ A_i \}_{i \in I} \subseteq B$ is a
  family then we have:
  \begin{enumerate}
    \item $\forall i \in I$ we have $A_i \subseteq \bigcup_{i \in I} A_i$
    
    \item $\forall i \in I$ we have $\bigcap_{i \in I} A_i \subseteq A_i$
    
    \item If $\forall i \in I$ we have that $A_i \subseteq C$ then $\bigcup_{i
    \in I} A_i \subseteq C$
    
    \item If $\forall i \in I$ we have $C \subseteq A_i$ then $C \subseteq
    \bigcap_{i \in I} A_i$
  \end{enumerate}
\end{theorem}

\begin{proof}
  
  \begin{enumerate}
    \item Let $i \in I$ and assume that $x \in A_i$ then $\exists i \in I$
    such that $x \in A_i$, so $x \in \bigcup_{i \in I} A_i$, proving that $A_i
    \subseteq \bigcup_{i \in I} A_i$.
    
    \item Let $i \in I$ then if $x \in \bigcap_{i \in I} A_i$ we have $\forall
    j \in I$ that $x \in A_j \Rightarrowlim_{i \in I} x \in A_i$, proving that
    $\bigcap_{i \in I} A_i \subseteq A_i$
    
    \item
    \begin{eqnarray*}
      x \in \bigcup_{i \in I} A_i & \Rightarrow & \exists i \in I \vdash x \in
      A_i\\
      & \Rightarrowlim_{A_i \subseteq C} & x \in C\\
      & \Rightarrow & \bigcup_{i \in I} A_i \subseteq C
    \end{eqnarray*}
    \item
    \begin{eqnarray*}
      x \in C & \Rightarrow & \forall i \in I \vDash x \in A_i\\
      & \Rightarrow & x \in \bigcap_{i \in I} A_i\\
      & \Rightarrow & C \subseteq \bigcap_{i \in I} A_i
    \end{eqnarray*}
  \end{enumerate}
\end{proof}

\begin{theorem}
  \label{family properties (2)}If $\{ A_i \}_{i \in I} \subseteq B$ is a
  family then
  \begin{enumerate}
    \item If $J \subseteq I$ then
    \begin{enumerate}
      \item $\bigcup_{i \in J} A_i \subseteq \bigcup_{i \in I} A_i$
      
      \item $\bigcap_{i \in I} A_i \subseteq \bigcap_{i \in J} A_i$
    \end{enumerate}
    \item If $I = J \bigcup K$ then
    \begin{enumerate}
      \item $\bigcup_{i \in I} A_i = \left( \bigcup_{i \in J} A_i \right)
      \bigcup \left( \bigcup_{i \in K} A_i \right)$
      
      \item $\bigcap_{i \in I} A_i = \left( \bigcap_{i \in J} A_i \right)
      \bigcap \left( \bigcap_{i \in K} A_i \right)$
    \end{enumerate}
  \end{enumerate}
\end{theorem}

\begin{proof}
  
  \begin{enumerate}
    \item 
    \begin{enumerate}
      \item If $x \in \bigcup_{i \in J} A_i$ then $\exists i \in J$ such that
      $x \in A_i$, as $J \subseteq I$ we have \ $i \in I$ with $x \in A_i$, so
      that $x \in \bigcup_{i \in I} A_i$.
      
      \item If $x \in \bigcap_{i \in I} A_i$ then $\forall i \in I$ we have $x
      \in A_i$, as $J \subseteq I$ we have also $\forall i \in J$ that $x \in
      A_i$, hence $x \in \bigcap_{i \in J} A_i$.
    \end{enumerate}
    \item 
    \begin{enumerate}
      \item As by [theorem: \ref{class intersection, union, inclusion}] $J, K
      \subseteq I$ we have using (1) that $\bigcup_{i \in J} A_i \subseteq
      \bigcup_{i \in I} A_i$ and $\bigcup_{i \in K} A_i \subseteq \bigcup_{i
      \in I} A_i$. Using [theorem: \ref{class intersection, union, inclusion}
      it follows that
      \begin{equation}
        \label{eq 2.25.004} \left( \bigcup_{i \in J} A_i \right) \bigcup
        \left( \bigcup_{i \in K} A_i \right) \subseteq \bigcup_{i \in I} A_i
      \end{equation}
      If $x \in \bigcup_{i \in I} A$ then $\exists i \in I$ such that $x \in
      A_i$, as $I = J \bigcup K$ we have $i \in J \Rightarrow x \in \bigcup_{i
      \in J} A_i$ or $i \in K \Rightarrow x \in \bigcup_{i \in K} A_i$, which
      proves that $x \in \left( \bigcup_{i \in J} A_i \right) \bigcup \left(
      \bigcup_{i \in K} A_i \right)$. Hence
      \[ \bigcup_{i \in I} A_i \subseteq \left( \bigcup_{i \in J} A_i \right)
         \bigcup \left( \bigcup_{i \in K} A_i \right) \]
      which combined with [eq: \ref{eq 2.25.004}] proves
      \[ \bigcup_{i \in I} A_i = \left( \bigcup_{i \in J} A_i \right) \bigcup
         \left( \bigcup_{i \in K} A_i \right) \]
      \item As by [theorem: \ref{class intersection, union, inclusion}] $J, K
      \subseteq I$ we have using (1) that $\bigcap_{i \in I} A_i \subseteq
      \bigcap_{i \in J} A_i$ and $\bigcap_{i \in I} A_i \subseteq \bigcap_{i
      \in K} A_i$. Using [theorem: \ref{class intersection, union, inclusion}]
      it follows that
      \begin{equation}
        \label{eq 2.26.004} \bigcap_{i \in I} A_i \subseteq \left( \bigcap_{i
        \in J} A_i \right) \bigcap \left( \bigcap_{i \in K} A_i \right)
      \end{equation}
      If $x \in \left( \bigcap_{i \in J} A \right) \bigcap \left( \bigcap_{i
      \in K} A \right)$ then $x \in \bigcap_{i \in J} A_i$ and $x \in
      \bigcap_{i \in K} A_i$. So $\forall i \in J$ we have $x \in A_i$ and
      $\forall i \in K$ we have $x \in A_i$. Hence as $\forall i \in I$ we
      have $i \in J \Rightarrow x \in A_i$ or $i \in K \Rightarrow x \in A_i$
      it follows that $x \in \bigcap_{i \in I} A_i$. So $\left( \bigcap_{i \in
      J} A \right) \bigcap \left( \bigcap_{i \in K} A \right) \subseteq
      \bigcap_{i \in I} A_i$ which combined with [eq: \ref{eq 2.26.004}]
      proves
      \[ \bigcap_{i \in I} A_i = \left( \bigcap_{i \in J} A_i \right) \bigcap
         \left( \bigcap_{i \in K} A_i \right) \]
    \end{enumerate}
  \end{enumerate}
  
\end{proof}

\begin{theorem}
  \label{family union intersection and inclusion}Let $\{ A_i \}_{i \in I}
  \subseteq C$ and $\{ B_i \}_{i \in I} \subseteq D$ be two families such that
  $\forall i \in I$ we have $A_i \subseteq B_i$ then
  \begin{enumerate}
    \item $\bigcup_{i \in I} A_i \subseteq \bigcup_{i \in I} B_i$
    
    \item $\bigcap_{i \in I} A_i \subseteq \bigcap_{i \in I} B_i$
  \end{enumerate}
\end{theorem}

\begin{proof}
  
  \begin{enumerate}
    \item If $x \in \bigcup_{i \in I} A_i$ there exist a $i \in I$ such that
    $x \in A_i \Rightarrowlim_{A_i \subseteq B_i} x \in B_i$, hence $x \in
    \bigcup_{i \in I} B_i$
    
    \item If $x \in \bigcap_{i \in I} A_i$ then $\forall i \in I$ we have $x
    \in A_i \Rightarrowlim_{A_i \subseteq B_i} x \in B_i$ proving $x \in
    \bigcap_{i \in I} B_i$
  \end{enumerate}
\end{proof}

We have also the distributive laws for union and intersection [theorem:
\ref{class class commutative,idempotent,associative,distributivity}]

\begin{theorem}[Distributivity]
  \label{family distributivity}Let $\{ A_i \}_{i \in I} \subseteq B$ be a
  family and $C$ a class then
  \begin{enumerate}
    \item $C \bigcap \left( \bigcup_{i \in I} A_i \right) = \bigcup_{i \in I}
    \left( C \bigcap A_i \right)$
    
    \item $C \bigcup \left( \bigcap_{i \in I} A_i \right) = \bigcap_{i \in I}
    \left( C \bigcup A_i \right)$
    
    \item $C \bigcap \left( \bigcap_{i \in I} A_i \right) = \bigcap_{i \in I}
    \left( C \bigcap A_i \right)$
    
    \item $C \bigcup \left( \bigcup_{i \in I} A_i \right) = \bigcup_{i \in I}
    \left( C \bigcup A_i \right)$
  \end{enumerate}
\end{theorem}

\begin{proof}
  
  \begin{enumerate}
    \item If $x \in C \bigcap \left( \bigcup_{i \in I} A_i \right)$ then $x
    \in C$ and $x \in \bigcup_{i \in I} A_i \Rightarrow \exists i \in I \text{
    such that } x \in A_i$. Hence $x \in C \bigcap A_i$, proving by [theorem:
    \ref{family properties (1)}] that $x \in \bigcup_{i \in I} A_i$. So
    \begin{equation}
      \label{eq 2.27.004} C \bigcap \left( \bigcup_{i \in I} A_i \right)
      \subseteq \bigcup_{i \in I} \left( C \bigcap A_i \right)
    \end{equation}
    If $x \in \bigcup_{i \in} \left( C \bigcap A_i \right)$ then there exist a
    $i \in I$ such that $x \in C$ and $x \in A_i \Rightarrow x \in \bigcup_{i
    \in I} A_i$, so $x \in C \bigcap \left( \bigcup_{i \in I} A_i \right)$,
    proving that $\bigcup_{i \in I} \left( C \bigcap A_i \right) \subseteq C
    \bigcap \left( \bigcup_{i \in I} A_i \right)$. Combining this with [eq:
    \ref{eq 2.27.004}] proves
    \[ C \bigcap \left( \bigcup_{i \in I} A_i \right) = \bigcup_{i \in I}
       \left( C \bigcap A_i \right) \]
    \item If $x \in C \bigcup \left( \bigcap_{i \in I} A_i \right)$ then we
    have the following cases to consider:
    \begin{description}
      \item[$x \in C$] then $\forall i \in I$ we have $x \in C \bigcup A_i$
      hence $x \in \bigcap_{i \in I} \left( C \bigcup A_i \right)$
      
      \item[$x \in \bigcap_{i \in I} A_i$] then $\forall i \in I$ we have $x
      \in A_i$ hence $x \in \bigcap_{i \in I} \left( C \bigcup A_i \right)$
    \end{description}
    which proves that
    \begin{equation}
      \label{eq 2.28.004} C \bigcup \left( \bigcap_{i \in I} A_i \right)
      \subseteq \bigcap_{i \in I} \left( C \bigcup A_i \right)
    \end{equation}
    If $x \in \bigcap_{i \in I} \left( C \bigcup A_i \right)$ then we have two
    cases to consider:
    \begin{description}
      \item[$x \in C$] then $x \in C \bigcup \left( \bigcap_{i \in I} A_i
      \right)$
      
      \item[$x \nin C$] then, as $\forall i \in I$ we have $x \in C \bigcup
      A_i \Rightarrowlim_{x \nin C} x \in A_i$, it follows that $x \in
      \bigcap_{i \in I} A_i$ hence $x \in C \bigcup \left( \bigcap_{i \in I}
      A_i \right)$
    \end{description}
    In all cases we have $x \in C \bigcup \left( \bigcap_{i \in I} A_i
    \right)$ proving that $\bigcap_{i \in I} \left( C \bigcup A_i \right)
    \subseteq C \bigcup \left( \bigcap_{i \in I} A_i \right)$, combining this
    with [eq: \ref{eq 2.28.004}] gives
    \[ C \bigcup \left( \bigcap_{i \in I} A_i \right) \subseteq \bigcap_{i
       \in I} \left( C \bigcup A_i \right) \]
    \item We have
    \begin{eqnarray*}
      x \in C \bigcap \left( \bigcap_{i \in I} A_i \right) & \Leftrightarrow &
      x \in C \wedge \forall i \in I \text{ we have } x \in A_i\\
      & \Leftrightarrow & \forall i \in I \text{ we have } x \in C \bigcap
      A_i\\
      & \Leftrightarrow & x \in \bigcap_{i \in I} \left( C \bigcap A_i
      \right)
    \end{eqnarray*}
    Proving
    \[ C \bigcap \left( \bigcap_{i \in I} A_i \right) = \bigcap_{i \in I}
       \left( C \bigcap A_i \right) \]
    \item We have
    \begin{eqnarray*}
      x \in C \bigcup \left( \bigcup_{i \in I} A_i \right) & \Leftrightarrow &
      x \in C \vee x \in \bigcup_{i \in I} A_i\\
      & \Leftrightarrow & x \in C \vee \exists i \in I \text{ with } x \in
      A_i\\
      & \Leftrightarrow & \exists i \in I \text{ with } (x \in C \vee x \in
      A_i)\\
      & \Leftrightarrow & \exists i \in I \text{ we have } x \in C \bigcup
      A_i
    \end{eqnarray*}
    proving that
    \[ C \bigcup \left( \bigcup_{i \in I} A_i \right) = \bigcup_{i \in I}
       \left( C \bigcup A_i \right) \]
  \end{enumerate}
  
\end{proof}

\begin{theorem}
  \label{family union of union of two families}Let $\{ A_i \}_{i \in I}
  \subseteq C$ and $\{ B_i \}_{i \in I} \subseteq D$ be two families then
  \begin{enumerate}
    \item $\left( \bigcup_{i \in I} A_i \right) \bigcup \left( \bigcup_{i \in
    I} B_i \right) = \bigcup_{i \in I} \left( A_i \bigcup B_i \right)$
    
    \item $\bigcup_{i \in I} \left( A_i \bigcap B_i \right) \subseteq \left(
    \bigcup_{i \in I} A_i \right) \bigcap \left( \bigcup_{i \in I} B_i
    \right)$
  \end{enumerate}
\end{theorem}

\begin{proof}
  
  \begin{enumerate}
    \item First as $\forall i \in I$ we have by [theorem: \ref{class
    intersection, union, inclusion}] that $A_i \subseteq A_i \bigcup B_i$ and
    $B_i \subseteq A_i \bigcup B_i$ so it follows using \ [theorem:
    \ref{family union intersection and inclusion}] that $\bigcup_{i \in I} A_i
    \subseteq \bigcup_{i \in I} \left( A_i \bigcup B_i \right)$ and
    $\bigcup_{i \in I} B_i \subseteq \bigcup_{i \in I} \left( A_i \bigcup B_i
    \right)$. Applying then [theorem: \ref{class intersection, union,
    inclusion}] gives
    \begin{equation}
      \label{eq 2.31.010} \left( \bigcup_{i \in I} A_i \right) \bigcup \left(
      \bigcup_{i \in I} B_i \right) \subseteq \bigcup_{i \in I} \left( A_i
      \bigcup B_i \right)
    \end{equation}
    If now $x \in \bigcup_{i \in I} A_i \bigcup B_i$ then $\exists i \in I$
    such that $x \in A_i \bigcup B_i$, then we have $x \in A_i \Rightarrow x
    \in \bigcup_{i \in I} A_i$ or $x \in B_i \Rightarrow x \in \bigcup_{i \in
    I} B_i$. So $x \in \left( \bigcup_{i \in I} A_i \right) \bigcup \left(
    \bigcup_{i \in I} B_i \right)$ proving that $\bigcup_{i \in I} \left( A_i
    \bigcup B_i \right) \subseteq \left( \bigcup_{i \in I} A_i \right) \bigcup
    \left( \bigcup_{i \in I} B_i \right)$ which combined with \ref{eq
    2.31.010} gives
    \[ \left( \bigcup_{i \in I} A_i \right) \bigcup \left( \bigcup_{i \in I}
       B_i \right) = \bigcup_{i \in I} \left( A_i \bigcup B_i \right) \]
    \item  As $\forall i \in I$ we have by [theorem: \ref{class intersection,
    union, inclusion}] that $A_i \bigcap B_i \subseteq A_i$ and $A_i \bigcap
    B_i \subseteq A_i$, $B_i \subseteq A_i \bigcup B_i$ it follows using \
    [theorem: \ref{family union intersection and inclusion}] that $\bigcup_{i
    \in I} \left( A_i \bigcap B_i \right) \subseteq \bigcup_{i \in I} A_i$ and
    $\bigcup_{i \in I} \left( A_i \bigcap B_i \right) \subseteq \bigcup_{i \in
    I} B_i$. Using then [theorem: \ref{class intersection, union, inclusion}]
    we have
    \[ \bigcup_{i \in I} \left( A_i \bigcap B_i \right) \subseteq \left(
       \bigcup_{i \in I} A_i \right) \bigcup \left( \bigcup_{i \in I} B_i
       \right) \]
  \end{enumerate}
\end{proof}

We have also a variant of the deMorgan's laws [theorem: \ref{class de Morgan's
law}]

\begin{theorem}[deMorgan's Law]
  \label{family de Morgan}Let $\{ A_i \}_{i \in I} \subseteq B$ be a family
  then we have
  \begin{enumerate}
    \item $\left( \bigcup_{i \in I} A_i \right)^c = \bigcap_{i \in I} (A_i)^c$
    
    \item $\left( \bigcap_{i \in I} A_i \right)^c = \bigcup_{i \in I} (A_i)^c$
    
    \item If $C$ is a class then $C\backslash \left( \bigcup_{i \in I} A_i
    \right) = \bigcap_{i \in I} (A_i \backslash C)$
    
    \item If $C$ is a class then $C \setminus \left( \bigcap_{i \in I} A_i
    \right) = \bigcup_{i \in I} (C\backslash A_i)$
  \end{enumerate}
\end{theorem}

\begin{proof}
  
  \begin{enumerate}
    \item 
    \begin{eqnarray*}
      x \in \left( \bigcup_{i \in I} A_i \right)^c & \Leftrightarrow & x \nin
      \left( \bigcup_{i \in I} A_i \right)\\
      & \Leftrightarrow & \neg \left( x \in \bigcup_{i \in I} A_i \right)\\
      & \Leftrightarrow & \neg \left( \exists i \in I \text{ with } x \in A_i
      \right)\\
      & \Leftrightarrow & \forall i \in I \text{ we have } \neg (x \in A_i)\\
      & \Leftrightarrow & \forall i \in I \text{ we have } x \nin A_i\\
      & \Leftrightarrow & \forall i \in I \text{ we have } x \in (A_i)^c\\
      & \Leftrightarrow & x \in \bigcap_{i \in I} (A_i)^c
    \end{eqnarray*}
    proving that
    \[ \left( \bigcup_{i \in I} A_i \right)^c = \bigcap_{i \in I} (A_i)^c \]
    \item 
    \begin{eqnarray*}
      x \in \left( \bigcap_{i \in I} A_i \right)^c & \Leftrightarrow & x \nin
      \left( \bigcap_{i \in I} A_i \right)^c\\
      & \Leftrightarrow & \neg \left( x \in \left( \bigcap_{i \in I} A_i
      \right) \right)\\
      & \Leftrightarrow & \neg \left( \forall i \in I \text{ we have } x \in
      A_i \right)\\
      & \Leftrightarrow & \exists i \in I \text{ we have } \neg (x \in A_i)\\
      & \Leftrightarrow & \exists i \in I \text{ we have } x \nin A_i\\
      & \Leftrightarrow & \exists i \in I \text{ we have } x \in (A_i)^c\\
      & \Leftrightarrow & x \in \bigcup_{i \in I} (A_i)^c
    \end{eqnarray*}
    proving that
    \[ \left( \bigcap_{i \in I} A_i \right)^c = \bigcup_{i \in I} (A_i)^c \]
    \item  We have
    \begin{eqnarray*}
      C\backslash \left( \bigcup_{i \in I} A_i \right) &
      \equallim_{\text{[theorem: \ref{class difference}]}} & C \bigcap \left(
      \bigcup_{i \in I} A_i \right)^c\\
      & \equallim_{(1)} & C \bigcap \left( \bigcap_{i \in I} (A_i)^c
      \right)\\
      & \equallim_{\text{[theorem: \ref{family distributivity}]}} &
      \bigcap_{i \in I} \left( C \bigcap (A_i)^c \right)\\
      & \equallim_{\text{[theorem: \ref{class difference}]}} & \bigcap_{i \in
      I} (C\backslash A_i)
    \end{eqnarray*}
    \item We have
    \begin{eqnarray*}
      C\backslash \left( \bigcap_{i \in I} A_i \right) &
      \equallim_{\text{[theorem: \ref{class difference}]}} & C \bigcap \left(
      \bigcap_{i \in I} A_i \right)^c\\
      & \equallim_{(2)} & C \bigcap \left( \bigcup_{i \in I} (A_i)^c
      \right)\\
      & \equallim_{\text{[theorem: \ref{family distributivity}]}} &
      \bigcup_{i \in I} \left( C \bigcap (A_i)^c \right)\\
      & = & \bigcup_{i \in I} (C\backslash A_i)
    \end{eqnarray*}
  \end{enumerate}
\end{proof}

\begin{theorem}
  \label{family properties (3)}If $\{ A_i \}_{i \in I} \subseteq B$ is a
  family and $A$ a class then we have
  \begin{enumerate}
    \item $\left( \bigcup_{i \in I} A_i \right) \backslash A = \bigcup_{i \in
    I} (A_i \backslash A)$
    
    \item $\left( \bigcap_{i \in I} A_i \right) \backslash A = \bigcap_{i \in
    I} (A_i \backslash A)$
    
    \item $\left( \bigcup_{i \in I} A_i \right) \times A = \bigcup_{i \in I}
    (A_i \times A)$
    
    \item $A \times \left( \bigcup_{i \in I} A_i \right) = \bigcup_{i \in I}
    (A \times A_i)$
    
    \item $\left( \bigcap_{i \in I} A_i \right) \times A = \bigcap_{i \in I}
    (A_i \times A)$
    
    \item $A \times \left( \bigcap_{i \in I} A_i \right) = \bigcap_{i \in I}
    (A \times A_i)$
  \end{enumerate}
\end{theorem}

\begin{proof}
  
  \begin{enumerate}
    \item 
    \begin{eqnarray*}
      \left( \bigcup_{i \in I} A_i \right) \backslash A &
      \equallim_{\text{[theorem: \ref{class difference}]}} & \left( \bigcup_{i
      \in I} A_i \right) \bigcap A^c\\
      & \equallim_{\text{[theorem: \ref{class class
      commutative,idempotent,associative,distributivity}]}} & A^c \bigcap
      \left( \bigcup_{i \in I} A_i \right)\\
      & \equallim_{\text{[theorem: \ref{family distributivity}]}} &
      \bigcup_{i \in I} \left( A^c \bigcap A_i \right)\\
      & \equallim_{\text{[theorem: \ref{class class
      commutative,idempotent,associative,distributivity}]}} & \bigcup_{i \in
      I} \left( A_i \bigcap A^c \right)\\
      & \equallim_{\text{[theorem: \ref{class difference}]}} & \bigcup_{i \in
      I} (A_i \backslash A)
    \end{eqnarray*}
    \item 
    \begin{eqnarray*}
      \left( \bigcap_{i \in I} A_i \right) \backslash A &
      \equallim_{\text{[theorem: \ref{class difference}]}} & \left( \bigcap_{i
      \in I} A_i \right) \bigcap A^c\\
      & \equallim_{\text{[theorem: \ref{class class
      commutative,idempotent,associative,distributivity}]}} & A^c \bigcap
      \left( \bigcap_{i \in I} A_i \right)\\
      & \equallim_{\text{[theorem: \ref{family distributivity}]}} &
      \bigcap_{i \in I} \left( A^c \bigcap A_i \right)\\
      & \equallim_{\text{[theorem: \ref{class class
      commutative,idempotent,associative,distributivity}]}} & \bigcap_{i \in
      I} \left( A_i \bigcap A^c \right)\\
      & \equallim_{\text{[theorem: \ref{class difference}]}} & \bigcap_{i \in
      I} (A_i \backslash A)
    \end{eqnarray*}
    \item  \
    \begin{eqnarray*}
      (x, y) \in \left( \bigcup_{i \in I} A_i \right) \times A &
      \Leftrightarrow & x \in \bigcup_{i \in I} A_i \wedge y \in A\\
      & \Leftrightarrow & y \in A \wedge \exists i \in I \text{ with } x \in
      A_i\\
      & \Leftrightarrow & \exists i \in I \tmop{with} (x \in A_i \wedge y \in
      A)\\
      & \Leftrightarrow & \exists i \in I \text{ with } (x, y) \in A_i \times
      A\\
      & \Leftrightarrow & (x, y) \in \bigcup_{i \in I} (A_i \times A)
    \end{eqnarray*}
    \item 
    \begin{eqnarray*}
      (x, y) \in A \times \left( \bigcup_{i \in I} A_i \right) &
      \Leftrightarrow & x \in A \wedge y \in \bigcup_{i \in I} A_i\\
      & \Leftrightarrow & x \in A \wedge \exists i \in I \text{ with } y \in
      A_i\\
      & \Leftrightarrow & \exists i \in I \tmop{with} (x \in A \wedge y \in
      A_i)\\
      & \Leftrightarrow & \exists i \in I \text{ with } (x, y) \in A \times
      A_i\\
      & \Leftrightarrow & (x, y) \in \bigcup_{i \in I} (A \times A_i)
    \end{eqnarray*}
    \item  \
    \begin{eqnarray*}
      (x, y) \in \left( \bigcap_{i \in I} A_i \right) \times A &
      \Leftrightarrow & x \in \bigcap_{i \in I} A_i \wedge y \in A\\
      & \Leftrightarrow & \left( \forall i \in I \text{ we have } x \in A_i
      \right) \wedge y \in A\\
      & \Leftrightarrow & \forall i \in I \text{ we have } (x \in A_i \wedge
      y \in A)\\
      & \Leftrightarrow & \forall i \in I \text{ we have } (x, y) \in A_i
      \times A\\
      & \Leftrightarrow & (x, y) \in \bigcap_{i \in I} (A_i \times A)
    \end{eqnarray*}
    \item
    \begin{eqnarray*}
      (x, y) \in A \times \left( \bigcap_{i \in I} A_i \right) &
      \Leftrightarrow & x \in A \wedge y \in \bigcap_{i \in I} A_i\\
      & \Leftrightarrow & \left( \forall i \in I \text{ we have } y \in A_i
      \right) \wedge x \in A\\
      & \Leftrightarrow & \forall i \in I \text{ we have } (y \in A_i \wedge
      x \in A)\\
      & \Leftrightarrow & \forall i \in I \text{ we have } (x, y) \in A
      \times A_i\\
      & \Leftrightarrow & (x, y) \in \bigcap_{i \in I} (A \times A_i)
    \end{eqnarray*}
  \end{enumerate}
\end{proof}

\begin{theorem}
  \label{family union intersection and empty set}Let $\{ A_i \}_{i \in I}
  \subseteq B$ a family then
  \begin{enumerate}
    \item If $j \in I$ then $\left( \bigcup_{i \in I\backslash \{ j \}} A_i
    \right) \bigcup A_j = \bigcup_{i \in I} A_i$
    
    \item $\bigcup_{i \in I} A_i = \bigcup_{i \in \{ j \in I|A_j \neq
    \varnothing \}} A_i$
    
    \item If $\exists i \in I$ such that $A_i = \varnothing$ then $\bigcap_{i
    \in I} A_i = \varnothing$
  \end{enumerate}
\end{theorem}

\begin{proof}
  
  \begin{enumerate}
    \item  If $x \in \left( \bigcup_{i \in I\backslash \{ j \}} A_i \right)
    \bigcup A_j$ then either $x \in A_j \subseteq \bigcup_{i \in I} A_i$ [see:
    \ref{family properties (1)}], so that $x \in \bigcup_{i \in I} A_i$ or $x
    \in \bigcup_{i \in I\backslash \{ j \}} A_i \Rightarrow \exists k \in
    I\backslash \{ j \}$ with $x \in A_k$ which as $k \in I$ proves $x \in
    \bigcup_{i \in I} A_i$. If $x \in \bigcup_{i \in I} A_i$ then $\exists i
    \in I$ such that $x \in A_i$, we have then for $i$ either $i \in
    I\backslash \{ j \}$ so that $x \in \bigcup_{i \in I\backslash \{ j \}}
    A_i$ or $i = j$ giving $x \in A_j$, proving that $x \in \left( \bigcup_{i
    \in I\backslash \{ j \}} A_i \right) \bigcup A_j .$
    
    \item As $\{ j \in I|A_j \neq \varnothing \} \subseteq I$ we have by
    [theorem: \ref{family properties (2)}] that
    \begin{equation}
      \label{eq 2.29.004} \bigcup_{i \in \{ j \in I|A_j \neq \varnothing \}}
      A_i \subseteq \bigcup_{i \in I} A_i
    \end{equation}
    Further if $x \in \bigcup_{i \in I} A_i$ then there exist a $i \in I$ such
    that $x \in A_i$. As $x \in A_i$ we must have that $A_i \neq \varnothing$
    or $i \in \{ j \in I|A_j \neq \varnothing \}$, proving that $x \in
    \bigcup_{i \in \{ j \in I|A_j \neq \varnothing \}} A_i$. So
    \[ \bigcup_{i \in I} A_i \subseteq \bigcup_{i \in \{ j \in I|A_j \neq
       \varnothing \}} A_i \]
    combining this with [eq: \ref{eq 2.29.004}] proves
    \[ \bigcup_{i \in I} A_i \subseteq \bigcup_{i \in \{ j \in I|A_j \neq
       \varnothing \}} A_i \]
    \item Assume that $i \in I$ such that $A_i = \varnothing$. If $x \in
    \bigcap_{j \in I} A_j$ we have $\forall j \in I$ that $x \in A_j$, so for
    sure $x \in A_i$ which contradicts $A_i = \varnothing$. Hence we have that
    $\bigcap_{j \in I} A_j = \varnothing$.
  \end{enumerate}
  
\end{proof}

\begin{theorem}
  \label{family union of family of families}If $\{ A_i \}_{i \in I} \subseteq
  C$ a family and $\forall i \in I$ $\{ B_{i, j} \}_{j \in J} \subseteq C$ a
  family such that $A_i = \bigcup_{j \in J} B_{i, j}$ then
  \[ \bigcup_{i \in I} A_i = \bigcup_{(i, j) \in I \times J} B_{i, j} \]
  in othere words 
\end{theorem}

\begin{proof}
  If $x \in \bigcup_{i \in I} A_i$ then $\exists i \in I$ such that $x \in A_i
  = \bigcup_{j \in J} B_i$, hence $\exists j \in J$ such that $x \in B_{i,
  j}$. So as $(i, j) \in I \times J$ we have that $x \in \bigcup_{(i, j) \in I
  \times J} B_{i, j}$. Further if $x \in \bigcup_{(i, j) \in I \times J} B_{i,
  j}$ then $\exists (i, j) \in I \times J$ such that $x \in B_{i, j}$, which,
  as $A_i = \bigcup_{j \in J} B_{i, j}$, proves that $x \in A_i$, hence $x \in
  \bigcup_{i \in I} A_i$. So we conclude that
  \[ \bigcup_{i \in I} A_i = \bigcup_{(i, j) \in I \times J} B_{i, j} \]
  
\end{proof}

\begin{theorem}
  \label{family image and preimage}If $f : A \rightarrow B$ is a function, $\{
  A_i \}_{i \in} \subseteq \mathcal{P} (A)$ and $\{ B_i \}_{i \in I} \subseteq
  \mathcal{P} (B)$ are families of sub-classes of $A$ and $B$ then
  \begin{enumerate}
    \item $f \left( \bigcup_{i \in I} A_i \right) = \bigcup_{i \in I} f (A_i)$
    
    \item $f^{- 1} \left( \bigcup_{i \in I} B_i \right) = \bigcup_{i \in I}
    f^{- 1} (B_i)$
    
    \item $f \left( \bigcap A_{i \in I} \right) \subseteq \bigcap_{i \in I} f
    (A_i)$
    
    \item If $f$ is injective and $I \neq \varnothing$ then $f \left(
    \bigcap_{i \in I} A_i \right) = \bigcap_{i \in I} f (A_i)$
    
    \item $f^{- 1} \left( \bigcap_{i \in I} B_i \right) = \bigcup_{i \in I}
    f^{- 1} (B_i)$
  \end{enumerate}
\end{theorem}

\begin{proof}
  
  \begin{enumerate}
    \item If $y \in f \left( \bigcup_{i \in I} A_i \right)$ then $\exists x
    \in \bigcup_{i \in I} A_i$ such that $(x, y) \in f$, hence $\exists i \in
    I$ such that $x \in A_i$, which as $(x, y) \in f$ proves that $y \in f
    (A_i)$. So $y \in \bigcup_{i \in I} f (A_i)$ giving
    \begin{equation}
      \label{eq 2.30.004} f \left( \bigcup_{i \in I} A_i \right) \subseteq
      \bigcup_{i \in I} f (A_i)
    \end{equation}
    If $y \in \bigcup_{i \in I} f (A_i)$ then there exists a $i \in I$ such
    that $y \in f (A_i)$, hence $\exists x \in A_i$ such that $(x, y) \in f$,
    as $x \in A_i$ this implies $x \in \bigcup_{i \in I} A_i$, so we have that
    $y \in f \left( \bigcup_{i \in I} A_i \right)$. Hence $\bigcup_{i \in I} f
    (A_i) \subseteq f \left( \bigcup_{i \in I} A_i \right)$, which combined
    with [eq: \ref{eq 2.30.004}] gives
    \[ f \left( \bigcup_{i \in I} A_i \right) = \bigcup_{i \in I} f (A_i) \]
    \item If $x \in f^{- 1} \left( \bigcup_{i \in I} B_i \right)$ then there
    exists a $y \in \bigcup_{i \in I} B_i$ such that $(x, y) \in f$, hence
    $\exists i \in I$ such that $y \in B_i$. So $x \in f^{- 1} (B_i)$ which as
    $i \in I$ implies that $x \in \bigcup_{i \in I} f^{- 1} (B_i)$ or
    \begin{equation}
      \label{eq 2.31.004} f^{- 1} \left( \bigcup_{i \in I} B_i \right)
      \subseteq \bigcup_{i \in I} f^{- 1} (A_i)
    \end{equation}
    If $x \in \bigcup_{i \in I} f^{- 1} (A_i)$ then there exists a $i \in I$
    such that $x \in f^{- 1} (A_i)$, so $\exists y \in A_i$ with $(x, y) \in
    f$. As from $y \in A_i$ we have $y \in \bigcup_{i \in I}$ it follows that
    $x \in f^{- 1} \left( \bigcup_{i \in I} A_i \right)$. This proves that
    $\bigcup_{i \in I} f^{- 1} (A_i) \subseteq f^{- 1} \left( \bigcup_{i \in
    I} A_i \right)$ which combined with [eq: \ref{eq 2.31.004}] gives
    \[ f^{- 1} \left( \bigcup_{i \in I} B_i \right) = \bigcup_{i \in I} f^{-
       1} (B_i) \]
    \item If $y \in f \left( \bigcap_{i \in I} A_i \right)$ then there exists
    a $x \in \bigcap_{i \in I} A_i$ such that $(x, y) \in f$. From $x \in
    \bigcap_{i \in I} A_i$ it follows that $\forall i \in I$ $x \in A_i$,
    which as $(x, y) \in f$ proves that $\forall i \in I$ $x \in f (A_i)$ or
    $x \in \bigcap_{i \in I} f (A_i)$. So
    \[ f \left( \bigcap_{i \in I} A_i \right) \subseteq \bigcap_{i \in I} f
       (A_i) \]
    \item Let $y \in \bigcap_{i \in I} f (A_i)$ then $\forall i \in I$ we have
    $y \in f (A_i)$. As $I \neq \varnothing$ there exists a $i \in I$ and we
    must thus have that $y \in f (A_i)$. So there exists a $x \in A_i$ such
    that $(x, y) \in f$. Assume that $x \nin \bigcap_{i \in I} A_i$ then
    $\exists j \in I$ such that $x \nin A_j$. However as $j \in I$ we must
    have that $y \in f (A_j)$, so there exists a $x' \in A_j$ such that $(x',
    y) \in f$. As $f$ is injective and $(x, y), (x', y) \in f$ we must have $x
    = x'$, but this means that $x \in A_j$ contradicting $x \nin A_j$. So the
    assumption that $x \nin \bigcap_{i \in I} A_i$ is wrong, hence $x \in
    \bigcap A_i$. As $(x, y) \in f$ we have $y \in f \left( \bigcap_{i \in I}
    A_i \right)$, proving that $\bigcap_{i \in I} f (A_i) \subseteq f \left(
    \bigcap_{i \in I} A_i \right)$, which combined with (3) proves
    \[ f \left( \bigcap_{i \in I} A_i \right) = \bigcap_{i \in I} f (A_i) \]
    \item If $x \in f^{- 1} \left( \bigcap_{i \in I} B_i \right)$ then there
    exists a $y \in \bigcap_{i \in I} B_i$ such that $(x, y) \in f$. Hence
    $\forall i \in I$ we have that $y \in B_i \Rightarrowlim_{(x, y) \in f} x
    \in f^{- 1} (B_i)$ proving that $x \in \bigcap_{i \in I} f^{- 1} B_i$. So
    \begin{equation}
      \label{eq 2.32.004} f^{- 1} \left( \bigcap_{i \in I} B_i \right)
      \subseteq \bigcap_{i \in I} f^{- 1} (B_i)
    \end{equation}
    If $x \in \bigcap_{i \in I} f^{- 1} (B_i)$ then $\forall i \in I$ we have
    $x \in f^{- 1} (B_i)$ or $\exists y \in B_i \text{ with } (x, y) \in f$.
    So $y \in \bigcap_{i \in I} B_i$ which as $(x, y) \in f$ proves that $x
    \in f^{- 1} \left( \bigcap_{i \in I} B_i \right)$. So $\bigcap_{i \in I}
    f^{- 1} (B_i) \subseteq f^{- 1} \left( \bigcap_{i \in I} B_i \right)$
    which combined with \ref{eq 2.32.004} gives
    \[ f^{- 1} \left( \bigcap_{i \in I} B_i \right) = \bigcup_{i \in I} f^{-
       1} (B_i) \]
  \end{enumerate}
\end{proof}

\section{Product of a family of sets}

The Cartesian product $A \times B$ consists of all the possible pairs that you
can form, where the first element is a element of $A$ and the second element
is a element of $B$. We want now to construct a generalized product of a
family of classes consisting of tuples whose elements are indexed by the index
of the family.

\begin{definition}[Product of a family of sets]
  \label{product}{\index{$\prod_{i \in I} A_i$}}Let $\{ A_i \}_{i \in I}
  \subseteq B$ a family then the \tmtextbf{product of $\{ A_i \}_{i \in I}$}
  noted as $\tmmathbf{\prod_{i \in I} A_i}$ is defined by
  \[ \prod_{i \in I} A_i = \left\{ f : f \in \left( \bigcup_{i \in I} A_i
     \right)^I \text{ where } \forall i \in I \text{ we have } f (i) \in A_i
     \right\} \]
  If $x \in \prod_{i \in I} A_i$ then $x_i$ is defined as
  \[ x_i = x (i) \]
  Here $\left( \bigcup_{i \in I} A_i \right)^I$ is the class of function
  graphs of functions between $I$ and $\bigcup_{i \in I} A_i$ [definition:
  \ref{function B^A}] and $f (i)$ is the unique $y$ such that $(i, y) \in f$.
  So $\prod_{i \in I} A_i$ is the class of graphs of functions from $I$ to
  $\bigcup_{i \in I} A_i$ such that $\forall i \in I$ $f_i = f (i) \in A_i$.
\end{definition}

\begin{theorem}
  \label{product of a empty set is empty}Let $\{ A_i \}_{i \in I} \subseteq B$
  be such that $\exists i_0 \in I$ with $A_{i_0} = \varnothing$ then $\prod_{i
  \in I} A_i = \varnothing$
\end{theorem}

\begin{proof}
  Let $x \in \prod_{i \in I} A_i$ then we have that $x : I \rightarrow
  \bigcup_i A_i$ is a function such that $x (i_0) \in A_{i_0}$, contradicting
  $A_{i_0} = \varnothing$. Hence we must have that $\prod_{i \in I} A_i =
  \varnothing$.
\end{proof}

The following shows that the product of a family of only one class is `almost`
that class itself.

\begin{example}
  \label{product of family with one element}Let $\{ A_i \}_{i \in \{ 0 \}}
  \subseteq \{ B \}$ be the family in [example: \ref{family trivial}] defined
  by $A : \{ 0 \} \rightarrow \{ B \}$ where $A = \{ (0, B) \}$ then there
  exists a bijection between $B$ and $\bigsqcap_{i \in \{ 0 \}} A_i$ or as
  $A_0 = B$ there exists a bijection between $A_0$ and $\prod_{i \in \{ 0 \}}
  A_i$.
\end{example}

\begin{proof}
  First using [example: \ref{family trivial}] we have
  \begin{equation}
    \label{eq 2.34.005} B = \bigcup_{i \in \{ 0 \}} A_i
  \end{equation}
  hence
  \begin{equation}
    \label{eq 2.35.005} \left( \bigcup_{i \in \{ 0 \}} A_i \right)^{\{ 0 \}} =
    B^{\{ 0 \}}
  \end{equation}
  Let $f \in B^{\{ 0 \}} \equallim_{\text{[eq: \ref{eq 2.35.005}]}} \left(
  \bigcup_{i \in \{ 0 \}} A_i \right)$ then if $i \in \{ 0 \}$ we must have $i
  = 1$ hence $f (i) = f (0) \in B = A (0) = A_0$ proving that $\forall i \in
  \{ 0 \}$ we have $f (i) \in A_i$. Hence $f \in \prod_{i \in \{ 0 \}} A_i$
  from which it follows that $B^{\{ 0 \}} \subseteq \prod_{i \in \{ 0 \}}
  A_i$. As clearly $\prod_{i \in \{ 0 \}} A_i \subseteq \left( \bigcup_{i \in
  \{ 0 \}} A_i \right)^{\{ 0 \}} \equallim_{\text{[eq: \ref{eq 2.35.005}]}}
  B^{\{ 0 \}}$ we have that
  \[ \prod_{i \in \{ 0 \}} A_i = B^{\{ 0 \}} \]
  Now by [theorem: \ref{function and power}] there exists a bijection between
  $B$ and $B^{\{ 0 \}}$ which by the above proves the example.
\end{proof}

The next theorem shows that the product of a family of two classes is `almost`
the Cartesian product of these classes.

\begin{theorem}
  \label{product of family with two classes}Let $\{ A_i \}_{i \in \{ 0, 1 \}}
  \subseteq \{ C, D \}$ be the family in [example: \ref{family union{A,B}}]
  defined by $A : \{ 0, 1 \} \rightarrow \{ C, D \} \tmop{where}$ $A = \{ (0,
  C), (1, D) \}$ then there exists a bijection between $A \times B$ and
  $\prod_{i \in \{ 0, 1 \}} A_i$
\end{theorem}

\begin{proof}
  First using [example: \ref{family union{A,B}}]: we have that
  \begin{equation}
    \label{eq 2.38.006} \bigcup_{i \in \{ 0, 1 \}} A_i = C \bigcup D
  \end{equation}
  so that
  \begin{equation}
    \label{eq 2.39.006} \left( \bigcup_{i \in \{ 0, 1 \}} A_i \right)^{\{ 0, 1
    \}} = \left( C \bigcup D \right)^{\{ 0, 1 \}}
  \end{equation}
  So
  \begin{equation}
    \label{eq 2.40.007} \prod_{i \in \{ 0, 1 \}} A_i = \left\{ f|f \in \left(
    C \bigcup D \right)^{\{ 0, 1 \}} \text{ where $f (0) \in C \wedge f (1)
    \in D$} \right\}
  \end{equation}
  Given $(c, d) \in C \times D \Rightarrow c \in C \wedge d \in D$, define
  $f_{c, d} = \{ (0, c), (1, d) \}$. If $(x, y) \in f_{c, d}$ we have either
  \[ (x, y) = (0, c) \Rightarrow x = 0 \in \{ 0, 1 \} \wedge y = c \in C
     \subseteq C \bigcup D \Rightarrow (x, y) \in \{ 0, 1 \} \times \left( C
     \bigcup D \right) \]
  or
  \[ (x, y) = (1, d) \Rightarrow x = 1 \in \{ 0, 1 \} \wedge y = d \in D
     \subseteq C \bigcup D \Rightarrow (x, y) \in \{ 0, 1 \} \times \left( C
     \bigcup D \right) \]
  proving that
  \begin{equation}
    \label{eq 2.40.006} f_{a, b} \subseteq \{ 0, 1 \} \times \left( C \bigcup
    D \right) \wedge f_{a, b} (0) \in C \wedge f_{a, b} (1) \in D
  \end{equation}
  If $(x, y), (x, y') \in f_{c, d}$ then either
  \[ (x, y) = (0, c) \Rightarrow x = 0 \Rightarrow (0, y') \in f_{c, d}
     \Rightarrow (0, y') = (0, c) \Rightarrow y' = c \Rightarrow y = y' \]
  or
  \[ (x, y) = (1, d) \Rightarrow x = 1 \Rightarrow (1, y') \in f_{c, d}
     \Rightarrow (1, y') = (1, d) \Rightarrow y' = d \Rightarrow y = y' . \]
  Together with [eq: \ref{eq 2.40.006}] this proves that
  \begin{equation}
    \label{eq 2.41.006} f_{a, b} : \{ 0, 1 \} \rightarrow C \bigcup D \text{
    is a partial function}
  \end{equation}
  If $x \in \{ 0, 1 \}$ then either $x = 0 \Rightarrow (0, c) \in f_{c, d}$ or
  $x = 1 \Rightarrow (1, d) \in f_{c, d}$ proving that $\{ 0, 1 \} \subseteq
  \tmop{dom} (f_{c, d})$ which by [theorem: \ref{function condition (1)}]
  proves that
  \begin{equation}
    \label{eq 2.42.006} f_{c, d} : \{ 0, 1 \} \rightarrow C \bigcup D \text{
    is a function}
  \end{equation}
  As by [eq: \ref{eq 2.40.006}] $f_{c, d} (0) \in C \wedge f_{c, d} (1) \in D$
  proving that
  \begin{equation}
    \label{eq 2.44.007} f_{c, d} \in \prod_{i \in \{ 0, 1 \}} A_i
  \end{equation}
  Define now $\gamma$ by $\gamma = \{ ((c, d), f_{c, d}) | (c, d) \in C \times
  D \}$. If $(x, y) \in \gamma$ then $x = (c, d) \in C \times D$ and $y =
  f_{c, d} \Rightarrowlim_{\text{[eq: \ref{eq 2.44.007}]}}$, hence $y \in
  \left( C \bigcup D \right)^{\{ 0, 1 \}}$. This proves that $(x, y) \in (C
  \times D) \times \left( \prod_{i \in \{ 0, 1 \}} A_i \right)$ or
  \begin{equation}
    \label{eq 2.43.006} \gamma \subseteq (C \times D) \times \left( \prod_{i
    \in \{ 0, 1 \}} A_i \right)
  \end{equation}
  If $(x, y), (x, y') \in \gamma$ then $\exists (c, d) \in C \times D$ such
  that $(x, y) = ((c, d), f_{c, d})$ and $(x, y') = ((c, d), f_{c, d})$ so
  that $y = f_{c, d} = y'$ hence $y = y'$. Combining this with [eq:\ref{eq
  2.43.006}] proves that
  \begin{equation}
    \label{eq 2.44.006} \gamma : C \times D \rightarrow \left( \prod_{i \in \{
    0, 1 \}} A_i \right) \text{ is a partial function}
  \end{equation}
  If $(c, d) \in C \times D$ then by definition of $\gamma$ we have $((c, d),
  f_{c, d}) \in \gamma$ so that $(c, d) \in \tmop{dom} (\gamma)$ proving that
  $C \times D \subseteq \tmop{dom} (\gamma)$. By [theorem: \ref{function
  condition (1)}] and [eq: \ref{eq 2.44.006}] we have
  \begin{equation}
    \label{eq 2.45.006} \gamma : C \times D \rightarrow \left( \prod_{i \in \{
    0, 1 \}} A_i \right) \text{ is a function}
  \end{equation}
  If $(x, y), (x', y) \in \gamma$ then there exists $(c, d), (c', d') \in C
  \times D$ such that $x = (c, d) \wedge x' = (c', d')$ and $f_{c, d} = y =
  f_{c', d'}$. As $(0, c) \in f_{c, d} = f_{c', d'}$ we have $(0, c) = (0,
  c')$ giving $c = c'$ and from $(1, d) \in f_{c, d} = f_{c', d'}$ we have
  $(1, d) = (1, d') \tmop{giving}$ $d = d'$. So $(c, d) = (c', d')$ proving
  that
  \[ \gamma : C \times D \rightarrow \left( \prod_{i \in \{ 0, 1 \}} A_i
     \right) \text{ is a injection} \]
  If $g \in \prod_{i \in \{ 0, 1 \}} A_i$ then $g : \{ 0, 1 \} \rightarrow C
  \bigcup D$ is a function and $g (0) \in C \wedge g (1) \in D$ So there
  exists a $c \in C$ such that $(0, c) \in g$ and there exists a $d \in D$
  such that $(1, d) \in g$. So $g = \{ (0, c), (1, d) \} = f_{c, d}$ which
  proves that
  \[ \gamma : C \times D \rightarrow \left( \prod_{i \in \{ 0, 1 \}} A_i
     \right) \text{ is a surjection} \]
\end{proof}

\begin{theorem}
  \label{product inclusion}Let $\{ A_i \}_{i \in I} \subseteq A$ and $\{ B_i
  \}_{i \in I} \subseteq B$ classes such that $\forall i \in I$ $A_i \subseteq
  B_i$ then
  \[ \prod_{i \in I} A_i \subseteq \prod_{i \in I} B_i \]
\end{theorem}

\begin{proof}
  Let $x \in \prod_{i \in I} A_i$ then $x \in \left( \bigcup_{i \in I} A_i
  \right)^I$ and $\forall i \in I$ $x (i) \in A_i$. Using [theorem:
  \ref{family union intersection and inclusion}] it follows that $\bigcup_{i
  \in I} A_i \subseteq \bigcup_{i \in I} B_i$, applying [theorem:
  \ref{function B^A and inclusion}] proves then $\left( \bigcup_{i \in I} A_i
  \right)^I \subseteq \left( \bigcup_{i \in I} B_i \right)^I$, so that
  \[ x \in \left( \bigcup_{i \in I} B_i \right)^I \]
  If $i \in I$ then $x (i) \in A_i$, which as $A_i \subseteq B_i$ gives $x (i)
  \in B_i$, combining this with the above proves that $x \in \prod_{i \in I}
  B_i$. Hence we have
  \[ \prod_{i \in I} A_i \subseteq \prod_{i \in I} B_i \]
\end{proof}

\begin{theorem}
  \label{product and intersection}Let $\{ A_i \}_{i \in I} \subseteq C$ and
  $\{ B_i \}_{i \in I} \subseteq D$ are two families then
  \[ \left( \prod_{i \in I} A_i \right) \bigcap \left( \prod_{i \in I} B_i
     \right) = \prod_{i \in I} \left( A_i \bigcap B_i \right) \]
\end{theorem}

\begin{proof}
  First, as $\forall i \in I$ we have by [theorem: \ref{class intersection,
  union, inclusion}] $A_i \bigcap B_i \subseteq A_i$ and $A_i \bigcap B_i
  \subseteq B_i$ it follows that by [theorem: \ref{product inclusion}]
  \[ \prod_{i \in I} \left( A_i \bigcap B_i \right) \subseteq \prod_{i \in I}
     A_i \text{ and } \prod_{i \in I} \left( A_i \bigcap B_i \right) \subseteq
     \prod_{i \in I} B_i \]
  so that by [theorem: \ref{class intersection, union, inclusion}]
  \begin{equation}
    \label{eq 2.48.008} \prod_{i \in I} \left( A_i \bigcap B_i \right)
    \subseteq \left( \prod_{i \in I} A_i \right) \bigcap \left( \bigcup_{i \in
    I} B_i \right)
  \end{equation}
  Now for the opposite inclusion Let $x \in \left( \prod_{i \in I} A_i \right)
  \bigcap \left( \prod_{i \in I} B_i \right)$ then $x \in \prod_{i \in I} A_i$
  and $x \in \prod_{i \in I} B_i$. So $x \in \left( \bigcup_{i \in I} A_i
  \right)^I \wedge \forall i \in I \vDash x (i) \in A_i$ and $x \in \left(
  \bigcup_{i \in I} B_i \right)^I \wedge \forall i \in I \vDash x (i) \in
  B_i$. Hence
  \begin{eqnarray*}
    & x : I \rightarrow \bigcup_{i \in I} A_i & \text{ is a function}\\
    & x : I \rightarrow \bigcup_{i \in I} B_i & \tmop{is} a \tmop{function}\\
    & \forall i \in I & \tmop{we} \tmop{have} x (i) \in A_i \bigcap B_i
  \end{eqnarray*}
  Now if $(i, y) \in x$ we have $i \in I$ [as $x \subseteq I \times \left(
  \bigcup_i A_i \right)$] and $y = x (i) \in A_i \bigcap B_i \subseteq
  \bigcup_{i \in I} \left( A_i \bigcap B_i \right)$ so that $(i, y) \in I
  \times \left( \bigcup_{i \in I} \left( A_i \bigcap B_i \right) \right)$
  giving
  \begin{equation}
    \label{eq 2.51.010} x \subseteq I \times \left( \bigcup_{i \in I} \left(
    A_i \bigcap B_i \right) \right) \text{ and } \forall i \in I \text{ we
    have $x (i) \in A_i \bigcap B_i$}
  \end{equation}
  Further as $x : I \rightarrow \bigcup_{i \in I} A_i$ is a function we have
  $\forall (i, y), (i, y')$ that $y = y'$ and that $\tmop{dom} (x) = I$.
  Combining this with [eq: \ref{eq 2.51.010}] proves that $f : I \rightarrow
  \bigcup_{i \in I} (A_i \times B_i)$ is a function and $\forall i \in I$ we
  have $x (i) \in A_i \bigcap B_i$. This proves that $x \in \prod_{i \in I}
  \left( A_i \bigcap B_i \right)$ giving $\left( \prod_{i \in I} A_i \right)
  \bigcap \left( \prod_{i \in I} B_i \right) \subseteq \prod_{i \in I} \left(
  A_i \bigcap B_i \right)$ which combined with \ref{eq 2.48.008} gives finally
  \[ \prod_{i \in I} \left( A_i \bigcap B_i \right) \subseteq \left( \prod_{i
     \in I} A_i \right) \bigcap \left( \bigcup_{i \in I} B_i \right) \]
\end{proof}

We can easily generalize the above theorem.

\begin{theorem}
  \label{product intersection of a product}Let $I$, $J$ be sets and $\{ \{
  A_{i, j} \}_{j \in J} \}_{i \in I}$ be a family of sets then
  \[ \bigcap_{i \in I} \left( \prod_{j \in J} A_{i, j} \right) = \prod_{j \in
     J} \left( \bigcap_{i \in I} A_{i, j} \right) \]
\end{theorem}

\begin{proof}
  As $\forall i \in I, j \in I$ we have $\bigcap_{i \in I} A_{i, j} \subseteq
  A_{i, j}$ it follows from [theorem: \ref{product inclusion}] that $\prod_{j
  \in J} \left( \bigcap_{i \in I} A_{i, j} \right) \subseteq \prod_{j \in J}
  A_{i, j}$. Hence by [theorem: \ref{family properties (1)}]
  \begin{equation}
    \label{eq 2.57.146} \prod_{j \in J} \left( \bigcap_{i \in I} A_{i, j}
    \right) \subseteq \bigcap_{i \in I} \left( \prod_{j \in J} A_{i, j}
    \right)
  \end{equation}
  If $x \in \bigcap_{i \in I} \left( \prod_{j \in J} A_{i, j} \right)$ then
  $\forall i \in I$ $x \in \prod_{j \in J} A_{i, j}$ hence $x : J \rightarrow
  A_{i, j}$ is a function such that $\forall j \in J$ we have $x (j) \in A_{i,
  j}$. Hence $x (j) \in \bigcap_{i \in I} A_{i, j}$ $\forall j \in J$ so that
  $x \in \prod_{j \in J} \left( \bigcap_{i \in I} A_{i, j} \right)$, so
  \[ \bigcap_{i \in I} \left( \prod_{j \in J} A_{i, j} \right) \subseteq
     \prod_{j \in J} \left( \bigcap_{i \in I} A_{i, j} \right) \]
  which combined with [eq: \ref{eq 2.57.146}] gives
  \[ \bigcap_{i \in I} \left( \prod_{j \in J} A_{i, j} \right) = \prod_{j \in
     J} \left( \bigcap_{i \in I} A_{i, j} \right) \]
  
\end{proof}

\

The following theorem is a motivation for the notation $A^B$ for the graphs
of functions from $B$ to $A$.

\begin{theorem}
  \label{product and power}Let $I, B$ be classes and consider the family $\{
  A_i \}_{i \in I} \subseteq \{ B \}$ based on the constant function $A : I
  \rightarrow \{ B \}$ where $A = C_B = I \times \{ B \}$ so that $\forall i
  \in I$ $A (i) = B$ [see example: \ref{function constant function}] then
  $\prod_{i \in I} A_i = A^I$
\end{theorem}

\begin{proof}
  For $I$ we have the following cases to consider:
  \begin{description}
    \item[$I = \varnothing$] Using [example: \ref{function A^empty is empty}]
    we have that
    \[ \left( \bigcup_{i \in \varnothing} A_i \right)^{\varnothing} = \{
       \varnothing \} \]
    Further $\forall i \in \varnothing$ we have $\varnothing (i) \in A_i$ is
    satisfied vacuously proving that $\varnothing \in \prod_{i \in
    \varnothing} A_i$ so that $\{ \varnothing \} \subseteq \prod_{i \in
    \varnothing} A_i \subseteq \left( \bigcup_{i \in \varnothing} A_i
    \right)^{\varnothing} = \{ \varnothing \}$ or taking $I = \varnothing$
    \[ \prod_{i \in I} A_i = A^I \]
    \item[$I \neq \varnothing$] If $y \in \tmop{range} (A)$ then $\exists x$
    such that $(x, y) \in C_B = I \times \{ B \}$, so that $y \in \{ B \}$.
    Hence
    \begin{equation}
      \label{eq 2.48.007} \tmop{range} (A) \subseteq \{ B \}
    \end{equation}
    As $I \neq \varnothing$ there exists a $i \in I$, which by the definition
    of $C_B$ means that $(i, B) \in C_B$, hence $B \in \tmop{range} (A)$. So
    if $y \in \{ B \}$ then $y = B \in \tmop{range} (A)$ proving that $\{ B \}
    \subseteq \tmop{range} (A)$ which combined with [eq: \ref{eq 2.48.007}]
    gives
    \[ \tmop{range} (A) = \{ B \} \]
    hence
    \[ \bigcup_{i \in I} A_i = \bigcup (\tmop{range} (A)) = \bigcup \{ B \}
       \equallim_{\text{[example: \ref{class trivial union intersection}]}} B
    \]
    so that
    \begin{equation}
      \label{eq 2.49.007} \left( \bigcup_{i \in I} A_i \right)^I = B^I
    \end{equation}
    Now if $f \in B^I$ then $\forall i \in I$ we have $f (i) \in B = A (i) =
    A_i$ proving that
    \[ f \in \{ f|f \in B^i \wedge \forall i \in I f (i) \in A_i \}
       \equallim_{\text{[eq: \ref{eq 2.49.007}]}} \left\{ f|f \in \left(
       \prod_{i \in I} A_i \right)^I \wedge \forall i \in I f (i) \in A_i
       \right\} = \prod_{i \in I} A_i \]
    proving that
    \begin{equation}
      \label{eq 2.50.007} B^I \subseteq \prod_{i \in I} A_i
    \end{equation}
    Further
    \[ \prod_{i \in I} A_i = \left\{ f|f \in \left( \prod_{i \in I} A_i
       \right)^I \wedge \forall i \in I f (i) \in A_i \right\} \subseteq
       \left\{ f|f \in \left( \prod_{i \in I} A_i \right)^I \right\} = \{ f|f
       \in B^I \} = B^I \]
    which combined with [eq: \ref{eq 2.50.007}] proves that
    \[ B^I = \prod_{i \in I} A_i \]
    
  \end{description}
\end{proof}

\begin{theorem}
  \label{family product and index transformation}Let $I, J, B$ be classes, $f
  : I \rightarrow J$ a bijection and $\{ A_j \}_{j \in J}$ then
  \[ \beta : \prod_{j \in J} A_j \rightarrow \prod_{i \in I} A_{f (i)} \text{
     where } \beta (x) = x \circ f \]
  is a bijection.
\end{theorem}

\begin{proof}
  First as $f : I \rightarrow J$ is a bijection, hence surjective, we have by
  [theorem: \ref{family union of family set and surjections}] that
  \begin{equation}
    \label{eq 2.59.022} \bigcup_{j \in J} A_j = \bigcup_{i \in I} A_{f (i)}
  \end{equation}
  Let $x \in \prod_{j \in J} A_j$ then $x \in \left( \bigcup_{j \in J} A_j
  \right)^J$, which is equivalent with $x : J \rightarrow \bigcup_{j \in J}
  A_j$ is a function, and $\forall j \in J$ we have $x (j) \in A_j$. So $x
  \circ f : I \rightarrow \bigcup_{j \in J} A_j \equallim_{\text{[eq: \ref{eq
  2.59.022}]}} \bigcup_{i \in I} A_{f (i)}$ is a function, proving that $x
  \circ f \in \left( \bigcup_{i \in I} A_{f (i)} \right)^I$, further if $i \in
  I$ then $(x \circ f) (i) = x (f (i)) \in A_{f (i)}$, hence
  \begin{equation}
    \label{eq 2.60.022} x \circ f \in \prod_{i \in I} A_{f (i)}
  \end{equation}
  So
  \[ \beta : \prod_{j \in J} A_j \rightarrow \prod_{i \in I} A_{f (i)}
     \text{} \]
  is indeed a function. To prove that it is a bijection note:
  \begin{description}
    \item[injectivity] Assume that $\beta (x) = \beta (y)$ then
    \begin{eqnarray*}
      x \circ f = y \circ f & \Rightarrowlim_{f \text{ is bijective}} & (x
      \circ f) \circ f^{- 1} = (y \circ f) \circ f^{- 1}\\
      & \Rightarrow & x \circ (f \circ f^{- 1}) = y \circ (f \circ f^{- 1})\\
      & \Rightarrow & x \circ \tmop{Id}_J = y \circ \tmop{Id}_J\\
      & \Rightarrow & x = y
    \end{eqnarray*}
    \item[surjectivity] If $y \in \prod_{i \in I} A_{f (i)}$ then $y : I
    \rightarrow \bigcup_{i \in I} A_{f (i)} \equallim_{\text{[eq: \ref{eq
    2.59.022}]}} \bigcup_{j \in J} A_j$ is a function and $\forall i \in I$ we
    have $y (i) \in A_{f (i)}$. As $f^{- 1} : J \rightarrow I$ is a bijection
    we have that $y \circ f^{- 1} : J \rightarrow \bigcup_{j \in J} A_j$ is a
    function, so that $y \circ f^{- 1} \in \left( \bigcup_{j \in J} A_j
    \right)^J$, and $(y \circ f^{- 1})  (j) = y (f^{- 1} (j)) \in A_{f (f^{-
    1} (j))} = A_j$. So that
    \[ y \circ f^{- 1} \in \prod_{j \in J} A_j \]
    Finally $\beta (y \circ f^{- 1}) = (y \circ f^{- 1}) \circ f = y \circ
    (f^{- 1} \circ f) = y \circ \tmop{Id}_I = y$ proving surjectivity.
  \end{description}
\end{proof}

\begin{definition}
  \label{product sub-product}Let $\{ A_i \}_{i \in I} \subseteq B$ be a family
  and $J \subseteq I$ then $\prod_{i \in J} A_i$ is the product based on the
  sub-family $\{ A_i \}_{i \in J} \subseteq B$ [see definition: \ref{family
  definition (2)}] or equivalently
  \[ \prod_{i \in J} A_i = \left\{ f : f \in \left( \bigcup_{i \in J} A_i
     \right)^J \text{ where } \forall i \in J \text{ we have } f (i) \in A_i
     \right\} \]
\end{definition}

The following theorem will be used later in induction arguments.

\begin{theorem}
  \label{product extension}Let $\{ A_i \}_{i \in I} \subseteq B$, $i \in I$
  and $b \in A_i$ then
  \[ \tmop{if} x \in \prod_{j \in I\backslash \{ i \}} A_j \text{ we have $y
     \in \prod_{i \in I} A_j$}  \]
  where $y$ is defined by
  \[ y_j = y (j) = \left\{\begin{array}{l}
       b \text{ if } j = i\\
       x_j \text{ if $j \in I\backslash \{ i \}$}
     \end{array}\right. \equallim_{\tmop{def}} \left\{\begin{array}{l}
       b \text{ if } j = i\\
       x (j) \text{ if $j \in I\backslash \{ i \}$}
     \end{array}\right. \]
  
\end{theorem}

\begin{proof}
  If $x \in \prod_{j \in I\backslash \{ i \}} A_j$ then $x \in \left(
  \bigcup_{j \in I\backslash \{ i \}} A_i \right)^{I\backslash \{ i \}}$ so
  that $x : I\backslash \{ i \} \rightarrow \bigcup_{j \in I\backslash \{ i
  \}} A_j$ is a function. As $i \nin (I\backslash \{ i \})$, $I = (I\backslash
  \{ i \}) \bigcup \{ i \}$ and $\bigcup_{j \in I} A_j
  \equallim_{\text{[theorem: \ref{family union intersection and empty set}]}}
  A_i \bigcup \left( \bigcup_{j \in I\backslash \{ i \}} A_j \right)$ we have
  by [theorem: \ref{function extending funtion domain}] that
  \[ y : I \rightarrow \bigcup_{i \in I} A_i \text{ where } y (j) =
     \left\{\begin{array}{l}
       b \text{ if } j = i\\
       x (j) \text{ if } j \in I\backslash \{ i \}
     \end{array}\right. \]
  is a function, so
  \begin{equation}
    \label{eq 2.59.018} y \in \left( \bigcup_{i \in I} A_i \right)^I
  \end{equation}
  Further if $j \in I$ then either $j = i$ so that $y_j = y (i) = b \in A_i =
  A_j$ or $j \in I\backslash \{ i \}$ then $y_j = y (j) = x (j) = x_j \in
  A_j$. Hence
  \begin{equation}
    \label{eq 2.60.018} \forall j \in I \text{ we have } y_j \in A_j
  \end{equation}
  From [eq: \ref{eq 2.59.018}] and [eq: \ref{eq 2.60.018}] it follows by
  \[ y \in \prod_{i \in I} A_i \]
  
\end{proof}

We introduce now the projection operator

\begin{definition}
  \label{product projection function}{\index{$\pi_i$}}Let $\{ A_i \}_{i \in I}
  \subseteq B$ be family then for $i \in I$ we define the projection function
  \[ \pi_i : \prod_{j \in I} A_j \rightarrow A_i \]
  where
  \[ \pi_i = \left\{ z|z = (x, x (i)) |x \in \prod_{j \in I} A_j \right\} \]
  In other words $(x, y) \in \pi_i \Leftrightarrow x \in \prod_{j \in I} A_j$
  and $y = x (i) \Leftrightarrow (i, y) \in x$
\end{definition}

\begin{proof}
  This definition only make sense if $\forall i \in I$ that $\pi_i : \prod_{j
  \in I} A_j \rightarrow A_i$ is a function. First if $(x, y) \in \pi_i$ we
  have that $x \in \prod_{j \in I} A_j$ and $y = x (i)$ giving $y \in A_i$, so
  $(x, y) \in \left( \prod_{i \in I} A_i \right) \times A_i$. Hence
  \begin{equation}
    \label{eq 2.51.007} \pi_i \subseteq \left( \prod_{i \in I} A_i \right)
    \times A_i
  \end{equation}
  If $(x, y), (x, y') \in \pi_i$ then $y = x (i) \wedge y' = x (i) $ proving
  that $y = y'$ or
  \[ \pi_i : \prod_{j \in I} A_j \rightarrow A_i \text{ is a partial function}
  \]
  If $x \in \prod_{j \in I} A_j$ then by definition $(x, x (i)) \in \pi_i$
  proving that $x \in \tmop{dom} (\pi_i)$ proving that $\prod_{j \in I} A_i
  \subseteq \tmop{dom} (\pi_i)$, which by [theorem: \ref{function condition
  (1)}] gives
  \[ \pi_i : \prod_{j \in I} A_j \rightarrow A_i \text{ is a function} \]
\end{proof}

We are not yet finished with the product of a family of classes, however for
some of the theorems we need the Axiom of Choice. For example to prove that
the projection function is a surjection we need the Axiom of Choice.

\chapter{Relations}

\section{Relation}

The idea of a relation is that we can specify which elements of a class are
related to each other.\.{}You do this by specifying a class of pairs.

\begin{definition}
  \label{relation}{\index{relation}}Let $A$ be a class then a relation in $A$
  is a sub-class of $A \times A$
\end{definition}

\begin{notation}
  So a relation is a set of pairs from elements of the same class, to avoid
  confusion with the graph of a function we use the following notation:
  
  If $R \subseteq A \times A$ is relation then instead of writing $(x, y) \in
  R$ we write $x R y$
\end{notation}

\begin{example}
  \label{relation trivial}Let $A$ be a class then $A \times A$ is a relation
  [as $A \times A \subseteq A \times A$]
\end{example}

We define now the following properties that a relation can have

\begin{definition}
  \label{relation properties}If $A$ is a class and $R \subseteq A \times A$ a
  relation then we say that $R$ is
  \begin{description}
    \item[reflexive] iff $\forall x \in A$ we have
    \[ x R x \]
    in other words every element is related to itself.
    
    \item[symmetric] iff
    \[ x R y \Rightarrow y R x \]
    in other words if one element is related to a second element then the
    second element is related to the first element.
    
    \item[anti symmetric] iff
    \[ x R y \wedge y R x \Rightarrow x = y \]
    in other words if on element is related to a second element and the second
    element is related to the first element then the two elements are the
    same.
    
    \item[transitive] iff
    \[ x R y \wedge y R z \Rightarrow x R z \]
    in other words if one element is related to a second element and the
    second element is related to the third element then the first element is
    also related to the third element.
  \end{description}
\end{definition}

\section{Equivalence relations}

\subsection{Equivalence relations and equivalence classes}

Note that for classes and equality we have by [theorem: \ref{class properties
(1)}] that
\begin{itemizedot}
  \item $A = A$
  
  \item $A = B \Rightarrow B = A$
  
  \item $A = B \wedge B = C \Rightarrow A = C$
\end{itemizedot}
If we want to create a relation that defines a kind of equality then it must
behave in the same way as the equality for classes. This it he idea behind the
following definition.

\begin{definition}[Equivalence Relation]
  \label{equivalence relation}{\index{equivalence relation}}If $A$ is a class
  then a relation $R$ is a \tmtextbf{equivalence} \tmtextbf{relation} iff it
  is reflexive, symmetric and transitive or in other words if
  \begin{description}
    \item[reflectivity] $\forall x \in A$ $x R x$
    
    \item[$\tmop{symetricity}$] $x R y \Rightarrow y R x$
    
    \item[transitivity] $x R y \wedge y R z \Rightarrow x R z$
  \end{description}
\end{definition}

Given a set $A$ and a equivalence relation in $A$ then it is useful to
partition the set in subsets containing all the elements that are equivalent
with each other. To do this we must first define what a partition of a set is.

\begin{definition}
  \label{equivalence relation partition}{\index{partition of a set}}Let $A$ be
  a set then a \tmtextbf{partition} of $A$ is a family $\{ A_i \}_{i \in I}
  \subseteq \mathcal{P} (A)$ of non empty subsets of $A$ [$\forall i \in I$ we
  have $A_i \neq \varnothing$] such that:
  \begin{enumerate}
    \item $\bigcup_{i \in I} A_i = A$
    
    \item $\forall i, j \in I$ we have $A_i \bigcap A_j = \varnothing \vee A_i
    = A_j$
  \end{enumerate}
\end{definition}

\begin{note}
  Condition (2) in the above definition is a weaker condition that pairwise
  disjointedness. For example if we define the family $(A_i)_{i \in \{ 1, 2, 3
  \}}$ by $A_1 = \{ 1 \}, A_2 = \{ 1 \}$ and $A_3 = \{ 2 \}$ then this family
  is not pairwise disjoint as $1 \neq 2$ and $A_1 \bigcap A_2 \neq
  \varnothing$, however $(2)$ is clearly satisfied.
\end{note}

We can also reformulate the definition of a partition of $A$ in the following
way

\begin{theorem}
  \label{equivalence relation partition alternative}Let $A$ be a set and $\{
  A_i \}_{i \in I} \subseteq \mathcal{P} (A)$ a family of non empty subsets of
  $A$ then we have the following equivalences
  \begin{enumerate}
    \item $\{ A_i \}_{i \in I} \subseteq \mathcal{P} (A)$ is a partition of
    $A$
    
    \item  $\{ A_i \}_{i \in I} \subseteq \mathcal{P} (A)$ satisfies
    \begin{enumerate}
      \item $\forall x \in A$ there exists a $i \in I$ such that $x \in A_i$
      
      \item $\forall i, j \in I$ with $A_i \bigcap A_j \neq \varnothing$ we
      have $A_i = A_j$
    \end{enumerate}
    \item $\{ A_i \}_{i \in I} \subseteq \mathcal{P} (A)$ satisfies
    \begin{enumerate}
      \item $\forall x \in A$ there exists a $i \in I$ such that $x \in A_i$
      
      \item $\forall i, j \in I$ with $A_i \neq A_j$ we have $A_i \bigcap A_j
      = \varnothing$
    \end{enumerate}
  \end{enumerate}
\end{theorem}

\begin{proof}
  
  \begin{description}
    \item[$1 \Rightarrow 2$] 
    \begin{enumeratealpha}
      \item If $x \in A$ then as $A = \bigcup_{i \in I} A_i$ there exists a $i
      \in I$ such that $x \in A_i$
      
      \item Let $i, j \in I$ with $A_i \bigcap A_j \neq \varnothing$. As by
      definition of a partition $A_i \bigcap A_j = \varnothing \vee A_i = A_j$
      we must have that $A_i = A_j$.
    \end{enumeratealpha}
    \item[$2 \Rightarrow 3$] 
    \begin{enumeratealpha}
      \item This is trivial
      
      \item Let $i, j \in I$ with $A_i \neq A_j$. Assume that $A_i \bigcap A_j
      \neq \varnothing$ then by (2.b) we have $A_i = A_j$ contradicting $A_i =
      A_j$, so we must have that $A_i \bigcap A_j = \varnothing$
    \end{enumeratealpha}
    \item[$3 \Rightarrow 1$] 
    \begin{enumeratealpha}
      \item Using (3.a) it follows that $A \subseteq \bigcup_{i \in I} A_i$.
      If $z \in \bigcup_{i \in I} A_i$ then there exists a $i \in I$ such that
      $x \in A_i$ [theorem: \ref{family union (2)}], hence as $A_i \in
      \mathcal{P} (A) \Rightarrow A_i \subseteq A$ it follows that $x \in A$,
      proving that $\bigcup_{i \in I} A_i \subseteq A$. So we have that
      \[ \bigcup_{i \in I} A_i = A \]
      \item Let $i, j \in I$ then if $A_i \neq A_j$ we have by (3b) that $A_i
      \bigcap A_j = \varnothing$, so we have that $A_i = A_j \vee A_i \bigcap
      A_j = \varnothing$.
    \end{enumeratealpha}
  \end{description}
\end{proof}

We show now how a equivalence relation can be used to partition a set.

\begin{definition}
  \label{equivalence relation class}{\index{$R [x]$}}Let $A$ be a set and $R$
  a equivalence relation in $A$ then given $x$ we define the
  \tmtextbf{equivalence} \tmtextbf{class} of $x$ noted by $R [x]$ by
  \[ R [x] = \{ y \in A|x R y \} \subseteq A \]
\end{definition}

\begin{note}
  Because $R [x] \subseteq A$ and $A$ is a set we have by the axiom of subset
  \ref{axiom of subsets} that $R [x]$ is a set.
\end{note}

We have the following important property for equivalence classes

\begin{theorem}
  \label{equivalence relation R[x]=R[y]}Let $A$ be a set with a equivalence
  relation $R$ in $A$ then
  \begin{enumerate}
    \item $\forall x \in A$ we have $x \in R [x]$
    
    \item $\forall x, y \in A$ we have
    \[ x R y \Leftrightarrow R [x] = R [y] \]
    \item $\forall x \in A$ we have
    \[ y \in R [x] \Leftrightarrow R [x] = R [y] \]
  \end{enumerate}
\end{theorem}

\begin{proof}
  
  \begin{enumerate}
    \item If $x \in A$ then using reflexivity we have $x R x$ so that $x \in R
    [x]$
    
    \item 
    \begin{description}
      \item[$\Rightarrow$] Let $z \in R [x]$ then $x R z$, further from $x R
      y$ we have $y R x$, so using transitivity it follows that $y R z$ or $z
      \in R [y]$. If $z \in R [y]$ then $y R z$ so as $x R y$ we have by
      transitivity that $x R z$ or that $z \in R$.
      
      \item[$\Leftarrow$] Using (1) $x \in R [x] \Rightarrowlim_{R [x] = R
      [y]} x \in R [y]$ proving that $r R y$
    \end{description}
    \item 
    \begin{description}
      \item[$\Rightarrow$] If $y \in R [x]$ then $y R x$ hence by (2) $R [x] =
      R [y]$
      
      \item[$\Leftarrow$] If $R [x] = R [y]$ then $y R x$ proving that $y \in
      R [x]$
    \end{description}
  \end{enumerate}
\end{proof}

We define now a function that maps a element of as set on its equivalence
class and use it to define a family of equivalence classes indexed by the
elements of the set.

\begin{definition}
  Let $A$ be a set and $R$ a equivalence relation in $A$ then $\{ R [x] \}_{x
  \in A} \subseteq \mathcal{P} (X)$ is the family defined by the function $R
  [] : A \rightarrow \mathcal{P} (A)$ where $R [] (x) = R [x]$
\end{definition}

\begin{note}
  As $x \in R [x]$ we have that $\{ R [x] \}_{x \in A}$ is a non empty family
  of subsets of $A$
\end{note}

\begin{proof}
  We must of course prove that this a function. First $R [x]$ is defined for
  every $x \in A$ and calculates a unique set, further $R [x] \subseteq A
  \Rightarrow R [x] \in \mathcal{P} (A)$. So by [proposition: \ref{function
  simple definition}] $R [] : A \rightarrow \mathcal{P} [A]$ is a function.
\end{proof}

\begin{theorem}
  \label{equivalence relation defines a partition}Let $A$ be a set and $R$ a
  equivalence relation in $A$ then $\{ R [x] \}_{x \in A}$ is a partition of
  $A$
\end{theorem}

\begin{proof}
  We use [theorem: \ref{equivalence relation partition alternative}] to prove
  this
  \begin{enumerate}
    \item If $x \in A$ then by [theorem: \ref{equivalence relation R[x]=R[y]}]
    we have that $x \in R [x]$ so that $x \in \bigcup_{x \in A} R [x]$
    
    \item Let $x, y \in A$ such that $R [x] \bigcap R [y] \neq \varnothing$
    then there exists a
    \[ z \in R [x] \bigcap R [y] \Rightarrow z R x \wedge z R y
       \Rightarrowlim_{\tmop{symmetry}} x R z \wedge z R y
       \Rightarrowlim_{\tmop{transitivity}} x R y \]
    Using the above together with [theorem: \ref{equivalence relation
    R[x]=R[y]}] we have then that $R [x] = R [y]$
  \end{enumerate}
  So by \ [theorem: \ref{equivalence relation partition alternative}] it
  follows that $\{ R [x] \}_{x \in A} \subseteq \mathcal{P} (A)$ is a
  partition of $A$
\end{proof}

We have also the opposite of the above theorem in that a partition defines a
equivalence relation that generates the same partition.

\begin{theorem}
  Let $A$ be a set and $\{ A_i \}_{i \in I} \subseteq \mathcal{P} (A)$ a
  partition of $A$. Define $R \subseteq A \times A$ by
  \[ R = \left\{ (x, y) | \exists i \in I \text{ such that } x \in A_i \wedge
     y \in A_i \right\} \]
  then we have:
  \begin{enumerate}
    \item $R$ is a equivalence relation
    
    \item $\forall i \in I$ there exists a $x \in A$ such that $R [x] = A_i$
    
    \item $\forall x \in A$ there exists a $i \in I$ such that $R [x] = A_i$
  \end{enumerate}
  we call $R$ is the called the \tmtextbf{equivalence relation associated with
  the partition }$\{ A_i \}_{i \in I} \subseteq \mathcal{P} (A)$
\end{theorem}

\begin{proof}
  
  \begin{enumerate}
    \item We have:
    \begin{enumerate}
      \item If $x \in A = \bigcup_{i \in I} A_i$ then $\exists i \in I$ such
      that $x \in A_i$ so that $(x, x) \in R$ or $x R x$
      
      \item If $x R y$ or $(x, y) \in R$ then $\exists i \in I$ such that $x
      \in A_i \wedge y \in A_i \Rightarrow y \in A_i \wedge x \in A_i$. Hence
      $(y, x) \in R$ or $y R x$.
      
      \item If $x R y \wedge y R z$ then $\exists i \in I$ such that $x, y \in
      A_i$ and $\exists j \in I$ such that $y, z \in A_j$. So $y \in A_i
      \bigcap A_j$ or $A_i \bigcap A_j \neq \varnothing$, by [theorem:
      \ref{equivalence relation partition alternative}] we have that $A_i =
      A_j$, hence $x, z \in A_i$ proving that $(x, z) \in R$ or $x R z$.
    \end{enumerate}
    \item If $i \in I$ then as $A_i \neq \varnothing$ [a partition is a family
    of non empty subsets] there exists a $x \in A_i$. Take $y \in A_i$ then
    $x, y \in A_i$ or $y R x$ proving that $y \in R [x]$. So
    \[ A_i \subseteq R [x] \]
    Take $y \in R [x]$ then $y R x$ so there exist a $j \in I$ such that $x, y
    \in A_j$, hence $A_i \bigcap A_j \neq \varnothing$ which by [theorem:
    \ref{equivalence relation partition alternative}] proves that $A_i = A_j$,
    so that $y \in A_i$. So $R [x] \subseteq A_i$ giving
    \[ A_i = R [x] \]
    \item If $x \in A$ then $\exists i \in I$ such that $x \in A_i$. Take $y
    \in A_i$ then $x, y \in A_i$ or $y R x$ proving that $y \in R [x]$, hence
    \[ A_i \subseteq R [x] \]
    Take $y \in R [x]$ then $y R x$ so there exist a $j \in I$ such that $x,
    y \in A_j$, hence $A_i \bigcap A_j \neq \varnothing$ which by [theorem:
    \ref{equivalence relation partition alternative}] proves that $A_i = A_j$,
    so that $y \in A_i$. So $R [x] \subseteq A_i$ giving
    \[ A_i = R [x] \]
  \end{enumerate}
  
\end{proof}

\begin{definition}
  \label{equivalence relation A/R}{\index{$A / R$}}Let $A$ be a set and $R$ a
  equivalence relation then $A / R$ is defined by
  \[ A / R = \{ R [x] |x \in A \} \]
\end{definition}

\begin{note}
  As $\forall x \in X$ $R [x] \in \mathcal{P} (A)$ it follows that
  \[ R / X \in \mathcal{P} (A) . \]
  As $A$ is a set it follows from the Axiom Power [axiom: \ref{axiom of
  power}] that $P (A)$ is a set, applying the Axiom of Subsets [axiom:
  \ref{axiom of subsets}] we have
  \[ R / X \text{ is a set} \]
\end{note}

\begin{theorem}
  \label{equivalence relation defines a partition (1)}Let $A$ be a set and $R$
  a equivalence relation then
  \[ A = \bigcup_{I \in A / R} I \text{ and $\forall I, J \in A / R$ with } I
     \neq J \text{ we have } I \bigcap J = \varnothing \]
\end{theorem}

\begin{proof}
  If $x \in A$ then by [theorem: \ref{equivalence relation R[x]=R[y]}] \ $x
  \in R [x] \in A / R$ so that $x \in \bigcup_{I \in A / R} I$ hence
  \[ A \subseteq \bigcup_{I \in A / R} I \]
  Further, as $\forall I \in A / R$ we have that $\exists x \in A$ such that
  $I = R [x] \subseteq A$, it follows that $\bigcup_{I \in A / R} I \subseteq
  A$ which combined with the above gives
  \[ A = \bigcup_{I \in A / R} I \]
  Further if $I, J \in A / R$ with $I \neq J$ then $\exists x, y \in A$ such
  that $I = R [x]$ and $J = R [y]$. If $z \in R [x] \bigcap R [y]$ then by
  [theorem: \ref{equivalence relation R[x]=R[y]}] we have $R [x] = R [z] = R
  [y]$ so that $I = J$ contradicting $I \neq J$. Hence we must have that
  \[ I \bigcap J = \varnothing \]
\end{proof}

\subsection{Functions and equivalence relations}

In this section we show how a function can be decomposed as the composition of
a surjection, a bijection and injection. First we examine the relation between
functions and equivalence relations.

We can use functions to generate a equivalence relation on the domain of the
function based on a equivalence relation on the target of the function.

\begin{theorem}
  \label{equivalence relation function}$f : A \rightarrow B$ a function and
  $R$ a equivalence relation in $A$ then
  \[ f \langle R \rangle = \{ (x, y) |f (x) R f (y) \} \subseteq A \times A
  \]
  is a equivalence relation in A
\end{theorem}

\begin{proof}
  
  \begin{description}
    \item[reflectivity] If $x \in A$ then $f (x) \in B$ so that $f (x) R f
    (x)$ hence by definition $x R x$
    
    \item[symmetric] If $x R y$ then $f (x) R f (y)$ so that $f (y) R f (x)$
    proving $y R x$
    
    \item[transitivity] If $x R y \wedge y R z$ then $f (x) R f (y) \wedge f
    (y) R f (z)$ so that $f (x) R f (x)$ proving that $x R z$.
  \end{description}
\end{proof}

A equivalence relation on a set induce a equivalence relation on a subset

\begin{theorem}
  \label{equivalence relation subsets}Let $A$ be a class, $B \subseteq A$ a
  sub-class and $R$ a equivalence relation in $R$ then $R_{|B}$ defined by
  \[ R_{|B} = \{ (x, y) |x \in B \wedge y \in B \wedge x R y \} = R \bigcap
     (B \times B) \]
  is a equivalence relation.
\end{theorem}

\begin{proof}
  
  \begin{description}
    \item[reflexivity] If $x \in B$ then $x R x$ so that $x R_{|B} x$
    
    \item[symmetric] If $x R_{| B \nobracket} y \Rightarrow x \in B \wedge y
    \in B \wedge \tmop{xRy} \Rightarrow \tmop{yR}_{| B \nobracket} x$
    
    \item[transitivity] If $x R_{|B} y \wedge y R_{|B} z$ then $x, y, z \in
    \mathbb{R}$ and $x R y \wedge y R z$ so that $x, z \in B$ and $x R z$
    proving $x R_{|B} z$
  \end{description}
\end{proof}

\begin{theorem}
  \label{equivalence relation determined by a function}If $f : A \rightarrow
  B$ is a function then $R_f$ defined by
  \[ R_f = \{ (x, y) \in A \times A|f (x) = f (y) \} \]
  is a relation. $R_f$ is called the \tmtextbf{equivalence relation determined
  by f}
\end{theorem}

\begin{proof}
  
  \begin{description}
    \item[reflexivity] If $x \in A$ then $f (x) = f (x)$ proving that $x R_f
    x$
    
    \item[symmetric] If $x R_f y$ then $f (x) = f (y) \Rightarrow f (y) = f
    (x)$ proving that $y R_f x$
    
    \item[transitivity] If $x R_f y \wedge y R_f x$ then $f (x) = f (y)$ and
    $f (y) = f (z)$ so that $f (x) = f (x)$ hence $x R_f z$
  \end{description}
\end{proof}

We can also do the opposite and associate a function with a equivalence
relation

\begin{theorem}[Canonical Function]
  \label{equivalence relation canical function}{\index{canonical function}}Let
  $A$ be a set and $R$ a equivalence relation in $A$ then:
  \begin{enumerate}
    \item $f_R : A \rightarrow A / R$ defined by $f_R (x) = R [x]$ is a
    surjective function.
    
    \item $R_{R_f} = R$
  \end{enumerate}
\end{theorem}

$f_R : A \rightarrow A / R$ is called the \tmtextbf{Canonical function
associated with $R$}

\begin{proof}
  
  \begin{enumerate}
    \item As for every $x \in A$ we have the unique $R [x] \in R / X$ it
    follows from [proposition: \ref{function simple definition}] that
    \[ f_R : A \rightarrow A / R \text{ is a function} \]
    Let $y \in R / X$ then $\exists x \in A$ such that $y = R [x]$ so that
    $(x, y) = (x, R [x]) \in f_R$ proving that $y \in \tmop{range} (f_R)$. So
    $R / X \subseteq \tmop{range} (f_R)$ which by [theorem: \ref{function
    surjection condition}] proves that
    \[ f_R : A \rightarrow A / R \text{ is surjective} \]
    \item We have
    \begin{eqnarray*}
      (x, y) \in R & \Leftrightarrow & x R y\\
      & \Leftrightarrowlim_{\text{[theorem: \ref{equivalence relation
      R[x]=R[y]}]}} & R [x] = R [y]\\
      & \Leftrightarrow & f_R (x) = f_R (y)\\
      & \Leftrightarrow & (x, y) \in R_{f_R}
    \end{eqnarray*}
  \end{enumerate}
\end{proof}

We use the above to decompose every function as the composition of a
surjection, bijection and injection.

\

\begin{theorem}
  \label{equivalence relation canonical decomposition of a function}Let $A, B$
  be sets and $f : A \rightarrow B$ a function and define the following
  functions:
  \begin{enumeratealpha}
    \item $s_f : A / R_f \rightarrow f (A)$ where $s_f = \{ (R_f [x], f (x))
    |x \in A \}$
    
    \item $i_{f (A)} : f (A) \rightarrow B$ where $i_{f (A)} = \{ (x, x) |x
    \in f (A) \}$ [the inclusion function see [example: \ref{function
    inclusion function}]
    
    \item $f_{R_f} : A \rightarrow A / R_f$ where $f_{R_f} (x) = R_f [x]$
    [theorem: \ref{equivalence relation canical function}]
  \end{enumeratealpha}
  then
  \begin{enumerate}
    \item $s_f : A / R_f \rightarrow f (A)$ is a bijection
    
    \item $i_{f (A)} : f (A) \rightarrow B$ is a injective function
    
    \item $f_{R_f} : A \rightarrow A / R_f$\quad is a surjective function
    
    \item $f = i_{f (A)} \circ (s_f \circ f_{R_f}) \equallim_{\text{[theorem:
    \ref{partial function associativity}]}} (i_{f (A)} \circ s_f) \circ
    f_{R_f}$
  \end{enumerate}
\end{theorem}

\begin{proof}
  Using \ [example: \ref{function inclusion function}] and [theorem:
  \ref{equivalence relation canical function}] we have that
  \[ i_{f (A)} : f (A) \rightarrow B \text{ is a injective function} \]
  and
  \[ f_{R_f} : A \rightarrow A / R_f \text{ is surjective function} \]
  We proceed now to prove that $s_f$ is a bijection. If $(x, y) \in s_f$ then
  there exists a $a \in A$ such that $(x, y) = (R_f [a], f (a))$ hence $x =
  R_f [a] \in A / R_f$ and $y = f (a) \Rightarrow (a, y) \in f \Rightarrow y
  \in f (A)$. So that $(x, y) \in (A / R_f) \times f (A)$ or
  \[ s_f \subseteq (A / R_f) \times f (A) \]
  If $(x, y), (x, y') \in s_f$ then there exists $a, a' \in A$ such that
  \[ (x, y) = (R_f [a], f (a)) \wedge (x, y') = (R_f [a'], f (a')) \]
  or
  \begin{equation}
    \label{eq 3.1.009} x = R_f [a] \wedge y = f (a) \wedge x = R_f [a'] \wedge
    y' = f (a')
  \end{equation}
  From the above $R_f [a] = x = R_f [a']$, which using [theorem:
  \ref{equivalence relation R[x]=R[y]}] means that $a R_f a'$, so by the
  definition of $R_f$ [theorem: \ref{equivalence relation determined by a
  function}] we have $f (a) = f (a')$. As by [eq: \ref{eq 3.1.009}] $y = f (a)
  \wedge y' = f (a')$ it follows that $y = y'$. So
  \[ s_f : A / R_f \rightarrow f (A) \text{ is a partial function} \]
  If $x \in A / R_f$ then $\exists a \in A$ such that $x = [a]$, hence if we
  take $y = f (A)$ we have that $(x, y) = ([a], f (a)) \in s_f$ proving that
  $x \in \tmop{dom} (s_f)$. So $A / R_f \subseteq \tmop{dom} (f)$ which by
  [proposition: \ref{function condition (1)}] proves that
  \[ s_f : A / R_f \rightarrow f (A) \text{ is a function} \]
  Let $(x, y), (x', y) \in s_f$ then $\exists a, a' \in A$ such that $(x, y) =
  (R_f [a], f (a))$ and $(x', y) = (R_f [a'], f (a'))$, hence
  \begin{equation}
    \label{eq 3.2.009} x = R_f [a] \wedge x' = R_f [a'] \wedge y = f (a)
    \wedge y = f (a')
  \end{equation}
  From $f (a) = y = f (a')$ it follows that $f (a) = f (a')$, which by the
  definition of $R_f$ [theorem: \ref{equivalence relation determined by a
  function}] proves that $a R_f a'$. Using [theorem: \ref{equivalence relation
  R[x]=R[y]} it follows that $R_f [a] = R_f [a']$ or using [eq: \ref{eq
  3.2.009}] that $x = x'$. So we have proved that
  \begin{equation}
    \label{eq 3.3.009} s_f : A / R_f \rightarrow f (A) \text{ is injective}
  \end{equation}
  Let $y \in f (A)$ then there exist a $a \in A$ such that $(a, y) \in f
  \Rightarrow y = f (a)$. But then $(R_f [a], y) = (R_f [a], f (a)) \in s_f$
  proving that $y \in \tmop{range} (s_f)$. So $A / R_f \subseteq \tmop{range}
  (s_f)$ which by [proposition: \ref{function surjection condition}] proves
  that
  \begin{equation}
    \label{eq 3.4.009} s_f : A / R_f \rightarrow f (A) \text{ is surjective}
  \end{equation}
  Combining [eq: \ref{eq 3.3.009}] and [eq: \ref{eq 3.4.009}] it follows that
  \[ s_f : A / R_f \rightarrow f (A) \text{ is a bijection} \]
  Now we proceed to prove that $f = (i_{f (A)} \circ s_f) \circ f_{R_f}$. Let
  $(x, u) \in (i_{f (A)} \circ s_f) \circ f_{R_f}$ then $\exists y$ such that
  $(x, y) \in f_{R_f} \wedge (y, u) \in i_{f (A)} \circ s_f$, from $(y, u) \in
  i_{f (A)} \circ s_f$ {\exists}z such that $(y, z) \in s_f \wedge (z, u) \in
  i_{f (A)}$, summarized
  \begin{equation}
    \label{eq 3.5.009} (x, y) \in f_{R_f} \wedge (y, z) \in s_f \wedge (z, u)
    \in i_{f (A)}
  \end{equation}
  From $(x, y) \in f_{R_f}$ it follows that $\exists a \in A$ such that $(x,
  y) = (a, R_f [a])$ or
  \begin{equation}
    \label{eq 3.6.009} x = a \wedge y = R_f [a]
  \end{equation}
  From $(y, z) \in s_f$ it follows that $\exists a' \in A$ such that $(y, z) =
  (R_f [a'], f (a'))$ or $y = R_f [a'] \wedge z = f (a')$. As $y
  \equallim_{\text{[eq: \ref{eq 3.6.009}]}} R_f [a]$ we have that $R_f [a] =
  R_f [a']$, which by [theorem: \ref{equivalence relation R[x]=R[y]}] proves
  that $a R_f a'$, so by the definition of $R_f$ we have $f (a) = f (a')$
  hence $z = f (a)$. From $(z, u) \in i_{f (A)}$ it follows that $z = u$ hence
  $u = f (a)$. As $x \equallim_{\text{[eq: \ref{eq 3.6.009}]}} a$ it follows
  that $(x, u) = (a, f (a)) \in f$. Hence
  \begin{equation}
    \label{eq 3.7.009} (i_{f (A)} \circ s_f) \circ f_{R_f} \subseteq f
  \end{equation}
  Finally if $(x, y) \in f$ then as $f \subseteq A \times B$ proves that $x
  \in A$ and $f (x) = y \in f (A)$. Hence $(R_f [x], f (x)) \in s_f$, $(x, R_f
  [x]) \in f_{R_f}$ and $(f (x), y) = (f (x), f (x)) \in i_{f (A)}$. So that
  $(R_f [x], y) \in i_{f (A)} \circ s_f$ and $(x, R_f [x]) \in f_{R_f}$
  proving that $(x, y) \in (i_{f (A)} \circ s_f) \circ f_{R_f}$. So $f
  \subseteq (i_{f (A)} \circ s_f) \circ f_{R_f}$ which combined with [eq:
  \ref{eq 3.7.009}] gives
  \[ f = (i_{f (A)} \circ s_f) \circ f_{R_f} \]
\end{proof}

\begin{notation}
  For the rest of this book we use the standard convention of noting a
  equivalence relation as $\sim$, The definition of $\sim$ should then be
  clear from the context. If many equivalence relations are used in the same
  context we use superscripts like $\sim_{\mathbb{R}}$ and $\sim_{\mathbb{Z}}$
  to avoid conflicts.
\end{notation}

\section{Partial ordered classes}

\subsection{Order relation}

First we define a partial order relation that allows us to compare two
elements and specify which element 'lies before' another element. \

\begin{definition}[Pre-order]
  \label{order preorder}{\index{preorder}}Let $A$ be a class then a relation
  $R \subseteq A \times A$ in $A$ is a pre-order if it is \tmtextbf{reflexive}
  and \tmtextbf{transitive} or in other words:
  \begin{description}
    \item[reflectivity] $\forall x \in A$ we have $x R x$
    
    \item[transitivity] If $x R y \wedge y R z$ then $x R z$
  \end{description}
\end{definition}

\begin{definition}
  \label{order preordered class}{\index{pre-ordered class}}$\langle A, R
  \rangle$ is a pre-ordered class iff $A$ is a class and $R$ is a pre-order in
  $A$
\end{definition}

A order relation is a pre-order with one extra condition

\begin{definition}[Order relation]
  \label{order order relation}{\index{order relation}}If $A$ is a class then a
  relation $R \subseteq A \times A$ in $A$ is a \tmtextbf{order} if it is a
  pre-order that is anti-symmetric or in other words:
  \begin{description}
    \item[reflectivity] $\forall x \in A$ we have $x R x$
    
    \item[anti-symmetry] If $x R y \wedge y R x$ then $x = y$
    
    \item[transitive] If $x R y \wedge y R z$ then $x R z$
  \end{description}
\end{definition}

\begin{definition}[Partial ordered class]
  \label{order partial ordered class}{\index{partial ordered class}}$\langle
  A, R \rangle$ is a \tmtextbf{partial ordered class} if $A$ is a class and
  $R$ is a order.
\end{definition}

\begin{notation}
  {\index{$\leqslant$}}We use the standard convention of noting a pre-order
  relation as $\leqslant$, The definition of $\leqslant$ should then be clear
  from the context. If many pre-order relations are used in the same context
  we use superscripts like $\leqslant_{\mathbb{R}}$ and
  $\leqslant_{\mathbb{Z}}$ or $\preccurlyeq$ to avoid conflicts.
\end{notation}

\begin{definition}
  \label{order a<less>=b<less>=c}If $\langle A, \leqslant \rangle$ is a
  pre-ordered or partial class and $x, y, z \in A$ then we define:
  \begin{eqnarray*}
    x \leqslant y \leqslant z & \text{ is the same as } & x \leqslant y \wedge
    y \leqslant z\\
    x \leqslant y < z & \text{ is the same as } & x \leqslant y \wedge y < z\\
    x < y \leqslant z & \text{ is the same as } & x < y \wedge y \leqslant z\\
    x < y < z & \text{ is the same as } & x < y \wedge y < z
  \end{eqnarray*}
\end{definition}

\begin{definition}
  {\index{$<$}}If $\langle A, \leqslant \rangle$ is a pre-ordered class [or
  partial ordered class] then $x < y$ is equivalent with $x \leqslant y \wedge
  x \neq y$
\end{definition}

\begin{theorem}
  \label{order strict order}If $\langle A, \leqslant \rangle$ is a partially
  ordered set then
  \begin{enumerate}
    \item $x \leqslant y \wedge y < z \Rightarrow x < z$
    
    \item $x < y \wedge y \leqslant z \Rightarrow x < z$
    
    \item $x < y \wedge y < z \Rightarrow x < z$
    
    \item $(x < y \vee x = y) \Leftrightarrow (x \leqslant y)$
  \end{enumerate}
  or in other words
  \begin{enumerate}
    \item $x \leqslant y < z \Rightarrow x < z$
    
    \item $x < y \leqslant z \Rightarrow x < z$
    
    \item $x < y < z \Rightarrow x < z$
    
    \item $(x < y \vee x = y) \Leftrightarrow x \leqslant y$
  \end{enumerate}
\end{theorem}

\begin{proof}
  
  \begin{enumerate}
    \item If $x \leqslant y \wedge y < z$ then $x \leqslant y \wedge y
    \leqslant z \wedge y \neq z$, so that $x \leqslant z$ and $y \neq z$.
    Assume that $x = z$ then $z \leqslant y \equallim_{y \leqslant z} z = y$
    contradicting $y \neq z$, so we must have $x \neq z$, which together with
    $x \leqslant z$ gives
    \[ x < z \]
    \item If $x < y \wedge y \leqslant z$ then $x \leqslant y \wedge y
    \leqslant z \wedge x \neq y$, so that $x \leqslant z$ and $x \neq y$.
    Assume that $x = z$ then $y \leqslant x \Rightarrowlim_{x \leqslant y} y =
    x$ contradicting $x \neq y$, so we must have $x \neq z$, which together
    with $x \leqslant z$ gives
    \[ x < z \]
    \item If $x < y \wedge y < z$ then $x \neq y \wedge x \leqslant y \wedge y
    < z$ so that by (1) we have $x < z$
    
    \item We have
    \begin{eqnarray*}
      (x < y \vee x = y) & \Leftrightarrow & ((x \leqslant y \wedge x \neq y)
      \vee x = y)\\
      & \Leftrightarrow & ((x \leqslant y \vee x = y) \wedge (x \neq y \vee x
      = y))\\
      & \Leftrightarrow & x \leqslant y \vee x = y\\
      & \Leftrightarrow & x \leqslant y
    \end{eqnarray*}
  \end{enumerate}
\end{proof}

\begin{example}
  \label{order inclusion is a order}Let $A$ be a class of classes and
  $\leqslant$ defined by $\leqslant = \{ (x, y) \in \mathcal{A} \times
  \mathcal{A}|x \subseteq y \}$ then $\langle \mathcal{A}, \leqslant \rangle$
  is a partial ordered class
\end{example}

\begin{proof}
  
  \begin{description}
    \item[reflectivity] If $A \in \mathcal{C}$ then by [theorem: \ref{class
    properties (1)}] $A \subseteq A$ so that $A \leqslant A$
    
    \item[anti-symmetric] If $A \leqslant B$ and $B \leqslant A$ then $A
    \subseteq B \wedge B \subseteq A$ so that by [theorem: \ref{class
    properties (1)}] $A = B$
    
    \item[transitivity] If $A \leqslant B \wedge B \leqslant C$ then $A
    \subseteq B \wedge B \subseteq C$ so that by [theorem: \ref{class
    properties (1)}] $A \subseteq C$ or $A \leqslant C$
  \end{description}
\end{proof}

Every pre-order can be used as the base to create a order relation as is
expressed in the following theorem. The basic idea is that $x \leqslant y
\wedge y \leqslant x \Rightarrow x = y$ is missing from a pre-order. By
defining a equivalence relation $\sim$ such that $x \sim y$ if $x \leqslant y
\wedge y \leqslant x$ we turn this in equality of equivalence classes. This is
a typical example about the use of equivalence relations, they allow you to
define a new type of equality, so that objects that are not equal have
associated equivalence classes that are equal.

\begin{theorem}
  \label{order eq order preorder to order}Let $\langle A, \leqslant \rangle$
  be a pre-ordered set then we have
  \begin{enumerate}
    \item $\sim \subseteq A \times A$ defined by $\sim = \{ (x, y) \in A|x
    \leqslant y \wedge y \leqslant x \}$ is a equivalence relation
    
    \item Define $\preccurlyeq \subseteq (A / \sim) \times (A / \sim)$ by
    \[ \preccurlyeq = \left\{ (x, y) \in (A / \sim) \times (A / \sim) |
       \exists x' \in \sim [x] \text{ and } \exists y' \in \sim [y] \text{
       such that $x' \leqslant y'$} \right\} \]
    then $\preccurlyeq$ is a order relation in $A / \sim$. So $\langle A /
    \sim, \preccurlyeq \rangle$ is a partial ordered set
    
    \item $\forall x, y \in A$ we have $x \leqslant y \Leftrightarrow \sim [x]
    \preccurlyeq \sim [y]$
  \end{enumerate}
\end{theorem}

\begin{proof}
  
  \begin{enumerate}
    \item To prove that $\sim$ is a equivalence relation note:
    \begin{description}
      \item[reflectivity] If $x \in A$ then $x \leqslant x$ proving that $x
      \sim x$
      
      \item[symmetric] If $x \sim y$ then $x \leqslant y \wedge y \leqslant x
      \Rightarrow y \leqslant x \wedge x \leqslant y$ so that $y \sim x$
      
      \item[transitive] If $x \sim y \infixand y \sim z$ then $x \leqslant y
      \wedge y \leqslant x \wedge y \leqslant z \wedge z \leqslant y$ so that
      $x \leqslant z$ and $z \leqslant x$ or $x \sim z$
    \end{description}
    \item To prove that $\preccurlyeq$ is a order relation we must prove
    reflectivity, symmetry and transitivity:
    \begin{description}
      \item[reflexivity] Take $\sim [x]$ then as $x \leqslant x$ there exists
      a $u \in \sim [x]$ and $v \in \sim [x]$ such that $u \leqslant v$ [just
      take $u = x = v$] so that
      \[ \sim [x] \preccurlyeq \sim [x] \]
      \item[symmetry] Let $\sim [x] \leqslant \sim [y]$ and $\sim [y]
      \leqslant \sim [x]$ then $\exists x', x'' \in \sim [x]$, $\exists y' y''
      \in \sim [y]$ such that
      \[ x' \leqslant y' \wedge y'' \leqslant x'' \]
      From $\exists x', x'' \in \sim [x]$, $\exists y' y'' \in \sim [y]$ we
      have
      \[ x' \leqslant x \wedge x \leqslant x' \wedge x'' \leqslant x \wedge x
         \leqslant x'' \wedge y' \leqslant y \wedge y \leqslant y' \wedge y''
         \leqslant y \wedge y \leqslant y'' \]
      From $x \leqslant x'$ and $x' \leqslant y'$ we have $x \leqslant y'$, as
      $y' \leqslant y$ we have
      \[ x \leqslant y \]
      From $y \leqslant y''$ and $y'' \leqslant x''$ we have $y \leqslant
      x''$, as $x'' \leqslant x$ it follows that
      \[ y \leqslant x \]
      Finally from $x \leqslant y$ and $y \leqslant x$ we have that $x \sim y
      \tmop{which} \tmop{by} \left[ \tmop{theorem} : \ref{equivalence relation
      R[x]=R[y]} \right]$ gives
      \[ \sim [x] = \sim [y] \]
      \item[transitivity] Assume that $\sim [x] \preccurlyeq \sim [y]$ and
      $\sim [y] \preccurlyeq \sim [z]$ then we have the existence of $x' \in
      \sim [x]$, $y', y'' \in \sim [y]$ and $z' \in \sim [z]$ such that
      \[ x' \leqslant y' \wedge y'' \leqslant z' \]
      From $x' \in \sim [x]$, $y', y'' \in \sim [y]$ and $z' \in \sim [z]$ it
      follows that
      \[ x' \leqslant x \wedge x \leqslant x' \wedge y' \leqslant y \wedge y
         \leqslant y' \wedge y'' \leqslant y \wedge y \leqslant y'' \wedge z'
         \leqslant z \wedge z \leqslant z' \]
      From $x \leqslant x'$ and $x' \leqslant y'$ we have $x \leqslant y'$, as
      $y' \leqslant y$ we have $x \leqslant y$, as $y \leqslant y''$ it
      follows that $x \leqslant y''$, from $y'' \leqslant z'$ we have that $x
      \leqslant z'$ and finally from $z' \leqslant z$ it follows that $x
      \leqslant z$. Hence
      \[ \sim [x] \preccurlyeq \sim [z] \]
    \end{description}
    \item 
    \begin{description}
      \item[$\Rightarrow$] If $x \leqslant y$ then as $x \in \sim [x]$ and $y
      \in \sim [y]$ we have $\sim [x] \preccurlyeq \sim [y]$
      
      \item[$\Leftarrow$] If $\sim [x] \preccurlyeq \sim [y]$ then $\exists x'
      \in \sim [x]$ and $\exists y' \in \sim [y]$ such that
      \[ x' \leqslant y' \]
      From $x' \in \sim [x]$ and $y' \in \sim [y]$ we have that
      \[ x' \leqslant x \wedge x \leqslant x' \wedge y' \leqslant y \wedge y
         \leqslant y' \]
      From $x \leqslant x'$ and $x' \leqslant y'$ it follows that $x \leqslant
      y'$ and as $y' \leqslant y$ it follows that
      \[ x \leqslant y \]
    \end{description}
  \end{enumerate}
\end{proof}

Given a partial ordered class then we can induce the order on a sub-class
making the sub-class also a partial ordered class.

\begin{theorem}
  \label{order partial order on sub class}If $\langle A, \leqslant \rangle$ is
  a partial ordered sets and $B \subseteq A$ then $\leqslant_{|B}$ defined by
  \[ \leqslant_{|B} = \leqslant \bigcap B \times B = B \]
  is a order relation in $B$ making $\langle B, \leqslant_{|B} \rangle$ a
  partial ordered set.\quad
\end{theorem}

\begin{proof}
  
  \begin{description}
    \item[reflectivity] If $x \in B$ then $x \leqslant x$ or $(x, x) \in
    \leqslant \Rightarrowlim_{x \in B} (x, x) \in \leqslant \bigcap (B \times
    B)$ hence $x \leqslant_{|B} y$
    
    \item[symmetry] If $x \leqslant_{| B \nobracket} y \wedge y \leqslant_{| B
    \nobracket} x \Rightarrow x \leqslant y \wedge y \leqslant x \Rightarrow x
    = y$
    
    \item[transitivity] If $x \leqslant_{| B \nobracket} y \wedge y
    \leqslant_{| B \nobracket} z \Rightarrow x \leqslant y \wedge y \leqslant
    z \Rightarrow x \leqslant z \Rightarrowlim_{x, z \in B} x \leqslant_{| B
    \nobracket} z$
  \end{description}
\end{proof}

\begin{convention}
  To avoid excessive usage notation we write $\langle B, \leqslant \rangle$
  instead of $\langle B, \leqslant_{|B} \rangle$
\end{convention}

The following shows a technique of defining a partial order on the Cartesian
product of partial ordered set.

\begin{theorem}[Lexical ordering]
  \label{order lexical order}Let $\langle A, \leqslant_A \rangle$ and $\langle
  B, \leqslant_B \rangle$ be partial ordered classes then $\leqslant_{A \times
  B}$ defined by
  \[ \leqslant_{A \times B} = \{ ((x, y), (u, v)) \in (A \times B) \times (A
     \times B) | (x \neq u \wedge x \leqslant_A u) \vee (x = y \wedge y
     \leqslant_B v \} \nobracket \]
  is a order in $A \times B$ making $\langle (A \times B) \times (A \times B),
  \leqslant_{A \times B} \rangle$ a partial ordered set
\end{theorem}

\begin{proof}
  
  \begin{description}
    \item[reflexivity] If $(x, y) \in A \times B$ then $x \leqslant_A x \wedge
    y \leqslant_B y$ proving that $(x, y) \leqslant_{A \times B} (x, y)$
    
    \item[symmetry] Let $(x, y) \leqslant_{A \times B} (u, v) \wedge (u, v)
    \leqslant_{A \times B} (x, y)$. If $x \neq u$ we would have $x \leqslant_A
    u \wedge u \leqslant_A x \Rightarrow x = u$ a contradiction. So we must
    have that $x = u$ but then $y \leqslant_B v \wedge v \leqslant_{|B} y
    \Rightarrow y = v$ proving that
    \[ (x, y) = (u, v) \]
    \item[transitivity] Let $(x, y) \leqslant_{A \times B} (u, v) \wedge (u,
    v) \leqslant_{A \times B} (r, s)$ then we have to consider the following
    cases:
    \begin{description}
      \item[$x = u$]  Then $y \leqslant_B v$ and we have the following
      possibilities
      \begin{description}
        \item[$u = r$] Then $v \leqslant_B s$ so that $y \leqslant_B s$ which
        as $x = r$ proves that
        \[ (x, y) \leqslant_{A \times B} (r, s) \]
        \item[$u \neq r$] Then $u \leqslant_A r \Rightarrowlim_{x = u} x
        \leqslant_A r$ which as $x \neq r$ proves that
        \[ (x, y) \leqslant_{A \times B} (r, s) \]
      \end{description}
      \item[$x \neq u$] Then $x \leqslant_A u$ and we have the following
      possibilities
      \begin{description}
        \item[$u = r$] Then $x \leqslant_A u \Rightarrowlim_{u = r} x
        \leqslant_A r$ and $x \neq r$ so that
        \[ (x, y) \leqslant_{A \times B} (r, s) \]
        \item[$u \neq r$] Then $u \leqslant_A r$ so that $x \leqslant_A r$. If
        $x = r$ then we would have $x \leqslant_A u \wedge u \leqslant_A x$
      \end{description}
      giving $x = u$ contradicting $x \neq u$. So we must have $x \neq r$
      which as $x \leqslant_A r$ gives
      \[ (x, y) \leqslant_{A \times B} (r, s) \]
    \end{description}
  \end{description}
\end{proof}

\begin{definition}
  \label{order comparable}{\index{comparable elements}}Let $\langle A,
  \leqslant \rangle$ be a partial ordered class then $x, y \in A$ are
  \tmtextbf{comparable} if $x \leqslant y$ or $y \leqslant x$
\end{definition}

\begin{theorem}
  \label{order comparable property}\label{order comparable
  property}{\index{comparable elements}}Let $\langle A, \leqslant \rangle$ be
  a partial ordered class and $x, y \in A$ comparable elements then we have
  either $x \leqslant y$ or $y < x$
\end{theorem}

\begin{proof}
  As $x, y$ are comparable then we have $x \leqslant y \vee y \leqslant x$,
  consider the following cases:
  \begin{description}
    \item[$x \leqslant y$] hen $x \leqslant y$
    
    \item[$\neg (x \leqslant y)$] then we must have $y \leqslant x$. If $x =
    y$ then as $x \leqslant x$ we have $x \leqslant y$ contradicting $\neg (x
    \leqslant y)$ so that $x \neq y$ proving $y < x$.
  \end{description}
  Hence we have
  \[ x \leqslant y \vee y < x \]
  
\end{proof}

\begin{definition}
  \label{totally ordered class}{\index{totally ordered class}}{\index{fully
  ordered class}}{\index{linear ordered class}}A pre-ordered class $\langle A,
  \leqslant \rangle$ is a \tmtextbf{totally ordered class} iff
  \[ \forall x, y \in A \text{ we have } x \leqslant y \vee y \leqslant x \]
  In other words $\langle A, \leqslant \rangle$ is a \tmtextbf{totally ordered
  class} if every pair of elements are comparable. Other names used in the
  literature are \tmtextbf{fully ordered class} or \tmtextbf{linear ordered
  class}.
\end{definition}

\begin{definition}[chain]
  \label{order chain}{\index{chain}}Let $\langle A, \leqslant \rangle$ be a
  partial ordered class and $C \subseteq A$ then $C$ is called a
  \tmtextbf{chain} if $\forall x, y \in C$ we have that $x \leqslant y$ or $y
  \leqslant x$.
\end{definition}

\begin{example}
  \label{order empty set is a chain}Let $\langle A, \leqslant \rangle$ be a
  partial ordered class then $\varnothing$ is a chain
\end{example}

\begin{proof}
  The condition $\forall x, y \in \varnothing$ we have that $x, y$ are
  comparable is satisfied vacuously.
\end{proof}

\begin{theorem}
  \label{order chain is a totally ordered class}Let $\langle A, \leqslant
  \rangle$ be a partial ordered class and $B \subseteq A$ a chain then
  $\langle B, \leqslant_{|B} \rangle$ is a totally ordered class
\end{theorem}

\begin{proof}
  Using [theorem: \ref{order partial order on sub class}] we have that
  $\langle B, \leqslant_{|B} \rangle$ is a partial ordered class. Let $x, y
  \in B$ then as $B$ is a chain we have that $\forall x, y \in B$ $x \leqslant
  y \vee y \leqslant x$ or using the definition of $\leqslant_{|B}$ that $x
  \leqslant_{|B} y \vee y \leqslant_{|B} x$.
\end{proof}

\begin{theorem}
  \label{order totally ordered subclass}Let $\langle A, \leqslant \rangle$ be
  a totally ordered class and $B \subseteq A$ then $B$ is a chain [hence by
  [theorem: \ref{order chain is a totally ordered class}] $\langle B,
  \leqslant_{|B} \rangle$ is a totally ordered class]
\end{theorem}

\begin{proof}
  If $x, y \in B$ then $x, y \in A$ and as $A$ is totally ordered we have $x
  \leqslant y \vee y \leqslant x$ so $B$ is a chain
\end{proof}

\begin{theorem}
  \label{order totally lexicol ordering}Let $\langle A, \leqslant_A \rangle$
  and $\langle B, \leqslant_B \rangle$ be totally ordered classes then
  $\langle A \times B, \leqslant_{A \times B} \rangle$ is a totally ordered
  class.
\end{theorem}

\begin{proof}
  First $\langle A \times B, \leqslant_{A \times B} \rangle$ is a partially
  ordered class by [theorem: \ref{order lexical order}]. If \ $(x, y), (x',
  y') \in A \times B$ then we have for $x, x'$ either
  \begin{description}
    \item[$x = x'$] As $\langle B, \leqslant_B \rangle$ is fully ordered we
    have either
    \begin{description}
      \item[$y \leqslant y'$] then $(x, y) \leqslant (x', y')$
      
      \item[$y' \leqslant y$] then $(x', y') \leqslant (x, y)$
    \end{description}
    \item[$x \neq x'$] As $\langle A, \leqslant_A \rangle$ is fully ordered we
    have either
    \begin{description}
      \item[$x \leqslant x'$] then $(x, y) \leqslant (x', y')$
      
      \item[$x' \leqslant x$] then $(x', y') \leqslant (x, y)$
    \end{description}
  \end{description}
\end{proof}

\begin{definition}[Initial Segment]
  \label{order initial segement}{\index{initial
  segment}}{\index{$\mathcal{S}_a$}}If $\langle A, \leqslant \rangle$ is a
  partial ordered class, $a \in A$ then a \tmtextbf{initial segment of A
  determined by a} noted as $S_{A, a}$ is defined by
  \[ S_{A, a} = \{ x \in A|x < a \} \]
\end{definition}

We have the following trivial result for initial segments.

\begin{proposition}
  \label{order initial segement inclusion}If $\langle A, \leqslant \rangle$ is
  a partial ordered class and $a, b \in A$ such that $a \leqslant b$ then
  $S_{A, a} \subseteq S_{A, b}$
\end{proposition}

\begin{proof}
  If $x \in S_{A, a}$ then $x < a \Rightarrowlim_{a \leqslant b} x < b$
  proving that $x \in S_{A, b}$
\end{proof}

\begin{theorem}
  \label{order intial sergment property}If $\langle A, \leqslant \rangle$ is a
  partial ordered class and $P$ is a initial segment of $A$ and $Q$ is a
  initial segment of $P$ [using the induced order $\leqslant_{|P}$] then $A$
  is a initial segment of $A$
\end{theorem}

\begin{proof}
  Using the hypothesis there exists $a \in A$ such that $P = \{ x \in A|x < a
  \}$ and a $b \in P$ such that $Q = \{ x \in P|x < b \}$. Consider then the
  initial segment $S_{A, b} = \{ x \in A|x < b \}$ of $A$ determined by $a$
  then we have
  \begin{eqnarray*}
    x \in S_{A, b} & \Rightarrow & x \in A \wedge x < b\\
    & \Rightarrowlim_{b < a \Rightarrow x < b \Rightarrow x < a} & x \in A
    \wedge x < a \wedge x < b\\
    & \Rightarrow & x \in P \wedge x < b\\
    & \Rightarrow & x \in P \wedge x <_{|P} b\\
    & \Rightarrow & x \in Q\\
    x \in Q & \Rightarrow & x \in P \wedge x <_{|P} b\\
    & \Rightarrow & x \in P \wedge x < b\\
    & \Rightarrowlim_{P \subseteq A} & x \in A \wedge x < b\\
    & \Rightarrow & x \in S_{A, b}
  \end{eqnarray*}
  Hence $Q = S_{A, b}$ a initial segment of $A$
\end{proof}

\subsection{Order relations and functions}

Functions between two partial ordered classes can be classified based on the
fact that they preserve or not preserve the order relation. This is expressed
in the next definition.

\begin{definition}
  \label{order increasing, decreasing}{\index{increasing
  function}}{\index{decreasing function}}{\index{order homomorphism}}Let
  $\langle A, \leqslant_A \rangle$, $\langle B, \leqslant_B \rangle$ be
  partial ordered classes and $f : A \rightarrow B$ a function then:
  \begin{enumerate}
    \item $f : \langle A, \leqslant_A \rangle \rightarrow B$ is
    \tmtextbf{increasing} if $\forall x, y \in A$ with $x \leqslant y$ we have
    $f (x) \leqslant f (y)$. Another name that is used is \tmtextbf{a order
    homeomorphism} [a homeomorphism is a function that preserver a certain
    operation, in this case the order relation]
    
    \item $f : \langle A, \leqslant_A \rangle \rightarrow \langle B,
    \leqslant_B \rangle$ is \tmtextbf{strictly increasing} if $\forall x, y
    \in A$ with $x < y$ we have $f (x) < f (y)$
    
    \item $f : \langle A, \leqslant_A \rangle \rightarrow \langle B,
    \leqslant_B \rangle$ is \tmtextbf{decreasing} if $\forall x, y \in A$ with
    $x \leqslant y$ we have $f (y) \leqslant f (x)$
    
    \item $f : \langle A, \leqslant_A \rangle \rightarrow \langle B,
    \leqslant_B \rangle$ is \tmtextbf{strictly decreasing} if $\forall x, y
    \in A$ with $x < y$ we have $f (y) < f (x)$
    
    \item $f : \langle A, \leqslant_A \rangle \rightarrow \langle B,
    \leqslant_B \rangle$ is a \tmtextbf{order isomorphism} if $\forall x, y
    \in A$ with $x \leqslant y \Leftrightarrow f (x) \leqslant f (y)$
  \end{enumerate}
\end{definition}

\begin{definition}
  \label{order A isomorphism B}{\index{$A \cong B$}}Two partial ordered
  classes $\langle A, \leqslant_A \rangle$ and $\langle B, \leqslant_B
  \rangle$ are \tmtextbf{order isomorphic} noted as $A \cong B$ if there
  exists order isomorphism between $A$ and $B$.
\end{definition}

\begin{theorem}
  \label{order homeomorphism extending}Let $\langle A, \leqslant_A \rangle$,
  $\langle B, \leqslant_B \rangle$ be two partial ordered classes, $D
  \subseteq B$ and
  \[ f : \langle A, \leqslant_A \rangle \rightarrow \langle D,
     (\leqslant_B)_{|D} \rangle \text{ be a order homeomorphism [see theorem:
     \ref{order partial order on sub class} for $\langle D, (\leqslant_B)_{|D}
     \rangle$} \]
  then
  \[ f : \langle A, \leqslant_A \rangle \rightarrow \langle B, \leqslant_B
     \rangle \text{ is a order homeomorphism} \]
\end{theorem}

\begin{proof}
  The proof is trivial for if $x, y \in A$ with $x \leqslant_A y$ then $f (x)
  (\leqslant_B)_{|D} f (y) \Rightarrowlim_{\text{\ref{order partial order on
  sub class}}} f (x) \leqslant_B f (y)$
  
  \ 
\end{proof}

\begin{theorem}
  \label{order composition of functions}Let $\langle A, \leqslant_A \rangle$,
  $\langle B, \leqslant_B \rangle$, $\langle C, \leqslant_C \rangle$ be
  partial ordered classes, $D \subseteq B$
  \begin{enumerate}
    \item If $D \subseteq B$ is equiped with the induced order from $\langle
    B, \leqslant_B \rangle$ [see theorem: \ref{order partial order on sub
    class}] and
    \[ f : \langle A, \leqslant_A \rangle \rightarrow \langle D, \leqslant_B
       \rangle \text{ and } g : \langle B, \leqslant_B \rangle \rightarrow
       \langle C, \leqslant_C \rangle \text{ are order homeomorphisms} \]
    then
    \[ g \circ f : \langle A, \leqslant_A \rangle \rightarrow \langle C,
       \leqslant_C \rangle \text{ is a order homeomorphism} \]
    \item If $D \subseteq B$ is equiped with the induced order from $\langle
    B, \leqslant_B \rangle$ [see theorem: \ref{order partial order on sub
    class}] and
    \[ f : \langle A, \leqslant_A \rangle \rightarrow \langle D, \leqslant_B
       \rangle \text{ and } g : \langle B, \leqslant_B \rangle \rightarrow
       \langle C, \leqslant_C \rangle \text{ are strictly increasing} \]
    then
    \[ g \circ f : \langle A, \leqslant_A \rangle \rightarrow \langle C,
       \leqslant_C \rangle \text{ is stritly increasing} \]
    \item If $D \subseteq B$ is equiped with the induced order from $\langle
    B, \leqslant_B \rangle$ [see theorem: \ref{order partial order on sub
    class}] and
    \[ f : \langle A, \leqslant_A \rangle \rightarrow \langle D, \leqslant_B
       \rangle \text{ and } g : \langle B, \leqslant_B \rangle \rightarrow
       \langle C, \leqslant_C \rangle \text{ are order isomorphism} \]
    then
    \[ g \circ f : \langle A, \leqslant_A \rangle \rightarrow \langle g (f
       (A)), \leqslant_C \rangle \text{ is a order isomorphism} \]
    or as $D \equallim_{f : a \rightarrow D \text{ is bijective}} f (A)$
    \[ g \circ f : \langle A, \leqslant_A \rangle \rightarrow \langle g (D),
       \leqslant_C \rangle \text{ is a order isomorphism} \]
  \end{enumerate}
\end{theorem}

\begin{proof}
  
  \begin{enumerate}
    \item Let $x, y \in A$ with $x \leqslant_A y$ then $f (x) \leqslant f_B
    (y)$ hence $(g \circ f) (x) = g (f (x)) \leqslant_C g (f (y)) = (g \circ
    f) ()$.
    
    \item Let $x, y \in A$ with $x <_A y$ then $f (x) <_B f (y)$ hence $(g
    \circ f) (x) = g (f (x)) <_C g (f (y)) = (g \circ f) ()$.
    
    \item Using [theorem: \ref{function composition injectivity, surjectivity
    and bijectivity}] we have that $g \circ f : A \rightarrow g (D) = g (f
    (A))$ is a bijection. Let $x, y \in A$. If $x \leqslant_A y$ then $f (x)
    \leqslant_B f (y)$ hence $(g \circ f) (x) = g (f (x)) \leqslant_C g (f
    (y)) = (g \circ f) (y)$. Also if $(g \circ f) (x) \leqslant_C (g \circ f)
    (y)$ then $g (f (x)) \leqslant_C g (f (y))$ so that $f (x) \leqslant_B f
    (y)$, giving $x \leqslant_A y$.
  \end{enumerate}
\end{proof}

\begin{theorem}
  \label{order isomorphism strictly}If $\langle A, \leqslant_A \rangle$ and
  $\langle B, \leqslant_B \rangle$ are partially ordered classes and
  \[ f : \langle A, \leqslant_A \rangle \rightarrow \langle B, \leqslant_B
     \rangle \text{ a order isomorphism} \]
  then $\forall x, y \in A$ we have
  \[ x <_A y \Leftrightarrow f (x) <_B f (y) \]
\end{theorem}

\begin{proof}
  
  \begin{description}
    \item[$\Rightarrow$] If $x <_A y$ then $x \neq y$ and $x \leqslant_A y
    \Rightarrow f (x) \leqslant_B f (y)$. Assume that $f (x) = f (y)$ then as
    $f$ is a bijection we would have $x = y$ contradicting $x \neq y$. So we
    must have that $f (x) \neq f (y)$ hence
    \[ f (x) <_B f (y) \]
    \item[$\Leftarrow$] As $f (x) <_B f (y)$ we have that $f (x) \neq f (y)$
    so that we must have $x \neq y$. Further as $f$ is a isomorphism we have
    $x \leqslant_A y$. So
    \[ x <_A y \]
  \end{description}
\end{proof}

\begin{theorem}
  \label{order condition for isomorphism}If $\langle A, \leqslant_A \rangle$
  and $\langle B, \leqslant_B \rangle$ are partially ordered classes and $f :
  A \rightarrow B$ a bijection then
  \[ f : \langle A, \leqslant_A \rangle \rightarrow \langle B, \leqslant_B
     \rangle \text{ is a order isomorphism } \Leftrightarrow \text{ } f :
     \langle A, \leqslant_A \rangle \rightarrow \langle B, \leqslant_B \rangle
     \text{ and } f^{- 1} : \langle B, \leqslant_B \rangle \rightarrow \langle
     A, \leqslant_A \rangle \text{ are increasing functions} \]
\end{theorem}

\begin{proof}
  As $f : A \rightarrow B$ is a bijection we have by [theorems: \ref{function
  bijection has a inverse}, \ref{function bijection and inverse}] that $f^{=
  1} : B \rightarrow A$ is a bijection.
  \begin{description}
    \item[$\Rightarrow$] As $f : \langle A, \leqslant_A \rangle \rightarrow
    \langle B, \leqslant_B \rangle$ is a isomorphism we have that $\forall x,
    y \in A$ with $x \leqslant_A y \Rightarrow f (x) \leqslant f (b)$ hence $f
    : A \rightarrow B$ is increasing. If $x, y \in B$ with $x \leqslant_B y$
    then
    \[ f (f^{- 1} (x)) = (f \circ f^{- 1}) (x) \equallim_{\text{[theorem:
       \ref{function bijection f,f-1}}} x \leqslant_B y = (f \circ f^{- 1})
       (y) = f (f^{- 1} (y)) \]
    which as $f$ is a isomorphism proves that $f^{- 1} (x) \leqslant_A f^{- 1}
    (y)$, hence $f^{- 1}$ is increasing.
    
    \item[$\Leftarrow$] Suppose that $f, f^{- 1}$ are increasing functions
    then if $x \leqslant_A y \Rightarrowlim_{f \tmop{is} \tmop{increasing}} f
    (x) \leqslant_B f (y)$. Further if $f (x) \leqslant_B f (y)
    \Rightarrowlim_{f^{- 1} \tmop{is} \tmop{increasing}} f^{- 1} (f (x))
    \leqslant_A f^{- 1} (f (y)) \Rightarrow x \leqslant y$
  \end{description}
\end{proof}

\begin{theorem}
  \label{order isomorphism condition (2)}If $\langle A, \leqslant_A \rangle,
  \langle C, \leqslant_C \rangle$ and $\langle B, \leqslant_B \rangle$ are
  partially ordered classes then
  \begin{enumerate}
    \item $1_A : \langle A, \leqslant_A \rangle \rightarrow \langle A,
    \leqslant_A \rangle$ is a order isomorphism
    
    \item If $f : \langle A, \leqslant_A \rangle \rightarrow \langle B,
    \leqslant_B \rangle$ is a order isomorphism then $f^{- 1} : \langle B,
    \leqslant_B \rangle \rightarrow \langle A, \leqslant_A \rangle$ is a order
    isomorphism
    
    \item If $f : \langle A, \leqslant_A \rangle \rightarrow B$ and $g :
    \langle B, \leqslant_B \rangle \rightarrow \langle C, \leqslant_C \rangle$
    are order isomorphism's then
    \[ g \circ f : \langle A, \leqslant_A \rangle \rightarrow \langle C,
       \leqslant_C \rangle \text{ is a order isomorphism} \]
  \end{enumerate}
\end{theorem}

\begin{proof}
  
  \begin{enumerate}
    \item By \ref{function identity function} we have that $\tmop{Id}_A : A
    \rightarrow A$ is a bijection then, as $x = I_A (x)$ and $y = \tmop{Id}_A
    (y)$, we have $x \leqslant y \Leftrightarrow \tmop{Id}_A (x) \leqslant
    \tmop{Id}_A (y)$.
    
    \item If $f : A \rightarrow B$ is a isomorphism then by [theorem:
    \ref{function bijection and inverse}] we have that $f^{- 1} : B
    \rightarrow A$ is a bijection. By the previous theorem [theorem:
    \ref{order condition for isomorphism}] we have that $f^{- 1}$ is
    increasing. Further as by \ref{function inverse of a bijection is unique}
    $f = (f^{- 1})^{- 1}$ and by [theorem: \ref{order condition for
    isomorphism}] $f$ is increasing it follows that $(f^{- 1})^{- 1}$ is
    increasing. Using then [theorem: \ref{order condition for isomorphism}] it
    follows that $f^{- 1}$ is a isomorphism.
    
    \item This follows from [theorem: \ref{order composition of functions}]
  \end{enumerate}
\end{proof}

\begin{theorem}
  \label{order properties of the isomorph relation}If $\langle A, \leqslant_A
  \rangle$, $\langle B, \leqslant_B \rangle$ and $\langle C, \leqslant_C
  \rangle$ are partially ordered classes then we have
  \begin{enumerate}
    \item $A \cong A$
    
    \item If $A \cong B$ then $B \cong A$
    
    \item If $A \cong B$ and $B \cong D$ then $B \cong D$
  \end{enumerate}
\end{theorem}

\begin{proof}
  This follows easily from the previous theorem [theorem: \ref{order
  isomorphism condition (2)}]
\end{proof}

\begin{theorem}
  \label{order condition for isomorphism in a totallu ordered set}Let $\langle
  A, \leqslant_A \rangle$. be a totally ordered class and $\langle B,
  \leqslant_B \rangle$ is a partially ordered class then a bijective and
  increasing function $f : \langle A, \leqslant_A \rangle \rightarrow \langle
  B, \leqslant_B \rangle$ is a isomorphism
\end{theorem}

\begin{proof}
  Suppose that $f (x) \leqslant_B f (y)$ then since $A$ is fully ordered we
  have that $x, y$ are comparable therefore by [theorem: \ref{order
  comparable}] we have the following exclusive cases
  \begin{enumerate}
    \item $x \leqslant_A y$ in this case our theorem is proved
    
    \item $y <_A x$ in this case we would have $f (y) \leqslant_B f (x)
    \Rightarrow f (y) = f (x) \Rightarrowlim_{f \tmop{is} \tmop{injective}} x
    = y$ a contradiction. So this case does not occurs.
  \end{enumerate}
\end{proof}

\subsection{Min, max, supremum and infinum}

\

\begin{definition}
  \label{order maximal minimal element}{\index{maximal
  element}}{\index{minimal element}}Let $\langle X, \leqslant \rangle$ be a
  pre-ordered class and $A \subseteq X$ then
  \begin{enumerate}
    \item $m$ is a \tmtextbf{maximal element} of $A$ iff $m \in A$ and if
    $\forall x \in A$ with $m \leqslant x$ we have $x = m$
    
    \item $m$ is a \tmtextbf{minimal element} of $A$ iff $m \in A$ and if
    $\forall x \in A$ with $x \leqslant m$ we have $x = m$
  \end{enumerate}
\end{definition}

\begin{definition}
  \label{order greatest lowest element}{\index{greatest
  element}}{\index{lowest element}}If $\langle X, \leqslant \rangle$ is a
  partial ordered class and $A \subseteq X$ then
  \begin{enumerate}
    \item $m$ is the \tmtextbf{greatest element }of $A$ iff $m \in A$ and
    $\forall x \in A$ we have $x \leqslant m$
    
    \item $m$ is the \tmtextbf{least element} of $A$ iff $m \in A$ and
    $\forall x \in A$ we have $m \leqslant x$
  \end{enumerate}
\end{definition}

\begin{note}
  There is a subtle difference between the definition of a maximal (minimal)
  element and the greatest (least) element. If $m$ is the greatest (least)
  element of $A$ then every element in $A$ is comparable with $m$, which is
  not the case if $m$ is a maximal (minimal) element of $A$.
\end{note}

\begin{note}
  The empty set $\varnothing$ can not have a maximal, minimal element,
  greatest element or least element.
\end{note}

\begin{theorem}
  \label{order greatest and lowest element are unique}{\index{$\max
  (A)$}}{\index{$\min (A)$}}If $\langle X, \leqslant \rangle$ is a partial
  ordered class and $A \subseteq X$ then
  \begin{enumerate}
    \item If $m, m'$ are greatest elements of $A$ then $m = m'$
    
    \item If $m, m'$ are least elements of $A$ then $m = m'$
  \end{enumerate}
  The unique greatest element of $A$ (if it exist) is called the maximum of
  $A$ and noted as $\max (A)$, the unique least element of $A$ (if it exist)
  is called the minimum of $A$ and noted as $\min (A)$
\end{theorem}

\begin{proof}
  
  \begin{enumerate}
    \item If $m, m'$ are greatest elements of $A$ then as $m, m' \in A$ we
    have $m \leqslant m' \wedge m' \leqslant m$ so that $m = m'$.
    
    \item If $m, m'$ are least elements of $A$ then as $m, m' \in A$ we have
    $m \leqslant m' \wedge m' \leqslant m$ so that $m = m'$.
  \end{enumerate}
\end{proof}

\begin{theorem}
  \label{order min(A)<less>=max(A)}If $\langle X, \leqslant \rangle$ is a
  partial ordered class and $A \subseteq X$ such that $\min (A)$ and $\max
  (A)$ exist then $\min (A) \leqslant \max (A)$
\end{theorem}

\begin{proof}
  As $\min (A) \in A$ we have by definition that $\min (A) \leqslant \max
  (A)$.
\end{proof}

\begin{theorem}
  \label{order maximum of class with bigger elements}Let $\langle X, \leqslant
  \rangle$ be a partial ordered class, $A \subseteq X$, $B \subseteq X$ then
  \begin{enumerate}
    \item If $\max (A)$ and $\max (B)$ exist and $\forall x \in A$ $\exists y
    \in B$ such that $x \leqslant y$ then $\max (A) \leqslant \max (B)$
    
    \item If $\min (A)$ and $\min (B)$ exist $\forall x \in B$ $\exists y \in
    A$ such that $y \leqslant x$ then then $\min (A) \leqslant \min (B)$
  \end{enumerate}
\end{theorem}

\begin{proof}
  
  \begin{enumerate}
    \item As $\max (A) \in A$ there exist a $y \in B$ such that $\max (A)
    \leqslant y$, as $y \leqslant \max (B)$ we have
    \[ \max (A) \leqslant \max (B) \]
    \item As $\min (B) \in A$ there exist a $y \in A$ such that $y \leqslant
    \min (B)$, as $\min (A) \leqslant y$ we have
    \[ \min (A) \leqslant \max (A) \]
  \end{enumerate}
\end{proof}

\begin{definition}
  \label{order upport lower bound}{\index{upper bound}}{\index{lower bound}}If
  $\langle X, \leqslant \rangle$ is a partial ordered class and $A \subseteq
  X$ then
  \begin{enumerate}
    \item $u \in X$ is a \tmtextbf{upper bound} of $A$ if $\forall a \in A$ $a
    \leqslant u$.
    
    \item $A$ is \tmtextbf{bounded above} if it has a upper bound.
    
    \item $l \in X$ is a \tmtextbf{lower bound} of $A$ if $\forall x \in A$ $l
    \leqslant a$
    
    \item $A$ is \tmtextbf{bounded below} if it has a lower bound.
    
    \item $\upsilon (A) = \left\{ x \in X|x \text{ is a upper bound of } A
    \right\}$ [the class of upper bound of $A$]
    
    \item $\lambda (A) = \left\{ x \in X|x \text{ is a lower bound of } A
    \right\}$ [the class of lower bounds of $A$]
  \end{enumerate}
\end{definition}

\begin{example}
  \label{order lower upper bounds of empty set}If $\langle X, \leqslant
  \rangle$ then $\upsilon (\varnothing) = X$ and $\lambda (\varnothing) = X$
\end{example}

\begin{proof}
  Let $x \in X$ then as $\forall a \in \varnothing$ $a \leqslant x$ [or $x
  \leqslant a$] is vacuously satisfied $X \subseteq \upsilon (A)$ and $X
  \subseteq \lambda (A)$, \ which as $\upsilon (X) \subseteq X$ and $\lambda
  (X) \subseteq X$ proves $\upsilon (A) = X = \lambda (A)$.
\end{proof}

\begin{definition}
  \label{order supremum
  infinum}{\index{supremum}}{\index{infinum}}{\index{$\inf (A)$}}{\index{$\sup
  (A)$}}If $\langle X, \leqslant \rangle$ is a partial ordered class and $A
  \subseteq X$ then
  \begin{enumerate}
    \item If $\min (\upsilon (A))$ exists then $\min (\upsilon (A))$ is called
    the supremum of $A$ and noted as $\sup (A)$.
    
    \item If $\max (\lambda (A))$ exists then $\max (\lambda (A))$ is called
    the infinum of $A$ and noted as $\inf (A)$
  \end{enumerate}
\end{definition}

In other words if $\upsilon (A)$ has a least element then the supremum of $A$
is this unique, by [theorem: \ref{order greatest and lowest element are
unique}], element. So $\sup (A)$ is the least upper bound of $A$ [if it exist]
and it is itself a upper bound. If $\lambda (A)$ has a least element then the
infinum of $A$ is this unique, by [theorem: \ref{order greatest and lowest
element are unique}], element. So $\inf (A)$ is the greatest lower bound [if
it exist] and it is itself a lower bound.

\begin{example}
  \label{order example inclusion order and sup, inf}Let $\mathcal{A}$ be a
  class of classes and $\langle \mathcal{A}, \leqslant \rangle$ the partial
  class where
  \[ \leqslant = \{ (x, y) \in \mathcal{A} \times \mathcal{A}|x \subseteq y \}
  \]
  [see example: \ref{order inclusion is a order}] and $\mathcal{B} \subseteq
  \mathcal{A}$ we have that
  \begin{enumerate}
    \item If $\bigcap \mathcal{B} \in \mathcal{A}$ then $\inf (\mathcal{B})$
    exist and $\inf (\mathcal{B}) = \bigcap \mathcal{B}$
    
    \item If $\bigcup \mathcal{B} \in \mathcal{A} \text{ then }$ $\sup
    (\mathcal{B})$ exist and $\sup (\mathcal{B}) = \bigcup \mathcal{B}$
  \end{enumerate}
\end{example}

\begin{proof}
  
  \begin{enumerate}
    \item If $B \in \mathcal{B}$ then by [theorem: \ref{class general
    intersection}] $\bigcap \mathcal{B} \subseteq B \Rightarrow \bigcap
    \mathcal{B} \leqslant B$ so that $\bigcap \mathcal{B} \in \lambda
    (\mathcal{B})$. Now if $C \in \lambda (\mathcal{B})$ then $\forall B \in
    \mathcal{B}$ we have that $C \leqslant B \Rightarrow C \subseteq B$, so
    that by [theorem: \ref{class general intersection}] we have $C \subseteq
    \bigcap \mathcal{B} \Rightarrow C \leqslant \bigcap \mathcal{B}$ so that
    $\bigcap \mathcal{B}$ is the greatest element of $\lambda (\mathcal{B})$
    proving that $\inf (\mathcal{B})$ exists and $\inf (\mathcal{B}) = \bigcap
    \mathcal{B}$.
    
    \item If $B \in \mathcal{B}$ then by [theorem: \ref{class general
    intersection}] $B \subseteq \bigcup \mathcal{B} \Rightarrow B \leqslant
    \bigcup \mathcal{B}$ so that $\bigcup \mathcal{B} \in \upsilon
    (\mathcal{B})$. Now if $C \in \upsilon (\mathcal{B})$ then $\forall B \in
    \mathcal{B}$ we have that $B \leqslant C \Rightarrow B \subseteq C$, so
    that by [theorem: \ref{class general intersection}] we have $\bigcup
    \mathcal{B} \subseteq C \Rightarrow \bigcup \mathcal{B} \leqslant C$ so
    that $\bigcup \mathcal{B}$ is the lowest element of $\upsilon
    (\mathcal{B})$ proving that $\sup (\mathcal{B})$ exists and $\sup
    (\mathcal{B}) = \bigcup \mathcal{B}$.
  \end{enumerate}
\end{proof}

The following theorem will be used a lot of time when dealing with supremums
and infinums.

\begin{theorem}
  \label{order sup, inf property}Let $\langle X, \leqslant \rangle$ be a
  totally ordered set and $A \subseteq X$ then
  \begin{enumerate}
    \item If $\sup (A)$ exists then $\forall x \in X$ with $x < \sup (A)$
    there $\exists a \in A$ such that $x < a \wedge a \leqslant \sup (A)$
    
    \item If $\inf (A)$ exist then $\forall x \in X$ with $\inf (A) < x$ there
    $\exists a \in A$ such that $\inf (A) \leqslant a \wedge a < x$
  \end{enumerate}
\end{theorem}

\begin{proof}
  First as $\langle X, \leqslant \rangle$ is totally ordered we have $\forall
  x, y \in X$ that $x, y$ are comparable, hence by [theorem: \ref{order
  comparable property}], we have $x \leqslant y \wedge y < x$/
  \begin{enumerate}
    \item Let $x \in X$ such that $x < \sup (A)$. Assume that $\forall a \in
    A$ we have $\neg (x < a)$ so that $a \leqslant x$, so $x$ is a upper bound
    of $A$, hence $x \in \upsilon (A)$, so that $\sup (A) = \min (\upsilon
    (A)) \leqslant x$, which, as $x < \sup (A),$leads to the contradiction $x
    < x$. So we must have that $\exists a \in A$ such that $x < a$, further as
    $\sup (A)$ is a upper bound we have that $a \leqslant \sup (A)$. So
    \[ \exists a \in A \text{ } x < a \wedge a \leqslant \sup (A) \]
    \item Let $x \in X$ such that $\inf (A) < x$. Assume that $\forall a \in
    A$ we have $\neg (a < x)$ so that $x \leqslant a$, so $x$ is a lower bound
    of $A$, hence $x \in \lambda (A)$, so that $x \leqslant \max (\lambda (A))
    = \inf (A)$, which, as $\inf (A) < x$, leads to the contradiction $x < x$.
    So we must have that $\exists a \in A$ such that $a \leqslant x$, further
    as $\inf (A)$ is a lower bound we have we have that $\inf (A) \leqslant
    a$. So
    \[ \exists a \in A \text{ } \inf (A) \leqslant a \wedge a < x \]
  \end{enumerate}
\end{proof}

\begin{lemma}
  \label{order inclusion and greatest and least element}If $\langle X,
  \leqslant \rangle$ is a partially ordered class and $A \subseteq X, B
  \subseteq X$ with $A \subseteq B$ then
  \begin{enumerate}
    \item If $\max (A)$ and $\max (B)$ exist then $\max (A) \leqslant \max
    (B)$
    
    \item If $\min (A)$ and $\min (B)$ exists then $\min (B) \leqslant \min
    (A)$
  \end{enumerate}
\end{lemma}

\begin{proof}
  
  \begin{enumerate}
    \item As $\max (A) \in A$ and $A \subseteq B$ we have that $\max (A) \in
    B$ so that $\max (A) \leqslant \max (B)$
    
    \item As $\min (A) \in A$ and $A \subseteq B$ we have that $\min (A) \in
    B$ so that $\min (B) \leqslant \min (A)$
  \end{enumerate}
\end{proof}

\begin{lemma}
  \label{order lower upper bound and inclusion}If $\langle X, \leqslant
  \rangle$ is a partially ordered class and $A \subseteq X, B \subseteq X$
  with $A \subseteq B$ then
  \begin{enumerate}
    \item $\upsilon (B) \subseteq \upsilon (A)$
    
    \item $\lambda (B) \subseteq \lambda (A)$
  \end{enumerate}
\end{lemma}

\begin{proof}
  
  \begin{enumerate}
    \item Let $x \in \upsilon (B)$ then $\forall a \in A$ we have, as $A
    \subseteq B$ that $a \in B$ hence $a \leqslant x$ proving that $x$ is a
    upper bound of $A$ or $x \in \upsilon (A)$.
    
    \item Let $x \in \lambda (B)$ then $\forall a \in A$ we have as $A
    \subseteq B$ hat $a \in B$ hence $x \leqslant a$ proving that $x$ is a
    lower bound of $A$ or $x \in \lambda (A)$.
  \end{enumerate}
\end{proof}

\begin{theorem}
  \label{order sup,inf and inclusion}Let $\langle X, \leqslant \rangle$ be a
  partial ordered class and \ $A \subseteq X$, $B \subseteq Y$ such that $A
  \subseteq B$ then
  \begin{enumerate}
    \item If $\sup (A)$ and $\sup (B)$ exist then $\sup (A) \leqslant \sup
    (B)$
    
    \item If $\inf (A)$ and $\inf (B)$ exist then $\inf (B) \leqslant \inf
    (A)$
  \end{enumerate}
\end{theorem}

\begin{proof}
  
  \begin{enumerate}
    \item Using [lemma: \ref{order lower upper bound and inclusion}] we have
    that $\upsilon (B) \subseteq \upsilon (A)$ so that by [lemma: \ref{order
    inclusion and greatest and least element}]
    \[ \sup (A) = \min (\upsilon (A)) \leqslant \min (\upsilon (B)) = \sup (B)
    \]
    \item Using [lemma: \ref{order lower upper bound and inclusion}] we have
    that $\lambda (B) \subseteq \lambda (A)$ so that by [lemma: \ref{order
    inclusion and greatest and least element}]
    \[ \inf (B) = \max (\lambda (B)) \leqslant \max (\lambda (A)) = \inf (A)
    \]
  \end{enumerate}
\end{proof}

\begin{theorem}
  \label{order sup and inf and bigger elements}Let $\langle X, \leqslant
  \rangle$ be a partial ordered class and $A \subseteq X, B \subseteq X$ then
  \begin{enumerate}
    \item If $\sup (A)$, $\sup (B)$ exists and $\forall a \in A$ $\exists b
    \in B \text{}$ such that $a \leqslant b$ then $\sup (A) \leqslant \sup
    (B)$
    
    \item If $\inf (A)$ and $\inf (B)$ exist and $\forall a \in A$ $\exists b
    \in B$ such that $b \leqslant a$ then $\inf (B) \leqslant \inf (A)$
  \end{enumerate}
\end{theorem}

\begin{proof}
  
  \begin{enumerate}
    \item Let $a \in A$ then $\exists b \in B$ such that $a \leqslant b$, as
    $b \leqslant \sup (B)$ it follows that $a \leqslant \sup (B)$. Hence $\sup
    (B) \in \upsilon (A)$. So $\sup (A) = \min (\upsilon (A)) \leqslant \sup
    (A)$, hence
    \[ \sup (A) \leqslant \sup (B) \]
    \item Let $a \in A$ then $\exists b \in B$ such that $b \leqslant a$, as
    $\inf (B) \leqslant b$ it follows that $\inf (B) \leqslant a$. Hence $\inf
    (B) \in \lambda (A)$, So $\inf (B) \leqslant \max (\lambda (A)) = \inf
    (A)$, hence
    \[ \inf (B) \leqslant \inf (A) \]
  \end{enumerate}
\end{proof}

We have by definition that $\sup (A)$ exists if $\min (\upsilon (A))$ exists
and $\inf (A)$ exist if $\max (\lambda (A))$ exist. The following theorem
shows that there is a weaker condition for the existence of $\sup (A)$ and
$\inf (A)$.

\begin{theorem}
  \label{order sup inf condition}Let $\langle X, \leqslant \rangle$ be a
  partial ordered class and $A \subseteq X$ then
  \begin{enumerate}
    \item If $\lambda (A)$ has a supremum then $A$ has a infinum and $\sup
    (\lambda (A)) = \inf (A)$
    
    \item If $\upsilon (A)$ has a infinum then $A$ has a supremum and $\inf
    (\upsilon (A)) = \sup (A)$
  \end{enumerate}
\end{theorem}

\begin{proof}
  
  \begin{enumerate}
    \item If $a \in A$ then $\forall y \in \lambda (A)$ we have $y \leqslant
    a$ so that $a \in \upsilon (\lambda (A))$. As $\sup (\lambda (A)) = \min
    (\upsilon (\lambda (A)))$ we have that $\sup (\lambda (A)) \leqslant a$.
    As $a \in A$ was arbitrary chosen we have that
    \begin{equation}
      \label{eq 3.8.011} \sup (\lambda (A)) \in \lambda (A)
    \end{equation}
    If $x \in \lambda (A)$, then, as $\sup (\lambda (A))$ is a upper bound of
    $\lambda (A)$, we have $x \leqslant \sup (\lambda (A))$. So
    \begin{equation}
      \label{eq 3.9.011} \forall x \in \lambda (A) \text{ we have $x \leqslant
      \sup (\lambda (A))$}
    \end{equation}
    Using [eq: \ref{eq 3.8.011}] and [eq: \ref{eq 3.9.011}] it follows that
    $\sup (\lambda (A)) = \max (\lambda (A)) = \inf (A)$ or
    \[ \sup (\lambda (A)) = \inf (A) \]
    \item If $a \in A$ then $\forall y \in \upsilon (A)$ we have $a \leqslant
    y$ so that $a \in \lambda (\upsilon (A))$. As $\inf (\upsilon (A)) = \max
    (\lambda (\upsilon (A)))$ we have that $a \leqslant \inf (\upsilon (A))$.
    As $a \in A$ was arbitrary chosen we have that
    \begin{equation}
      \label{eq 3.10.012} \inf (\upsilon (A)) \in \upsilon (A)
    \end{equation}
    If $x \in \upsilon (A)$, then, as $\inf (\upsilon (A))$ is a lower bound
    of $\upsilon (A)$, we have $\inf (\upsilon (A)) \leqslant x$. So we have
    that
    \begin{equation}
      \label{eq 3.11.012} \forall x \in \upsilon (A) \text{ we have that $\inf
      (\upsilon (A)) \leqslant x$}
    \end{equation}
    Using [eq: \ref{eq 3.10.012}] and [eq: \ref{eq 3.11.012}] it follows that
    $\inf (\upsilon (A)) = \min (\upsilon (A)) = \sup (A)$ or
    \[ \inf (\upsilon (A)) = \sup (A) \]
  \end{enumerate}
\end{proof}

In general it is not guaranteed that $\sup (A)$ or $\inf (A)$ exists. However
there exists partial order classes that guarantees the existence of a supremum
for non empty sub-classes that are bounded above.

\begin{definition}[Conditional Completeness]
  \label{order conditional complete order}{\index{conditional completeness}}A
  partial ordered class $\langle X, \leqslant \rangle$ is
  \tmtextbf{conditional complete }if every non empty sub-class of $A$ that is
  bounded above has a supremum.
\end{definition}

The next theorem shows that conditional completeness can also be defined based
on bounded below and infinum.

\begin{theorem}
  \label{order conditional complete alternatives}If $\langle A, \leqslant
  \rangle$ is a partial ordered class then the following are equivalent
  \begin{enumerate}
    \item Every non empty sub-class of $X$ that is bounded above has a
    supremum [$\langle X, \leqslant \rangle$ is conditional complete]
    
    \item Every non empty sub-class of $X$ that is bounded below has a infinum
    
  \end{enumerate}
\end{theorem}

\begin{proof}
  
  \begin{description}
    \item[$1 \Rightarrow 2$] Let $A \subseteq X$ a non empty sub-class that is
    bounded below. As $A \neq \varnothing$ there exists a $a \in A$, further
    by definition of $\lambda (A)$ we have $\forall y \in \lambda (A)$ that $y
    \leqslant a$ so $\lambda (A)$ is bounded above. As $A$ is bounded below we
    have that $\lambda (A) \neq \varnothing$. So by the hypothesis $\sup
    (\lambda (A))$ exist. Applying then [theorem: \ref{order sup inf
    condition}] proves
    \[ \inf (A) \text{ exist} \]
    \item[$2 \Rightarrow 1$] Let $A \subseteq X$ a non empty sub-class that is
    bounded above. As $A \neq \varnothing$ there exists a $a \in A$, further
    by definition of $\upsilon (A)$ we have $\forall y \in \upsilon (A)$ that
    $a \leqslant y$ so $\upsilon (A)$ is bounded below. As $A$ is bounded
    above we have that $\upsilon (A) \neq \varnothing$. So by the hypothesis
    $\inf (\upsilon (A))$ exist. Applying then [theorem: \ref{order sup inf
    condition}] proves
    \[ \sup (A) \text{ exist} \]
  \end{description}
\end{proof}

Next we show that a order isomorphism preserves the concepts of greatest
element, least element, upper bound, lower bound, supremum and infinum.

\begin{lemma}
  \label{order isomorphism preservers sup and inf}Let $\langle X, \leqslant_X
  \rangle$, $\langle Y, \leqslant_Y \rangle$ be partial ordered classes, $f :
  \langle X, \leqslant_X \rangle \rightarrow \langle Y, \leqslant_Y \rangle$
  is a order isomorphism, $A \subseteq X$ and $B \subseteq Y$ then
  \begin{enumerate}
    \item If $u$ is a upper bound of $B$ then $(f^{- 1}) (u)$ is a upper bound
    of $f^{- 1} (B)$
    
    \item If $l$ is a lower bound of $B$ then $(f^{- 1}) (l)$ is a lower bound
    of $f^{- 1} (B)$
    
    \item If $u$ is a upper bound of $A$ then $f (u)$ is a upper bound of $f
    (A)$
    
    \item If $l$ is a lower bound of $A$ then $f (u)$ is a lower bound of $f
    (A)$
    
    \item $f (\upsilon (A)) = \upsilon (f (A))$
    
    \item $f (\lambda (A)) = \lambda (f (A))$
    
    \item If $\max (A)$ exist then $\max (f (A))$ exist and $\max (f (A)) = f
    (\max (A))$
    
    \item If $\min (A)$ exist then $\min (f (A))$ exist and $\min (f (A)) = f
    (\min (A))$
    
    \item If $\sup (A)$ exist then $\sup (f (A))$ exist and $\sup (f (A)) = f
    (\sup (A))$
    
    \item If $\inf (A)$ exist then $\inf (f (A))$ exist and $\inf (f (A)) = f
    (\inf (A))$
  \end{enumerate}
\end{lemma}

\begin{proof}
  First using [theorem: \ref{order condition for isomorphism}] we have that $f
  : \langle X, \leqslant_X \rangle \rightarrow \langle Y, \leqslant_Y \rangle$
  and $f^{- 1} : \langle Y, \leqslant_Y \rangle \rightarrow \langle X,
  \leqslant_X \rangle$ are increasing.
  \begin{enumerate}
    \item Let $x \in f^{- 1} (B)$ then $\exists y \in B$ such that $y = f
    (x)$, as $u$ is a upper bound of $B$, we have that $y \leqslant_B u$. So
    $x \equallim_{\text{[theorem: \ref{function inverse function and f(x)}]}}
    (f^{- 1}) (f (x)) = (f^{- 1}) (y) \leqslant_A (f^{- 1}) (u)$, proving that
    $(f^{- 1}) (u)$ is a upper bound of $f^{- 1} (B)$.
    
    \item Let $x \in f^{- 1} (B)$ then $\exists y \in B$ such that $y = f
    (x)$, as $l$ is a lower bound of $B$ we have that $l \leqslant_B y$. So
    $(f^{- 1}) (l) \leqslant_A (f^{- 1}) (y) = (f^{- 1}) (f (x))
    \equallim_{\text{[theorem: \ref{function inverse function and f(x)}]}} x$,
    proving that $(f^{- 1}) (l)$ is a lower bound of $f^{- 1} (B)$.
    
    \item If $y \in f (A)$ then $\exists x \in A$ such that $y = f (x)$. As
    $u$ is a upper bound of $A$ we have that $x \leqslant_A u$, so $y = f (x)
    \leqslant_B f (u)$ proving that $f (u)$ is a upper bound of $f (A)$.
    
    \item If $y \in f (A)$ then $\exists x \in A$ such that $y = f (x)$, As
    $l$ is a lower bound of $A$ we have that $l \leqslant_A x$, so $f (l)
    \leqslant_B f (x) = y$ proving that $f (l)$ is a lower bound of $f (A)$.
    
    \item  If $y \in f (\upsilon (A))$ then there $\exists x \in \upsilon (A)$
    such that $y = f (x)$. As $x \in \upsilon (A)$, $x$ is a upper bound of
    $B$, so that by (3) $y = f (x)$ is a upper bound of $f (A)$. Hence
    \begin{equation}
      \label{eq 3.12.012} f (\upsilon (A)) \subseteq \upsilon (f (A))
    \end{equation}
    If $y \in v (f (A))$ then by (1) $(f^{- 1}) (y)$ is a upper bound of $f^{-
    1} (f (A)) \equallim_{\text{[theorem: \ref{function preimage of image}]}}
    A$ so that $(f^{- 1}) (y) \in \upsilon (A)$. So $y
    \equallim_{\text{[theorem: \ref{function inverse function and f(x)}}} f
    ((f^{- 1}) (y)) = y \in f (\upsilon (A))$. Hence $\upsilon (f (A))
    \subseteq f (\upsilon (A))$ which combined with [eq: \ref{eq 3.12.012}]
    proves
    \[ f (\upsilon (A)) = \upsilon (f (A)) \]
    \item If $y \in f (\lambda (A))$ then there $\exists x \in \lambda (A)$
    such that $y = f (x)$. As $x \in \lambda (A)$, $x$ is a lower bound of
    $A$, so that by (4) $y = f (x)$ is a lower bound of $f (A)$. Hence
    \begin{equation}
      \label{eq 3.14.012} f (\lambda (A)) \subseteq \lambda (f (A))
    \end{equation}
    If $y \in \lambda (f (A))$ then by (2) $(f^{- 1}) (y)$ is a lower bound of
    $f^{- 1} (f (A)) \equallim_{\text{[theorem: \ref{function preimage of
    image}]}} A$ so that $(f^{- 1}) (y) \in \lambda (A)$. So $y
    \equallim_{\text{[theorem: \ref{function inverse function and f(x)}}} f
    ((f^{- 1}) (y)) = y \in f (\lambda (A))$. Hence $\lambda (f (A)) \subseteq
    f (\lambda (A))$ which combined with [eq: \ref{eq 3.12.012}] proves
    \[ f (\lambda (A)) = \lambda (f (A)) \]
    \item If $\max (A)$ exist then $\max (A) \in A$ giving $f (\max (A)) \in f
    (A)$. Let $y \in f (A)$ then $\exists x \in A$ such that $y = f (x)$, as
    $\max (A)$ exist we have $x \leqslant_A \max (A)$ so that $y = f (x)
    \leqslant_B f (\max (A))$. So
    \[ \max (f (A)) \text{ exist and } \max (f (A)) = f (\max (A)) \]
    \item If $\min (A)$ exist then $\min (A) \in A$ giving $f (\min (A)) \in f
    (A)$. Let $y \in f (A)$ then $\exists x \in A$ such that $y = f (x)$, as
    $\min (A)$ exist we have $\min (A) \leqslant_A x$ so that $f (\min (A))
    \leqslant_B f (x) = y$. So
    \[ \min (f (A)) \text{ exist and } \min (f (A)) = f (\min (A)) \]
    \item If $\sup (A)$ exists then $\min (\upsilon (A))$ exists and $\sup (A)
    = \min (\upsilon (A))$. Using (8) $\min (f (\upsilon (A)))$ exist, As $f
    (\upsilon (A)) \equallim_{(5)} \upsilon (f (A))$ we have that $\min
    (\upsilon (f (A)))$ exist and
    \[ \sup (f (A)) = \min (\upsilon (f (A))) \equallim_{(5)} \min (f
       (\upsilon (A))) \equallim_{(8)} f (\min (\upsilon (A))) = f (\sup (A))
    \]
    \item If $\inf (A)$ exists then $\max (\lambda (A))$ exists and $\inf (A)
    = \max (\lambda (A))$. Using (7) $\max (f (\lambda (A)))$ exist, As $f
    (\lambda (A)) \equallim_{(6)} \lambda (f (A))$ we have that $\max (\lambda
    (f (A)))$ exist and
    \[ \inf (f (A)) = \max (\lambda (f (A))) \equallim_{(6)} \max (f (\lambda
       (A))) \equallim_{(7)} f (\max (\lambda (A))) = f (\inf (A)) \]
  \end{enumerate}
\end{proof}

\begin{theorem}
  \label{order isomorphism and conditional complete}Let $\langle X,
  \leqslant_X \rangle$ be a conditional complete partial ordered set, $\langle
  Y, \leqslant_Y \rangle$ a partial ordered class and $f : \langle X,
  \leqslant_X \rangle \rightarrow \langle Y, \leqslant_Y \rangle$ a order
  isomorphism then $\langle Y, \leqslant_Y \rangle$ is conditionally complete.
\end{theorem}

\begin{proof}
  Let $A \subseteq Y$ be such that $A$ is bounded above and non empty. Let $u$
  be a upper bound of $A$ then by [lemma: \ref{order isomorphism preservers
  sup and inf}] we have that $(f^{- 1}) (u)$ is a upper bound of $f^{- 1}
  (A)$. As $A \neq \varnothing$ there exists a $a \in A$ which as $f$ is
  surjective means that $\exists x$ such that $a = f (x)$ hence $x \in f^{- 1}
  (A)$ proving that $f^{- 1} (A) \neq \varnothing$. As $\langle X, \leqslant_X
  \rangle$ is conditional complete $\sup (f^{- 1} (A))$ exist. Using \ [lemma:
  \ref{order isomorphism preservers sup and inf}] $\sup (f (f^{- 1} (A)))$
  exist which as $A \equallim_{\text{[theorem: \ref{function preimage of
  image}]}} f (f^{- 1} (A)) $ proves that $\sup (A)$ exist. So $\langle X,
  \leqslant_Y \rangle$ is conditional complete.
\end{proof}

\subsection{Well ordering}

\begin{definition}
  \label{order well-rodered class}{\index{well-ordered class}}A partial
  ordered class $\langle X, \leqslant \rangle$ is \tmtextbf{well ordered} is
  every non empty sub-class of $X$ has a least element. In other words if
  $\forall A \in \mathcal{P} (X)$ $\min (A)$ exist.
\end{definition}

\begin{theorem}
  \label{order well ordering and order isomorphism}If $\langle X, \leqslant_X
  \rangle$, $\langle Y, \leqslant_Y \rangle$ are partial ordered sets, $f :
  \langle X, \leqslant_X \rangle \rightarrow \langle Y, \leqslant_Y \rangle$ a
  order isomorphism then if $\langle X, \leqslant_X \rangle$ is well ordered
  $\langle Y, \leqslant_Y \rangle$ is well ordered.
\end{theorem}

\begin{proof}
  Let $A \subseteq Y$ be a non empty subclass of $Y$. Then $\exists a \in A$
  and as $f$ is a bijection there exist a $x \in X$ such that $y = f (x)$,
  from which it follows that $x \in f^{- 1} (A)$. So
  \[ f^{- 1} (A) \neq \varnothing \]
  As $\langle X, \leqslant_X \rangle$ is well ordered we have that $f^{- 1}
  (A)$ has a least element, hence
  \[ \exists m' \in f^{- 1} (A) \tmop{such} \tmop{that} \forall a \in f^{- 1}
     (A) \text{ we have } m' \leqslant_X a \]
  Take now $m = f (m')$ then as $m' \in f^{- 1} (A)$ we have that
  \begin{equation}
    \label{eq 3.14.026} m \in A
  \end{equation}
  Further if $a \in A$ then as $f$ is surjective there exists a $b \in X$ such
  that $a = f (b)$ or $b \in f^{- 1} (A)$, so that $m' \leqslant_X b$. As $f$
  is a order isomorphism we have $m = f (m') \leqslant_Y f (b) = a$. Hence we
  have proved that
  \begin{equation}
    \label{eq 3.15.026} \forall a \in A \text{ we have } m \leqslant a
  \end{equation}
  From [eq: \ref{eq 3.14.026}] and [eq: \ref{eq 3.15.026}] we conclude finally
  that $\langle Y, \leqslant_Y \rangle$ is well ordered.
\end{proof}

\begin{theorem}
  \label{order total/well-order inclusion}If $\langle X, \leqslant \rangle$ is
  a partial ordered class, $B \subseteq X$ then for $\langle B, \leqslant_{|B}
  \rangle$ [see theorem: \ref{order partial order on sub class}] we have
  \begin{enumerate}
    \item If $\langle X, \leqslant \rangle$ is totally ordered then $\langle
    B, \leqslant_{|B} \rangle$ is totally ordered
    
    \item If $\langle X, \leqslant \rangle$ is well ordered then $\langle B,
    \leqslant_{|B} \rangle$ is totally ordered
  \end{enumerate}
\end{theorem}

\begin{proof}
  
  \begin{enumerate}
    \item If $x, y \in B \Rightarrow x, y \in X$ hence $x \leqslant y \vee y
    \leqslant x$ so that $x \leqslant_{|B} y \vee y \leqslant_{|B} x$.
    
    \item If $C \subseteq B$ is a non empty class then as $B \subseteq X$ we
    have $\varnothing \neq C \subseteq X$. So there exists a least element $c$
    of $C$. So $c \in C$ and $\forall x \in C$ we have $c \leqslant x
    \Rightarrowlim_{x \in B} c \leqslant_{|B} x$ proving that $c$ is a least
    element of $C$ using the order relation $\leqslant_{|B}$.
  \end{enumerate}
\end{proof}

Well ordering is a stronger condition then conditional completeness and
totally ordering

\begin{theorem}
  \label{order well order implies conditional complete and totally
  ordering}Let $\langle X, \leqslant \rangle$ is a well ordered class then
  \begin{enumerate}
    \item $\langle X, \leqslant \rangle$ is totally ordered
    
    \item $\langle X, \leqslant \rangle$ is conditional complete
    
    \item $\forall x, y \in X$ we have $x \leqslant y$ or $y < x$
  \end{enumerate}
\end{theorem}

\begin{proof}
  
  \begin{enumerate}
    \item If $x, y \in X$ then $\{ x, y \}$ is a non empty sub-class of $X$
    and must have a least element. If $x$ is the least element then $x
    \leqslant y$ and if $y$ is the least element then $y \leqslant x$, so
    $\langle X, \leqslant \rangle$ is totally ordered.
    
    \item If $A$ is a non empty sub-class of $X$ that is bounded above then
    $\upsilon (A) \neq \varnothing$. Using well ordering we have that $\sup
    (A) = \min (\upsilon (A))$ exist.
    
    \item As by (1) $\langle X, \leqslant \rangle$ is totally ordered we have
    that $x \text{ and } y$ are comparable, hence by [theorem: \ref{order
    comparable property}] we have $x \leqslant y \vee y < x$.
  \end{enumerate}
\end{proof}

One difference between the order relation on the set of whole numbers
$\mathbb{Z}$ and the set of real numbers $\mathbb{R}$ is that there does not
exist a whole number between 1 and 2 while for the real numbers there is the
real number $1.5$ between $1$ and $2$. This leads to the following definition.

\begin{definition}[Immediate successor]
  \label{order immediate successor}{\index{immediate successor}}Let $\langle
  X, \leqslant \rangle$ be a partial ordered set and $x, y \in X$ then $y$ is
  the \tmtextbf{immediate} successor of $x$ iff
  \begin{enumerate}
    \item $x < y$
    
    \item $\neg \left( \exists z \in X \text{ such that } x < z \wedge z < y
    \right)$ [in words there does not exists a $x \in X$ such that $x < z <
    y$]
  \end{enumerate}
\end{definition}

\begin{theorem}
  \label{order well order and immediate successor}Let $\langle X, \leqslant
  \rangle$ be a well ordered class then every element that is not a greatest
  element of $X$ has a immediate successor.
\end{theorem}

\begin{proof}
  Using [theorem: \ref{order well order implies conditional complete and
  totally ordering}] we have that $\langle X, \leqslant \rangle$ is totally
  ordered. Let $x \in X$ such that $x$ is not a greatest element in $X$. Take
  $B = \{ y \in X|x < y \}$ then if $B = \varnothing$ we have that
  $X\backslash B = X$ so $\forall r \in X$ we have $r \nin B$ or $\neg (x <
  r)$, by [theorem: \ref{order well order implies conditional complete and
  totally ordering}] we have that $r \leqslant x$, proving that $x$ is a
  greatest element of $X$ which contradicts or hypothesis.. So we must have
  that $B \neq \varnothing$, by well ordering there exist a least element $b$
  of $B$, which as $b \in B$ gives $x < b$. Assume that there exist a $a \in
  X$ such that $x < a \wedge a < b$, then we must have that $a \in B$ and $a <
  b$. As $b$ is the least element of $B$ and $a \in B$ we have $b < a$ leading
  to the contradiction $a < a$. So $b$ is a immediate successor of $x$/
\end{proof}

\begin{definition}
  \label{order section}{\index{section}}Let $\langle X, \leqslant \rangle$ be
  a partial ordered class then $B \subseteq A$ is a \tmtextbf{section} of $X$
  if
  \[ \forall x \in X \tmop{we} \tmop{have} \text{ } \forall y \in B \text{
     with } x \leqslant y \text{ that } x \in B \]
\end{definition}

\begin{lemma}
  \label{order section and well ordering}Let $\langle X, \leqslant \rangle$ be
  a well ordered class and $B \subseteq X$ then
  \[ B \text{ is a section } \Leftrightarrow \text{ } B = X \text{ or } B
     \text{ is a initial segment of } X \text{ [definition: \ref{order initial
     segement}]} \]
\end{lemma}

\begin{proof}
  
  \begin{description}
    \item[$\Rightarrow$] Let $B$ be a section of $X$ then if $B = X$ we are
    done. So we must prove the theorem for $B \neq X$ or equivalently
    $X\backslash B \neq \varnothing$. Because $X$ is well ordered, there a
    exists a least element $l \in X\backslash B$. Consider the initial segment
    $S_{X, l} = \{ x \in X|x < l \}$ [see definition: \ref{order initial
    segement}]. Let $x \in S_{X, l}$ so that $x < l$. Assume that $x \nin B$
    then $x \in X\backslash B$ so, as $l$ is a least element of $X\backslash
    B$, we have $l \leqslant x$ which combined with $x < l$ leads to the
    contradiction $l < l$. So we must have that $x \in B$ which proves that
    \begin{equation}
      \label{eq 3.14.013} S_{X, l} \subseteq B
    \end{equation}
    Let $x \in B$, as $X$ is well ordered we have by [theorem: \ref{order well
    order implies conditional complete and totally ordering}] that $l
    \leqslant x \vee x < l$. Assume that $l \leqslant x$ then, as $B$ is a
    section, we have $l \in B$ contradicting $l \in X\backslash B$ [as $l$ is
    least element of $X\backslash B$]. So we must have $x < l$ or $x \in S_{X,
    l}$ so $B \subseteq S_{X, l}$. Combining this result with [eq: \ref{eq
    3.14.013}] proves
    \[ S_{X, l} = B \]
    \item[$\Leftarrow$] If $X = B$ then $\forall x \in X$ we have $\forall y
    \in B = X$ with $x \leqslant y$ that trivially $x \in X = B$, so $B$ is a
    section. If $B$ is initial segment then there exist a $l \in X$ such that
    $B = \{ y \in X|y < l \}$. Take $x \in X$ then if $y \in B$ with $x
    \leqslant y$ we have $y < l$ so that $x < l$ hence $x \in B$, proving that
    $B$ is a section.
  \end{description}
\end{proof}

A application of the above lemma is Transfinite Induction.

\begin{theorem}[Transfinite Induction]
  \label{order transfinite induction}{\index{transfinite induction}}Let
  $\langle X, \leqslant \rangle$ be a well ordered class and let $P (x)$ a
  proposition about $x$ [a statement about $x$ that can be true or false] such
  that
  \begin{equation}
    \label{eq 3.15.013} \forall x \in X \text{ such that, if $P (y)$ is true
    for every $y < x$ then $P (x) \text{ is true}$}
  \end{equation}
  then
  \[ \forall x \in X \text{ } P (x) \text{ is true} \]
\end{theorem}

\begin{proof}
  We prove this by contradiction. Assume that $\exists x \in X$ such that $P
  (x)$ is false, then $B = \left\{ x \in X|\mathcal{P} (x) \text{ is false}
  \right\}$ is non empty. \ As $X$ is well ordered there exist a least element
  $l \in B$. Take $x \in X$ with $x < l$ then $x \nin B$ [for if $x \in B$
  then $l \leqslant x$, which combined with $x < l$ gives the contradiction $l
  < l$] so that $P (x)$ is true. By the hypothesis [eq: \ref{eq 3.15.013}] we
  have that $P (l)$ is true, which means that $l \nin B$ contradicting $l \in
  B$. So we must have that $\forall x \in X$ $P (x)$ is true.
\end{proof}

\begin{lemma}
  \label{order well ordered and order isomorphism}Let $\langle X, \leqslant
  \rangle$ be a well ordered class, $B \subseteq X$ and $f : \langle X,
  \leqslant \rangle \rightarrow \langle B, \leqslant \rangle$ a order
  isomorphism then $\forall x \in X$ we have $x \leqslant f (x)$
\end{lemma}

\begin{proof}
  We prove this by contradiction. Assume that that$\exists x \in X$ such that
  $\neg (x \leqslant f (x))$. As $\langle X, \leqslant \rangle$ if well
  ordered we have by \ [theorem: \ref{order well order implies conditional
  complete and totally ordering}] that $f (x) < x$, hence $C = \{ x \in X|f
  (x) < x \} \neq \varnothing$. By well ordering there exists a least element
  $c$ of $C$. As $c \in C$ we have that $f (c) < c$, hence by [theorem:
  \ref{order isomorphism strictly}] $f (f (c)) < f (c)$ so that $f (c) \in C$.
  As $c$ is the least element of $C$ we have $c \leqslant f (c)$, which
  combined with $f (c) < c$ gives the contradiction $c < c$. So we must have
  $\forall x \in X$ that $x \leqslant f (x)$.
\end{proof}

\begin{theorem}
  \label{order well ordered class and isomorphism}Let $\langle X, \leqslant
  \rangle$ be a well ordered class then there does not exist a order
  isomorphism from $X$ to a sub-class of an initial segment of $X$.
\end{theorem}

\begin{proof}
  We prove this by contradiction. So assume that there exists a initial
  segment $S_{X, a} = \{ y \in X|y < a \}$ of $X$, a $B \subseteq S_{X,
  \alpha}$ and a order isomorphism $f : \langle X, \leqslant \rangle
  \rightarrow \langle B, \leqslant \rangle$. Using the previous lemma [lemma:
  \ref{order well ordered and order isomorphism}] we have that $a \leqslant f
  (a)$, so $f (a) \nin S_{X, a}$ [for if $f (a) \in S_{X, a}$ then $f (a) < a$
  leading to the contradiction $a < a$]. However as $\tmop{range} (f) = B
  \subseteq S_{X, a}$ we must have that $f (a) \in S_{X, a}$ and we reach a
  contradiction.
\end{proof}

\begin{corollary}
  \label{order well ordered is not isomorph to a initial segment }Let $\langle
  X, \leqslant \rangle$ be a well ordered class then there does not exist a
  order isomorphism between $X$ and initial segment of $X$
\end{corollary}

\begin{proof}
  As a initial segment is a sub-class of itself this follows from the previous
  theorem [theorem: \ref{order well ordered class and isomorphism}]
\end{proof}

\begin{theorem}
  \label{order well ordered isomorphism property}If $\langle X, \leqslant_X
  \rangle$, $\langle Y, \leqslant_Y \rangle$ are well ordered classes then if
  $X$ is order isomorphic with an initial segment of $Y$ we have that $Y$ is
  not order isomorphic with any sub-class of $X$.
\end{theorem}

\begin{proof}
  Let $S_{Y, y}$ be a initial segment of $Y$ and $f : \langle X, \leqslant_X
  \rangle \rightarrow \langle S_{Y, y}, \leqslant_Y \rangle$ a order
  isomorphism. Assume that there exist a $A \subseteq X$ and a order
  isomorphism $g : \langle Y, \leqslant_Y \rangle \rightarrow \langle A,
  \leqslant_A \rangle$, As by [lemma: \ref{function extend target}],[theorem:
  \ref{function injectivity, surjectivity}] and the fact that 'increasing' is
  a property of the graph of a function,we have that $g : \langle Y,
  \leqslant_Y \rangle \rightarrow \langle X, \leqslant_X \rangle$ is a
  injective increasing function. Using [theorem: \ref{function composition
  injectivity, surjectivity and bijectivity}],[theorem: \ref{order composition
  of functions}] $\tmop{we} \tmop{have} \tmop{that} f \circ g : \langle Y,
  \leqslant_Y \rangle \rightarrow \langle S_{Y, y}, \leqslant_Y \rangle$ is a
  injective increasing function, hence $f \circ g : Y \rightarrow (f \circ g)
  (Y)$ is a bijective function [see theorem: \ref{function injectivity to
  bijection}] which is increasing, hence by [theorem: \ref{order condition for
  isomorphism in a totallu ordered set}] we have that $f \circ g : \langle Y,
  \leqslant_Y \rangle \rightarrow \langle (f \circ g) (Y), \leqslant_Y
  \rangle$ is a order isomorphism. As $(f \circ g) (Y) \subseteq \tmop{range}
  (f)$ [see theorem: \ref{partial function domain range composition}] and
  $\tmop{range} (f) \subseteq S_{Y, y}$ we have a order isomorphism \ between
  $Y$ and a sub-class of a initial segment of $Y$. By [theorem: \ref{order
  well ordered class and isomorphism}] this is impossible so the assumption is
  false, hence $Y$ is not order isomorphic to a an initial segment of $Y$.
\end{proof}

\begin{corollary}
  \label{order well ordered isomorphic property (3)}If $\langle X, \leqslant_X
  \rangle$, $\langle Y, \leqslant_Y \rangle$ are well ordered classes such
  that $X$ is order isomorphic with $Y$ then
  \begin{enumerate}
    \item $X$ can not be order isomorphic with a initial segment of $Y$
    
    \item $Y$ can not be order isomorphic with a initial segment of $X$
  \end{enumerate}
\end{corollary}

\begin{proof}
  We prove this by contradiction. First by the hypothesis we have $X \cong Y$
  and by [theorem: \ref{order properties of the isomorph relation}] $Y \cong
  X$.
  \begin{enumerate}
    \item If $X$ is order isomorphic with a initial segment of $Y$ then as $Y
    \cong X$ we have that $Y$ is order isomorphic with a sub-class of $X$,
    which by [theorem: \ref{order well ordered isomorphism property}] is not
    allowed.
    
    \item If $Y$ is order isomorphic with a initial segment of $X$ then as $X
    \cong Y$ we have that $X$ is order isomorphic with a sub-class of $Y$,
    which by [theorem: \ref{order well ordered isomorphism property}] is not
    allowed.
  \end{enumerate}
\end{proof}

\begin{lemma}
  \label{order initial segement a<less>b}Let $\langle X, \leqslant \rangle$ be
  a well ordered class and $a, b \in X$ with $a < b$ then $S_{X, a}$ is a
  initial segment of $\mathcal{S}_{X, b}$ [using the order $\leqslant_{|S_{X,
  y}}$]
\end{lemma}

\begin{proof}
  First if $x \in S_{X, a}$ then $x < a \Rightarrowlim_{\text{a<b}} x < b $ so
  that $x \in S_{X, b}$, hence
  \[ S_{X, a} \subseteq S_{X, b} \]
  Now if $x \in S_{X, b}$ and $y \in S_{X, a}$ is such that $x
  \leqslant_{|S_{X_B}} y \text{ then } x \leqslant y \Rightarrowlim_{y \in
  S_{X, a} \Rightarrow y < a} x < a \text{ hence } x \in S_{X, a}$. So $S_{X,
  a}$ is a section of $S_{X, b}$, as $a \nin S_{X, a} \wedge a \in S_{X, b}$
  [for $a < b$] we have $S_{X, a} \neq S_{X, b}$ so that, using [theorem:
  \ref{order section and well ordering}], $S_{X, a}$ is a initial segment of
  $S_{X, b}$.
\end{proof}

\begin{theorem}
  \label{order well ordering and isomorphism (2)}Let Let $\langle X,
  \leqslant_X \rangle$ and $\langle Y, \leqslant_Y \rangle$ be well ordered
  classes then exactly one of the following cases hold
  \begin{enumerate}
    \item $X$ is order isomorphic with $Y$
    
    \item $X$ is order isomorphic with an initial segment of $Y$
    
    \item $Y$ is order isomorphic with an initial segment of $X$
  \end{enumerate}
\end{theorem}

\begin{proof}
  Define
  \begin{equation}
    \label{eq 3.16.016} C = \left\{ x \in X| \exists y \in Y \text{ such that
    } S_{X, x} \cong S_{Y, y} \right\}
  \end{equation}
  and
  \begin{equation}
    \label{eq 3.17.016} F = \{ (x, y) \in C \times Y|S_{X, x} \cong S_{Y, y})
  \end{equation}
  We prove now that $F$ is the graph of a order isomorphism between $C$ and $F
  (C)$. We have trivially from the definition of $F$ that
  \begin{equation}
    \label{eq 3.16.014} F \subseteq C \times Y
  \end{equation}
  Let $(x, y), (x, y') \in F$, then $\mathcal{S}_{X, x} \cong S_{Y, y}$ and
  $S_{x, x} \cong S_{Y, y'}$ so by [theorem: \ref{order properties of the
  isomorph relation}]
  \begin{equation}
    \label{eq 3.17.014} S_{Y, y} {\cong S_{Y, y'}} 
  \end{equation}
  Assume that $y \neq y'$ then, as $\langle Y, \leqslant_Y \rangle$ is well
  ordered we have by [theorem: \ref{order well order implies conditional
  complete and totally ordering}] either:
  \begin{description}
    \item[$y \leqslant y'$] then $y < y'$ so that by the previous lemma
    [lemma: \ref{order initial segement a<less>b}] we have that $S_{Y, y}$ is
    a initial segment of $S_{Y, y'}$. Using [corollary: \ref{order well
    ordered is not isomorph to a initial segment }] we have then that $S_{Y,
    y'}$ is not order isomorphic with $S_{Y, y}$ contradicting [eq: \ref{eq
    3.17.014}].
    
    \item[$y' < y$] then by the previous lemma [lemma: \ref{order initial
    segement a<less>b}] we have that $S_{Y, y'}$ is a initial segment of
    $S_{Y, y}$. Using [corollary: \ref{order well ordered is not isomorph to a
    initial segment }] we have then that $S_{Y, y}$ is not order isomorphic
    with $S_{Y, y'}$ contradicting [eq: \ref{eq 3.17.014}].
  \end{description}
  as in all cases we have a contradiction, the assumption must be wrong. Hence
  \begin{equation}
    \label{eq 3.18.014} \text{If } (x, y), (x, y') \in F \text{ then } y = y'
  \end{equation}
  Further if $x \in C$ then by definition of $C$ there exists a $y \in Y$ such
  that $S_{X, x} = S_{Y, y}$ hence $(x, y) \in F$ proving that
  \begin{equation}
    \label{eq 3.19.014} C \subseteq \tmop{dom} (F)
  \end{equation}
  If $(x, y), (x', y) \in F$ then $S_{X, x} \cong S_{Y, y}$ and
  $\mathcal{S}_{X, x'} \cong S_{Y, y}$ so by \ [theorem: \ref{order properties
  of the isomorph relation}] we have that
  \begin{equation}
    \label{eq 3.20.014} S_{X, x} \cong S_{X, x'}
  \end{equation}
  Assume that $x \neq x'$ then, as $\langle X, \leqslant_X \rangle$ is well
  ordered we have by \ [theorem: \ref{order well order implies conditional
  complete and totally ordering}] either:
  \begin{description}
    \item[$x \leqslant x'$] then $x < x'$ so that by the previous lemma
    [lemma: \ref{order initial segement a<less>b}] we have that $S_{X, x}$ is
    a initial segment of $S_{X, x'}$. Using [corollary: \ref{order well
    ordered is not isomorph to a initial segment }] we have then that $S_{X,
    x'}$ is not order isomorphic with $S_{X, x}$ contradicting [eq: \ref{eq
    3.20.014}].
    
    \item[$x' \leqslant x$] then by the previous lemma [lemma: \ref{order
    initial segement a<less>b}] we have that $S_{X, x'}$ is a initial segment
    of $S_{X, x}$. Using [corollary: \ref{order well ordered is not isomorph
    to a initial segment }] we have then that $S_{X, x}$ is not order
    isomorphic with $S_{X, x'}$ contradicting [eq: \ref{eq 3.20.014}].
  \end{description}
  as in all cases we have a contradiction, the assumption must be wrong. Hence
  \begin{equation}
    \label{eq 3.21.014} \text{If } (x, y), (x', y) \in F \text{ we have } x =
    x'
  \end{equation}
  Combining [eq: \ref{eq 3.16.014}], [eq: \ref{eq 3.18.014}], [eq: \ref{eq
  3.19.014}] and [eq: \ref{eq 3.21.014}] it follows that $F : C \rightarrow Y
  \text{ is a injective \ function}$. Applying then [proposition:
  \ref{function injectivity to bijection}] gives if we define $D = F (C)$
  \begin{equation}
    \label{eq 3.22.014} F : C \rightarrow D \text{ is a bijection}
  \end{equation}
  Take $x, y \in C$ such that $x \leqslant_X y $ then by definition of $F$ we
  have
  \begin{equation}
    \label{eq 3.23.014} S_{X, x} \cong S_{Y, F (x)} \text{ and } S_{X, y}
    \cong S_{Y, F (y)}
  \end{equation}
  Assume now that $\neg (F (x) \leqslant_Y F (y))$ then as $\langle Y,
  \leqslant_Y \rangle$ is well ordered we have by [theorem: \ref{order well
  order implies conditional complete and totally ordering}] that $F (y) <_Y F
  (x)$. So using [theorem: \ref{order initial segement a<less>b}] we have that
  $S_{Y, F (y)}$ is a initial segment of $S_{Y, F (x)}$. As $x \leqslant_X y$
  it follows that $S_{X, x} \subseteq S_{X, y}$ [see proposition: \ref{order
  initial segement inclusion}]. So we have using [eq: \ref{eq 3.23.014}]
  \begin{enumeratealpha}
    \item $S_{X, y}$ is order isomorphic with $S_{Y, F (y)}$ a initial segment
    of $S_{Y, F (x)}$
    
    \item $S_{F (x)}$ is order isomorphic with $S_{X, x}$ a sub-class of
    $S_{X, y}$
  \end{enumeratealpha}
  Using [theorem: \ref{order well ordered isomorphism property}] we see that
  (a) and (b) can not be all true, hence our assumption is false so that $F
  (x) \leqslant F (y)$. Hence we have that $F : C \rightarrow D$ is a
  increasing bijection which by [theorem: \ref{order condition for isomorphism
  in a totallu ordered set}] proves that
  \begin{equation}
    \label{eq 3.21.014.1} F : \langle C, \leqslant_X \rangle \rightarrow
    \langle D, \leqslant_Y \rangle \text{ is a order isomorphism or } C \cong
    D
  \end{equation}
  Next we prove that
  \begin{equation}
    \label{eq 3.25.015} C \text{ is a section of } X
  \end{equation}
  \begin{proof}
    Let $x \in X$ and take $c \in C$ such that $x \leqslant_X c$. As $S_{X, c}
    \cong S_{Y, F (c)}$ there exist a order isomorphism
    \begin{equation}
      \label{eq 3.25.014} g : S_{X, c} \rightarrow S_{Y, F (c)}
    \end{equation}
    Now as $x \leqslant_X c$ we have by [proposition: \ref{order initial
    segement inclusion}] that $S_{X, x} \subseteq S_{X, c}$. Hence by
    \ref{function restriction of a function} we have that
    \begin{equation}
      \label{eq 3.26.014} g_{|S_{X, x}} : S_{X, x} \rightarrow S_{X, c} \text{
      is a function}
    \end{equation}
    Further if $y \in S_{X, x}$ we have that $y <_X x$, so as $g$ is a order
    isomorphism we have $g (y) <_Y g (x)$ proving that $g_{|S_{X, x}} (y) = g
    (y) \in S_{Y, g (x)}$ or $\tmop{range} (g_{|S_{X, x}}) \subseteq S_{Y, g
    (x)}$. So by [theorem: \ref{function range restriction}] it follows that
    \begin{equation}
      \label{eq 3.27.014} g_{|S_{X, x}} : S_{X, x} \rightarrow S_{Y, g (x)}
      \text{ is a function}
    \end{equation}
    As $g$ is a isomorphism and thus injective it follows from [theorem:
    \ref{function restricted function properties}] that
    \begin{equation}
      \label{eq 3.28.014} g_{|S_{X, x}} : S_{X_x} \rightarrow S_{Y, g (x)}
      \text{ is injective}
    \end{equation}
    Further if $y \in S_{Y, g (x)}$ then $y <_Y g (x)$, as $g (x) \in S_{Y, F
    (c)}$ [see eq: \ref{eq 3.25.014}] we have $g (x) <_Y F (c)$ so that $y <_Y
    F (c)$ proving $y \in S_{Y, F (c)}$. As $g$ is surjective there exist a $u
    \in S_{X, c}$ such that $y = g (u)$. Assume that $x \leqslant_X u$ then $g
    (x) \leqslant_Y g (u) = y$, as $y <_Y g (x)$ this gives the contradiction
    $g (x) < g (x)$. So we have $\neg (x \leqslant u)$ which, as $\langle X,
    \leqslant_X \rangle$ is well ordered, gives by [theorem: \ref{order well
    order implies conditional complete and totally ordering}] that $u <_X x$
    so that $u \in S_{X, x}$. So for $y \in S_{Y, g (x)}$ we found a $u \in
    S_{X, x}$ such that $g_{|S_{X, x}} (u) = g (u) = y$ proving that
    \begin{equation}
      \label{eq 3.29.014} g_{|S_{X, x}} : S_{X, x} \rightarrow S_{Y, g (x)}
      \text{ is surjective}
    \end{equation}
    Further if $u, v \in S_{X, x}$ are such that $u \leqslant_X v$ so that
    $g_{|S_{X, x}} (u) = g (u) \leqslant_X g (v) = g_{|S_{X, x}} (v)$ proving
    that
    \begin{equation}
      \label{eq 3.30.014} g_{|S_{X, x}} : S_{X, x} \rightarrow S_{Y, g (x)}
      \text{ is increasing}
    \end{equation}
    Combining [eq: \ref{eq 3.26.014}], [eq: \ref{eq 3.27.014}], [eq: \ref{eq
    3.29.014}], [eq: \ref{eq 3.30.014}] we have that $g_{|S_{X, x}} : \langle
    S_{X, x}, \leqslant_X \rangle \rightarrow \left\langle {S_{Y, g (x)}} ,
    \leqslant_Y \right\rangle$ is a order isomorphism so that $S_{X, x} \cong
    S_{Y, g (x)}$ hence $x \in C$. Proving that $C$ is as section of $X$.
  \end{proof}
  
  Next we prove that
  \begin{equation}
    \label{eq 3.31.014} D \text{ is a section of } Y
  \end{equation}
  \begin{proof}
    Let $y \in Y$ and take $d \in D$ such that $y \leqslant_Y d$. As $d \in D
    = \tmop{range} (F)$ there exist a $c \in C$ such that $F (c) = d$, so
    $S_{X, c} \cong S_{Y, d} \Rightarrowlim_{\text{[theorem: \ref{order
    properties of the isomorph relation}]}} S_{Y, d} \cong S_{X, c}$. So there
    exist a order isomorphism
    \begin{equation}
      \label{eq 3.32.014} f : S_{Y, d} \rightarrow S_{X, c}
    \end{equation}
    Now from $y \leqslant_D d$ we have by [theorem: \ref{order initial
    segement inclusion}] $S_{Y, y} \subseteq S_{Y, d}$. Hence by \ref{function
    restriction of a function} we have that
    \begin{equation}
      \label{eq 3.33.014} f_{|S_{Y, y}} : S_{Y, y} \rightarrow S_{X, c} \text{
      is a function}
    \end{equation}
    If $x \in S_{Y, y}$ then $x <_Y y$ so, as $f$ is a order isomorphism,
    $f_{|S_{Y, y}} (x) = f (x) <_X f (y)$, we have that $f_{|S_{Y, y}} (x) \in
    S_{Y, f (y)}$, so $\tmop{range} (f_{|S_{Y, y}}) \subseteq S_{X, f (y)}$.
    By [theorem: \ref{function range restriction}] it follows that
    \begin{equation}
      \label{eq 3.34.014} f_{|S_{Y, y}} : S_{Y, y} \rightarrow S_{X, f (y)}
      \text{ is a function}
    \end{equation}
    As $f$ is a isomorphism and injective it follows from [theorem:
    \ref{function restricted function properties}] that
    \begin{equation}
      \label{eq 3.35.014} f_{|S_{Y, y}} : S_{Y, y} \rightarrow S_{X, f (y)}
      \text{ is injective} 
    \end{equation}
    If $x \in S_{X, f (y)}$ then $x <_X f (y)$, as by [eq: \ref{eq 3.32.014}]
    $f (y) \in S_{X, c}$, we have $f (y) < c$, so that $x <_X c$ or $x \in
    S_{X, c}$. As $f$ is surjective there exists a $u \in S_{Y, d}$ such that
    $f (u) = x$. As $u \in S_{Y, d}$ we have that $u <_Y d$. Assume now that
    $y \leqslant_Y u$ then, as $f$ is a order isomorphism, $f (y) \leqslant_X
    f (u) = x$, which as $x <_X f (y)$ gives the contradiction $x <_X x$. So
    we must have that $\neg (y \leqslant_Y u)$, which, as $\langle Y,
    \leqslant_Y \rangle$ is well ordered, gives by [theorem: \ref{order well
    order implies conditional complete and totally ordering}] that $u <_Y y$
    or $u \in S_{Y, y}$. So for $x \in S_{X, f (y)}$ there exist a $u \in
    S_{Y, y}$ such that $f (u) = x$, proving that
    \begin{equation}
      \label{eq 3.36.014} f_{|S_{Y, y}} : S_{Y, y} \rightarrow S_{X, f (y)}
      \text{ is surjective}
    \end{equation}
    Further if $u, v \in S_{Y, y}$ is such that $u \leqslant v$ then
    $f_{|S_{Y, y}} (u) = f (u) \leqslant f (v) = f_{|S_{U, y}} (v)$ proving
    that
    \begin{equation}
      \label{eq 3.37.014} f_{|S_{Y, y}} : S_{Y, y} \rightarrow S_{X, f (y)}
      \text{ is increasing}
    \end{equation}
    Combining [eq: \ref{eq 3.34.014}], [eq: \ref{eq 3.35.014}], [eq: \ref{eq
    3.36.014}] and [eq: \ref{eq 3.37.014}] we have that
    \[ f_{|S_{Y, y}} : \langle S_{Y, y}, \leqslant  \rangle \rightarrow
       \langle S_{X, f (y)}, \leqslant_X \rangle \text{ is a order
       isomorphism}, \]
    hence $S_{Y, y} \cong S_{X, f (y)}$. As $f (y) \in S_{X, c} \subseteq X$
    and $y \in Y$ it follows from the definition of $C$ that $f (y) \in C$,
    hence by definition of $F$ $(f (y), y) \in F$ or $y = F (f (y)) \in F (C)
    = D$, giving $y \in D$. Proving that $D$ is a section of $Y$.
  \end{proof}
  
  To summarize [eq: \ref{eq 3.21.014.1}], [eq: \ref{eq 3.25.015}] and [eq:
  \ref{eq 3.31.014}] we have
  \begin{equation}
    \label{eq 3.39.015} C \cong D \wedge C \text{ is a segment of } X \wedge D
    \text{ is a segment of } Y
  \end{equation}
  Assume now that $C$ is a initial segment of $X$ and $D$ is a initial segment
  of $Y$ then there exist a $r \in X$ and a $s \in Y$ such that $C = S_{X, r}$
  and $D = S_{Y, s}$. By \ref{eq 3.39.015} we have that $S_{X, r} \cong S_{Y,
  s}$ which by definition of $C$ means that $r \in C$ or as $C = S_{X, r}$
  that $r < r$ a contradiction. So we have that
  \begin{equation}
    \label{eq 3.42.015} \neg \left( C \text{ is a initial segment of $X \wedge
    D \text{ is a initial segment of } Y$} \right)
  \end{equation}
  As $C$ is a section of $X$ we have by [theorem: \ref{order section and well
  ordering}] that
  \begin{equation}
    \label{eq 3.43.015} X = C \text{ or } C \text{ is a initial segment of } X
  \end{equation}
  Like wise, as $D$ is a section of $Y$ we have by \ [theorem: \ref{order
  section and well ordering}] that
  \begin{equation}
    \label{eq 3.44.015} Y = D \text{ or } D \text{ is a initial segment of } Y
  \end{equation}
  We have taking [eq: \ref{eq 3.43.015}] and [eq: \ref{eq 3.44.015}] in
  account that either:
  \begin{description}
    \item[$X = C \wedge Y = D$] then by [eq: \ref{eq 3.39.015}]
    \[ X \cong Y \]
    Using theorem [theorem: \ref{order well ordered isomorphic property (3)}]
    and the above we have that
    \[ X \text{ is not order isomorphic with a sub-class of Y} \]
    \[ Y \text{ is not order isomorphic with a sub-class of X} \]
    \item[$X = C \wedge Y \neq D$] then by [eq: \ref{eq 3.44.015}] we have
    that $D$ is a initial segment of $Y$, which as by [eq: \ref{eq 3.39.015}]
    $X = C \cong D$ prove that
    \[ X \text{ is order isomorphic with a initial segment of $Y$} \]
    If $Y$ is order isomorphic with a initial segment of $X$ then by [theorem:
    \ref{order well ordered isomorphism property}] we have that $X$ is not
    order isomorphic to a subset of $Y$ contradicting $X \cong D$ and $X \cong
    Y$. So \
    \[ Y \text{ is not } \tmop{order} \tmop{isomorphic} \tmop{to} a
       \tmop{initial} \tmop{segment} \tmop{of} X \]
    \[ X \text{ is not order isomorphic to } Y \]
    \item[$X \neq C \wedge Y = D$] then by [eq: \ref{eq 3.43.015}] we have
    that $C$ is a initial segment of $X$, which as by [eq: \ref{eq 3.39.015}]
    $C \cong D \Rightarrowlim_{\text{[theorem: \ref{order properties of the
    isomorph relation}]}} Y = D \cong C$ proves that
    \[ Y \text{ is order isomorphic with a initial segment of $X$} \]
    If $X$ is order isomorphic with a initial segment of $Y$ then by [theorem:
    \ref{order well ordered isomorphism property}] we have that $Y$ is not
    order isomorphic to a subset of $X$ contradicting $Y \cong C$ and $Y \cong
    X$. So \
    \[ X \text{ is not } \tmop{order} \tmop{isomorphic} \tmop{to} a
       \tmop{initial} \tmop{segment} \tmop{of} Y \]
    \[ X \text{ is not order isomorphic to } Y \]
    \item[$X \neq C \wedge Y \neq D$] Using [eq: \ref{eq 3.43.015}] and [eq:
    \ref{eq 3.44.015}] we have that $C$ is a initial segment of $X$ and $D$ is
    a initial segment of $Y$ which contradicts [eq: \ref{eq 3.42.015}]. Hence
    this case does not apply.
  \end{description}
\end{proof}

\begin{corollary}
  \label{order well order every subclass is isomorphic with A or a iitial
  segement}Let $\langle X, \leqslant \rangle$ be a well ordered class and $Y
  \subseteq X$ then we have either (but not both):
  \begin{enumerate}
    \item $Y$ is order isomorphic with $X$
    
    \item $X$ is order isomorphic with a initial segment of $X$
  \end{enumerate}
\end{corollary}

\begin{proof}
  If $Y \subseteq X$ then $\langle Y, \leqslant_{|Y} \rangle$ is a well
  ordered class [see theorem: \ref{order total/well-order inclusion}], so
  using the previous [theorem: \ref{order well ordering and isomorphism (2)}]
  we have either:
  \begin{enumerate}
    \item $Y$ is order isomorphic with X
    
    \item $Y$ is order isomorphic with a initial segment of $X$
    
    \item $X$ is order isomorphic with a initial segment of $Y$. By [theorem:
    \ref{order well ordered isomorphism property}] we may not have that $Y$ is
    order isomorphic with a sub-class of $X$. As by [theorem: \ref{order
    properties of the isomorph relation}] $Y \cong Y$ and $Y$ is a sub-class
    of $X$ we reach a contradiction, so this case never applies.
  \end{enumerate}
\end{proof}

\section{Axiom of choice}

The axiom of choice in it's many equivalent forms like
\begin{eqnarray*}
  & \tmop{Hausdorff}' s \tmop{Maximal} \tmop{Principle} & \\
  & \tmop{Zorn}' s \tmop{Lemma} & \\
  & \tmop{Well} - \tmop{Ordering} \tmop{Theorem} & 
\end{eqnarray*}
plays a major role in some fundamental theorems about the product of sets, the
existence of a basis for a vector space, etc.

\begin{definition}
  \label{choice P'(A)}{\index{$\mathcal{P}' (A)$}}Let $A$ be a class then
  $\mathcal{P}' (A)$ is defined as
  \[ \mathcal{P}' (A) =\mathcal{P} (A) \backslash \{ \varnothing \} \]
  In other words it is the collection of all non empty sub sets of a set
\end{definition}

It turns out that if $A$ is a set then $\mathcal{P}' (A)$ is also a set.

\begin{theorem}
  \label{choice P'(A) is a set}If $A$ is a set then $\mathcal{P}' (A)$ is a
  set
\end{theorem}

\begin{proof}
  Using the Axiom of Power [axiom \ref{axiom of power}] we have that
  $\mathcal{P} (A)$ is a set. As $\mathcal{P}' (A) \subseteq \mathcal{P} (A)$
  [see [theorem: \ref{class intersection, union, inclusion}] it follow from
  the Axiom of Subsets [axiom: \ref{axiom of subsets}] that $\mathcal{P}' (A)$
  is a set.
\end{proof}

\begin{definition}[Choice Function]
  \label{choice choice function}{\index{choice function}}Let $A$ be a set then
  a \tmtextbf{choice function for $A$} is a function $f : \mathcal{P}' (A)
  \rightarrow A$ such that $\forall B \in \mathcal{P}' (A)$ we have $f (B) \in
  B$
\end{definition}

So a choice function picks out one element out of each subset of $A$ and the
axiom of choice ensures the existence of a choice function for a set.

\begin{axiom}[Axiom of Choice]
  \label{axiom of choice}{\index{axiom of choice}}If $A$ is a set then there
  exist a choice function for $A$
\end{axiom}

As a application of the axiom of choice we have the following theorem

\begin{theorem}
  \label{function surjection and construction of inverse function}If $f : A
  \rightarrow B$ is a surjective function then there exists a injective
  function $g : B \rightarrow A$ such that $f \circ g = \tmop{Id}_B$
\end{theorem}

\begin{proof}
  By the axiom of choice there exists a choice function
  \[ c : \mathcal{P}' (A) \rightarrow A \text{ such that } \forall A \in
     \mathcal{P}' (A) \text{ we have } c (A) \in A \]
  If $f : A \rightarrow B$ is surjective. Then $\forall y \in B$ we have that
  $f^{- 1} (\{ y \})$ is a non empty subset of $A \Rightarrow f^{- 1} (\{ y
  \}) \in \mathcal{P}' (A)$. Define then the function
  \[ g : B \rightarrow Y \text{ by } g (y) = c (f^{- 1} (\{ y \})) \]
  Now if $y \in Y$ then, as $c$ is a choice function, $c (f^{- 1} (\{ y \}))
  \in f^{- 1} (\{ y \})$ so that $f (c (f^{- 1} (\{ y \}))) = y$. Hence we
  have that $(f \circ g) (y) = f (g (y)) = f (c (f^{- 1} (\{ y \}))) = y$ or
  \[ f \circ g = \tmop{Id}_B \]
  If $g (y) = g (y')$ then we have $f (g (y)) = f (g (y')) \Rightarrowlim_{f
  \circ g = \tmop{Id}_B} \tmop{Id}_B (y) = \tmop{Id}_B (y') \Rightarrow y =
  y'$ proving that
  \[ g : B \rightarrow Y \tmop{is} \tmop{injective} \]
\end{proof}

The important thing to remember in the above is that the axiom of choice
ensures the existence of $g : B \rightarrow A$ but does not give a way to
construct the function $g$ itself.

We have the following equivalent statements of the axiom of choice

\begin{theorem}
  \label{choice axiom of choice equivalences (1)}The following are equivalent
  \begin{enumerate}
    \item The Axiom of Choice
    
    \item Let $\mathcal{A}$ be a set of sets such that:
    \begin{enumerate}
      \item $\forall A \in \mathcal{A}$ we have $A \neq \varnothing$
      
      \item $\forall A, B \in \mathcal{A}$ with $A \neq B$ we have $A \bigcap
      B = \varnothing$
    \end{enumerate}
    then there exist a set $C$ called the \tmtextbf{choice set for
    $\mathcal{A}$} such that
    \begin{enumerate}
      \item $C \subseteq \bigcup \mathcal{A}$
      
      \item $\forall A \in \mathcal{A}$ we have $A \bigcap C \neq \varnothing$
      and if $y, y' \in A \bigcap C$ then $y = y'$
    \end{enumerate}
    In other words $C$ consists of exactly one element from each $A \in
    \mathcal{A}$.
    
    \item If $\{ A_i \}_{i \in I} \subseteq \mathcal{A}$ is a family of non
    empty sets [$\forall i \in I$ we have $A_i \neq \varnothing$] where $I,
    \mathcal{A}$ are sets then there exists a function $f : I \rightarrow
    \bigcup_{i \in I} A_i$ such that $\forall i \in I$ we have $f (i) \in A_i$
  \end{enumerate}
\end{theorem}

\begin{proof}
  
  \begin{description}
    \item[$1 \Rightarrow 2$] Take $U = \bigcup \mathcal{A}$ [see definition:
    \ref{class union}]. As $\mathcal{A}$ is a set we have by the Axiom of
    Union [axiom: \ref{axiom of union}] that $U$ is a set. So we can apply the
    Axiom of Choice [axiom: \ref{axiom of choice}] to get a function
    \[ c : \mathcal{P}' (U) \rightarrow U \text{ such that } \forall A \in
       \mathcal{P}' (U)  \text{ we have } c (A) \in A \]
    If $A \in \mathcal{A}$ then $A \neq \varnothing$ and using [theorem:
    \ref{class general intersection}] we have $A \subseteq U$ proving that $A
    \in \mathcal{P}' (U)$ hence
    \[ \mathcal{A} \subseteq \mathcal{P}' (U) \]
    so we can take the \tmtextbf{image} of $\mathcal{A}$ by $c$
    \[ C = c (\mathcal{A}) \]
    We have now:
    \begin{enumeratealpha}
      \item If $x \in C$ then $\exists A \in \mathcal{A}$ such that $x = (c)
      (A)$, which as $c$ is a choice function means that $x \in A$ hence, by
      [theorem: \ref{class general intersection}], we have that $x \in \bigcup
      \mathcal{A}$ proving that
      \[ C \subseteq \bigcup \mathcal{A} \]
      \item Let $A \in \mathcal{A}$ then $(c) (A) \in c (\mathcal{A}) = C$
      and, as $c$ is a choice function, $(c) (A) \in A$ [note: $(c) (A)$ is
      function application and $c (\mathcal{A})$ is the image of $\mathcal{A}$
      by $c$]. Hence
      \[ A \bigcap C \neq \varnothing \]
      If $y, y' \in A \bigcap C$ then as $y, y' \in C = c (\mathcal{A})$ there
      exist $Y, Y' \in \mathcal{A}$ such that $y = (c) (Y)$ and $y' = (c)
      (Y')$, as $c$ is a choice function we have $y = (c) (Y) \in Y$ and $y' =
      (c) (Y') \in Y'$. Assume that $Y \neq Y'$ then we have the contradiction
      $y, y' \in Y \bigcap Y' = \varnothing, \tmop{so} \tmop{we} \tmop{have}
      \tmop{that}$Y=Y' but then $y = c (Y) = c (Y') = y'$ proving that $y = y'
      .$ So
      \[ y, y' \in A \bigcap C \Rightarrow y = y' \]
    \end{enumeratealpha}
    so (2.a) and (2.b) is proved.
    
    \item[$2 \Rightarrow 1$] Let $A$ be a set and let $B \in \mathcal{P}' (A)$
    then $\varnothing \neq B \subseteq A$. Define now
    \begin{equation}
      P_B = \{ (B, x) |x \in B \}
    \end{equation}
    If $(B, x) \in P_B$ then as $B \in \mathcal{P}' (A)$ and $x \in B
    \subseteq A$ we have $(B, x) \in \mathcal{P}' (A) \times A$ or
    \begin{equation}
      \label{eq 3.46.016} P_B \subseteq \mathcal{P}' (A) \times A \text{ or
      $P_B \in \mathcal{P} (\mathcal{P}' (A) \times A)$}
    \end{equation}
    As $B \neq \varnothing$ we have that $\exists b \in B$ so that $(B, p) \in
    P_B$ proving that
    \begin{equation}
      \label{eq 3.47.016} \forall B \in \mathcal{P}' (A) \text{ we have } P_B
      \neq \varnothing
    \end{equation}
    If $x \in P_B \bigcap P_{B'}$ then $\exists b \in B$ and $b' \in B$ such
    that $(B, b) = x = (B', b')$ proving that $B = B'$, hence $P_B = P_{B'}$.
    From this it follows that
    \begin{equation}
      \label{eq 3.48.016} \forall B, B' \in \mathcal{P}' (A) \text{ we have }
      \tmop{If} P_B \neq P_{B'} \text{ then $P_B \bigcap P_{B'} =
      \varnothing$}
    \end{equation}
    Define
    \begin{equation}
      \label{eq 3.49.016} \mathcal{A}= \{ P_B |B \in \mathcal{P}' (A) \}
      \subseteq \mathcal{P} (\mathcal{P}' (A) \times A)
    \end{equation}
    As $A$ is a set we have by [theorem: \ref{choice P'(A) is a set}] that
    $\mathcal{P}' (A)$ is a set, using [theorem: \ref{set A*B}] it follow that
    $\mathcal{P}' (A) \times A$ is a set, applying \ the Axiom of Power sets
    [axiom: \ref{axiom of power}] proves that $\mathcal{P} (\mathcal{P}' (A)
    \times A)$ is a set. As by [eq: \ref{eq 3.49.016}] we have that
    $\mathcal{A} \subseteq \mathcal{P} (\mathcal{P}' (A) \times A)$ we can use
    the Axiom of Sub Sets [axiom: \ref{axiom of subsets}] giving
    \begin{equation}
      \label{eq 3.50.016} \mathcal{A} \tmop{is} a \tmop{set}
    \end{equation}
    So the conditions for the hypothesis (2) are satisfied by [eq: \ref{eq
    3.50.016}],[eq: \ref{eq 3.47.016}] and [eq: \ref{eq 3.48.016}] hence there
    exist a choice set $C$ for $\mathcal{A}$ such that:
    \begin{equation}
      \label{eq 3.51.016} C \subseteq \bigcup \mathcal{A} \text{ and } \forall
      B \in \mathcal{A} \text{ we have $B \bigcap C \neq \varnothing$ and if
      $y, y' \in B \bigcap C \text{ then } y = y'$}
    \end{equation}
    If $x \in C$ then $\exists y \in \mathcal{A}$ such that $x \in y$. As $y
    \in \mathcal{A}$ there exists a $B \in \mathcal{P}' (A)$ such that $y =
    P_B = \{ (B, x) |x \in B \}$, hence there exist a $b \in B$ such that $x =
    (B, b) \in P_B \subseteq \mathcal{P}' (A) \times A$ [see eq: \ref{eq
    3.46.016}] proving that
    \begin{equation}
      \label{eq 3.52.016} C \subseteq \mathcal{P}' (A) \times A
    \end{equation}
    If $(B, y), (B, y') \in C$ then $(B, y), (B, y') \in P_B \bigcap C
    \Rightarrowlim_{\text{[eq: \ref{eq 3.51.016}}} (B, y) = (B, y')$ proving
    that $y = y'$, so
    \begin{equation}
      \label{eq 3.53.016} \text{If } (B, y), (B, y') \in C \text{ then } y =
      y'
    \end{equation}
    Let $B \in \mathcal{P}' (A)$ then $P_B \in \mathcal{A}$ so that by [eq:
    \ref{eq 3.51.016}] $P_B \bigcap C \neq \varnothing$ hence there exist a $y
    \in B$ such that $(B, y) \in C$ proving that
    \begin{equation}
      \label{eq 3.54.016} \mathcal{P}' (A) \subseteq \tmop{dom} (C)
    \end{equation}
    From [eq: \ref{eq 3.52.016}], [eq: \ref{eq 3.53.016}] and [eq: \ref{eq
    3.54.016}] it follows that
    \begin{equation}
      \label{eq 3.55.016} C : \mathcal{P}' (A) \rightarrow A \text{ is a
      function}
    \end{equation}
    Let $B \in \mathcal{P}' (A)$ then $(B, C (B)) \in C \subseteq \bigcup
    \mathcal{A}$ so that $\exists B' \in \mathcal{P}' (A)$ such that $(B, C
    (B)) \in P_{B'}$ hence $B = B'$ and $C (B) \in B' = B$ proving that
    $\forall B \in \mathcal{P}' (A) \text{ we have } C (B) \in B$, so that
    \[ C : \mathcal{P}' (A) \rightarrow A \text{ is a choice function} \]
    proving (1)
    
    \item[$1 \Rightarrow 3$] Let $\{ A_i \}_{i \in I} \subseteq \mathcal{A}$
    be a family of non empty \ sets where $I, \mathcal{A}$ are sets. Then
    using [theorem: \ref{family union condition set}] it follows that
    $\bigcup_{i \in I} A_i$ is a set. Using the Axiom of Choice [axiom:
    \ref{axiom of choice}] there exist a choice function
    \[ c : \mathcal{P}' \left( \bigcup_{i \in I} A_i \right) \rightarrow
       \bigcup_{i \in I} A_i \text{ where } \forall A \in \mathcal{P}' \left(
       \bigcup_{i \in I} A_i \right) \text{ } c (A) \in A \]
    Let $A : I \rightarrow \mathcal{A}$ be the function that defines $\{
    A_i)_{i \in I} \subseteq \mathcal{A}$ then $\forall i \in I$ we have that
    $A (i) = A_i \subseteq \bigcup_{i \in I} A_i$ [see: theorem: \ref{family
    properties (1)}] or $A (i) \in \mathcal{P} \left( \bigcup_{i \in I} A_i
    \right)$, further as $A_i \neq \varnothing$ we have that $A_i \in
    \mathcal{P}' \left( \bigcup_{i \in I} A_i \right)$, hence $\tmop{range}
    (A) \subseteq \mathcal{P}' \left( \bigcup_{i \in I} A_i \right)$. Using
    [theorem: \ref{function range restriction}] it follows that $A : I
    \rightarrow \mathcal{P}' \left( \bigcup_{i \in I} A_i \right)$ is also a
    function. If we take $f = c \circ A$ then
    \[ f : I \rightarrow \bigcup_{i \in I} A_i \text{ is a function and }
       \forall i \in I \text{ we have } f (i) = c (A (i)) = c (A_i) {\in A_i} 
    \]
    proving (3).
    
    \item[$3 \Rightarrow 1$] Let $A$ be a set and define the family $\{ B_C
    \}_{C \in \mathcal{P}' A} \subseteq \mathcal{P}' (A)$ by $B =
    \tmop{Id}_{\mathcal{P}' (A)} : \mathcal{P}' (A) \rightarrow \mathcal{P}'
    (A)$ [see example: \ref{function identity function}]. For every $C \in
    \mathcal{P}' (A)$ we have $B_C = \tmop{Id} (C) = C \neq \varnothing$,
    further as $A$ is a set we have by [theorem: \ref{choice P'(A) is a set}]
    that $\mathcal{P}' (A)$ is a set. So the conditions for (3) are satisfied
    and by (3) there exist a function
    \begin{equation}
      \label{eq 3.56.016} f : \mathcal{P}' (A) \rightarrow \bigcup_{C \in
      \mathcal{P}' (A)} B_C \text{ such that $\forall C \in \mathcal{P}' (A)
      \text{ we have $f (C) \in B_C = \tmop{Id} (C) = C$}$}
    \end{equation}
    Let $x \in \bigcup_{C \in \mathcal{P}' (A)} B_C$ then $\exists C \in
    \mathcal{P}' (A)$ such that $x \in B_C = \tmop{Id}_{\mathcal{P}' (A)} (C)
    = C \subseteq A \Rightarrow x \in A$. So $\bigcup_{C \in \mathcal{P}' (A)}
    B_C \subseteq A$. Using then [theorem: \ref{function extend target}] we
    have
    \[ f : \mathcal{P}' (A) \rightarrow A \text{ is a function with } \forall
       C \in \mathcal{P}' (A) \text{ we have $f (C) \in C$} \]
    which proves that $f : \mathcal{P}' (A) \rightarrow A$ is a choice
    function for $A$, proving (1).
  \end{description}
\end{proof}

TODO

\begin{theorem}
  \label{choice function generating}Let $A, B$ be sets such that $\forall a
  \in A$ there exist a $b \in B$ satisfying $P (a, b)$ [where $P (a, b)$ is a
  predicate] then there exist a function $f : A \rightarrow B$ such that
  $\forall a \in A$ $P (a, f (a))$. Furthermore as a function defines a family
  you can also say that there is a family $\{ f_a \}_{a \in A} \subseteq B$
  such that $\forall a \in A$ $P (a, f_a)$.
\end{theorem}

\begin{proof}
  Let $a \in A$ then by the hypothesis the set $\{ b \in B|P (a, b) \}$ is non
  empty. This allows us to define the function
  \[ \mathcal{A}: A \rightarrow \mathcal{P}' (B) \text{ by } \mathcal{A} (a) =
     \{ b \in B|P (a, b) \} \subseteq B \]
  defining the family
  \[ \{ \mathcal{A}_a \}_{a \in A} \subseteq \mathcal{P}' (B) \]
  Applying then [theorem: \ref{choice axiom of choice equivalences (1)} (3)]
  there exist a function
  \[ f : A \rightarrow \bigcup_{a \in A} \mathcal{A}_a \]
  such that $\forall a \in A$ we have $f (a) \in \mathcal{A}_a$ so that $f
  (a)$ satisfies $P (a, f (a))$. Further as $\forall a \in A$ $\mathcal{A}_a
  \subseteq B$ we have that $\bigcup_{a \in A} \mathcal{A}_a \subseteq B$,
  hence we have that
  \[ f : A \rightarrow B \]
  is a function such that $\forall a \in A$ $P (a, f (a))$.
\end{proof}

As a application of the Axiom of Choice we have the following theorems about
the product of a family of sets. First we prove that the projection function
is surjective.

\begin{theorem}
  \label{product projection is surjective}Let $\{ A_i \}_{i \in I} \subseteq
  \mathcal{A}$ be a family of \tmtextbf{non empty sets} \ where $I,
  \mathcal{A}$ are sets then $\forall i \in I$ we have that the projection
  function
  \[ \pi_i : \prod_{j \in I} A_j \rightarrow A_i \text{ defined by } \pi_j (x)
     = x (j) \text{ [see definition: } \ref{product projection function}
     \text{]} \]
  is a surjection.
\end{theorem}

\begin{proof}
  Let $i \in I$ and take $x \in A_i$. Consider the family $\{ A_j \}_{j \in
  I\backslash \{ i \}}$ [see definition: \ref{family definition (2)}] then
  $\forall j \in I\backslash \{ i \}$ we have $A_j \neq \varnothing$. So we
  can use [theorem: \ref{choice axiom of choice equivalences (1)} (3)] to find
  a function
  \[ f : I\backslash \{ i \} \rightarrow \bigcup_{j \in I\backslash \{ i \}}
     A_j \text{ such that $\forall j \in I\backslash \{ i \} \text{ we have $f
     (j) \in A_j$}$} \]
  By the definition of the product of a family of sets we have that
  \[ f \in \prod_{j \in I\backslash \{ i \}} A_j \]
  Define now $g : I \rightarrow \bigcup_{j \in I} A_j$ by $g (j) =
  \left\{\begin{array}{l}
    x \text{ if } j = i\\
    f (j) \text{ if } j \in I\backslash \{ i \}
  \end{array}\right.$ then by [theorem: \ref{product extension}] we have that
  $g \in \prod_{i \in I} A_i$. Finally by $\pi_i (g) = g (i) = x$ proving
  surjectivity.
\end{proof}

Second we prove that the product of a family of sets is not empty if and only
if every set in the family is non empty.

\begin{theorem}
  \label{product product is not empty}Let $\{ A \}_{i \in I} \subseteq
  \mathcal{A}$ be a family of sets where $I, \mathcal{A}$ are sets then we
  have
  \[ \prod_{i \in I} A_i \neq \varnothing \Leftrightarrow \forall i \in I
     \text{ we have } A_i \neq \varnothing \]
\end{theorem}

\begin{proof}
  
  \begin{description}
    \item[$\Rightarrow$] We prove this by contradiction, so assume that
    $\exists i \in I$ such that $A_i = \varnothing$. As $\prod_{i \in I} A_i
    \neq \varnothing$ there exists a $x \in \prod_{i \in I} A_i$ such that
    $\forall j \in I$ $x_j \in A_j$, in particular \ we would have $x_i \in
    A_i$ contradicting $A_i = \varnothing$. So we must have that $\forall i
    \in I$ we have $A_i \neq \varnothing$.
    
    \item[$\Leftarrow$] If $\forall i \in I$ we have $A_i \neq \varnothing$ we
    have by [theorem: \ref{choice axiom of choice equivalences (1)} (3)] that
    there exist a function
    \[ f : I \rightarrow \bigcup_{i \in I} A_i \text{ such that } \forall i
       \in I \text{ we have } f (i) \in \Alpha_i \]
    which by definition of the product means that $f \in \prod_{i \in I} A_i$
    proving that
    \[ \prod_{i \in I} A_i \neq \varnothing \]
  \end{description}
\end{proof}

We can rephrase the above theorem in another way.

\begin{theorem}
  Let $I$ be a set, $B$ a set and $\forall i \in I$ there exist a $A_i
  \subseteq A_i$
\end{theorem}

The Axiom of Choice has also important consequences for partial ordered sets.

\begin{theorem}
  \label{choice existence of successor}Let $\langle X, \leqslant \rangle$ be a
  partial ordered \tmtextbf{set} such that:
  \begin{enumerate}
    \item $X$ has a least element $p$
    
    \item Every chain [see definition:\ref{order chain}] of $X$ has a supremum
  \end{enumerate}
  then there is a element $x \in X$ which has no immediate successor [see
  definition: \ref{order immediate successor}]
\end{theorem}

\begin{proof}
  We prove this by contradiction, so assume that $\forall x \in X$ there exist
  a immediate successor. Given $x \in X$ define $T_x = \left\{ y|y \text{ is a
  immediate successor of } x \right\}$ then $T_x \neq \varnothing$ so that
  $T_x \in \mathcal{P}' (X)$. Using the Axiom of Choice [axiom: \ref{axiom of
  choice}] there exist a choice function
  \begin{equation}
    \label{eq 3.57.018} c : \mathcal{P}' (A) \rightarrow A \text{ such that
    $\forall A \in \mathcal{P}' (X)$ we have } c (A) \in A
  \end{equation}
  As $\forall x \in X \text{ we have } T_x \in \mathcal{P}' (X)$ so that $c
  (T_x)$ is well defined we can use [proposition: \ref{function simple
  definition}] to define the function
  \[ \tmop{succ} : X \rightarrow X \text{ by } \tmop{succ} (x) = c (T_x) . \]
  If $x \in X$ then $\tmop{succ} (x) = c (T_x) \in T_x$ so that $\tmop{succ}
  (x)$ is a immediate successor of $x$, to summarize
  \begin{equation}
    \label{eq 3.58.018} \tmop{succ} : X \rightarrow X \text{ is a function
    such that } \forall x \in X \text{ } \tmop{succ} (x) \text{ is a immediate
    successor of } x
  \end{equation}
  Before we can reach the contradiction we need to have some definitions and
  sub lemmas.
  
  \begin{definition}
    \label{choice lemma p-sequence}$A \subseteq X$ is a \tmtextbf{p-sequence}
    iff
    \begin{enumerate}
      \item $p \in A$
      
      \item If $x \in A \text{ then } \tmop{succ} (x) \in A$
      
      \item If $C \subseteq A$ is a chain then $\sup (C) \in A$ [note that by
      hypothesis (2) $\sup (C)$ exist]
    \end{enumerate}
  \end{definition}
  
  \begin{note}
    $X$ is a p-sequence so there exist p-sequences.
  \end{note}
  
  \begin{proof}
    First $p \in X$ by the hypothesis (1), second if $x \in X$ then by [eq:
    \ref{eq 3.58.018}] $\tmop{succ} (X) \in X$ and finally if $C$ is chain
    then by definition of the supremum $\sup (C) \in X$
  \end{proof}
  
  \begin{lemma}
    \label{choice intersection of p-sewuences}Every intersection of a set of
    p-sequences is a p-sequence
  \end{lemma}
  
  \begin{proof}
    Let $\mathcal{A}$ be a set of p-sequences then
    \begin{enumerate}
      \item $\forall A \in \mathcal{A}$ $A$ is a p-sequence hence $p \in A$ so
      that $p \in \bigcap \mathcal{A}$
      
      \item If $x \in \bigcap \mathcal{A}$ then $\forall A \in \mathcal{A}$ we
      have $p \in A$ which as $A$ is a p-sequence gives that $\tmop{succ} (x)
      \in A$ hence $\tmop{succ} (x) \in \bigcap \mathcal{A}$
      
      \item If $C \subseteq \bigcap \mathcal{A}$ is a chain then $\forall A
      \in \mathcal{A}$ we have $C \subseteq A$ and as $A$ is a p-sequence we
      have that $\sup (C) \in A$ so that $\sup (A) \in \bigcap \mathcal{A}$
    \end{enumerate}
    so by definition of a p-sequence we have that
    \[ \bigcap \mathcal{A} \text{ is a p-sequence} \]
  \end{proof}
  
  From the above lemma [lemma: \ref{choice intersection of p-sewuences}] we
  have that $\bigcap \left\{ A \in \mathcal{P} (X) |A \text{ is a p-sequence}
  \right\}$ is a p-sequence and by definition $p \in \bigcap \left\{ A \in
  \mathcal{P} (X) |A \text{ is a p-sequence} \right\}$. Further if $A$ is a
  p-sequence then $\bigcap \left\{ A \in \mathcal{P} (X) |A \text{ is a
  p-sequence} \right\} \subseteq A$. Summarized
  \begin{equation}
    \label{eq 3.59.018} P = \bigcap \left\{ B \in \mathcal{P} (X) |B \text{ is
    a p-sequence } \right\} \text{ is a p-sequence} \wedge p \in P \wedge
    \text{} \tmop{If} A \text{ is a p-sequence} \Rightarrow P \subseteq A
  \end{equation}
  \begin{definition}
    A element $x \in P$ is \tmtextbf{select }if $x$ is comparable with every
    element in $P$. 
  \end{definition}
  
  \begin{lemma}
    \label{choice lemma property of select elements}If $x \in P$ is select
    then $\forall y \in P$ with $y < x$ have $\tmop{succ} (y) \leqslant x$
  \end{lemma}
  
  \begin{proof}
    If $y \in P$ with $y < x$ then as $P$ is a p-sequence we have by
    [definition: \ref{choice lemma p-sequence} (2)] that $\tmop{succ} (y) \in
    P$. Now as $x$ is select we have that $x, \tmop{succ} (y)$ are comparable,
    hence by [theorem: \ref{order comparable property}] we have either
    $\tmop{succ} (y) \leqslant x$ or $x < \tmop{succ} (y)$. If $x <
    \tmop{succ} (y)$ then from $y < x$ it follows that $y < x \wedge x <
    \tmop{succ} (y)$ contradicting the fact that by [eq: \ref{eq 3.58.018}]
    $\tmop{succ} (y)$ is the immediate successor of $y$. Hence we must have
    that
    \[ \tmop{succ} (y) \leqslant x \]
  \end{proof}
  
  \begin{lemma}
    \label{choice lemma p-sequence generation}If $x$ is select then $A_x = \{
    y \in P|y \leqslant x \vee \tmop{succ} (x) \leqslant y \}$ is a p-sequence
  \end{lemma}
  
  \begin{proof}
    
    \begin{enumerate}
      \item As $p$ is a least element of $X$ we have that $p \leqslant x$ so
      that $p \in A_x$
      
      \item Let $y \in A_x$ Then we have either:
      \begin{description}
        \item[$y = x$] Then $\tmop{succ} (x) = \tmop{succ} (y) \Rightarrow
        \tmop{succ} (x) \leqslant \tmop{succ} (y)$ so that $\tmop{succ} (y)
        \in A_x$.
        
        \item[$y < x$] Then as $y \in A_x \subseteq P$ we have by the previous
        lemma [lemma: \ref{choice lemma property of select elements}] that
        $\tmop{succ} (y) < x \Rightarrow \tmop{succ} (y) \leqslant x$ so that
        $\tmop{succ} (y) \in A_x$.
        
        \item[$\tmop{succ} (x) \leqslant y$] As $\tmop{succ} (y)$ is the
        immediate successor of $y$ we have $y < \tmop{succ} (y)$ so that
        $\tmop{succ} (x) < \tmop{succ} (y) \Rightarrow \tmop{succ} (x)
        \leqslant \tmop{succ} (y)$ proving that $\tmop{succ} (y) \in A_x$.
      \end{description}
      so in all cases we have
      \[ \tmop{succ} (y) \in A_x \]
      \item If $C \subseteq A_x$ is a chain then we have the following
      excluding cases:
      \begin{description}
        \item[$\exists y \in C \text{ with } \tmop{succ} (x) \leqslant y$]
        Then as $y \leqslant \sup (C)$ we have that $\tmop{succ} (x) \leqslant
        \sup (C)$ so that $\sup (C) \in A_x .$
        
        \item[$\forall y \in C \text{ we have } \neg (\tmop{succ} (x)
        \leqslant y)$] Now $\forall y \in C$ as $y \in C \subseteq A_x$ we
        have either $y \leqslant x$ or $\tmop{succ} (y) \leqslant y$. As $\neg
        (\tmop{succ} (x) \leqslant y)$ is true we must have $y \leqslant x$
        and thus $x$ is a upper bound of $C$. So by definition of the supremum
        as the least upper bound of $C$ we must have that $\sup (C) \leqslant
        x$, hence $\sup (C) \in A_x$
      \end{description}
      So in all cases we have
      \[ \sup (C) \in A_x \]
    \end{enumerate}
    From (1),(2) and (3) it follows then that
    \[ A_x \text{ is a p-sequence} \]
  \end{proof}
  
  \begin{corollary}
    \label{choice lemma properties of select}If $x$ is select then $\forall y
    \in P$ we have $y \leqslant x$ or $\tmop{succ} (x) \leqslant y$
  \end{corollary}
  
  \begin{proof}
    As $A_x$ is a p-sequence by the previous lemma [lemma: \ref{choice lemma
    p-sequence generation}] we have by [eq: \ref{eq 3.59.018}] that $P
    \subseteq A_x$ and as by definition of $A_x$ $A_x \subseteq P$ it follows
    that
    \[ P = A_x \]
  \end{proof}
  
  \begin{lemma}
    \label{choice lemma select elements froms a p-sewuence}The set $\left\{ x
    \in X|x \text{ is select} \right\}$ is \ a p-sequence.
  \end{lemma}
  
  \begin{proof}
    
    \begin{enumerate}
      \item As $p$ is a least element of $X$ we have $\forall x \in P$ that $p
      \leqslant x$ so it is comparable with every element of $p$, hence $p$ is
      select, so \ $p \in \left\{ x \in X|s \text{ is select} \right\}$.
      
      \item If $x \in \left\{ x \in X|x \text{ is select} \right\}$ then $x$
      is select and by [corollary: \ref{choice lemma properties of select}] we
      have $\forall y \in P$ either:
      \begin{description}
        \item[$y \leqslant x$] Then as $\tmop{succ} (x)$ is the immediate
        successor of $x$ we have $x < \tmop{succ} (x)$ so that $y <
        \tmop{succ} (x) \Rightarrow y \leqslant \tmop{succ} (x)$ proving that
        $\tmop{succ} (x)$ is comparable with $y$
        
        \item[$\tmop{succ} (x) \leqslant y$] Then $\tmop{succ} (x)$ is
        comparable with $y$
      \end{description}
      from the above it follows that $\tmop{succ} (x)$ is comparable with
      every $y \in P$ hence
      \[ \tmop{succ} (x) \in \left\{ x \in X|x \text{ is selected} \right\} \]
      \item Let $C \subseteq \left\{ x \in X|x \text{ is select} \right\}$ be
      a chain. Then as $C \subseteq X$ we have the hypothesis (3) that $\sup
      (C)$ exist. Then $\forall y \in P$ we have the following possibilities
      for $C$:
      \begin{description}
        \item[$\exists x \in C \text{ with } y \leqslant x$] Then $x \leqslant
        \sup (C)$ so that $y \leqslant \sup (C)$ so that $\sup (C)$ is
        comparable with $y$
        
        \item[$\forall x \in C$ we have $\neg (y \leqslant x)$] Then given $x
        \in C$ we have as $C \subseteq \left\{ x \in X|x \text{ is select}
        \right\}$ that $x$ is select. By [corollary: \ref{choice lemma
        properties of select}] we have either $y \leqslant x$ which is not
        allowed or $\tmop{succ} (x) \leqslant y$. As $\tmop{succ} (x)$ is a
        immediate successor of $x$ we have $x < \tmop{succ} (x)$ so that $x <
        y$ proving that $y$ is a upper bound of $C$. Hence $\sup (C) \leqslant
        y$ proving that $\sup (C)$ is comparable with $y$
      \end{description}
      So in all cases we have that $\sup (C)$ is comparable with $y$ proving
      that $\sup (C)$ is select and thus that $\sup (C) \in \left\{ x \in X|x
      \text{ is select} \right\}$
    \end{enumerate}
    From (1),(2),(3) it follows then that $\left\{ x \in X|x \text{ is select}
    \right\}$ is a p-sequence.
  \end{proof}
  
  Now for the last corollary in the proof.
  
  \begin{corollary}
    $P$ is a chain
  \end{corollary}
  
  \begin{proof}
    As by the previous lemma [lemma: \ref{choice lemma select elements froms a
    p-sewuence}] $\left\{ x \in X|x \text{ is select} \right\}$ is a
    p-sequence it follows from [eq: \ref{eq 3.59.018}] that $P \subseteq
    \left\{ x \in X|x \text{ is select} \right\}$. So if $x, y \in P$ then $x$
    is select and as $y \in P$ comparable with $y$, proving that $P$ is a
    chain.
  \end{proof}
  
  We are now finally able to reach a contradiction and prove the theorem. As
  $P$ is a chain we have by hypothesis (2) that $\sup (P)$ exist. Now as $P$
  is a p-sequence [see eq: \ref{eq 3.59.018}] we have by [definition:
  \ref{choice lemma p-sequence} (3)] that $\sup (P) \in P$ and by [definition:
  \ref{choice lemma p-sequence} (2)] that $\tmop{succ} (\sup (P)) \in P$ so
  that $\tmop{succ} (\sup (P)) \leqslant \sup (P)$. As $\tmop{succ} (\sup
  (P))$ is the immediate successor of $\sup (P)$ we have that $\sup (P) <
  \tmop{succ} (\sup (P))$. Hence $\sup (P) < \sup (P)$ which is a
  contradiction.
\end{proof}

This was a long proof but it will be used in the following important theorem.

\begin{definition}
  \label{choice Hausdorff maximal principle}{\index{Hausdorff's maximality}}A
  partial ordered set $\langle X, \leqslant \rangle$ is \tmtextbf{Hausdorff
  maximal} if there exist a chain $C$ such that if $D$ is a chain with $C
  \subseteq D$ then $C = D$. In other words $C$ is maximal when using the
  order relation defined by $\subseteq$.
\end{definition}

We show now that as a consequence of the Axiom of choice every partial ordered
set is Hausdorff maximal.

\begin{theorem}[Hausdorff's Maximal Theorem]
  \label{choice Hausdorff's Maximal Principle}Let $\langle X, \leqslant
  \rangle$ be a partial ordered set then it is Hausdorff maximal. In other
  words there exists a chain $C$ such that if $D$ is a chain such that $C
  \subseteq D$ then $C = D$. 
\end{theorem}

\begin{proof}
  Define the set of all chain of $X$
  \[ \mathcal{C}= \left\{ A \in \mathcal{P} (X) |A \text{ is a chain in }
     \langle X, \leqslant \rangle \right\} \]
  Using the fact $\mathcal{P} (X)$ is a set by the Axiom of Power Sets [axiom:
  \ref{axiom of power}] we have by the Axiom of Subsets [axiom: \ref{axiom of
  subsets}] and the fact that $\mathcal{C} \subseteq \mathcal{P} (X)$ it
  follows that
  \begin{equation}
    \label{eq 3.61.018} \mathcal{C} \text{ is a set}
  \end{equation}
  Using [example: \ref{order inclusion is a order}] we have that
  \[ \langle \mathcal{C}, \preccurlyeq \rangle \text{ where } \preccurlyeq =
     \{ (x, y) \in \mathcal{C} \times \mathcal{C}|x \subseteq y \} \text{ is a
     partial ordered set} \]
  As $\forall A \in \mathcal{C}$ we have $\varnothing \subseteq A \Rightarrow
  \varnothing \preccurlyeq A$ and $\varnothing$ is a chain [see example:
  \ref{order empty set is a chain}] in $\langle X, \leqslant \rangle \text{}
  \tmop{it} \tmop{follows} \tmop{that} \text{}$
  \begin{equation}
    \label{eq 3.62.018} \mathcal{C} \text{ has a least element [using
    $\preccurlyeq$]}
  \end{equation}
  Let $\mathcal{D}$ a chain in $\langle \mathcal{C}, \preccurlyeq \rangle$
  then if $x, y \in \bigcup \mathcal{D}$ there exists $A, B \in \mathcal{D}
  \subseteq \mathcal{C}$ such that $x \in A \wedge y \in B$ where $A, B$ are
  chains in $\langle X, \leqslant \rangle$. \ As $\mathcal{D}$ is a chain we
  have either:
  \begin{description}
    \item[$A \subseteq B$] Then $x, y \in B$ which as $B$ is a chain [using
    $\leqslant$] means that $x, y$ are comparable [using the order
    $\leqslant$]
    
    \item[$B \subseteq A$] Then $x, y \in A$ which as $A$ is a chain [using
    $\leqslant$] \ means that $x, y$ are comparable [using the order
    $\leqslant$]
  \end{description}
  From the above it follows that $\bigcup \mathcal{D}$ is a chain in $\langle
  X, \leqslant \rangle$ hence $\bigcup \mathcal{D} \in \mathcal{C}$. Hence by
  [example: \ref{order example inclusion order and sup, inf}] it follows that
  $\bigcup \mathcal{D}= \sup (\mathcal{D})$ [using $\preccurlyeq$]. So we have
  proved that
  \begin{equation}
    \label{eq 3.63.018} \text{Every chain of } \langle \mathcal{C},
    \preccurlyeq \rangle \text{ has a supremum}
  \end{equation}
  Now the conditions for [theorem: \ref{choice existence of successor}] are
  satisfied by [eq: \ref{eq 3.61.018}], [eq: \ref{eq 3.62.018}] and [eq:
  \ref{eq 3.63.018}] so we have
  \begin{equation}
    \label{eq 3.64.018} \exists C \in \mathcal{C} \text{ [so $C \text{ is a
    chain in $\langle X, \leqslant \rangle$]}$ which has no immediate
    successor}  [\tmop{using} \preccurlyeq] 
  \end{equation}
  Let now $D$ be a chain in $\langle X, \leqslant \rangle$ [so that $D \in
  \mathcal{C}$] such that $C \subseteq D$. Take $d \in D$ and assume that $d
  \nin C$ then $C \subset C \bigcup \{ d \}$ [as $C \bigcup \{ d \} \nsubseteq
  C \Rightarrow C \neq C \bigcup \{ d \}$] so that $C \prec C \bigcup \{ d
  \}$. As $C$ has no immediate successor [using $\prec$] there must be a $H
  \in \mathcal{C}$ such that $C \prec H \wedge H \prec C \bigcup \{ d \}$ or
  $C \subset H \wedge H \subset C \bigcup \{ d \}$. As $C \subset H$ there
  exists a $h \in H$ such that $h \nin C$, but then as $H \subset C \bigcup \{
  d \}$ we must have $h \in \{ d \}$ or $h = d$, so $d \in H$. Now as $H
  \subset C \bigcup \{ d \}$ there exists a $y \in C \bigcup \{ d \}$ such
  that $y \nin H$, we can not have $y = d$ [as $d \in H$] so we must have $y
  \in C$ but then as $C \subset H$ we have $y \in H$ contradicting $y \in H$.
  So we must have $d \in C$. As $d \in D$ was chosen arbitrary we have that $D
  \subseteq C$ or $C = D$ which proves maximality.
\end{proof}

We state now Zorn's lemma but not prove it yet, it will be show to be directly
dependent on the Hausdorff maximal principle, which in turn depends on the
Axiom of Choice. So if we accept the Axiom of Choice [which we do as it is a
expressed as a Axiom] then Zorn's lemma applies.

\begin{lemma}[Zorn's Lemma]
  \label{choice Zorn's lemma}{\index{Zorn's Lemma}}Let $\langle X, \leqslant
  \rangle$ be a partial ordered set such that every chain has a upper bound
  then $X$ has a maximal element.
\end{lemma}

We prove now that the Hausdorff Maximal principle implies Zorn's lemma.

\begin{theorem}
  \label{choice Hausdorff's implies Zorn's}Let $\langle X, \leqslant \rangle$
  be Hausdorff Maximal then Zorn's lemma follows.
\end{theorem}

\begin{proof}
  Let $\langle X, \leqslant \rangle$ be a partial ordered set such that every
  chain in $X$ has a upper bound. As $\langle X, \leqslant \rangle$ is
  Hausdorff maximal [definition: \ref{choice Hausdorff maximal principle}]
  there exist a chain $C$ such that for every chain $D$ with $C \subseteq D$
  we have $C = D$. As $C$ is a chain it has by the hypothesis a upper bound
  $u$ for $C$. Assume now that $u$ is not a maximal element of $X$, then by
  the definition of a maximal element [definition: \ref{order maximal minimal
  element}] there exist a $x \in X$ with $u \leqslant x$ and $u \neq x$ so
  that $u < x$. If $x \in C$ then as $u$ is a upper bound of $C$ we have $x
  \leqslant u$ so that $u < u$ a contradiction. So we must have that $x \nin
  C$. Consider now $r, s \in C \bigcup \{ x \}$ then we have to consider the
  following possibilities:
  \begin{description}
    \item[$r = x \wedge s = x$] Then by reflectivity we have $r \leqslant s$,
    so $r, s$ are comparable.
    
    \item[$r = x \wedge s \neq x$] Then $s \in C$ so that $s \leqslant u$,
    which as $u \leqslant x$ proves that $s \leqslant x \Rightarrowlim_{r = x}
    s \leqslant r$, so $r, s$ are comparable.
    
    \item[$r \neq x \wedge s = x$] Then $r \in C$ so that $r \leqslant u$,
    which as $u \leqslant x$ proves that $r \leqslant x \Rightarrowlim_{s = x}
    r \leqslant s$, so $r, s$ are comparable.
    
    \item[$r \neq x \wedge s \neq x$] Then $r, s \in C$, which as $C$ is a
    chain proves that $r, s$ are comparable
  \end{description}
  From the above it follows that $C \bigcup \{ x \}$ is a chain such that $C
  \subseteq C \bigcup \{ x \}$ giving by maximality of $C$ that $C = C \bigcup
  \{ x \}$ contradicting $x \nin C$. Hence the assumption that $u$ is not a
  maximal element of $X$ is false. So $u$ is a maximal element of $X$.
\end{proof}

We show now that Zorn's lemma implies well ordering.

\begin{theorem}
  \label{choice Zorn implies welll ordering}Zorn's lemma implies that given a
  set $X$ there exist a order relation $\leqslant$ on $X$ such that $\langle
  X, \leqslant \rangle$ is well ordered [see \ref{order well-rodered class}] 
\end{theorem}

\begin{proof}
  Just like the proof of [theorem: \ref{choice existence of successor}] this
  proof will consist of many sub lemma's.
  
  Let $X$ be a set and define the class
  \[ \mathcal{A}= \left\{ (B, R) |B \in \mathcal{P} (A) \wedge R \text{ a
     order relation on $B$ so that } \langle B, R \rangle \text{ is well
     ordered} \right\} \]
  Define now $\preccurlyeq \in \mathcal{A} \times \mathcal{A}$ by
  \[ \preccurlyeq = \left\{ ((B, R), (B', R')) |B \subseteq B' \wedge R
     \subseteq R' \wedge \text{If } x \in B \wedge \text{} y \in B' \backslash
     B \text{ then } (x, y) \in R' \right\} \]
  then we have that
  \begin{equation}
    \label{eq 3.65.018} \langle \mathcal{A}, \preccurlyeq \rangle \text{ is a
    order relation}
  \end{equation}
  \begin{proof}
    We have to prove reflexivity, anti-symmetry and transitivity:
    \begin{description}
      \item[reflectivity] If $(B, R) \in \mathcal{A}$ then we have
      \begin{enumerate}
        \item $B \subseteq B$
        
        \item $R \subseteq R$
        
        \item If $x \in B \wedge y \in B\backslash B
        \equallim_{\text{[theorem: \ref{class universal and empotyset
        properties}]}} \varnothing$ which can not occur so that $(x, y) \in R$
        is satisfied vacuously
      \end{enumerate}
      proving that $(B, R) \preccurlyeq (B, R)$
      
      \item[anti-symmetry] If $(B, R) \preccurlyeq (B', R') \wedge (B', R')
      \preccurlyeq (B, R)$ then $B \subseteq B' \wedge R \subseteq R' \wedge
      B' \subseteq B \wedge R' \subseteq R$ proving that $B = B'$ and $R = R'$
      so that $(B, R) = (B', R')$
      
      \item[transitivity] Let $(B, R) \preccurlyeq (B', R')$ and $(B', R')
      \preccurlyeq (B'', R'')$ then we have
      \begin{enumerate}
        \item $B \subseteq B' \wedge B' \subseteq B'' \Rightarrow B \subseteq
        B''$
        
        \item $R \subseteq R' \wedge R' \subseteq R'' \Rightarrow R \subseteq
        R''$
        
        \item If $x \in B \wedge y \in B'' \backslash B$ we have for $y$ to
        consider the following possibilities
        \begin{description}
          \item[$y \in B'$] Then $y \in B' \backslash B$ so that $(x, y) \in
          R' \Rightarrowlim_{R' \subseteq R''} (x, y) \in R''$
          
          \item[$y \nin B'$] Then $y \in B'' \backslash B'$ so that $(x, y)
          \in R''$
        \end{description}
        so in all cases we have $(x, y) \in R''$.
      \end{enumerate}
      proving $(B, R) \preccurlyeq (B'', R'')$.
    \end{description}
  \end{proof}
  
  We now have the following sub lemma:
  
  \begin{lemma}
    \label{choice lemma well ordering lemma (1)}If $\mathcal{C} \subseteq
    \mathcal{A}$ is a chain in $\langle \mathcal{A}, \preccurlyeq \rangle$
    then if
    \[ B_{\mathcal{C}} = \bigcup \left\{ B| \exists R \text{ such that } (B,
       R) \in \mathcal{C} \right\} \text{} \]
    \[ R_{\mathcal{C}} = \bigcup \left\{ R| \exists B \text{ such that } (B,
       R) \in \mathcal{C} \right\} \]
    then
    \[ \left( B_{\mathcal{C}} {, R_{\mathcal{C}}}  \right) \in \mathcal{A} \]
  \end{lemma}
  
  \begin{proof}
    First note that if $(B, R) \in \mathcal{C}$ then
    \[ B \in \left\{ B| \exists R \text{ such that } (B, R) \in \mathcal{C}
       \right\} \]
    and
    \[ R \in \left\{ R| \exists B \text{ such that } (B, R) \in \mathcal{C}
       \right\} \]
    or
    \begin{equation}
      \label{eq 3.66.019} \forall (B, R) \in \mathcal{C} \text{ we have } B
      \subseteq B_{\mathcal{C}} \wedge R \subseteq R_{\mathcal{C}}
    \end{equation}
    \begin{enumerate}
      \item If $x \in B_{\mathcal{C}}$ then $\exists (B, R) \in \mathcal{C}$
      such that $x \in B$, as $\mathcal{C} \subseteq \mathcal{A}$ we have $(B,
      R) \in \mathcal{C}$, so that $B \in \mathcal{P} (A)$, hence $B \subseteq
      A$, proving that $x \in A$. In other words $B_{\mathcal{C}} \subseteq A$
      or $B \in \mathcal{P} (A)$.
      
      \item We must prove that $R_{\mathcal{C}}$ is a a order relation on
      $B_{\mathcal{C}}$:
      \begin{description}
        \item[reflectivity] If $x \in B_{\mathcal{C}}$ then $\exists (B, R)
        \in \mathcal{C}$ such that $x \in B$, as $R$ is a order relation we
        have that $(x, x) \in R$ so that by [eq: \ref{eq 3.66.019}] $(x, x)
        \in R_{\mathcal{C}}$
        
        \item[anti-symmetry] If $(x, y) \in R_{\mathcal{C}} \wedge (y, x) \in
        R_{\mathcal{C}}$ then $\exists (B, R), (B', R') \in \mathcal{C}$ such
        that $(x, y) \in R$ and $(y, x) \in R'$. As $\mathcal{C}$ is a chain
        we have either:
        \begin{description}
          \item[$(B, R) \preccurlyeq (B', R')$] Then $R \subseteq R'$ so that
          $(x, y) \in R' \wedge (y, x) \in R'$, which as $R'$ is a order
          relation proves that $x = y$.
          
          \item[$(B', R') \preccurlyeq (B, R)$] Then $R' \subseteq R$ so that
          $(x, y) \in R \wedge (y, x) \in R$, which as $R$ is a order relation
          proves that $x = y$.
        \end{description}
        \item[transitivity] If $(x, y) \in R_{\mathcal{C}} \wedge (y, z) \in
        R_{\mathcal{C}}$ then $\exists (B, R), (B', R') \in \mathcal{C}$ such
        that $(x, y) \in R$ and $(y, x) \in R'$. As $\mathcal{C}$ is a chain
        we have either:
        \begin{description}
          \item[$(B, R) \preccurlyeq (B', R')$] Then $R \subseteq R'$ so that
          $(x, y) \in R' \wedge (y, z) \in R'$, which as $R'$ is a order
          relation proves that $(x, z) \in R'$, hence $(x, z) \in
          R_{\mathcal{C}}$ [see eq: \ref{eq 3.66.019}].
          
          \item[$(B', R') \preccurlyeq (B, R)$] Then $R' \subseteq R$ so that
          $(x, y) \in R \wedge (y, z) \in R$, which as $R$ is a order relation
          proves that $(x, z) \in R$, hence $(x, z) \in R_{\mathcal{C}}$ [see
          eq: \ref{eq 3.66.019}].
        \end{description}
      \end{description}
      \item Next we have to prove well ordering of $\langle B_{\mathcal{C}},
      R_{\mathcal{C}} \rangle$. Let $D \subseteq B_{\mathcal{C}}$ and $D \neq
      \varnothing$. Then there exist a $x \in D$ so that $x \in
      B_{\mathcal{C}}$, hence there exist a $(B, R) \in \mathcal{C}$ such that
      $x \in B$ or $x \in D \bigcap B$ proving that $D \bigcap B \neq
      \varnothing$. As $\mathcal{C} \subseteq \mathcal{A}$ we have by the
      definition of $\mathcal{A}$ that $\langle B, R \rangle$ is well ordered,
      hence there exist a least element $b \in B$. So
      \begin{equation}
        \label{eq 3.66.018} \forall y \in B \text{ we have } (b, y) \in R
      \end{equation}
      We prove now that
      \[ b \text{ is a least element of } D \]
      \begin{proof}
        If $x \in D$ then $\exists (B', R')$ such that $x \in B'$. For $x$ and
        $B$ we the following possible cases:
        \begin{description}
          \item[$x \in B$] Then by [eq: \ref{eq 3.66.018}] we have that $(b,
          x) \in R$ so that by [eq: \ref{eq 3.66.019}] \ $(b, x) \in
          R_{\mathcal{C}}$.
          
          \item[$x \nin B$] Then $x \in B' \backslash B \wedge b \in B$. As
          $\mathcal{C}$ is a chain we have the following cases:
          \begin{description}
            \item[$(B, R) \preccurlyeq (B', R')$] Then by definition of
            $\preccurlyeq$ we have $(b, x) \in R'$ so that by [eq: \ref{eq
            3.66.019}] \ $(b, x) \in R_{\mathcal{C}}$
            
            \item[$(B', R') \preccurlyeq (B, R)$] Then $B' \subseteq B$ and as
            $x \in B'$ we have $x \in B$ contradicting $x \nin B$. So this
            case never occurs.
          \end{description}
        \end{description}
        So in all cases that apply we have $(b, x) \in R_{\mathcal{C}}$
        proving that $b$ is a least element of $D$.
      \end{proof}
      
      As we have proved that every non empty $D \subseteq B_C$ has a least
      element [using the order $R_{\mathcal{C}}$ it follows that $\langle
      B_{\mathcal{C}}, R_{\mathcal{C}} \rangle$ is well ordered.
    \end{enumerate}
    From (1),(2) and (3) it follows that
    \[ (B_{\mathcal{C}}, R_{\mathcal{C}}) \in \mathcal{A} \]
  \end{proof}
  
  \begin{lemma}
    \label{choice lemma upper bound of chain}If $\mathcal{C}$ is a chain in
    $\langle \mathcal{A}, \preccurlyeq \rangle$ then $(B_{\mathcal{C}},
    R_{\mathcal{C}})$ is a upper bound of $\mathcal{C}$
  \end{lemma}
  
  \begin{proof}
    Let $(B, R) \in \mathcal{C}$ then
    \begin{enumerate}
      \item $B \subseteq B_{\mathcal{C}}$ [see eq: \ref{eq 3.66.019}]
      
      \item $R \subseteq R_{\mathcal{C}}$ [see eq: \ref{eq 3.66.019}]
      
      \item Let $x \in B$ and $y \in B_{\mathcal{C}} \backslash B$ then
      $\exists (B', R') \in \mathcal{C}$ such that $y \in B'$ or as $y \in
      B_{\mathcal{C}} \backslash B$ that
      \[ y \in B' \backslash B \]
      As $\mathcal{C}$ is a chain we have either $(B, R) \preccurlyeq (B',
      R')$ or $(B', R') \preccurlyeq (B, R)$. If $(B', R') \preccurlyeq (B,
      R)$ then $B' \subseteq B$, as $y \in B'$ we would have $y \in B$
      contradiction $y \in B_{\mathcal{C}} \backslash B$. So we have
      \[ (B, R) \preccurlyeq (B', R') \]
      As $x \in B$ and $y \in B' \backslash B$ we have by definition of
      $\preccurlyeq$ and the above that $(x, y) \in R'$ which as $R' \subseteq
      R_C$ [see eq: \ref{eq 3.66.019}] proves that $(x, y) \in
      R_{\mathcal{C}}$
    \end{enumerate}
    So by the definition of $\preccurlyeq$ we have by (1),(2) and (3) that
    \[ (B, R) \preccurlyeq (B_{\mathcal{C}}, R_{\mathcal{C}}) \]
  \end{proof}
  
  Using Zorn's [lemma: \ref{choice Zorn's lemma}] together with the above
  lemma [lemma: \ref{choice lemma upper bound of chain}] we have
  \begin{equation}
    \label{eq 3.68.019} \exists (B_m, R_m) \in \mathcal{A} \text{ such that }
    (B_m, R_m) \text{ is a maximum element of $\mathcal{A}$}
  \end{equation}
  We prove now by contradiction that
  \[ B_m = X \]
  \begin{proof}
    Assume that $X \neq B_m$. Then as $B_m \in \mathcal{P} (X) \Rightarrow B_m
    \subseteq X$ there exist a
    \[ x \in X\backslash B_m \Rightarrow x \nin B_m . \]
    Define
    \begin{equation}
      \label{eq 3.69.019} R^{\ast} = R_m \bigcup \{ (b, x) |b \in B_m \}
      \bigcup \{ (x, x) \}
    \end{equation}
    Then if $(r, s) \in R_m \bigcap \{ (b, x) |b \in B_m \}$ we have as $R_m
    \subseteq B_m \times B_m$ that $s \in B_m \wedge s = x \nin B_m$ a
    contradiction, if $(r, s) \in R_m \bigcap \{ (x, x) \}$ then $r \in B_m
    \wedge r = x \nin B_m$ a contradiction and finally if $(r, s) \in \{ (b,
    x) |b \in B_m \} \bigcap \{ (x, x) \}$ then $r \in B_m \wedge r = x \nin
    B_m$ a contradiction. So we have
    \begin{equation}
      \label{eq 3.70.019} R_m \bigcap \{ (b, x) |b \in B_m \} = \varnothing
      \wedge R_m \bigcap \{ (x, x) \} = \varnothing \wedge \{ (b, x) |b \in
      B_m \} \bigcap \{ (x, x) \} = \varnothing
    \end{equation}
    Further if $ (x, r) \in R^{\star}$ then we have either $(x, r) \in R_m
    \Rightarrow x \in B_m$ contradicting $x \nin B_m$,, $(x, r) \in \{ (b, x)
    |b \in B_m \} \Rightarrow x \in B_m \text{ contradicting } x \nin B_m$ or
    $(x, r) \in \{ (x, x) \} \Rightarrow r = x$. To summarize we have
    \begin{equation}
      \label{eq 3.71.019} \text{If } (x, r) \in R^{\ast} \text{ then } r = x
    \end{equation}
    We prove now that $\left\langle B_m \bigcup \{ x \}, R^{\ast}
    \right\rangle \text{}$ is well ordered.
    
    \begin{proof}
      First we have:
      \begin{description}
        \item[reflexivity] If $r \in B_m \bigcup \{ x \}$ then we have either:
        \begin{description}
          \item[$r \in B_m$] Then as $\langle B_m, R_m \rangle$ is a partial
          order we have $(r, r) \in R_m \subseteq R^{\ast}$.
          
          \item[$r \nin B_m$] Then $r \in \{ x \}$ so that $r = x$ hence $(r,
          r) = (x, x) \in \{ (x, x) \} \subseteq R^{\ast}$
        \end{description}
        proving that $(r, r) \in R^{\ast}$.
        
        \item[anti-symmetry] If $(r, s) \in R^{\ast}$ and $(s, r) \in
        R^{\ast}$ then we have by [eq: \ref{eq 3.69.019}] for $(r, s)$ either:
        \begin{description}
          \item[$(r, s) \in R_m$] Then as $R_m \subseteq B_m \times B_m$ we
          have $r, s \in B_m$ so that $r \neq x \neq s$ so that $(s, r) \in
          R_m$ [if $(s, r) \in \{ (b, x) |b \in B \} \bigcup \{ (x, x) \}$
          then $r = x$ contradicting $r \neq x$], which as $\langle B_m, R_m
          \rangle$ is a partial order gives that $r = s$.
          
          \item[$(r, s) \in \{ (b, x) |b \in B_m \}$] Then $s = x$ so that
          $(x, r) = (s, r) \in R^{\ast} \Rightarrowlim_{\text{[eq: \ref{eq
          3.71.019}]}} r = x = s$ hence $s = r$.
          
          \item[$(r, s) \in \{ (x, x) \}$] Then $r = x = s \Rightarrow r = s$.
        \end{description}
        proving $r = s$
        
        \item[transitivity] If $(r, s) \in R^{\ast} \wedge (s, t) \in
        R^{\ast}$ then we have by [eq: \ref{eq 3.69.019}] that:
        \begin{description}
          \item[$(r, s) \in R_m $] We have the following case for $(s, t)$:
          \begin{description}
            \item[$(s, t) \in R_m$] Then as $\langle B_m, R_m \rangle$ is a
            partial ordered we have $(r, t) \in R_m \subseteq R^{\ast}$.
            
            \item[$(s, t) \in \{ (b, x) |b \in B_m \}$] Then $t = x$ and $r
            \in B_m$ so that $(r, t) \in \{ (b, x) |b \in B_m \} \subseteq
            R^{\ast}$.
            
            \item[$(s, t) \in \{ (x, x) \}$] Then $t = x$ and $r \in B_m$ so
            that $(r, t) \in \{ (b, x) |x \in B_m \} \subseteq R^{\ast}$.
          \end{description}
          \item[$(r, s) \in \{ (b, x) |b \in B_m \}$] Then $s = x$ so that
          $(s, t) = (x, t) \in R^{\ast} \Rightarrowlim_{\text{[eq: \ref{eq
          3.71.019}]}} t = x$. As $r \in B_m$ we have $(r, t) \in \{ (b, x) |b
          \in B_m \} \subseteq R^{\ast}$.
          
          \item[$(r, s) \in \{ (x, x) \}$] Then $r = x \wedge t = x$ so that
          $(x, t) = (s, t) \in R^{\ast} \Rightarrowlim_{\text{[eq: \ref{eq
          3.71.019}]}} t = x$ hence $(r, t) = (x, x) \in \{ (x, x) \}
          \subseteq R^{\star} .$
        \end{description}
        proving $ (r, t) \in R^{\ast}$.
      \end{description}
      Hence
      \[ \left\langle B_m \bigcup \{ x \}, R^{\ast} \right\rangle \text{ is
         partial ordered } \]
      If $\varnothing \neq C \subseteq B_m \bigcup \{ x \}$ is non empty then
      we have for $C \bigcap B_m$ the following possibilities:
      \begin{description}
        \item[$C \bigcap B_m \neq \varnothing$] Then as $\varnothing \neq C
        \bigcap B_m \subseteq B_m$ and $\langle B_m, R_m \rangle$ is well
        ordered [see definition of $\mathcal{A}$] there exist a least element
        $l \in C \bigcap B_m$ so
        \begin{equation}
          \label{eq 3.72.019} \forall r \in C \bigcap B_m \text{ we have } (l,
          r) \in R_m
        \end{equation}
        Now if $r \in C$ we have either:
        \begin{description}
          \item[$r \in B_m$] then $r \in C \bigcap B_m$ so that by the above
          [eq: \ref{eq 3.72.019}] $(l, r) \in R_m \subseteq R^{\ast}$
          
          \item[$r \nin B_m$] then as $C \subseteq B_m \bigcup \{ x \}$ we
          have $r = x$ so $(l, r) \in \{ (b, x) |b \in B_m \} \bigcup \{ (x,
          x) \} \subseteq R^{\ast}$
        \end{description}
        proving that $(l, r) \in R^{\ast}$. Hence
        \[ C \text{ has a least element[} \tmop{using} \left\langle B \bigcup
           \{ x \}, R^{\ast} \right] \]
        \item[$C \bigcap B_m = \varnothing$] Then $C = \{ x \}$ so that
        $\forall r \in C$ we have $r = x$ so that $(r, x) = (x, x) \in \{ (x,
        x) \} \subseteq R^{\ast}$ proving that $x$ is a least element of $C$.
      \end{description}
      So in all cases we have that $C$ has a least element, hence
      \[ \left\langle B_m \bigcup \{ x \}, R^{\ast} \right\rangle \text{ is
         well ordered} \]
    \end{proof}
    
    Now as $B_m \bigcup \{ x \} \subseteq X$, we have by the definition of
    $\mathcal{A}$ and the above that
    \[ \left( B_m \bigcup \{ x \}, R^{\ast} \right) \in \mathcal{A} \]
    Next we have:
    \begin{enumerate}
      \item $B_m \subseteq B_m \bigcup \{ x \}$
      
      \item $R_m \subseteq R^{\ast}$
      
      \item If $r \in B_m$ and $s \in \left( B_m \bigcup \{ x \} \right)
      \backslash B_m$ then $s = x$ so that $(r, s) = (r, x) \in \{ (b, x) |b
      \in B_m \} \subseteq R^{\ast}$
    \end{enumerate}
    proving that $(B_m, R_m) \preccurlyeq \left( B_m \bigcup \{ x \}, R^{\ast}
    \right)$. As $(B_m, R_m)$ is a maximal element of $\langle \mathcal{A},
    \preccurlyeq \rangle$ we must have $(B_m, R_m) = \left( B_m \bigcup \{ x
    \}, R^{\ast} \right)$ so that $B = B \bigcup \{ x \} \tmop{which}
    \tmop{as} x \nin B_m$ leads to a contradiction. Hence the assumption that
    $X \neq B_m$ is wrong and we must have that
    \[ X = B_m \]
  \end{proof}
  
  As $\langle B_m, R_m \rangle$ is a well ordered the above proves that there
  exists a partial order $R_m$ such that
  \[ \langle X, R_m \rangle = \langle B_m, R_m \rangle  \text{ is
     well-ordered [by definition of $\mathcal{A}$ } B_m \text{ is well
     ordered]} \]
\end{proof}

We show now that Well Ordering implies the Axiom of Choice.

\begin{theorem}
  \label{choice well-order implies Axiom of Choice}Assume that for every $X$
  there exist a order relation such that $\langle X, \leqslant \rangle$ is
  well ordered then there exists a function $c : \mathcal{P}' (X) \rightarrow
  X$ such that $\forall A \in \mathcal{P}' (X)$ we have $c (A) \in A$ (Axiom
  of Choice).
\end{theorem}

\begin{proof}
  Let $X$ be a set then by the hypothesis there exist a order $\leqslant$ on
  $X$ such that $\langle X, \leqslant \rangle$ is well ordered. Define now $c
  = \left\{ (A, x) |A \in \mathcal{P}' (X) \wedge x \text{ is a least element
  of $A$} \right\}$. If $(A, x) \in c$ then $A \in \mathcal{P}' (X)$ and $x$
  is a least element of $A$, so that $x \in A \subseteq X$ proving that $(A,
  x) \in \mathcal{P}' (X) \times X$. So $c \subseteq \mathcal{P}' (X) \times
  X$. If $(A, x), (A, x') \in c$ then $x$ and $x'$ are least elements of $A$,
  which are unique by [theorem: \ref{order greatest and lowest element are
  unique}] so that $x = x'$. Hence we have that
  \[ c : \mathcal{P}' (X) \rightarrow X \text{ is a partial function} \]
  If $A \in \mathcal{P}' (X)$ then $A \neq \varnothing$ so by well ordering
  $A$ has a least element $l$ so that $(A, l) \in c$, so $\mathcal{P}' (A)
  \subseteq \tmop{dom} (c)$. Hence by [proposition: \ref{function condition
  (1)}] we have that
  \[ c : \mathcal{P}' (X) \rightarrow X \text{ is a function} \]
  If $(A, x) \in c$ then $x$ is the least element of $A$ so that $c (A) = x
  \in A$ proving that
  \[ c : \mathcal{P}' (X) \rightarrow X \text{ is a choice function}
     \tmop{for} X \]
\end{proof}

We are now ready to specify the different equivalent statements of the Axiom
of Choice

\begin{theorem}
  The following statements are equivalent
  \begin{enumerate}
    \item Axiom of Choice
    
    \item Hausdorff's Maximal Principle
    
    \item Zorn's Lemma
    
    \item Every set can be well ordered
  \end{enumerate}
\end{theorem}

\begin{proof}
  
  \begin{description}
    \item[$1 \Rightarrow 2$] This follows from [theorem: \ref{choice
    Hausdorff's Maximal Principle}]
    
    \item[$2 \Rightarrow 3$] This follows from [theorem: \ref{choice
    Hausdorff's implies Zorn's}]
    
    \item[$3 \Rightarrow 4$] This follows from [theorem: \ref{choice Zorn
    implies welll ordering}]
    
    \item[$4 \Rightarrow 1$] This follows from [theorem: \ref{choice
    well-order implies Axiom of Choice}]
  \end{description}
\end{proof}

As in most of works about mathematics we assume the Axiom of Choice. To
summarize the consequences of the Axiom of Choice we have [taking in account
[theorem: \ref{choice axiom of choice equivalences (1)}] that the following
statements are true.

\begin{theorem}
  \label{choice Axiom of choice consequences}{\tmdummy}
  
  \begin{description}
    \item[Axiom of Choice] Let $X$ be a set then there exist a function $c :
    \mathcal{P}' (X) \rightarrow X$ such that $\forall A \in \mathcal{P}' (X)$
    we have $c (A) \in A$.
    
    \item[Existence of Choice set] Let $\mathcal{A}$ be a set of sets such
    that
    \begin{enumeratealpha}
      \item $\forall A \in \mathcal{A} \text{ we have } A \neq \varnothing$
      
      \item $\forall A, B \in \mathcal{A} \text{ with } A \neq B$ we have $A
      \bigcap B = \varnothing$
    \end{enumeratealpha}
    then there exist a set $C$ [called the \tmtextbf{choice set of
    $\mathcal{A}$}] such that
    \begin{enumeratealpha}
      \item $C \subseteq \bigcup \mathcal{A}$
      
      \item $\forall A \in \mathcal{A}$ we have $A \bigcap C \neq \varnothing$
      and if $y, y' \in A \bigcap C$ then $y = y'$
    \end{enumeratealpha}
    \item[Axiom of Choice alternative] If $\{ A_i \}_{i \in I} \subseteq
    \mathcal{A}$ is a family of non empty sets [$\forall i \in I$ we have $A_i
    \neq \varnothing$] where $I, \mathcal{A}$ are sets then there exists a
    function $f : I \rightarrow \bigcup_{i \in I} A_i$ such that $\forall i
    \in I$ we have $f (i) \in A_i$
    
    \item[Hausdorff's Maximal Theorem] If $\langle X, \leqslant \rangle$ is a
    partial ordered set then there exists a chain $C \subseteq X$ such that
    for every chain $D \subseteq X$ with $C \subseteq D$ we have $C = D$
    
    \item[Zorn's Lemma] If $\langle X, \leqslant \rangle$ is a partial ordered
    set such that every chain has a upper bound then $X$ has a maximal
    element.
    
    \item[Well-Ordering Theorem] For every set there exists a order relation
    making $\langle X, \leqslant \rangle$ well-ordered.
  \end{description}
\end{theorem}

There is a kind of extension of Zorn's lemma to pre-ordered sets if change the
definition of maximal element slightly.

\begin{theorem}
  \label{choice Zorn's lemma for pre-order}Let $\langle X, \leqslant \rangle$
  be a pre-ordered set [see definitions: \ref{order preordered class},
  \ref{order preorder}] such that every chain has a upper bound then there
  exists a $m \in X$ such that $\forall x \in X$ with $m \leqslant x$ we have
  $x \leqslant m$
\end{theorem}

\begin{proof}
  Using [theorem: \ref{order eq order preorder to order}] we have the
  following
  \begin{enumerate}
    \item $\sim \subseteq X \times X$ defined by $\sim = \{ (x, y) \in X|x
    \leqslant y \wedge y \leqslant x \}$ is a equivalence relation
    
    \item Define $\preccurlyeq \subseteq (X / \sim) \times (X / \sim)$ by
    \[ \preccurlyeq = \left\{ (x, y) \in (X / \sim) \times (X / \sim) |
       \exists x' \in \sim [x] \text{ and } \exists y' \in \sim [y] \text{
       such that $x' \leqslant y'$} \right\} \]
    then $\preccurlyeq$ is a order relation in $X / \sim$. So $\langle X /
    \sim, \preccurlyeq \rangle$ is a partial ordered set
    
    \item $\forall x, y \in A$ we have $x \leqslant y \Leftrightarrow \sim [x]
    \preccurlyeq \sim [y]$
  \end{enumerate}
  Let $C \subseteq X / \sim$ be a chain [using the order $\preccurlyeq$] and
  construct $C' = \bigcup C$. If $x, y \in C'$ then $\exists \sim [x'], \sim
  [y']$ such that $x \in \sim [x']$ and $y \in \sim [y']$, so $x \sim x'$ and
  $y \sim y'$ or $x \leqslant x' \wedge x' \leqslant x$\quad and $y \leqslant
  y' \wedge y' \leqslant y$. As $C$ is a chain [using $\preccurlyeq$] we have
  the following possibilities:
  \begin{description}
    \item[$\sim [x'] \preccurlyeq \sim [y']$] then $x' \leqslant y'$ and as $x
    \leqslant x'$ and $y' \leqslant y$ we have $x \leqslant y$
    
    \item[$\sim [y'] \preccurlyeq \sim [x']$] then $y' \leqslant x'$ and as $y
    \leqslant y'$ and $x' \leqslant x$ we have $y \leqslant x$
  \end{description}
  proving that $x, y$ are comparable. Hence
  \[ C' \text{ is a chain [using } \leqslant \text{]} \]
  By the hypothesis we have that there exist a upper bound $u$ of $C'$ [using
  $\leqslant$], in other words
  \[ \exists u \in X \text{ such that } \forall x \in C' \text{ we have } x
     \leqslant u \]
  Take now $\sim [z] \in C$ then $z \in \sim [z] \subseteq C'$ so that $z
  \leqslant u$ and thus by (3) $\sim [z] \preccurlyeq \sim [u]$. So $\sim [u]$
  is a upper bound of $C$. As we just have proved that every chain in $X /
  \sim$ has a upper bound and $\langle X / \sim, \preccurlyeq \rangle$ is a
  partial order, it follows from Zorn's lemma that there exist a maximal
  element $\sim [m]$ in $X / \sim$. So by [definition: \ref{order maximal
  minimal element}] we have
  \[ \forall \sim [x] \in X / \sim \text{ with $\sim [m] \preccurlyeq \sim
     [x] \text{ we have $\sim [x] = \sim [m]$}$} \]
  If now $x \in X$ such that $x \leqslant m$ then by (3) we have $\sim [x]
  \preccurlyeq \sim [m]$ hence by the above we have $\sim [x] = \sim [m]$ so
  that $x \sim m$ hence $x \leqslant m$.
\end{proof}

From this point on we will gradually start to use the simpler notations for
functions and families that are mentioned in the references [definition:
\ref{function f(x)}], [theorem: \ref{function equality (2)}], [theorem:
\ref{function alternative for composition}], [theorem: \ref{function
injectivity, surjectivity}], [theorem: \ref{function simple definition}],
[notation: \ref{function simple definition notation}], [theorem: \ref{family
union (2)}] and [theorem: \ref{family intersection (2)}] without explicit
referring to them. This to avoid excessive notation and difference of notation
between this text and standard mathematical practice.

\

As a interesting application of the Axiom of Choice we prove that every
function can be restricted to a injection or bijection.

\begin{theorem}
  \label{choice function to injection/bijection}Let $X, Y$ be sets, $f : X
  \rightarrow Y$ a function then there exist a $Z \subseteq X$ such that:
  \begin{enumerate}
    \item $f_{|Z} : Z \rightarrow Y \text{ is a injection}$
    
    \item $f_{|Z} (Z) = f (X)$
    
    \item $f_{|Z} : Z \rightarrow f (X)$ is a bijection
  \end{enumerate}
\end{theorem}

\begin{proof}
  
  \begin{enumerate}
    \item Define
    \[ \mathcal{A}= \{ f^{- 1} (\{ y \}) |y \in f (X) \} . \]
    If $A \in \mathcal{A}$ then $\exists y \in f (X)$ such that $A = f^{- 1}
    (\{ y \}) \subseteq X$ and as $y \in f (X)$ there exists a $x \in X$ such
    that $f (x) = y \in \{ y \} \Rightarrow x \in f^{- 1} (\{ y \}) = A$,
    proving that $A \neq \varnothing$. So we have proved that
    \[ \mathcal{A} \subseteq \mathcal{P}' (X) \]
    By the Axiom of Choice [axiom: \ref{axiom of choice}] there exist a
    function
    \[ c : \mathcal{P}' (X) \rightarrow X \text{ such that } \forall A \in
       \mathcal{P}' (X) \text{ } (c) (A) \in A \]
    Take
    \[ Z = c (\mathcal{A}) \subseteq X \]
    and consider the restriction of $f$ to $Z$
    \[ f_{|Z} : Z \rightarrow Y \]
    Let $x, y \in Z$ such that $f_{|Z} (x) = f_{|Z} (y) \Rightarrowlim_{x, y
    \in Z} f (x) = f (y)$. As $x, y \in Z = c (\mathcal{A})$ there exists $A_x
    \in \mathcal{A} \wedge A_y \in \mathcal{A}$ such that $x = (c) (A_x) \in
    A_x$ and $y = (c) (A_y) \in A_x$. As $A_x, A_y \in \mathcal{A}$ there
    exist $x', y' \in f (X)$ such that $A_x = f^{- 1} (\{ x' \})$ and $A_y =
    f^{- 1} (\{ y' \})$. Then $f (x) \equallim_{x \in A_x} x'$ and $f (y)
    \equallim_{y \in A_y} y'$. As $f (x) = f (y)$ we have $x' = y'$ so that
    $A_x = f^{- 1} (\{ x' \}) = f^{- 1} (\{ y' \}) = A_y$. So $x = (c) (A_x) =
    (c) (A_y) = y$, proving that $x = y$.
    
    \item If $y \in f (X)$ then $f^{- 1} (\{ y \}) \in \mathcal{A}$ so to
    that x=$(c) (f^{- 1} (\{ y \})) \in c (\mathcal{A}) = Z$. Further as $(c)
    (f^{- 1} (\{ y \})) \in f^{- 1} (\{ y \})$ we have that $f (x) = f ((c)
    (f^{- 1} (\{ y \}))) \in \{ y \}$ so that $y = f (x) \in f (Z)$, proving
    that $f (X) \subseteq f (Z)$. As $Z \subseteq X$ we have by [theorem:
    \ref{partial functions image/preimage properties}] that $f (Z) \subseteq f
    (X)$ so that
    \[ f (X) = f (Z) \]
    \item From (2) we have that $f_{|Z} : Z \rightarrow f (X)$ is surjective
    which together with (1) proves bijectivity.
  \end{enumerate}
\end{proof}

Another application of the Axiom of Choice is the following.

\begin{theorem}
  \label{choice family on preimage}Let $I, X, Y$ be sets, $f : X \rightarrow
  Y$ a function and $\{ y_i \}_{i \in I} \subseteq f (X)$ then $\exists \{ x_i
  \}_{i \in I} \subseteq X$ such that $\forall i \in I$ we have $f (x_i) =
  y_i$
\end{theorem}

\begin{proof}
  Define
  \[ \{ A_i \}_{i \in I} \text{ by } A_i = f^{- 1} (\{ y_i \}) \subseteq X \]
  Let $i \in I$ then, as $y_i \in f (X)$, there exist a $x \in X$ such that $f
  (x) = y_i$, hence $x \in f^{- 1} (\{ y_i \}) = A_i$. This proves that
  \[ \forall i \in I \text{ } A_i \neq \varnothing \]
  Using the Axiom of Choice [axiom: \ref{choice axiom of choice equivalences
  (1)}(3)] there exist a function
  \[ x : I \rightarrow \bigcup_{i \in I} A_i \text{ such that } \forall i \in
     I \text{ we have } x (i) \in A_i \]
  If $i \in I$ we have that that $x (i) \in A_i$ so that $f (x (i)) \in \{ y_i
  \} \Rightarrow f (x_i) = y_i$, this together with the fact that we can
  extend the function $x : I \rightarrow \bigcup_{i \in I} A_i$ to $f : I
  \rightarrow X$ proves that we have found a function
  \[ x : I \rightarrow X \text{ such that } \forall i \in I \text{ we have }
     f (x_i) = y_i \]
  This function defines then the family $\{ x_i \}_{i \in I} \subseteq X$
  satisfying $\forall i \in I$ $f (x_i) = y_i$.
\end{proof}

\section{Generalized Intervals}

\begin{definition}
  \label{interval interval}Let $\langle A, \leqslant \rangle$ be a partial
  ordered class then if $a, b \in A$ we define
  \begin{enumerate}
    \item $[a, b] = \{ x \in A|a \leqslant x \wedge x \leqslant b \}$
    
    \item $] a, b] = \{ x \in A|a < x \wedge x \leqslant b \}$
    
    \item $[a, b [= \{ x \in A|a \leqslant x \wedge x < b \}$
    
    \item $] a, b [= \{ x \in A|a < x \wedge x < b \}$
    
    \item $] - \infty, a] = \{ x \in A|x \leqslant a \}$
    
    \item $] - \infty, a [= \{ x \in A|x < a \}$
    
    \item $[a, \infty [= \{ x \in A|a \leqslant x \}$
    
    \item $] a, \infty [= \{ x \in A|a < x \}$
    
    \item $] - \infty, \infty [= A$
  \end{enumerate}
  these are the intervals in $A$
\end{definition}

\begin{theorem}
  \label{interval condition to be empty}Let $\langle A, \leqslant \rangle$ be
  a partial ordered class and $a, b \in A$ then we have:
  \begin{enumerate}
    \item If $b < a$ then $[a, b] = [a, b [=] a, b] =] a, b [= \emptyset$
    
    \item If $b \leqslant a$ then $] a, b] = [a, b [=] a, b [= \emptyset$
  \end{enumerate}
\end{theorem}

\begin{proof}
  
  \begin{enumerate}
    \item Let $b < a$ then we have
    \begin{enumerate}
      \item If $z \in [a, b]$ then $b < a \wedge a \leqslant z \wedge z
      \leqslant b$ so that $b < z \wedge z \leqslant b$ a contradiction.
      
      \item If $z \in] a, b]$ then $b < a \wedge a < z \wedge z \leqslant b$
      so that $b < z \wedge z \leqslant b$ a contradiction.
      
      \item If $z \in [a, b [$ then $b < a \wedge a \leqslant z \wedge z < b$
      so that $b < z \wedge z < b$ a contradiction.
      
      \item If $z \in [a, b]$ then $b < a \wedge a < z \wedge z \leqslant b$
      so that $b < z \wedge z < b$ a contradiction.
    \end{enumerate}
    \item Let $b \leqslant a$ then we have
    \begin{enumerate}
      \item If $z \in [a, b [$ then $b \leqslant a \wedge a \leqslant z \wedge
      z < b$ so that $b \leqslant z \wedge z < b$ a contradiction.
      
      \item If $z \in] a, b]$ then $b \leqslant a \wedge a < z \wedge z
      \leqslant b$ so that $b < z \wedge z \leqslant b$ a contradiction.
      
      \item If $z \in] a, b [$ then $b \leqslant a \wedge a < z \wedge z < b$
      so that $b < z \wedge z < b$ a contradiction.
    \end{enumerate}
  \end{enumerate}
\end{proof}

\begin{definition}[Generalized Interval]
  \label{interval generalized interval}Let $\langle A, \leqslant \rangle$ be a
  partial ordered class then $I \subseteq A$ is a \tmtextbf{generalized
  interval} if $\forall x, y \in I$ we have $[x, y] \subseteq I$.
\end{definition}

We have the following equivalent conditions for a generalized interval.

\

\begin{theorem}
  \label{interval generalized interval condition}Let $\langle A, \leqslant
  \rangle$ be a partial ordered class that is conditional complete and $I
  \subseteq A$ then then
  \[ I \text{ is a generalized interval } \Leftrightarrow \text{ } \forall x,
     y \in I \text{ we have }] x, y [\subseteq I \]
\end{theorem}

\begin{proof}
  
  \begin{description}
    \item[$\Rightarrow$] Let $x, y \in I$ then if $z \in] x, y [$ we have $x <
    z \wedge z < y \Rightarrow x \leqslant z \wedge z \leqslant y \Rightarrow
    z \in [x, y] \subseteq I$ proving that $] x, y [\subseteq I$
    
    \item[$\Leftarrow$] Let $x, y \in I$ then if $z \in [x, y]$ we have either
    \begin{description}
      \item[$x = y$] Then $x \leqslant z \wedge z \leqslant x \Rightarrow z =
      x \in I$
      
      \item[$x < y$] Then we have either
      \begin{description}
        \item[$z = x$] Then $z \in I$
        
        \item[$z = y$] Then $z \in I$
        
        \item[$z \neq x \wedge z \neq y$] Then $x < z \wedge z < y$ proving
        that $z \in] x, y [\subseteq I$
      \end{description}
    \end{description}
    So in all cases we have that $z \in I$ proving that $[x, y] \subseteq I$
  \end{description}
\end{proof}

\begin{theorem}
  \label{interval and inf or sup}Let $\langle A, \leqslant \rangle$ be a fully
  ordered class that is conditional complete and $\emptyset \neq I \subseteq
  A$ a generalized interval then we have either
  \begin{enumerate}
    \item If $I$ is bounded below and above [so that by conditional
    completeness $\inf (I)$ and $\sup (I)$ exists] we have:
    \begin{enumerate}
      \item If $\inf (I) \in I \wedge \sup (I) \in I$ then $I = [\inf (I),
      \sup (I)]$
      
      \item If $\inf (I) \nin I \wedge \sup (I) \in I$ then $I =] \inf (I),
      \sup (I)]$
      
      \item If $\inf (I) \in I \wedge \sup (I) \nin I$ then $I = [\inf (I),
      \sup (I) [$
      
      \item If $\inf (I) \nin I \wedge \sup (I) \nin I$ then $I =] \inf (I),
      \sup (I) [$
    \end{enumerate}
    \item If $I$ is bounded below and not bounded above [so that by
    conditional completeness $\inf (I)$ exists] we have
    \begin{enumerate}
      \item If $\inf (I) \in I$ then $I = [\inf (I), \infty [$
      
      \item If $\inf (I) \nin I$ then $I =] \inf (I), \infty [$
    \end{enumerate}
    \item If $I$ is bounded above and not bounded below [so that by
    conditional completeness $\sup (I)$ exists] we have
    \begin{enumerate}
      \item If $\sup (I) \in I$ then $I =] - \infty, \sup (I)]$
      
      \item If $\sup (I) \nin I$ then $I =] - \infty, \sup (I) [$
    \end{enumerate}
    \item If $I$ is not bounded above and not bounded below we have $I = A$
  \end{enumerate}
  In other words a generalized interval $I$ has either one of the following
  forms
  \begin{eqnarray*}
    & [a, b] & a, b \in A \text{ and } a \leqslant b\\
    & ] a, b] & a, b \in A \text{ and } a < b\\
    & [a, b [ & a, b \in A \text{ and } a < b\\
    & ] a, b [ & a, b \in A \text{ and } a < b\\
    & [a, \infty [ & \\
    & ] a, \infty [ & \\
    & ] - \infty, a] & \\
    & ] - \infty, a [ & \\
    & ] - \infty, \infty [ & 
  \end{eqnarray*}
\end{theorem}

\begin{proof}
  
  \begin{enumerate}
    \item 
    \begin{enumerate}
      \item Let $z \in I$ then using the definition of a supremum and infinum
      as a upper and lowerbound that \ $\inf (I) \leqslant z \wedge z
      \leqslant \sup (I)$ so that
      \begin{equation}
        \label{eq 2.20.053} I \subseteq [\inf (I), \sup (I)]
      \end{equation}
      As $I$ is a generalized interval and $\sup (I) \in I \wedge \inf (I) \in
      I$we have further that $[\inf (I), \sup (I)] \subseteq I$ which using
      \ref{eq 2.20.053} proves
      \[ I = [\inf (I), \sup (I)] \]
      \item Let $z \in I$ then using the definition of a supremum and infinum
      as upper and lower bound and the fact that $\inf (I) \nin I$ we have
      that $\inf (I) < z \wedge z \leqslant \sup (I)$ so that
      \begin{equation}
        \label{eq 2.21.053} I \subseteq] \inf (I), \sup (I)]
      \end{equation}
      If $z \in] \inf (I), \sup (I)]$ we have $\inf (I) < z \wedge z \leqslant
      \sup (I)$. Using [theorem: \ref{order sup, inf property}] there exists a
      $x \in I$ such that $\inf (I) \leqslant x < z$ or $z \in [x, \sup I]
      \subseteq I$ [as $I$ is a generalized interval and $x, \sup (I) \in I$].
      This proves that $] \inf (I), \sup (I)] \subseteq I$. Combining this
      with [eq: \ref{eq 2.21.053}] proves
      \[ I =] \inf (I), \sup (I)] \]
      \item Let $z \in I$ then using the definition of a supremum and infinum
      as upper and lower bound and the fact that $\sup (I) \nin I$ we have
      that $\inf (I) \leqslant z \wedge z < \sup (I)$ so that
      \begin{equation}
        \label{eq 2.22.053} I \subseteq [\inf (I), \sup (I) [
      \end{equation}
      If $z \in [\inf (I), \sup (I) [$ we have $\inf (I) \leqslant z \wedge z
      < \sup (I)$. Using [theorem: \ref{order sup, inf property}] there exists
      a $y \in I$ such that $z < y \leqslant \sup (I)$ or $z \in [\inf (I), y]
      \subseteq I$ [as $I$ is a generalized interval and $\inf (I), y \in I$].
      This proves that $[\inf (I), \sup (I) [\subseteq I$. Combining this with
      [eq: \ref{eq 2.22.053}] proves
      \[ I = [\inf (I), \sup (I) [ \]
      \item Let $z \in I$ then using the definition of a supremum and infinum
      as upper and lower bound and the fact that $\inf (I) \nin I \wedge \sup
      (I) \nin I$ we have that $\inf (I) < z \wedge z < \sup (I)$ so that
      \begin{equation}
        \label{eq 2.23.053} I \subseteq] \inf (I), \sup (I) [
      \end{equation}
      If $z \in] \inf (I), \sup (I) [$ we have $\inf (I) < z \wedge z < \sup
      (I)$. Using [theorem: \ref{order sup, inf property}] there exists $x, y
      \in I$ such that $\inf (I) \leqslant x < z \wedge z < y \leqslant \sup
      (I)$ or $z \in] x, y [\subseteq I$ [as $I$ is a generalized interval and
      $x, y \in I$]. This proves that $] \inf (I), \sup (I) [\subseteq I$.
      Combining this with [eq: \ref{eq 2.23.053}] gives
      \[ I =] \inf (I), \sup (I) [ \]
    \end{enumerate}
    \item \quad
    \begin{enumerate}
      \item Let $z \in I$ then using the definition of the infinum as a lower
      bound we have that $\inf (I) \leqslant z$ proving that
      \begin{equation}
        \label{eq 2.24.053} I \subseteq [\inf (I), \infty [
      \end{equation}
      If $z \in [\inf (I), \infty [$ then $\inf (I) \leqslant z$. As $I$ is
      not bounded above there exists a $x \in I$ such that $\neg (x \leqslant
      z) \Rightarrowlim_{A \text{ is fully ordered}} z < x$. Hence $z \in
      [\inf (I), x] \subseteq I$ [as $I$ is a generalized interval and $\inf
      (I), x \in I$]. So $[\inf (I), \infty [\subseteq I$ and combining this
      with [eq: \ref{eq 2.24.053}] proves
      \[ I = [\inf (I), \infty [ \]
      \item Let $z \in I$ then using the definition of the infinum as a lower
      bound and the fact that $\inf (I) \nin I$ we have that $\inf (I) < z$
      proving that
      \begin{equation}
        \label{eq 2.25.053} I \subseteq] \inf (I), \infty [
      \end{equation}
      If $z \in] \inf (I), \infty [$ then $\inf (I) < z$. Using [theorem:
      \ref{order sup, inf property}] there exists a $x \in I$ such that $\inf
      (I) \leqslant x < z$. As $I$ is not bounded above there exists a $y \in
      I$ such that $\neg (y \leqslant z) \Rightarrowlim_{A \text{ is fully
      ordered}} z < y$. Hence $z \in] x, y [\subseteq I$ [as $I$ is a
      generalized interval and $x, y \in I$]. So $] \inf (I), \infty
      [\subseteq I$ and combining this with [eq: \ref{eq 2.25.053}] gives
      \[ I =] \inf (I), \infty [ \]
    \end{enumerate}
    \item 
    \begin{enumerate}
      \item Let $z \in I$ then using the definition of the supremum as a upper
      bound we have $z \leqslant \sup (I)$ proving that
      \begin{equation}
        \label{eq 2.26.053} I \subseteq] - \infty, \sup (I)]
      \end{equation}
      If $z \in] - \infty, \sup (I)]$ then $z \leqslant \sup (I)$. As $I$ is
      not bounded below there exists a $x \in I$ such that $\neg (z \leqslant
      x) \Rightarrowlim_{A \text{ is fully ordered}} x < z$. Hence $z \in [x,
      \sup (I)] \subseteq I$ [as $I$ is a generalized interval and $x, \sup
      (I) \in I$]. So $] - \infty, \sup (I)] \subseteq I$ which combined with
      [eq: \ref{eq 2.26.053}] gives
      \[ I =] - \infty, \sup (I)] \]
      \item Let $z \in I$ then using the definition of the supremum as a upper
      bound and the fact that $\sup (I) \nin I$ we have that $z < \sup (I)$
      proving that
      \begin{equation}
        \label{eq 2.27.053} I \subseteq] - \infty, \sup (I) [
      \end{equation}
      If $z \in] - \infty, \sup (I) [$ then $z < \sup (I)$. Using [theorem:
      \ref{order sup, inf property}] there exists a $y \in I$ such that $z < y
      \leqslant \sup (I)$. As $I$ us not bounded below there exists a $x \in
      I$ such that $\neg (z \leqslant x) \Rightarrowlim_{A \text{ is fully
      ordered}} x < z$. Hence $z \in] x, y [\subseteq I$ [as $I$ is a
      generalized interval and $x, y \in I$]. So $] - \infty, \sup (I)
      [\subseteq I$ which combined with [eq: \ref{eq 2.27.053}] proves
      \[ I =] - \infty, \sup (I) [ \]
    \end{enumerate}
    \item If $z \in A$ then as $I$ is not bounded below and not bounded above
    there exists $x, y \in I$ such that $x < z \wedge z < y$ giving $z \in] x,
    y [\subseteq I$ [as $I$ is a generalized interval and $x, y \in I$]. This
    proves that $A \subseteq I$ and as trivially we have $I \subseteq A$ it
    follows that
    \[ I = A \]
  \end{enumerate}
\end{proof}

We have also the opposite of the above theorem

\begin{theorem}
  \label{interval generalized condition (1)}Let $\langle A, \leqslant \rangle$
  be a fully ordered class that is conditional complete then we have given $a,
  b \in A$ that
  \begin{enumerate}
    \item $[a, b]$ is a generalized interval
    
    \item $[a, b [$ is a generalized interval
    
    \item $] a, b]$ is a generalized interval
    
    \item $] a, b [$ is a generalized interval
    
    \item $[a, \infty [$ is a generalized interval
    
    \item $] a, \infty [$ is a generalized interval
    
    \item $] - \infty, a]$ is a generalized interval
    
    \item $] - \infty, a [$ is a generalized interval
    
    \item $A$ is a generalized interval
  \end{enumerate}
\end{theorem}

\begin{proof}
  
  \begin{enumerate}
    \item If $x, y \in [a, b]$ and $z \in [x, y]$ then we have $x \leqslant z
    \Rightarrowlim_{a \leqslant x} a \leqslant z$ and $z \leqslant y
    \Rightarrowlim_{y \leqslant b} z \leqslant b$ proving that $z \in [a, b]$.
    Hence $[x, y] \subseteq [a, b]$ from which it follows that
    \[ [a, b] \text{ is a generalized interval} \]
    \item If $x, y \in [a, b [$ and $z \in [x, y]$ then we have $x \leqslant z
    \Rightarrowlim_{a \leqslant x} a \leqslant z$ and $z \leqslant y
    \Rightarrowlim_{y < b} z < b$ proving that $z \in [a, b [$. Hence $[x, y]
    \subseteq [a, b [$ from which it follows that
    \[ [a, b [\text{ is a generalized interval} \]
    \item If $x, y \in] a, b]$ and $z \in [x, y]$ then we have $x \leqslant z
    \Rightarrowlim_{a < x} a < z$ and $z \leqslant y \Rightarrowlim_{y
    \leqslant b} z \leqslant b$ proving that $z \in] a, b]$. Hence $[x, y]
    \subseteq] a, b]$ from which it follows that
    \[ ] a, b] \text{ is a generalized interval} \]
    \item If $x, y \in] a, b [$ and $z \in [x, y]$ then we have $x \leqslant z
    \Rightarrowlim_{a < x} a < z$ and $z \leqslant y \Rightarrowlim_{y < b} z
    < b$ proving that $z \in] a, b [$. Hence $[x, y] \subseteq] a, b [$ from
    which it follows that
    \[ ] a, b [\text{ is a generalized interval} \]
    \item If $x, y \in [a, \infty [$ and $z \in [x, y]$ we have $x \leqslant z
    \Rightarrowlim_{a \leqslant x} a \leqslant z$ proving that $z \in [a,
    \infty [$. Hence $[x, y] \subseteq [a, \infty [$from which it follows that
    \[ [a, \infty [\text{ is a generalized interval} \]
    \item If $x, y \in] a, \infty [$ and $z \in [x, y]$ we have $x \leqslant z
    \Rightarrowlim_{a < x} a < z$ proving that $z \in] a, \infty [$. Hence
    $[x, y] \subseteq] a, \infty [$ from which it follows that
    \[ ] a, \infty [\text{ is a generalized interval} \]
    \item If $x, y \in] - \infty, a]$ and $z \in [x, y]$ we have $z \leqslant
    y \Rightarrowlim_{y \leqslant a} z \leqslant a$ proving that $z \in] -
    \infty, a]$. Hence $[x, y] \subseteq] - \infty, a]$ from which it follows
    that
    \[ ] - \infty, a] \text{ is a generalized interval} \]
    \item If $x, y \in] - \infty, a [$ and $z \in [x, y]$ we have $z \leqslant
    y \Rightarrowlim_{y < a} z < a$ proving that $z \in] - \infty, a [$. Hence
    $[x, y] \subseteq] - \infty, a [$ from which it follows that
    \[ ] - \infty, a [\text{ is a generalized interval} \]
    \item If $x, y \in A$ then trivially $[x, y] \subseteq A$ hence $A$ is a
    generalized interval.
  \end{enumerate}
\end{proof}

\

\begin{theorem}
  \label{generalized intervals and boundaries}Let $\langle A, \leqslant
  \rangle$ be a totally ordered class then
  \begin{enumerate}
    \item $\forall a, b \in A$ we have if $] - \infty, a] =] - \infty, b]$
    then $a = b$
    
    \item $\forall a, b \in A$ we have if $] - \infty, a [=] - \infty, b [$
    then $a = b$
    
    \item $\forall a, b \in A$ we have if $[a, \infty [= [b, \infty [$ then $a
    = b$
    
    \item $\forall a, b \in A$ we have if $] a, \infty [=] b, \infty [$ then
    $a = b$
    
    \item $\forall a, b, c, d$ with $[a, b] \neq \emptyset$ then we have if
    $[a, b] = [c, d]$ then $a = c \wedge b = d$
    
    \item $\forall a, b, c, d$ with $] a, b] \neq \emptyset$ then we have if
    $] a, b] =] c, d]$ then $a = c \wedge b = d$
    
    \item $\forall a, b, c, d$ with $[a, b [\neq \emptyset$ then we have if
    $[a, b [= [c, d [$ then $a = c \wedge b = d$
    
    \item $\forall a, b, c, d$ with $] a, b [\neq \emptyset$ then we have if
    $] a, b [=] c, d [$ then $a = c \wedge b = d$
  \end{enumerate}
\end{theorem}

\begin{proof}
  
  \begin{enumerate}
    \item As $a \in] - \infty, a] =] - \infty, b]$ we have $a \leqslant b$,
    further from $b \in] - \infty, b] =] - \infty, a]$ we have $b \leqslant
    a$, hence $a = b$.
    
    \item As $A$ is totally ordered we have for $a, b$ either
    \begin{description}
      \item[$a = b$] then (2) is proved
      
      \item[$a < b$] then $a \in] - \infty, b [=] - \infty, a [$ giving the
      contradiction $a < a$
      
      \item[$b < a$] then $b \in] - \infty, a [=] - \infty, b [$ giving the
      contradiction $b < b$
    \end{description}
    so the only valid conclusion is that $a = b$
    
    \item As $a \in [a, \infty [= [b, \infty [$ we have $b \leqslant a$,
    further from $b \in [b, \infty [= [a, \infty [$ we have $a \leqslant b$,
    hence $a = b$
    
    \item As $A$ is totally ordered we have for $a, b$ either
    \begin{description}
      \item[$a = b$] then (4) is proved
      
      \item[$a < b$] then $b \in] a, \infty [=] b, \infty [$ giving the
      contradiction $b < b$
      
      \item[$b < a$] then $a \in] b, \infty [=] a, \infty [$ giving the
      contradiction $a < a$
    \end{description}
    so the only valid conclusion is that $a = b$
    
    \item As $[a, b] = [c, d] \neq \varnothing$ it follows from [theorem:
    \ref{interval condition to be empty}] that $a \leqslant b$ and $c
    \leqslant d$ so that $a, b \in [a, b] = [c, d] \Rightarrow c \leqslant a
    \wedge b \leqslant d$ and $c, d \in [c, d] = [a, b] \Rightarrow a
    \leqslant c \wedge d \leqslant b$ proving that $a = c \wedge b = d$.
    
    \item As $] a, b] =] c, d] \neq \varnothing$ it follows from [theorem:
    \ref{interval condition to be empty}] that $a < b \wedge c < d$ hence $b
    \in] a, b] \wedge d \in] c, d]$. From $b \in] a, b] =] c, d]$ we have $b
    \leqslant d$ and from $d \in] c, d] =] a, b]$ we have $d \leqslant b$
    hence we have
    \[ b = d \]
    Assume that $a < c$ then as $c < d = b$ we have $c \in] a, b] =] c, d]$ so
    that $c < c$ a contradiction, hence we must have
    \begin{equation}
      \label{eq 3.82.146} c \leqslant a
    \end{equation}
    Assume that $c < a$ then as $a < b = d$ we have $a \in] c, d] =] a, b]$ so
    that $a < a$ a contradiction hence we must have $a \leq c$ which combined
    with [eq: \ref{eq 3.82.146}] proves that
    \[ a = c \]
    \item As $[a, b [= [c, d [\neq \varnothing$ we have by [theorem:
    \ref{interval condition to be empty}] that $a < b \wedge c < d$ so that $a
    \in [a, b [\wedge c \in [c, d [$. From $a \in [a, b [= [c, d [$ it follows
    that $c \leqslant a$ and from $c \in [c, d [= [a, b [$ we have $a
    \leqslant c$ hence we have
    \[ a = c \]
    Assume that $b < d$ then as $c = a < b$ we have $b \in [c, d [= [a, b [$
    so that $b < b$ a contradiction, hence
    \begin{equation}
      \label{eq 3.83.146} b \leqslant d
    \end{equation}
    Assume that $d < b$ then as $a = c < d < b$ we have $d \in [a, b [= [c, d
    [$ so that $d < d$ a contradiction, hence $d \leqslant b$ which combined
    with [eq: \ref{eq 3.83.146}] proves
    \[ b = d \]
    \item As $\emptyset \neq] a.b [=] c, d [$ there exists a $x \in A$ such
    that $x \in] a, b [=] c, d [$ giving
    \begin{equation}
      \label{eq 3.84.146} a < x < b \wedge c < x < d
    \end{equation}
    Assume that $a < c$ then by [eq: \ref{eq 3.84.146}] $a < c < x < b$ so
    that $c \in] a, b [=] c, d [$ leading to the contradiction $c < c$ hence
    \begin{equation}
      \label{eq 3.85.146} c \leqslant a
    \end{equation}
    Assume that $c < a$ then by [eq: \ref{eq 3.84.146}] $c < a < x < d$ so
    that $a \in] c, d [=] a, b [$ leading to the contradiction $a < a$ hence
    $a \leqslant c$ which combined with [eq: \ref{eq 3.85.146}] gives
    \[ a = c \]
    Assume that $b < d$ then by [eq: \ref{eq 3.84.146}] $c < x < b < d$ so
    that $b \in] c, d [=] a, b [$ leading to the contradicton $b < b$, hence
    \begin{equation}
      \label{eq 3.86.146} d \leqslant b
    \end{equation}
    Assume that $d < b$ then by \ [eq: \ref{eq 3.84.146}] $a < x < d < b$ so
    that $d \in] a, b [=] c, d [$ leading to the contradiction $d < d$ proving
    that $b \leqslant d$. Combining this with [eq: \ref{eq 3.86.146}] results
    in
    \[ b = d \]
  \end{enumerate}
\end{proof}

\

\

\

\

\

\

\

\

\

\

\

\

\

\

\end{document}
